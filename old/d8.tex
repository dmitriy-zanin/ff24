\documentclass[a4paper]{amsart}
\usepackage{color}
\usepackage[sort]{natbib}
\usepackage{fullpage}

\begin{document}

I believe that my recent achievements in the area of noncommutative analysis and its applications and my publication record undoubtedly demonstrate my outstanding research abilities and potential. This has been recently confirmed by DECRA award (2015--2018). It is worth to point out that my DECRA application was ranked first among all successful applications in Mathematics in 2014.

I started my research in noncommutative functional analysis at Flinders University as a PhD student. Since then my contribution to this research field include my PhD Thesis  and a series of publications including research monograph \lq\lq Singular traces. Theory and Applications\rq\rq. My research articles are published in highly respected journals such as Journal f\"ur die Reine und Angewandte Mathematik (Crelle's Journal), Advances in Mathematics, Journal of Functional Analysis, Journal of Spectral Theory, Pacific Journal of Mathematics and my publication rate is outstanding. My PhD Thesis received the highest possible ranking from the referees reports. One of the referees wrote \lq\lq this is the best PhD thesis I ever refereed during my whole career\rq\rq. My subsequent work has appeared (or will appear) at the first class international mathematical journals. It is also worth to mention that my work on singular traces has found deep and non-trivial applications in the realm of Noncommutative (Differential) Geometry, in particular jointly with A. Carey, A. Rennie and F. Sukochev, I have substantially extended and strengthened Connes Character Theorem. I believe that my research and publication record is spectacular and that breadth of my research spanning noncommutative analysis, noncommutative geometry and noncommutative probability theory acknowledges my outstanding research capabilities.

Some of the principal achievements of my work now follow.  The primary objective of my PhD study was two-fold. One direction was studying orbits in symmetric function and sequence spaces. The particular question of interest was whether orbit is a norm-closed convex hull of its extreme points. The other direction concerned Khinchine-type inequalities. The principal tool in the studies of orbits is the vastly extended Birkhoff theorem on doubly stochastic matrices. By employing this technique in my PhD Thesis I was able to (i) develop a unified approach to orbits on symmetric function and sequence spaces; (ii) significantly broaden the setting of applicability of earlier results; (iii) to determine the conditions that are necessarily and sufficient for the affirmative answer to the above question. The principal tool in the studies of probabilistic inequalities (such as Khinchine inequality) is so-called Kruglov operator (an integration over the Poisson process). Using Kruglov operator, I radically extended the area where Khinchine-type inequalities hold true. Later, I published the results of my PhD research in the series of papers [21,32--35]. One of these papers is published jointly with Nigel Kalton, a late giant of functional analysis, whose interest to my research was a source of further inspiration.

Basing on the advances in the theory of symmetric function spaces made in my PhD Thesis I obtained several profound results in noncommutative analysis: one of them resolves a long standing question of existence of continuous traces on Banach ideals in $\mathcal{L}(H)$ (the $*-$algebra of all bounded operators on a separable Hilbert space $H$). It was already known to the experts that traces (even without continuity assumption) do not exist on an ideal which can be obtained from the couple $(\mathcal{L}_p,\mathcal{L}_{\infty}),$ $p>1,$ by interpolation. Using ideas from my PhD thesis, I proposed to study a weaker condition:
\begin{equation}\label{my condition}
\frac1n\|A^{\oplus n}\|_{\mathcal{I}}\to0\mbox{ as }n\to\infty\mbox{ for every operator }A\mbox{ in the ideal }\mathcal{I}.
\end{equation}
The condition \eqref{my condition} holds for every interpolation space as above and prevents the existence of a continuous trace on the Banach ideal $\mathcal{I}.$ In my paper [10] (joint with my former supervisor F. Sukochev, published in Crelle's Journal), the converse assertion is proved; namely, condition \eqref{my condition} holds for every Banach ideal $\mathcal{I}$ without continuous traces. In the same paper, I constructed (under the assumption that \eqref{my condition} fails) a positive continuous trace which respects the Hardy-Littlewood submajorization (and also the one that does not). 

It is a standard result in linear algebra that a trace of a matrix depends only on its eigenvalues. A corresponding result for the standard trace on the trace class ideal $\mathcal{L}_1$ is radically harder (it was resolved in the affirmative by V. Lidskii only in 1959). More precisely, the following formula holds.
$${\rm Tr}(A)={\rm Tr}({\rm diag}(\lambda(A)))\mbox{ for every operator }A\mbox{ in the ideal }\mathcal{L}_1.$$
Here, ${\rm diag}(\lambda(A))$ is a diagonal operator with eigenvalues of $A$ on the diagonal (repeated with multiplicities). In his famous 1990 paper, Pietsch asked  whether the same is true for an arbitrary trace on an arbitrary\footnote{An ideal is assumed to be a purely algebraic object, without any topology on it.} ideal in $\mathcal{L}(H).$ He characterized this question as \lq\lq extremely difficult\rq\rq. A partial answer to this question was given by Kalton in his 1998 paper. In my paper [9] (joint with F. Sukochev, published in Advances in Mathematics), a class of ideals closed with respect to the logarithmic submajorization is introduced. In this paper, it is proved that if an ideal $\mathcal{I}$ is closed with respect to the logarithmic submajorization, then
$$\varphi(A)=\varphi({\rm diag}(\lambda(A)))$$
for every operator $A\in\mathcal{I}$ and for every trace $\varphi$ on $\mathcal{I}.$ Hence, for the mentioned class of ideals Pietsch's question is answered in the affirmative. Conversely, if the ideal $\mathcal{I}$ is not closed with respect to the logarithmic submajorization, then there exists an operator $A\in\mathcal{I}$ such that ${\rm diag}(\lambda(A))\notin\mathcal{I}$ and the question is answered in the negative.

Classical result due to Schur states that every matrix is unitarily conjugate to an upper-triangular one. Thus, it is a sum of a normal matrix (diagonal part) and a nilpotent matrix (strictly upper-triangular part). A corresponding result for compact operators in $\mathcal{L}(H)$ is proved by Ringrose in his 1962 paper. Namely, he proved that every compact operator is a sum of a normal one and a quasi-nilpotent one. It is natural to ask whether a similar result holds in the setting of type II von Neumann algebras. In my paper [8] (joint with K. Dykema and F. Sukochev, published in Crelle's Journal), it is proved that every operator $T$ in type II$_1$ von Neumann algebra can be written as $T=N+Q,$ where $N$ is normal and $Q$ is \lq\lq almost\rq\rq nilpotent in the following sense
$$|Q^n|^{\frac1n}\to0\mbox{ as }n\to\infty\mbox{ in the strong operator topology.}$$

Connes Character Theorem is of paramount importance in the noncommutative geometry. It expresses a certain cocycle (called Chern character) in terms of Dixmier traces. This cocycle is then used to explicitly compute indices, Euler characteristics and other topological invariants via smooth expressions. In my paper [7] (joint with A. Rennie, A. Carey and F. Sukochev, published in Journal of Spectral Theory), a completely new approach to Connes Character Theorem is proposed. The result is now proved for an arbitrary trace (not only a Dixmier trace as in Connes' original approach).

A 50-year old problem due to Krein asks whether Lipschitz functions are operator Lipschitz on a given Banach ideal. For Schatten ideals $\mathcal{L}_p,$ $1<p<\infty,$ the assertion is known to be true. Namely, Potapov and Sukochev (Acta Math., 2011) proved that
$$\|f(A)-f(B)\|_p\leq c_p\|f'\|_{\infty}\|A-B\|_p$$
for every couple of self-adjoint operators $(A,B)$ such that $A-B\in\mathcal{L}_p.$ For $p=1$ (or for $p=\infty$), a similar result is false. Kato and Davies proved that even absolute value fails to be operator Lipschitz in $\mathcal{L}_1.$ Nazarov and Peller asked whether the following replacement
\begin{equation}\label{np conj}
\|f(A)-f(B)\|_{1,\infty}\leq c_{abs}\|f'\|_{\infty}\|A-B\|_1,
\end{equation}
holds, where $\mathcal{L}_{1,\infty}$ is the principal ideal generated by the operator ${\rm diag}(\{1,\frac12,\frac13,\cdots\})$ equipped with a natural quasi-norm. Very recently, I proved (in collaboration with M. Caspers, D. Potapov and F. Sukochev) that, indeed, \eqref{np conj} holds true. The manuscript [1] is accepted for publication in the American Journal of Mathematics. The result has been verified by the leading experts (G.Pisier, Q.Xu, V.Peller, M. Junge and many others) and is, by now, widely accepted as an outstanding achievement in operator calculus. The key machinery of our proof was developed in our earlier research papers.

Inequalities proved by Johnson and Schechtman relate the norm of the sum of independent random variables to that of their disjoint copies. Namely, if $E\supset L_p,$ $p<\infty,$ is a symmetric function space on $(0,1)$ and if $x_k\in E$ are mean zero independent random variables, then
$$\|\sum_{k\geq0}x_k\|_E\approx_E\|\bigoplus_{k\geq0}x_k\|_{Z_E^2},$$
where $Z_E^2$ is a certain symmetric function space on $(0,\infty).$ Astashkin and Sukochev (Isr. J. Math., 2005) proposed another method of proving Johnson-Schechtman inequalities by means of the so-called Kruglov operator (which is an integration with respect to the Poisson process). In my paper (joint with S. Astashkin and F. Sukochev, published in Pacific Journal of Mathematics), it is proved that Johnson-Schechtman inequalities hold in a symmetric (quasi-Banach) function space if and only if Kruglov operator is bounded on that space. It is tempting to extend the technique related to the Kruglov operator to the domain of the noncommutative probability. In my paper (joint with F. Sukochev, published in Journal of Functional Analysis), I constructed an analogue of the Kruglov operator in the Free Probability Theory. It appears that Kruglov operator behaves better in the setting of free probability than in the setting of a classical probability; and that Johnson-Schechtman inequalities are always true.

This breakthrough has inspired a question: what sort of independence suffices to guarantee the Johnson-Schechtman inequalities for sufficiently rich class of symmetric spaces?  I proposed this investigation as a subject for my DECRA application. This application was approved by the ARC in November 2014. After that, an international team based at UNSW (Y. Jiao, F. Sukochev and myself) obtained a significant achievement: if symmetric space is an interpolation space for the couple $(L_p,L_q),$ $1<p<q<\infty,$ then Johnson-Schechtman inequalities hold true in $E$ for an arbitrary sequence of mean zero random variables independent in the sense of Junge and Xu. This paper [6] is published in the Journal of the London Mathematical Society.

Furthermore, I have struck up a professional relationship with a genuine legend of modern mathematics, Fields Medal awardee and creator of noncommutative geometry, Professor Alain Connes, and this has led to an on-going joint project with him. This project has already resulted in $2$ joint papers ([3] and [13]). I delivered a keynote talk in the international conference celebrating Connes's 70'th birthday. Only those mathematicians whose contribution to the Noncommutative Geometry is substantial and universally acknowledged were invited to that conference.

It is not an exaggeration to state that I am already an outstanding and versatile mathematician whose full potential is enormous and whose appointment will contribute to the 2025 strategy of attracting the best researchers in enabling and fundamental sciences.



\end{document}
