\documentclass{article}
\usepackage{amsmath,amsfonts,cite,color}
\usepackage[a4paper,margin=2cm]{geometry}

\newtheorem{thm}{Theorem}

\renewcommand{\baselinestretch}{0.95}
\newcommand{\archeading}[1]{\vspace{.3cm} \noindent{\bfseries #1} \vspace{.1cm}   }
\renewcommand{\refname}{{\normalsize References}}


\begin{document}

\parindent=20pt
\pagestyle{empty}
\newpage


PROJECT TITLE
AIMS AND BACKGROUND
FUTURE FELLOWSHIP CANDIDATE
PROPOSED PROJECT QUALITY AND INNOVATION
FEASIBILITY AND STRATEGIC ALIGNMENT
BENEFIT AND COLLABORATION
COMMUNICATION OF RESULTS
MANAGEMENT OF DATA
REFERENCES
ACKNOWLEDGEMENTS (IF REQUIRED)


\archeading{Project Title} Actions of quantum groups and noncommutative manifolds.

\bigskip\archeading{Aims and Background}

\paragraph*{Broad Aim} This proposal covers a novel viewpoint on noncommutative Riemannian manifolds and aims bringing together an international team of experts with a new approach to fundamental and long-standing problems in geometry and harmonic analysis.

There exist two theories of Riemannian manifolds: the classical one (which treats manifold as a locally Euclidean space) and the modern one introduced by Connes (which treats manifold as a spectral triple). A triple consists of $\ast-$algebra $\mathcal{A}$ represented on a Hilbert space $H$ and an unbounded self-adjoint operator $D$ on $H.$ It is easy to compare both notions of a manifold: in the classical case, one lets $\mathcal{A}=C^{\infty}(X)$ (algebra of smooth functions on $X$), $H=L_2\Omega(X)$ (square-integrable forms on $X$) and lets $D$ to be a Hodge-Dirac operator. Famous Connes Reconstruction Theorem tells us that so-defined spectral triple captures all the geometric information about a manifold.

Manifolds are often equipped with a Lie group action. For instance, an orthogonal group acts on the real sphere. Moreover, rotations are exactly those automorphisms of the sphere which commute with Laplace-Beltrami operator. To take group action into account, one requires that Dirac-type operator commutes with the action of the group.

In the noncommutative realm, we have an action of a given Lie group (or even quantum group) on a $\ast-$algebra. Natural noncommutative analogue of "commutes with a group action" is an equivariance property.



The CI will investigate actions of groups (more broadly, quantum groups) on noncommutative manifolds. Those actions compel an equivariant Dirac operator which captures all the geometric information about the manifold (famous Connes Reconstruction Theorem). 


%  which naturally extends the classical independence in the commutative probability theory as well as the free independence developed by Voiculescu in the noncommutative probability theory; develop new approaches to the centrally important concept of cumulant (an analogue of the Fourier transform); find stable random variables; during the project further develop the Law of Large Numbers and the Central Limit Theorem.
%
%On a second front the project will investigate problems in the noncommutative analysis related to the isomorphic embeddings of the Banach spaces. The investigator will introduce the analogue of the Kruglov operator.
%
%This proposal suggests a new and innovative approach to the notion of independence and provides a new path to the solution of fundamental and long-standing problems in the probability theory.
%
%Broadly speaking, there are two distinct theories of probability: the classical theory of random variables (that is of measurable functions on a probability measure space) and that of free probability originating mainly in works of D. Voiculescu and his students and collaborators approximately 40 years ago. In both theories, the notion of independence plays an absolutely fundamental, central role. In fact, one could view a probability theory as a part of general integration theory where elements are subjected to a certain independence condition. The classical notion of independence, which forms the basis of the classical probability theory is replaced in Voiculescu’s theory with a notion of freeness, which carries deep connections with the theory of free groups and their representations (and, hence, various applications in operator algebra theory), random matrices and Wigner’s distribution law (and, hence, non-trivial connections to modern mathematical physics, which ultimately made success and interest to Voiculescu’s free probability theory so wide spread). It is not easy to compare both notions of independence, even though from our point of view such a comparison appears to be paramount to any attempt to unify both theories into a single “theory of probability”. This proposal is the first meaningful attempt of such a unified theory: we aim to present a single viewpoint and a single theory of independence which encompasses, as two extreme cases, the classical and free probability theories and reveals a number of intermediate theories, depending on certain numerical parameters. For various values of such parameters we also recover and, in fact, provide a deeper insight into Nica’s theory [7] of so-called q−convolutions which interpolates between the usual convolution and the free convolution. That theory was a first serious and well-aimed (albeit, eventually unsuccessful) attempt to exhibit various independence laws which range between two cases q = −1 (the free probability case) and q = 1 (the classical case). Another attempt, closely linked to Nica’s work, was made in the paper of Bozeiko and Speicher [1], where the authors discuss q-independent random variables. 
%
%In this project we introduce a new conceptual approach to the notion of independence and suggest new tools for its study. Our approach naturally extends the classical notion of independence in commutative probability theory as well as that of free independence developed by Voiculescu in the noncommutative probability theory. We will develop a new approach to the centrally important concept of cumulant (an analogue of the Fourier transform) and define a notion of stable random variables naturally extending the classical and free versions of that notion. We intend to prove a new (noncommutative) Law of Large Numbers and a new (again, noncommutative) Central Limit Theorem. What will emerge as a result of our work is a new theory of probability, uniting and extending the classical theory, free probability theory, Nica’s theory of q-cumulants and Bozeiko-Speicher’s theory of q-random variables and a number of other scattered results where various authors encounter effects which are not covered by the existing theories and attempt to explain such effects ad hoc.

Our proposal gives a firm and unified foundation to many such attempts.


\paragraph*{General Background}

Operator algebras provide a natural edifice to many areas of classical and modern Mathematics. Among them, a particular role is played by $C^{\ast}-$algebras (uniformly closed $\ast-$subalgebras in $B(H)$) and von Neumann algebras (weakly closed subalgebras in $B(H)$). These additional structures make the algebras suitable for studying noncommutative analysis. 

Gelfand-Naimark theorem delivers a duality between category of compact topological spaces and that of commutative $C^{\ast}-$algebras. So, topology is incorporated into a theory of $C^{\ast}-$algebras. A similar result due to von Neumann theorem provides a duality between category of measure spaces and that of commutative von Neumann algebras. In other words, measure theory is a part of von Neumann algebras theory.

A conventional wisdom suggests now a dictionary between classical and quantum worlds.  While topology and measure theory are supplied with natural quantum counter-parts,  geometry was left behind for a long time. Starting from von Neumann himslef, people tried to find a quantum analogue of geometry (that is, to construct a functor from useful geometric categories to treatable quantum categories). So far, the most successful attempt is due to Connes who introduced the notion of spectral triple and promoted it as an analogue of Riemannian manifold.  

Notion of a Riemannian manifold is a pillar of geometry. Usually, a manifold is defined as a topological space in which every point admits a neighborhood which is homeomorphic to a linear space. A manifold equipped with a smooth metric tensor is called Riemannian.

With every Riemannian manifold, one can associate (i) algebra $\mathcal{A}=C^{\infty}(X)$ of all smooth functions on $X;$ (ii) Hilbert space $H=L_2\Omega(X)$ of all square-integrable forms on $X;$ (iii) Hodge-Dirac operator $D$ on $L_2\Omega(X).$ The following properties of the triple $(\mathcal{A},H,D)$ are crucial:
\begin{enumerate}
\item if $\pi:\mathcal{A}\to B(H)$ is the representation of $\mathcal{A}$ by multiplication operators, then the commutator $[D,\pi(a)]$ is bounded for all $a\in\mathcal{A};$
\item the operator $\pi(a)(D+i)^{-1}$ is compact and its singular values decrease as $k^{-\frac1d},$ where $d={\rm dim}(X).$
\end{enumerate}
Recall that $-D^2$ is a Hodge-Laplace operator and its component acting on $0$ order forms is Laplace-Beltrami operator.

A triple $(\mathcal{A},H,D)$ satisfying the conditions above is called spectral. Celebrated Reconstruction Theorem due to Connes dictates that every spectral triple with commutative $\mathcal{A}$ (satisfying few natural conditions) comes from a $d-$dimensional Riemannian manifold $X.$

Hodge-Dirac operator allows an easy description of isometries of the manifold $X.$ If $\gamma:X\to X$ is an isometry and if $U:L_2\Omega(X)\to L_2\Omega(X)$ is a composition (with $\gamma$) operator, then $U$ is unitary, $U$ preserves $\mathcal{A}$ and commutes with $D.$ Conversely, every operator $U$ as above comes from an isometry $\gamma:X\to X.$


As is typical in applications, manifolds are equipped with isometric action of Lie groups. 


------------------------------------------

After the seminal work of Weyl, it become a tradition to measure geometric (and often topological) quantities in terms of heat semi-group expansion. If $(X,g)$ is a compact Riemannian manifold, then resolvent of $\Delta_g$ (Laplace-Beltrami operator) is compact. Moreover, heat semi-group $e^{t\Delta_g}$ belongs to the trace class for every $t>0.$ For every $f\in C^{\infty}(X),$ Minakshisundaram-Plejel theorem asserts an existence of an asymptotic expansion
\begin{equation}\label{heat asymptotic}
{\rm Tr}(M_fe^{t\Delta_g})\approx (4\pi t)^{-\frac{d}{2}}\cdot \sum_{n\geq0}a_n(f)t^n,\quad t\downarrow0.
\end{equation}
Here, $d$ is the dimension of $X.$ Here,
\begin{equation}\label{normality of coefficients}
a_k(f)=\int_X A_k\cdot f d{\rm vol}_g,
\end{equation}
where ${\rm vol}_g$ is the standard volume element on $X$ given in local coordinates by the formula
$$d{\rm vol}_g=({\rm det}(g))^{\frac12}(x)dx.$$

As established by Weyl, $A_0=1.$ Further computations show that
$$A_1=\frac16 R,$$
where $R$ is the scalar curvature.

For non-commutative manifolds, we no longer have coordinates. Suppose that a non-commutative manifold is equipped with a Laplacian $\Delta.$ The task now is to
\begin{enumerate}
\item find a normal state $h$ on $\mathcal{A}''$ such that
$${\rm Tr}(\pi(x)e^{t\Delta})\approx (4\pi t)^{-\frac{d}{2}} h(x),\quad t\downarrow0,$$
for every $x\in\mathcal{A}'';$
\item prove a non-commutative version of Minakshisundaram-Plejel theorem, i.e. generalise the formula \eqref{heat asymptotic} as follows
$${\rm Tr}(\pi(x)e^{t\Delta})\approx (4\pi t)^{-\frac{d}{2}} \sum_{n\geq0}a_n(x)t^n,\quad t\downarrow0,$$
for every $x\in\mathcal{A}'';$
\item prove normality of coefficients as in \eqref{normality of coefficients} with respect to the volume state, i.e., show that
$$a_k(x)=h(xA_k),\quad x\in\mathcal{A}'',$$
for every $k\geq0$ and for some $A_k\in\mathcal{A};$ 
\item compute respective $A_k;$
\end{enumerate}
When this mission is accomplished, one can {\it define} a scalar curvature of a non-commutative manifold by setting $R=6A_1.$



---------------------------------------------


One of the fundamental tools in noncommutative geometry is the Chern character. The Connes Character Formula (also known as the Hochschild character theorem) provides an expression for the class of the Chern character in Hochschild cohomology, and it is an important    tool in the computation of the Chern character. The formula has been applied to many areas     of noncommutative geometry and its applications: such as the local index formula \cite{Connes-Moscovici}, the spectral characterisation of manifolds \cite{Connes-reconstruction} and recent work in mathematical physics \cite{Connes-Chamseddine-Mukhanov-quanta-of-geometry-2015}.

In its original formulation, \cite{Connes-original-spectral-1995}, the Character Formula is stated as follows: Let $(\mathcal{A},H,D)$ be a $p$-dimensional compact spectral triple     with (possibly trivial) grading $\Gamma.$ By the definition of a spectral triple, for all $a \in \mathcal{A}$ the commutator $[D,a]$ has an extension to a bounded operator $\partial(a)$ on $H.$ Assume for simplicity that $\ker(D)=\{0\}$ and set $F={\rm sgn}(D).$ For all $a \in \mathcal{A}$ the commutator $[F,a]$ is a compact operator in the weak Schatten ideal $\mathcal{L}_{p,\infty}.$ 

Consider the following two linear maps on the algebraic tensor power $\mathcal{A}^{\otimes(p+1)}$,
    defined on an elementary tensor $c = a_0\otimes a_1\otimes \cdots \otimes a_p \in \mathcal{A}^{\otimes(p+1)}$ by setting
$$\mathrm{Ch}(c) := \frac{1}{2}\mathrm{Tr}(\Gamma F[F,a_0][F,a_1]\cdots[F,a_p]),\quad \Omega(c) := \Gamma a_0\partial a_1\partial a_2\cdots \partial a_p.$$
Then the Connes Character Formula states that if $c$ is a Hochschild cycle (as defined in Section \ref{hochschild subsection}) then
    \begin{equation*}
        \mathrm{Tr}_\omega(\Omega(c)(1+D^2)^{-p/2}) = \mathrm{Ch}(c)
    \end{equation*}
    for every Dixmier trace $\mathrm{Tr}_\omega$. In other words, the multilinear maps $\mathrm{Ch}$ and $c \mapsto \mathrm{Tr}_\omega(\Omega(c)(1+D^2)^{-p/2})$ define
    the same class in Hochschild cohomology.
    
There has been great interest in generalising the tools and results of noncommutative geometry to the \lq\lq non-compact\rq\rq (i.e., non-unital) setting. The definition of a spectral triple associated to a non-unital algebra originates with Connes \cite{Connes-reality}, was furthered by the work of Rennie \cite{Rennie-2003, Rennie-2004}     and Gayral, Gracia-Bond\'ia, Iochum, Sch\"ucker and Varilly \cite{gayral-moyal}. Earlier, similar ideas appeared in the work of Baaj and Julg \cite{Baaj-Julg}. Additional contributions to this area were made by Carey, Gayral, Rennie and Sukochev\cite{CGRS1,CGRS2}. The conventional definition  of a non-compact spectral triple is to replace the condition that $(1+D^2)^{-1/2}$ be compact with the assumption that for all $a \in \mathcal{A}$ the operator $a(1+D^2)^{-1/2}$ is compact.
    
This raises an important question: is the Connes Character Formula true for locally compact spectral triples? 
    
    In this project we aim to answer this question, using recently developed techniques of operator integration.
    

------------------------------------------

\paragraph*{Specific aims} There are {\bf put number} specific aims.


\noindent{\bf Aim 1:} Investigate when Chern character provides an asymptotic expansion for the heat semi-group. More precisely, when 
\begin{equation}\label{heat eq}
{\rm Tr}(\Omega(c)e^{-s^2D^2})={\rm Ch}(c)s^{-p}+O(s^{1-p}),\quad s\downarrow0,
\end{equation}
for every Hochschild cycle $c\in\mathcal{A}^{\otimes (p+1)}?$ This question is closely related (albeit, not equivalent) to the question about analyticity of the $\zeta-$function. Show that the function (defined a priori for $\Re(z)>p$)
$$z\to {\rm Tr}(\Omega(c)(1+D^2)^{-\frac{z}{2}})$$
admits an analytic extension to the half-plane $\Re(z)>p-1$ so that
$$\lim_{z\to p}(z-p){\rm Tr}(\Omega(c)(1+D^2)^{-\frac{z}{2}})=p{\rm Ch}(c)?$$


\noindent{\bf Aim 2:} The purpose of the Connes Character Formula is to compute the Hochschild class of the Chern character by a "local" formula, here stated in terms of singular traces. Show that
\begin{equation}\label{super main eq}
\varphi(\Omega(c)(1+D^2)^{-\frac{p}{2}})={\rm Ch}(c).
\end{equation}
for every (normalised) trace $\varphi$ on $\mathcal{L}_{1,\infty}$ and for every Hochschild cycle $c\in\mathcal{A}^{\otimes (p+1)}.$ Equivalently, show that
$$\sum_{k=0}^n \lambda(k,\Omega(c)(1+D^2)^{-p/2}) = \mathrm{Ch}(c)\log(n)+O(1),\quad n\to\infty.$$
Here, $\lambda(k,T)$ means the $k-$th eigenvalue (counted with algebraic multiplicity) of a compact operator $T.$

------------------------------------------


%
%
%A von Neumann algebra is an algebra of operators on Hilbert space with additional structures making it suitable for studying noncommutative analysis. A von Neumann algebra is said to be {\it finite} if it has a tracial state. Finite von Neumann algebras are typically infinite dimensional and they are useful in many disciplines like mathematical physics, free probability theory, measurable dynamical systems and noncommutative geometry.
%
%We propose to extend the concept of independence to the setting of a noncommutative probability theory.  In the latter theory, probability space becomes a von Neumann algebra $\mathcal{M}$ equipped with a faithful normal tracial state $\tau.$ Bounded random variables are replaced with the self-adjoint elements in $\mathcal{M}$ and unbounded random variables are the self-adjoint operators affiliated to $\mathcal{M}$ (that is, those commuting with the commutant of $\mathcal{M}$). Some authors do not require random variables to be self-adjoint. This choice has its advantages because random variables form an algebra, but we prefer here to follow physical tradition which imposes the restriction that every \lq\lq physically relevant\rq\rq object is hermitian.
%
%The joint distribution used in the definition of independence in the classical setting does not make much sense in the noncommutative setting. The approach using mixed moment seems to be much more reasonable. The definition of independence using characteristic function will be employed as an instrument, rather than a concept.
%
%Firstly, we introduce a concept of \lq\lq tensor independence\rq\rq which extends the classical independence to the realm of an arbitrary von Neumann algebra. Bounded random variables $x,y\in\mathcal{M}$ are said to be \lq\lq tensor independent\rq\rq if
%\begin{enumerate}
%\item $x$ and $y$ commute.
%\item For every $m,n\geq0,$ we have
%$$\tau(x^ny^m)=\tau(x^n)\tau(y^m).$$
%\end{enumerate}
%The reason for such a name is that the von Neumann subalgebra $\mathcal{M}_0$ generated by $x$ and $y$ is the $*-$isomorphic to the tensor product of von Neumann subalgebras $\mathcal{M}_1$ and $\mathcal{M}_2$ generated by $x$ and $y,$ respectively. This notion has been at the outset of the noncommutative probability theory.
%
%Next, we introduce a concept of \lq\lq free independence\rq\rq in an arbitrary von Neumann algebra. Let $x,y\in\mathcal{M}$ be random variables and let $\mathcal{A}_1$ (respectively, $\mathcal{A}_2$) be the von Neumann algebra generated by $x$ (respectively, by $y$). Random variables $x$ and $y$ are said to be \lq\lq freely independent\rq\rq if, for every choice of mean zero random variables $x_k\in\mathcal{A}_1,$ $y_k\in\mathcal{A}_2,$ $1\leq k\leq n,$ we have
%$$\tau(x_1y_1x_2y_2\cdots x_ny_n)=0.$$
%The reason for such a name is that the von Neumann subalgebra $\mathcal{M}_0$ generated by $x$ and $y$ is the $*-$isomorphic to the free product of von Neumann subalgebras $\mathcal{M}_1$ and $\mathcal{M}_2$ generated by $x$ and $y,$ respectively.
%
%The mixed moments of tensor and free are polynomial expressions. This analysis intuitively leads us to the following tentative definition which makes the basis of our approach to the notion of the noncommutative independence and makes the basis of this proposal: {\it random variables $x$ and $y$ are independent if mixed moments of $x$ and $y$ are polynomial expressions in terms of moments of $x$ and that of $y.$} However, this definition was shown by Speicher \cite{Speicher_bad} to be not viable. He proved that the only possible types of independence are the tensor one and the free one.
%
%Another attempt is due to Nica \cite{Nicaqconv}. In the cited paper, Nica introduced a $q-$convolution which interpolates between the usual convolution and the free convolution of Voiculescu. For decade, it was believed to be a good generalization; however, Oravecz \cite{Oravecz} broke this hopes by showing that Nica's convolution does not preserve positivity.
%
%------------------------------------------------------------------------------------------------
%
%We intend to formulate our new approach to independence by analyzing the minimal requirement on what we want from such an independence. The minimal possible requirement is that
%
%{\bf Condition 1:} If $x,y\in\mathcal{M}$ are random variables, then the distribution function of $x+y$ is uniquely determined by the distribution function of $x$ and that of $y.$
%
%
%
%
%% Haagerup and Schultz's result, and that of the investigators, are pivotal advances in the Schur problem and hyper-invariant subspace problem for type II$_1$ factors, \cite{HalCM}. Every operator in $\mathcal{M}$ that is not a s.o.t-quasinilpotent operator has hyper-invariant subspaces, and the projections onto those subspaces are all contained in $\mathcal{M}$.  The results allow the investigators to pursue Lidskii theorems for singular traces associated to a type II$_1$ factor $\mathcal{M}$, extending Brown's original 1986 result on the Lidksii formula for the finite normal trace on $\mathcal{M}$.
%
%The investigator, with his expertise in noncommutative integration theory, free probability theory, and singular traces, is looking to push the results {\bf TBD}
%
%\paragraph*{Specific Aims} There are five specific aims.
%
%\noindent {\bf Aim 1}  To develop new approaches to noncommutative independence. The currently available sorts of independence are restricted to tensor independence and free independence and simple combinatorial arguments do not work \cite{Speicher_bad}. This is a severe limitation since noncommutative integration theory in finite von Neumann algebras provides a very natural substitution for the classical measure theory and the only thing needed for the noncommutative probability theory is the suitable concept of independence. It is also a significant drawback of the current theory of free independence compared with the classical probability theory that its mixed momenta rules appear rather {\it ad hoc}\footnote{These rules in Voiculescu theory arrived from the appplications, not from the internals of the theory itself.}. We aim to define the notion of the noncommutative independence in terms of certain mixed momenta rules and precisely determine the special cases which lead to tensor independence and free 
%independence.
%
%\noindent {\bf Aim 2} To extend classical cumulant theory and free cumulant theory developed in \cite{Nica_Speicher} to the general setting. This will depend on results from Aim 1. Such an extension will provide an analogue of characteristic function for the noncommutative probability theory.
%
%\noindent {\bf Aim 3} To prove classical results of the probability theory in the setting of general noncommutative independence. In particular, we intend to treat results like Large Number Law and Central Limit Theorem. This should also extend the famous Voiculescu proof of Central Limit Theorem in the free probability theory.
%
%\noindent {\bf Aim 4} To extend the Johnson-Schechtman inequalities to the general setting of noncommutative independence. The now classical paper of Johnson and Schechtman \cite{JS_ineq} appeared in 1989 and its analogue for freely independent random variables \cite{SZ_free} was proved by Sukochev and the author. This will be based on the results of Aims 1,2. The latter results combined with the Prokhorov inequality recently proved by Junge {\it et al} will allow us to extend the seminal result of Johnson and Schechtman to its most general domain.
%
%\noindent {\bf Aim 5} To apply Johnson-Schechtman inequalities to the studying of isomorphic embeddings of symmetric operator spaces. In 1989, Johnson and Schechtman proved that every symmetric function space on the interval $(0,1)$ is isomorphic to a certain symmetric function space on the semi-axis. This approach allowed to study the isomorphic embeddings of Orlicz sequence spaces into $L_p(0,1)$ \cite{AS_embed}. Using the recent discovery of Junge \cite{Junge_poisson}, we propose to fully classify those Banach ideals in the algebra $\mathcal{L}(H)$ which admit the isomorphic embedding into $\mathcal{L}_p(\mathcal{R}).$ We expect the following assertion to be true: Banach ideal can be isomorphically embedded into $\mathcal{L}_p(\mathcal{R})$ if and only if the corresponding sequence space can be isomorphically embedded into $L_p(0,1).$ 

\bigskip\archeading{Future Fellowship Candidate}


%
%\bigskip\archeading{Research Project}
%
%\paragraph*{Significance}
%
%The problems are fundamental and at the forefront of their area. The complete resolution of these questions will be important for the noncommutative probability theory. This is a mature project with an experienced CI and a network of collaborators in place which is already making progress. Other researchers actively work in the field, so this is becoming a very competitive area of investigation.
%
%\paragraph*{Conceptual framework, design and methods}
%
%In what follows, $\mathcal{M}$ is a semifinite von Neumann algebra equipped with a faithful normal semifinite trace $\tau.$ Let $\mu(A)$ denote the generalised eigenvalue function of the operator $A\in\mathcal{M}$ defined by the formula
%$$\mu(t,A)=\inf\{\|A(1-p)\|_{\infty}:\ \tau(p)\leq t\},\quad t>0.$$
%If, in particular, $\mathcal{M}=\mathcal{L}(H)$ and if $A\in\mathcal{M}$ is a compact operator, then $\mu(A)$ is a step function whose values are the eigenvalues of the operator $|A|.$
%
%\paragraph*{Approach to Aim 1}
%
%If $\mathcal{M}$ is a finite von Neumann algebra, then self-adjoint operators in $\mathcal{M}$ are referred to as to the bounded random variables. The bare minimum one expects from the independence is that if $x,y\in\mathcal{M}$ are independent, then the distribution function of $x+y$ is determined by the distribution function of $x,y.$ In either words, momenta of $x+y$ admit a (polynomial) expression in terms of that of $x$ and $y.$ That is, there should exist a rule
%$$\tau((x+y)^n)=\sum_{k+l=n}\sum_{\pi_1\in S_k,\pi_2\in S_l}\omega_{\pi_1,\pi_2}\tau_{\pi_1}(x)\tau_{\pi_2}(y).$$
%We can show that it suffices to have a rule of the form
%$$\tau((xy)^n)=\sum_{\pi_1,\pi_2\in S_n}\alpha_{\pi_1,\pi_2}\tau_{\pi_1}(x)\tau_{\pi_2}(y).$$
%
%\paragraph*{Approach to Aim 2}
%
%It is easy to compute the first $3$ cumulants. For example, if $x,y\in\mathcal{M}$ are independent (in its most general form) random variables, then
%$$\tau(xy)=\tau(x)\tau(y).$$
%It follows that
%$$\tau((x+y)^2)-\tau^2(x+y)=\tau(x^2)-\tau^2(x)+\tau(y^2)-\tau^2(y).$$
%Therefore, the second cumulant can be written as follows
%$$\kappa_2(x)=\tau(x^2)-\tau^2(x).$$
%A very similar argument shows that
%$$\kappa_3(x)=\tau(x^3)-3\tau(x^2)\tau(x)+2\tau^3(x).$$
%However, the situation with $\kappa_4$ is more complicated. Indeed, if $x,y\in\mathcal{M}$ are mean zero random variables, then
%$$\tau(xyxy)=\alpha\tau(x^2)\tau(y^2),$$
%where $\alpha$ is a parameter which describes the particular type of independence. Tensor independence corresponds to $\alpha=1,$ while free independence corresponds to $\alpha=0.$ Thus, we always have
%$$\kappa_4(x)=\tau(x^4)-(2+\alpha)\tau^2(x^2)$$
%for every mean zero $x\in\mathcal{M}.$ Higher order cumulants depend on the parameter $\alpha$ but also on many more parameters. We propose to use the combinatorial approach developed in \cite{Nica_Speicher} to write the cumulants explicitly.
%
%\paragraph*{Approach to Aim 3}
%
%In the classical probability theory, the Central Limit Theorem is proved by taking the limit of the characteristic function of the random variable
%$$\frac1{\sqrt{n}}\sum_{k=1}^nx_k,$$
%where $x_k$ are mean zero identically distributed random variables. The latter has the form
%$$t\to \varphi_x(\frac{t}{\sqrt{n}})^n.$$
%This expression tends to the characteristic function of a Gaussian random variable provided that $\varphi_x$ is sufficiently smooth at $0.$ In the noncommutative probability theory, we do not have a characteristic function, but a collection of cumulants instead. Thus, for every $m\geq 1,$ we should consider
%$$\kappa_m(\frac1{\sqrt{n}}\sum_{k=1}^nx_k)=n\kappa_m(\frac{x}{\sqrt{n}})=n^{1-m/2}\kappa_m(x_1).$$
%Since $x_1$ is mean zero random variable, it follows that
%$$\kappa_m(\frac1{\sqrt{n}}\sum_{k=1}^nx_k)\to
%\begin{cases}
%\tau(x_1^2),\quad m=2\\
%0,\quad m=1\\
%0,\quad m>2
%\end{cases}
%\quad n\to\infty.
%$$
%
%The next step is to find an analogue of the Gaussian random variable. That is, we need a random variable $\xi$ satisfying the condition
%$$\kappa_m(\xi)=
%\begin{cases}
%1,\quad m=2\\
%0,\quad m=1\\
%0,\quad m>2
%\end{cases}
%$$
%It is now clear that, for every $m\geq 1,$ we have
%$$\kappa_m(\frac1{\sqrt{n}}\sum_{k=1}^nx_k)\to\kappa_m(\|x_1\|_2\xi),\quad n\to\infty.$$
%From the convergence of cumulants, we infer the convergence of moments. It is a classical fact that the convergence of moments implies the convergence in distribution. This should prove the Central Limit Theorem for mean zero identically distributed random variables. The case of non-identically distributed  random variables is the subject for investigation.
%
%\paragraph*{Approach to Aim 4}
%
%Johnson and Schechtman proved their inequalities in the setting of function spaces and classical independence. Their method cannot be extended to the noncommutative setup. In the recent paper \cite{SZ_free} of Sukochev and the author a new method appeared. It involves the so-called \lq\lq free Kruglov operator\rq\rq. Essentially, this operator is the integration operator over the free Poisson process. The latter random process is mentioned yet at \cite{VDN}. However, in \cite{SZ_free}, a more elegant construction due to Nica and Speicher \cite{Nica_Speicher_constr} was used. We propose to construct a suitable analogue of the Poisson process for each type of independence. This will allow us to prove the inequality
%$$\|\sum_{k=1}^nx_k\|_E\sim\|\bigoplus_{k=1}^nx_k\|_E$$
%for the independent random variables satisfying the condition
%$$\sum_{k=1}^n\tau({\rm supp}(x_k))\leq 1$$
%in the large class of symmetric operator spaces. This is not yet a Johnson-Schechtman inequality in its full glory. In the setting of function spaces, the latter was obtained in \cite{JS_ineq} by handcrafted method and \cite{Zanin_th} by means of Prokhorov inequality. In the setting of free independence, Prokhorov inequality is replaced by Voiculescu inequality. The latter was re-proved in \cite{SZ_free} via free cumulants. Recently, a paper \cite{Junge_prokhorov} appeared which proved the Prokhorov inequality in a very general setting. The combination of Poisson-based approach and the Prokhorov inequality should suffice for the Johnson-Schechtman inequality in its most general form.
%
%\paragraph*{Approach to Aim 5}
%
%Astashkin and Sukochev classified those Orlicz sequence spaces isomorphically embeddable into $L_p(0,1).$ Those spaces are necesssarily $p-$convex and $2-$concave. It was shown in \cite{AS_embed} that (with the only exception of the sequence space $l_p$) that the standard basis in such sequence space is equivalent to a sequence of independent identically distributed functions in $L_p(0,1).$ The crucial fact used in the latter paper is the Johnson-Schechtman inequality for the space $L_p(0,1).$
%
%We propose that a similar siatuation holds in the noncommutative domain. Obviously, if $\mathcal{I}$ is a Banach ideal in $\mathcal{L}(H),$ then $\mathcal{I}$ has no basis which consists of pairwise orthogonal operators. Thus, the approach of \cite{AS_embed} does not work. However, one can replace the integration with repsect to the Poisson process with the new powerful instrument discovered by Junge \cite{Junge_poisson}. The latter method named in \cite{Junge_poisson} the \lq\lq noncommutative Poisson process\rq\rq does the perfect job.
%
%Unifying the approach of \cite{AS_embed} and that of \cite{Junge_poisson}, we propose to prove that an Orlicz ideal $\mathcal{L}_M\neq\mathcal{L}_p$ in $\mathcal{L}(H)$ admits an isomorphic embedding into $\mathcal{L}_p(\mathcal{R})$ if and only if $M$ is $p-$convex and $2-$concave. Based on this result, we propose to classify the ideals isomorphically embeddable into $\mathcal{L}_p(\mathcal{R})$ beyond the scope of Orlicz ideals. Indeed, it was proved by Raynaud and Sch\"utt \cite{RS} that a sequence space embeddable into $L_1(0,1)$ has a norm which is an average of Orlicz norms. We propose to prove a similar representation for embeddings into $L_p(0,1)$ and, therefore, extend the embedding of a sequence space $I$ into $L_p(0,1)$ to the embedding of the corresponding ideal $\mathcal{I}$ into $\mathcal{L}_p(\mathcal{R}).$
%
%\bigskip\archeading{The expected results and timeframe}
%
%{\bf Construction of cumulants, year 1} There is a strong empirical evidence (part of it presented above) that the proposed method is significantly more efficient and reliable than previous tailor-made approaches. The first task of this project is then to build a theoretical framework and the precise conditions under which the methodology is expected to work well. Thus, for a complete theory we also need an honest theoretical investigation of the limitations of the methodology. Given the reliance of the method on the combinatorics of partitions (in the classical case) or the combinatorics of the non-crossing partitions (in the free case), it is expected that these limitations would be also related to the a certain collection of partitions. To achieve these theoretical outcomes we intend to use the deep combinatorial results of Nica and Speicher \cite{Nica_Speicher}.
%
%{\bf Proof of the Central Limit Theorem, year 2} In the Approach to Aim 3, we arrived at the \lq\lq gaussian\rq\rq random variable $\xi.$ This is the special example of stable random variables, which may be difficult to construct. Thus, we need to investigate conditions under which stable random variables do exist. In addition, we need to investigate the optimal choice of the stable random variable (gaussian or not). $R-$transform method suggested by Voiculescu has proved to be a reliable algorithm for such a choice.
%
%
%{\bf Proof of the Johnson-Schechtman inequalities, years 2-3} {\bf Professor Sukochev and ????} CI discovered \cite{SZ_free} that a free Kruglov operator allows to prove the Johnson-Schechtman inequalities in the setting of free probability with virtually no restrictions on the space. This exciting discovery gives a crucial link between the Johnson-Schechtman inequalities in the classical setting and in the free one. Both operators appear as the integration over the certain Poisson process, which will give further insight into Kruglov operator for generic independence.
%
%{\bf Isomorphic embeddings of symmetric operator spaces, year 3} So far the Banach space theory imported only the technique from the classical probability theory. The extra information must be exploited to obtain isomorphic embeddings results of the noncommutative spaces. The specific analytical knowledge obtained in Aim 4 can be implemented in the theory of emebddings of symmetric operator spaces.
%
%\paragraph*{Research Environment}
%
%The school at UNSW has a distinguished group in the Noncommutative Analysis. Within Australia the School of Mathematics and Statistics at UNSW is the best mentoring and research support environment for the CI, primarily because his research project in the field of the Noncommutative Probability Theory complements these existing strengths in the noncommutative integration at UNSW. This environment will give the CI the opportunity to produce independent research at the highest level, suitable for publication in the leading journals in the field (Annals of Probability, Crelle's Journal, Journal of Functional Analysis, Advances in Mathematics) and suitable for presentation at the leading conferences in the field ({\bf WHAT?}). The CI can also rely on the support of internationally known researchers and collaborators such as: Sergei Astashkin (Samara State University), Kenneth Dykema (Texas A\&M University), Marius Junge (University of Illinois), Evgeniy Semenov (Voronezh State University).
%
%\bigskip\archeading{Feasibility and Strategic Alignment}
%
%\paragraph*{Feasibility} The CI possesses the right skills to conduct this project successfully and design an efficient interdisciplinary approach to the independence in the Noncommutative Probability Theory. The project presents a realistic timeframe as seen from above. Confidence in the feasibility is enhanced by the following:
%
%\begin{enumerate}
%\item The CI has published research relevant to the project in the field of the Noncommutative Probability Theory (e.g. \cite{SZ_free}).
%\item The CI has authored a number of publications in the field of the Classical Probability Theory.
%\item The CI has authored a number of publications in the field of the Noncommutative Analysis, including the research monograph \cite{LSZ} jointly written with S. Lord and F. Sukochev. 
%\item The CI can rely on mentoring, support and expert advice from the members of the Noncommutative Analysis group at UNSW, and from his highly-distinguished international collaborators (e.g. K. Dykema).
%\end{enumerate}
%
%\paragraph*{Benefit} The project will ultimately contribute to the development of the Noncommutative Probability Theory at UNSW.
%
%
%\footnotesize
%
%\begin{thebibliography}{99}
%\setlength{\itemsep}{0 pt}
%\setlength{\parskip}{0pt}
%\setlength{\parsep}{0pt}
%
%\bibitem{AS_embed} Astashkin S., Sukochev F. {\it Orlicz sequence spaces spanned by identically distributed independent random variables in $L_p-$spaces.} J. Math. Anal. Appl. {\bf 413} (2014), no. 1, 1--19.
%\bibitem{JS_ineq} Johnson W., Schechtman G. {\it Sums of independent random variables in rearrangement invariant function spaces.} Ann. Probab. {\bf 17}, No.2, 789--808 (1989).
%\bibitem{Junge_poisson} Junge M. {\it Noncommutative Poisson process} to appear
%\bibitem{Junge_prokhorov} Junge M., Zeng Q. {\it Noncommutative Bennett and Rosenthal inequalities.} Ann. Probab. {\bf 41} (2013), no. 6, 4287--4316.
%\bibitem{LSZ} Lord S., Sukochev F., Zanin D. {\it Singular traces. Theory and applications.} De Gruyter Studies in Mathematics, {\bf 46}. De Gruyter, Berlin, 2013. xvi+452 pp. 
%\bibitem{Nicaqconv} Nica A. {\it A one-parameter family of transforms, linearizing convolution laws for probability distributions.} Comm. Math. Phys. {\bf 168} (1995), no. 1, 187--207.
%\bibitem{Nica_Speicher} Nica A., Speicher R. {\it Lectures on the combinatorics of free probability.} London Mathematical Society Lecture Note Series, Vol. 335 Cambridge University Press, 2006.
%\bibitem{Nica_Speicher_constr} Nica A., Speicher R. {\it On the multiplication of free N-tuples of noncommutative random variables.} Am. J. Math. {\bf 118}, No.4, 799--837 (1996).
%\bibitem{Oravecz} Oravecz F. {\it Nica's $q-$convolution is not positivity preserving.} Comm. Math. Phys. {\bf 258} (2005), no. 2, 475--478.
%\bibitem{RS} Raynaud Y., Sch\"utt C. {\it Some results on symmetric subspaces of $L_1.$} Studia Math. {\bf 89} (1988), no. 1, 27--35.
%\bibitem{Speicher_bad} Speicher R. {\it On universal products.} Fields Institute Communications, Vol. 12 (D. Voiculescu, ed.), AMS, 1997, pp. 257--266.
%\bibitem{SZ_free} Sukochev F., Zanin D. {\it Johnson-Schechtman inequalities in the free probability theory.} J. Funct. Anal. {\bf 263}, No. 10, 2921--2948 (2012).
%\bibitem{VDN} Voiculescu D., Dykema K., Nica A. {\it Free random variables. A noncommutative probability approach to free products with applications to random matrices, operator algebras and harmonic analysis on free groups.} CRM Monograph Series. 1. Providence, RI: American Mathematical Society (AMS). v, 70 p. (1992).
%\bibitem{Zanin_th} Zanin D. {\it Orbits and Khinchine-type inequalities in symmetric spaces}. PhD thesis, 2011, Flinders University.
%\end{thebibliography}

\end{document}

