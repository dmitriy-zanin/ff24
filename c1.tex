\documentclass[12pt]{article}
\usepackage{amsmath,amsfonts,cite,color,mathptmx}
\usepackage[a4paper,margin=0.5cm]{geometry}

\usepackage{enumitem}
\newlist{steps}{enumerate}{1}
\setlist[steps, 1]{label = Step \arabic*:}

\newtheorem{thm}{Theorem}

\renewcommand{\baselinestretch}{0.95}
\newcommand{\archeading}[1]{\vspace{.3cm} \noindent{\bfseries #1} \vspace{.1cm}   }
\renewcommand{\refname}{{\normalsize References}}

\newcommand{\hl}{\color{blue}}
\newcommand{\edcomment}{\color{red}}


\begin{document}

\parindent=20pt
\pagestyle{empty}
\newpage

\archeading{Project Title} Asymptotics of non-commutative Laplacians, quantum symmetries and the Chern character.

\bigskip\archeading{Project quality and innovation}


\paragraph*{Brief outline} 

Differential equations describe the world around us, from the behaviour of black holes and supernovae (Einstein field equations) to subatomic particles (Schr\"odinger's equation), and from evolutionary biology (replicator equations) to how the leopard got its spots (reaction-diffusion systems). Given the diversity of differential equations, one may be surprised to find that only a small handful of mathematical tools are necessary to build up this remarkable field. Amongst such tools, the Laplacian is arguably the most important, describing the movement of heat and fluids, the behaviour of electric fields, and more recently having found application in image processing and
machine learning. Here our interest lies in heat kernels, which describe the spectrum (resonant frequencies) of the Laplacian and its generalisations.

Unfortunately, the Laplacian is not panacea, and understanding the behaviour of quantum mechanics suggests to us that more involved mathematical tools inspired by these simple ideas are necessary. In the same way that you should not put on your shoes before your socks, Heisenberg's uncertainty principle tells us that you get different results depending on the order in which you measure momentum and position at the level of the sub-atomic. Mathematically, this is governed by the field of ''Non-commutative Analysis'', the study of objects in which changing the order changes the outcome.

The field arose in response to the then nascent field of quantum mechanics, seeking to establish a mathematical framework which could describe these strange new behaviours.
In 1930, John von Neumann introduced ''Rings of Operators'', now called von Neumann
algebras, to provide a tool with which we could describe quantum systems. This was followed by the development of non-commutative topology and integration in the '40s, '50s, and '60s, which sought to understand this strange new mathematical landscape, and culminated in Alain Connes' theory of non-commutative geometry in the '80s, which extended the purely commutative world of differential geometry to the non-commutative.

Connes' non-commutative geometry sets the scene for the project at hand, wherein we study the mathematical properties of non-commutative extensions of the Laplacian.

Our first objective is to develop a Connes Character Formula, which states
an equality between cocycles in non-commutative Riemannian geometry and
conformal geometry. While the formula was first stated in Connes' book ''Noncommutative Geometry'', he provided no proof, leading to a flurry of research in trying to fill this gap, with mixed success. For compact non-commutative manifolds, the final word is due to the CI and co-authors \cite{CRSZ}, and here we seek to extend these results to quantum
groups.

% One of the cocycles in the Connes Character Formula is defined via Dixmier trace. Those traces are very special case of singular traces. The CI co-authored world first monograph on singular traces and their applications and a number of his papers in this direction are published in highly prestigious journals: Journal f\"ur die reine und angewandte Mathematik (the oldest mathematical journal still in publishing), Advances in Mathematics, Journal of Functional Analysis, etc.

Our second objective is to extend the Minakshisundaram--Plejel theorem to the
non-commutative, providing a suitable analogue for the heat kernel expansion. In the commutative world, this describes the eigenvalues (which one may think of as resonant frequencies) of the Laplacian, and we seek to do the same for the non-commutative. 

Connes and collaborators managed to calculate the curvature term of the non-commutative heat expansion for the simplest non-flat non-commutative manifold, but only through tremendously difficult effort. Appealing to the theory of Double Operator Integrals, the CI and co-authors have developed a breakthrough method to calculate the curvature term for more general non-commutative manifolds, whilst greatly simplifying the complexity of the work.

Our final objective is to construct a spectral triple (the suitable description of non-commutative geometries) from a given quantum group. While Lie groups describe the geometrical structure for much of classical mechanics, quantum groups provide a non-commutative generalisation suitable for the structure of quantum mechanics.
Prior attempts to build such spectral triples have been met with two key failures
--- the dimension drop phenomena, in which spectral dimension is no longer equal
to homological dimension, and that the first term of the heat kernel expansion does not recover the Haar state as it should. We propose a new construction which rectifies both these issues.

The CIs believes (and his publication list supports this) that all the stated objectives are plausible.


\paragraph*{National/international progress in the field of research and its relationship to this Proposal}

At the heart of non-commutative geometry lies the Gelfand--Naimark theorem,
which gives a categorical equivalence between locally compact Hausdorff spaces,
and commutative C$^{\ast}$-algebras. This means that we are able to recover all of the information about topology and geometry in the setting of operator algebras, and naturally leads one to ask what happens when we perform the same calculations for non-commutative algebras. In much the same way, the results of von Neumann and Irving Segal allow us to translate measure and integration theory to operator algebras.
Using these ideas, we may study geometry and integration in the non-commutative realm, giving us a mathematical framework in which to properly understand and describe the quantum world around us.

While these ideas are fairly simply stated, the mathematics is difficult and
subtle. Attempts to describe non-commutative geometry stem back to von Neumann's
original research in the '30s, but only successfully manifested with Connes'
development of non-commutative geometry in the '80s. This gave a suitable non-commutative extension of differential geometry, built out of spectral triples, which encode the information carried by Riemannian manifolds.

How do we replace a Riemannian manifold and its metric tensor $g$? Take a Hilbert space $H$, and let $\mathcal{A}$ be a $^{\ast}$-algebra represented on $H.$ The three components $(\mathcal{A},H,D)$ form a (compact) spectral triple if:
\begin{enumerate}
\item $D$ is a self-adjoint bounded operator on $H$;
\item For any operator $a\in\mathcal{A},$ the commutator $[D,a]=Da-aD$ has bounded extension;
\item And lastly, the operator $(D+i)^{-1}$ must be compact.
\end{enumerate}
Here, the operator $D$ (called the Dirac operator) stands in place of the metric
tensor $g.$ If the singular values of $(D+i)^{-1}$ decay proportionally to the sequence $(k^{-\frac{1}{d}})_{k=1}^{\infty}$, then the spectral triple is $d$-dimensional.

Given a Riemannian manifold $(M,g)$, we can build a spectral triple
$(\mathcal{A}, H, D)$ to recover its geometry as follows.
\begin{enumerate}
\item Let $\mathcal A=C^{\infty}(M)$ be the algebra of all smooth functions on $M$.
\item Let $H=L^2\Omega(M)$ be the space of all square integrable functions on $M.$ The representation $\pi:\mathcal A\to B(H)$ is then given by pointwise multiplication.
\item The Dirac operator $D$ is then given by the classical Hodge--Dirac operator (see \cite{BGV}, for example).
\end{enumerate}
This gives a $d$-dimensional spectral triple, where $d$ is the dimension of
the original manifold $M$ (as one should expect). Conversely, Connes' Reconstruction theorem \cite{Connes-reconstruction} tells us that every spectral triple over a commutative algebra $\mathcal A$ corresponds to some Riemannian manifold $(M,g).$ In light of this, spectral triples form non-commutative manifolds, and set the stage for a geometry of the quantum world.

% A Hodge-Dirac operator $D$ is a convenient tool to identify and describe the isometries of the manifold $M.$ If $\gamma:M\to M$ is an isometry and if $U:L_2\Omega(M)\to L_2\Omega(M)$ is the  operator of composition with $\gamma$, then (i) $U$ is unitary on $L_2\Omega(M)$; (ii) $U$ preserves $\pi(\mathcal{A})$ (that is, $U^{-1}\pi(\mathcal{A})U=\pi(\mathcal{A})$); (iii) $U$ commutes with $D.$ Conversely, every diffeomorphism $\gamma:M\to M$ which commutes with $D$ is an isometry. This simple but revealing result can be found in \cite{helgason}.

In typical applications to harmonic analysis \cite{helgason} and mathematical physics \cite{c2dft}, a manifold is equipped with some isometric action of a Lie group. Moreover, in examples of serious interest and importance, the manifold will actually be a homogeneous space of some Lie group. For example, the sphere $\mathbb{S}^{n-1}$ is a homogeneous space of the Lie group $SO(n)$.
Hence it is natural to consider not just a manifold, but the manifold equipped
with an isometric group action.
In parallel, we may consider the action of a Lie group or a quantum group on
a spectral triple.
In this setting, it is most sensible to consider the actions of quantum groups,
the natural non-commutative extension of Lie groups.
The need for such groups arose in the early 1980s, trying to find a suitable
notion of duality to extend the Pontrjagin duality theorem to non-abelian groups.
In response to these questions, Woronowicz developed a general theory of
quantum compact groups, including an extension of the Peter--Weyl theory.
The most famous example of one of these quantum groups is the group
$SU_q(n)$, the $q$-deformation of the Lie group $SU(n)$.

The rich and extensive interplay between Lie groups and differential geometry
naturally begs the question --- what is the interaction between quantum groups
and spectral triples?
Initial attempts to reconcile these structures, such as
\cite{ChakrabortyPal, NeshTus}, drew the attention of Connes (see
\cite{Connes-suq2}), who further developed the results of Chakraborty and Pal.
These developments help to motivate the aims of this proposal.

In this project, we aim to construct spectral triples for certain quantum groups
(e.g. compact quantum groups like $SU_q(n)$ and $SO_q(n)$),
as well as non-compact quantum groups like $SL_q(n)$, and their homogeneous 
spaces.
This would allow us to study the extension of major results in Non-commutative
Geometry (such as Connes' Character Formula) for quantum groups, and even
to be able to compute topological invariants (such as the $K$-theory) for these
examples.

Returning to our starting point, we see that a spectral triple is naturally
equipped with a semi-group action, given by the heat semigroup.
If $(M,g)$ is a $d$-dimensional Riemannian manifold, and $(\mathcal A,H,D)$ the
associated spectral triple, then $D^2$ \textit{is} the Hodge-Laplace operator,
denoted by $\Delta_g$ \cite{Rosenberg}.
This suggests to us how we can use the Dirac operator for a given spectral triple
to build a \textit{non-commutative Laplacian}, the object which lies at the
heart of this proposal.
In this setting, the heat semi-group is now defined by $t\mapsto e^{-t\Delta_g}$,
for any $t>0$.

In his seminal work \cite{Weyl}, Weyl proved that, for a compact manifold,
\begin{equation}\label{weyl formula}
\lim_{t\downarrow0}(4\pi t)^{\frac{d}{2}}{\rm Tr}(e^{-t\Delta_g})={\rm Vol}(M).
\end{equation}
Following Weyl's work, it became an established custom to measure various geometric (and often topological) quantities associated with a Riemannian manifold $M$ in terms of its heat semi-group expansion $t\mapsto e^{-t\Delta_g},\quad t>0.$ The mere existence of such an expansion is a famous theorem of Minakshisundaram and Plejel (among all approaches to that theorem, a particularly detailed account is given in \cite{Rosenberg}). 

For every $f\in C^{\infty}(M),$ the Minakshisundaram--Plejel theorem asserts an existence of an asymptotic expansion 
\begin{equation}\label{heat asymptotic}
{\rm Tr}(M_fe^{-t\Delta_g})\approx (4\pi t)^{-\frac{d}{2}}\cdot \sum_{n\geq0}a_n(f)t^n,\quad t\downarrow0.
\end{equation}
Here, $d$ is the dimension of $M$ and $M_f:L_2(M)\to L_2(M)$ is the operator of pointwise multiplication by $f.$ Moreover, there exist functions $A_k\in C^{\infty}(M)$ such that
\begin{equation}\label{normality of coefficients}
a_k(f)=\int_M A_k\cdot f d{\rm vol}_g,
\end{equation}
where ${\rm vol}_g$ is the Riemannian volume on $M.$

As follows from \eqref{weyl formula}, $A_0=1.$ Further computations (see e.g. Proposition 3.29 in \cite{Rosenberg}) show that $A_1=\frac16 R,$ where $R$ is the scalar curvature. In particular, $a_1(1)$ is the Einstein--Hilbert action (see e.g. \cite{Connes-book}).

Note that $a_0$ extends to a normal state $h$ on $L_{\infty}(M)$ by the obvious formula
$$h(f)=\int_M fd{\rm vol}_g,\quad f\in L_{\infty}(M).$$
Equation \eqref{normality of coefficients} can be re-written as
$$a_k(f)=h(A_k\cdot f),\quad f\in C^{\infty}(M).$$

One of the {\it primary targets} of this project is to find suitable extensions 
of the Minakshisundaram--Plejel theorem (and, consequently, of the Weyl theorem) 
for a vast class of non-commutative manifolds (such as non-commutative tori with 
generic, non-flat, metric tensor). 
This grand program began with \cite{ConnesTretkoff} 
(although this was only published in 2011, the concepts and theorems therein
stretch back to the '90s), where special $2-$dimensional non-commutative manifolds
(conformal deformations of a flat non-commutative torus) were considered.
The authors of \cite{ConnesTretkoff} proved that Euler characteristic of such 
manifold is $0$ by means of the Gauss--Bonnet theorem (which asserts that the Euler 
characterstic of the $2-$dimensional Riemannian manifold equals the average of
its scalar curvature). 
Subsequently, the authors were able to explicitly calculate the scalar curvature for these manifolds in \cite{ConnesMoscovici_curvature} and \cite{FathizadehKhalkhali}.
Later, the term $a_2$ (the first place where the Riemann curvature tensor manifests itself beyond the scalar curvature) was computed in \cite{ConnesFathizadeh} (intermediate computations include about a million terms!).

The process for these calculations, as in \cite{ConnesTretkoff}, is as follows.
Let $(\mathcal A,H,D)$ be a $d$-dimensional spectral triple, and for simplicity
we assume $\mathcal A$ to be unital.
\begin{steps}
\item\label{raph2} Verify that the following generalisation of \eqref{heat asymptotic} (the asymptotic heat expansion) holds:
$${\rm Tr}(xe^{-tD^2})\approx (4\pi t)^{-\frac{d}{2}} \sum_{n\geq0}a_n(x)t^n,\quad t\downarrow0,\quad x\in\mathcal{A}''.$$
\item\label{raph3} Verify the normality of coefficients as in \eqref{normality of coefficients} with respect to the volume state, or, more precisely, to show that
$$a_k(x)=h(xA_k),\quad x\in\mathcal{A}'',\mbox{ for some }A_k\in\mathcal{A}'',\quad k\geq0.$$
\item\label{raph4} Compute $A_k$ explicitly. (Far more easily said than done.)
\end{steps}
When this mission is accomplished, one can {\it define} a scalar curvature of a non-commutative manifold by setting $R=6A_1.$

We acknowledge that this program sets reasonable objectives but we claim that the tools based on on pseudo-differential calculus in various guises applied up-to-date to accomplish them have been inadequate. Our main technical innovation which we bring here is based on the novel Double Operator Integration techniques developed by the CI recently in close collaboration with Alain Connes and Fedor Sukochev. The particular integral representations (see Approach to Aims 1,2,3 below) arose in \cite{Connes_team} when geometric measures on limit sets of Quasi-Fuchsian groups were recovered by means of singular traces. They form the core contribution of UNSW team (including the CI) to \cite{Connes_team}.


One of the fundamental tools in noncommutative geometry is the Chern character. The Connes Character Formula provides an expression for the class of the Chern character in Hochschild cohomology, and it is an important tool in the computation of the Chern character. The formula has been applied to many areas of noncommutative geometry and its applications: such as the local index formula \cite{ConnesMoscovici}, the spectral characterisation of manifolds \cite{Connes-reconstruction} and recent work in mathematical physics \cite{Connes-Chamseddine-Mukhanov-quanta-of-geometry-2015}. We point out to its applications in \cite{Connes-reconstruction} as particularly relevant to the theme of this application.

In its original formulation, \cite{Connes-original-spectral-1995}, the Character Formula is stated as follows: Let $(\mathcal{A},H,D)$ be a $p$-dimensional compact spectral triple     with (possibly trivial) grading $\Gamma.$ By the definition of a spectral triple, for all $a \in \mathcal{A}$ the commutator $[D,a]$ has an extension to a bounded operator $\partial(a)$ on $H.$ Assume for simplicity that $\ker(D)=\{0\}$ and set $F={\rm sgn}(D).$ For all $a \in \mathcal{A}$ the commutator $[F,a]$ is a compact operator in the weak Schatten ideal $\mathcal{L}_{p,\infty}$ (see e.g. \cite{Connes-book, book}).

Consider the following two linear maps on the algebraic tensor power $\mathcal{A}^{\otimes(p+1)},$ defined on an elementary tensor $c = a_0\otimes a_1\otimes \cdots \otimes a_p \in \mathcal{A}^{\otimes(p+1)}$ by setting
$$\mathrm{Ch}(c) := \frac{1}{2}{\rm Tr}(\Gamma F[F,a_0][F,a_1]\cdots[F,a_p]),\quad \Omega(c) := \Gamma a_0\partial a_1\partial a_2\cdots \partial a_p.$$
Then the Connes Character Formula states that if $c$ is a Hochschild cycle then
$$\mathrm{Tr}_\omega(\Omega(c)(1+D^2)^{-p/2}) =\mathrm{Ch}(c)$$
for every Dixmier trace $\mathrm{Tr}_\omega$. In other words, the multilinear maps $\mathrm{Ch}$ and $c \mapsto \mathrm{Tr}_\omega(\Omega(c)(1+D^2)^{-p/2})$ define the same class in Hochschild cohomology. We mention, in passing, that the generality of the result just stated was achieved fairly recently by the CI and his co-authors, see \cite{CRSZ}.
    
There has been great interest in generalising the tools and results of noncommutative geometry to the \lq\lq non-compact\rq\rq (i.e., non-unital) setting. The definition of a spectral triple associated to a non-unital algebra originates with Connes \cite{Connes-reality}, was furthered by the work of Rennie \cite{Rennie} and Gayral, Gracia-Bond\'ia, Iochum, Sch\"ucker and Varilly \cite{gayral-moyal}. Earlier, similar ideas appeared in the work of Baaj and Julg \cite{Baaj-Julg}. Additional contributions to this area were made by Carey, Gayral, Rennie and Sukochev\cite{CGRS}. The conventional definition  of a non-compact spectral triple is to replace the condition that $(D+i)^{-1}$ be compact with the assumption that for all $a \in \mathcal{A}$ the operator $a(D+i)^{-1}$ is compact. This raises an important question: is the Connes Character Formula true for locally compact spectral triples? This question was suggested to the CI by Professor Connes himself during Shanghai conference celebrating the 70-th anniversary of A. Connes (held at Fudan University). During lengthy discussions covering a substantial range of topics of importance in non-commutative geometry, Professor Connes, in particular had emphasized to the CI that such an extension could be an excellent starting point for several new directions in non-commutative geometry. Here are two such directions. Firstly, the Connes Character Formula plays  an important role in the reconstruction theorem for closed Riemannian manifolds, and it is natural to expect that its extension would play a similar role for locally compact Rimannian manifolds. Secondly, developing suitable new techniques needed for a self-contained theory of locally compact spectral triples should also open an avenue for treating the case of noncommutative manifolds with boundary or even incomplete (e.g. punctured) manifolds. This suggestion by Professor Connes has been taken seriously by the CI and preliminary work in this direction has already brought substantial fruits. The ground-breaking manuscript co-authored by Professor Connes and the CI (and a number of collaborators from UNSW) the new approach to spectral triples involving {\it symmetric, non-self-adjoint} operators has been developed \cite{Connes_team_symmetric}. These are precisely the required tools allowing the possibility to develop a new theory for non-commutative Riemannian manifolds with boundary. This development indicates the importance and timeliness of the present proposal which has already achieved substantial progress and leads to a unified theory of locally compact non-commutative manifolds with boundary and incomplete manifolds. We emphasize that the progress achieved involves joint work with luminaries like Alain Connes, and local top experts in noncommutative analysis and geometry like Alan Carey, Adam Rennie and Fedor Sukochev.
    

\paragraph*{Significance of the project} This project intertwines the small scale geometry (studies in terms of calculus and differential equations) with global geometry (which aims to comprehend the shape of a manifold). The significance of the specific Aims proposed below is to establish links between different fields of mathematics (in particular, between Operator Algebras and Differential Geometry).

Heat semi-group in the classical Differential Geometry provides a solution of a parabolic PDE. For non-commutative manifold, heat semi-group essentially delivers the time evolution of an irreversible quantum system whose Hamiltonian is a (non-commutative) Laplace-Beltrami operator.

The project will concentrate on non-commutative geometries with high degree of symmetry provided by quantum groups, the feature which is paramount both in Mathematics and Physics. The research impact of this project will be enhancement of the Australian profile in the crucial area of Quantised Calculus and its applications to Non-commutative Geometry. The project will maintain and foster international collaborative links already built by the CI (Connes, Junge, Higson, Dykema) simultaneously contributing to training the next generation of Australian mathematicians. Certain parts of this proposal were discussed in the past 3 years with each of just listed collaborators in order to ensure that Aims below are cutting edge research.

Generally speaking, the development of noncommutative geometry and its applications is hindered by the paucity of non-trivial examples demonstrating its richness and ability to {\it compute} quantities of geometric significance for genuinely non-commutative manifolds. The project will generate an original approach to this computational task which will interact with and improve on the work of top class practitioners of Non-commutative Analysis/Geometry such as Marius Junge (University of Illinois) and Nigel Higson (Penn State University). The CI is in regular contact with both (they both visited UNSW during last years) and this collaboration is to continue unabated.

The project will have scholarly impact on the crucial parts of Non-commutative Differential Geometry with links to several other areas of Mathematics. 





\paragraph*{Aims of the project} There are 7 specific aims.


\noindent{\bf Aim 1:} Investigate when Chern character provides an asymptotic expansion for the heat semi-group (arising from a locally compact spectral triple). More precisely, when 
\begin{equation}\label{heat eq}
{\rm Tr}(\Omega(c)e^{-s^2D^2})={\rm Ch}(c)s^{-p}+O(s^{1-p}),\quad s\downarrow0,
\end{equation}
for every Hochschild cycle $c\in\mathcal{A}^{\otimes (p+1)}?$ It seems plausible that Hochschild cochain on the left hand side is cohomologous to the one in the right hand side (modulo $O(s^{1-p})$). We expect to achieve Aim 1 during Year 1.

\noindent{\bf Aim 2:} Aim 1 above is closely related (albeit, not equivalent) to a question concerning analyticity of a suitable $\zeta-$function. The latter is an analytic function defined by the formula
$$z\to {\rm Tr}(\Omega(c)(1+D^2)^{-\frac{z}{2}}),\quad \Re(z)>p.$$
We aim to find an analytic extension to the half-plane $\Re(z)>p-1$ so that
$$\lim_{z\to p}(z-p){\rm Tr}(\Omega(c)(1+D^2)^{-\frac{z}{2}})=p{\rm Ch}(c).$$

\noindent{\bf Aim 3:} To compute the Hochschild class of the Chern character by a ``local'' formula, which is customarily  stated in terms of singular traces on the ideal $\mathcal{L}_{1,\infty}$ ($\mathcal{L}_{1,\infty}$ is the principal ideal in $B(H)$ generated by an operator with singular values $(\frac1k)_{k\geq1}$). Here, trace $\varphi:\mathcal{L}_{1,\infty}\to\mathbb{C}$ is a unitarily invariant linear functional; it can be seen from CI's results in \cite{book} that such a functional is automatically singular. Our third aim is to show that
\begin{equation}\label{super main eq}
\varphi(\Omega(c)(1+D^2)^{-\frac{p}{2}})={\rm Ch}(c).
\end{equation}
for every (normalised) trace $\varphi$ on $\mathcal{L}_{1,\infty}$ and for every Hochschild cycle $c\in\mathcal{A}^{\otimes (p+1)}.$ 

Aims 2 and 3 are intimately connected with Aim 1, however, the amount of analytical complications which arise when one navigates between formulas stated in all three aims is enormous and requires a very careful treatment and certainly warrants a separation of these aims.

We rely on the the deep theory of singular traces and its connections with operator $\zeta-$functions which the CI (with various collaborators) has been developing since 2009 \cite{book,SUZ-indiana}. The CI is in the unique position to apply numerous and well-developed techniques from that theory in order to achieve this aim.  We expect to achieve Aim 2 during Year 2.

\noindent{\bf Aim 4:} Introduce a generic (i.e., not necessarily a conformal deformation of a flat one) Laplace-Beltrami operator
$$\Delta_g x=M_{G^{-\frac12}}\sum_{i,j=1}^dD_iM_{G^{\frac14}(g^{-1})_{ij}G^{\frac14}}D_j\mbox{ where }G^{-\frac12}=({\rm det}(g_{ij}))^{-\frac12}\stackrel{def}{=}\pi^{-\frac{d}{2}}\int_{\mathbb{R}^d}e^{-\sum_{i,j}g_{ij}t_it_j}dt,$$
on the non-commutative torus and non-commutative Euclidean space. Prove a non-commutative version of Minakshisundaram--Plejel theorem (for these manifolds) as outlined above.

Aim 4, in turn, depends on Aims 1 and 2, or rather on the technical instruments which are to be developed to deal with those aims in full generality.

\noindent{\bf Aim 5:} Compute explicitly the curvature for a generic Riemannian metric on the non-commutative torus and non-commutative Euclidean space.

By using Double Operator Integration technique as developed in \cite{PotapovSukochev}, we aim to prove analyticity of the right hand side and, hence, of the left hand side. The application of these techniques is our trump card in the joint work with Professor Connes \cite{Connes_team}, in which we have completed the work started by such giants as Connes and Sullivan, and which left dormant for more than 20 years, until our new techniques were brought to bear on that problem.

\noindent{\bf Aim 6:} Construct a spectral triple (or possibly, a twisted spectral triple) on quantum groups (like ${\rm SU}_q(n)$) and on their homogeneous spaces (like Podle\'s sphere). Previous attempts \cite{ChakrabortyPal} suffer from a ``dimension drop'' pathology (that is, where spectral dimension differs from cohomological dimension). We aim to have an equivariance property for the Dirac operator in a way that prevents ``dimension drop'' pathology. 

The spectral triple constructed in \cite{ChakrabortyPal} is equivariant and its spectral dimension is $3.$ However, every $3-$cocycle is cohomologous to $0$ and, therefore, the homological dimension is strictly less than $3.$ 

We expect that the reason for this phenomenon is the wrong choice of $q-$deformed left regular representation. Our aim is to find a suitable $q-$deformation which permit the spectral triple to be equivariant, $3-$dimensional and, at the same time to allow certain $3-$cocycles to be non-trivial. A natural candidate for such a $3-$cocycle is the Hochschild class of the Chern Character. Successful resolution of Aims 1,2,3 will deliver the required technical tools to determine the ``correct representation'' and such cocycles.


\noindent{\bf Aim 7:} Design a version of Connes Character Formula for the above spectral triples which recovers a non-trivial cocycle.

It is extremely probable that ordinary spectral triples in this setting should be replaced by twisted ones (where commutators $[D,a]$ may be unbounded, but a certain ``twisted'' commutator is bounded). At the moment, no satisfactory theory is available which allows the derivation of a Connes Character formula for such triples. The only attempt is made in \cite{MasF}. However, we expect that a combination of approaches from \cite{MasF} and \cite{CRSZ} would yield (at least a germ of) such a theory.


\bigskip\archeading{Benefit}

The CI expects the project to produce significant results and to publish them in the most reputed journals. In addition, the CI expects the project to be beneficial in the following ways.

a. Australia has strong research profile in Non-Commutative Geometry and Operator Algebras. This area develops rapidly and the CI enthusiastically contributes to this development. The outcomes of this project will therefore be of interest to a number of research groups in Australia (Wollongong, Adelaide, Sydney and Canberra). The project will broaden existing strengths and introduce new directions in a highly active and internationally competitive area of research endeavour. 

%of asymptotics for non-commutative Laplacians and geometry associated to quantum groups.


%Geometry and Functional Analysis. , specifically geometric and homological results many of which are based on the properties of the partition function associated to the heat kernel asymptotic expansion on classical Riemannian manifolds and the spectral and algebraic properties of the Laplacian and associated operators. 

%Australian mathematics also has a strong groups in symmetries and Lie groups and operator algebras and non-commutative geometry. 

b. The project will enhance international collaboration in research. The CI is collaborating with highly-distinguished international experts (Alain Connes, Kenneth Dykema, Nigel Higson, Marius Junge). This collaboration already resulted in a number of papers in prestigious journals. Involvement of mathematicians of such a calibre would be beneficial not only for this project, but also to other directions of mathematical research in UNSW.

It should be pointed out that Professor Connes is very enthusiastic about the suggested direction of research and this strong endorsement from the world leading expert and his on-going commitment to the joint research efforts is a strong acknowledgement of the depth and importance of the current research proposal.

c. Improve the international competitiveness of Australian research. The area of Quantised Calculus is the cutting edge research toolkit in the modern analysis. This proposal is on par with efforts of top world experts in the area and thus will strengthen the Australian leadership in this area.

d. Due to the breadth and depth of the proposal which involves groundbreaking research in several mathematical disciplines, the CI expects to attract the top Australian students for postgraduate studies who otherwise might be heading overseas.

\bigskip\archeading{Mentoring and capacity building}
The CI has already demonstrated his capacity to make significant, original and innovative contributions to wide range of Non-commutative Analysis and Non-commutative Geometry. He has co-authored the very first monograph on singular traces \cite{book}, which has albeit in embrionic form, some necessary components to attack Aim 3. The project presents a realistic timeframe as seen from above. Confidence in the feasibility is enhanced by the following:

1. The CI has already demonstrated his research credentials in the field of Noncommutative Analysis. His numerous papers in this area are published in the highly ranked journals like Crelle's Journal, Advances in Mathematics, Journal of Functional Analysis.

2. The CI wrote a number of publications in the field of Mathematical Physics. For example, paper \cite{SZ-cmp} is published in prestigious journal Communications in Mathematical Physics.

3. The CI has co-authored a number of publications in the field of Classical Analysis.

4. The CI has authored a research monograph \cite{book} jointly written with S. Lord and F. Sukochev. The substantial portion of this monograph describes CI's contribution to the field of Non-commutative Geometry and related parts of Quantised Calculus. At the moment, a second edition of the book is in preparation.

5. The CI can rely on support and expert advice from the members of the Noncommutative
Analysis group at UNSW (M. Cowling, I. Doust, D. Potapov, F. Sukochev).

It should also be pointed out that some parts of the current proposal have been thoroughly discussed with Professor Alain Connes (Fields medalist) who is the originator of (and leading expert in) Non-commutative Geometry. Professor Connes strongly and enthusiastically endorsed the ideas and methods underlying this proposal and the approach.

\archeading{Communication of results} Results of this project will be published in peer-reviewed journals (as well as on arxiv). CI and collaborators will also deliver the results in thematic international conferences.

\small

\begin{thebibliography}{99}
\setlength{\itemsep}{0pt}
\setlength{\parskip}{0pt}
\setlength{\parsep}{0pt}
\bibitem{Baaj-Julg} Baaj S., Julg P. {\it Th\'eorie bivariante de Kasparov et op\'erateurs non born\'es dans les $C^*$-modules hilbertiens.} C. R. Acad. Sci. Paris S\'er. I Math. {\bf 296} (1983), no. 21, 875--878.
\bibitem{BGV} Berline N., Getzler E., Vergne M. {\it Heat kernels and Dirac operators.} Grundlehren der Mathematischen Wissenschaften, {\bf 298}. Springer-Verlag, Berlin, 1992.
\bibitem{CGRS} Carey A., Gayral V., Rennie A., Sukochev F. {\it Integration on locally compact noncommutative spaces.} J. Funct. Anal. {\bf 263} (2012), no. 2, 383--414; Carey A., Gayral V., Rennie A., Sukochev F. {\it Index theory for locally compact noncommutative geometries.} Mem. Amer. Math. Soc. {\bf 231} (2014), no. 1085, vi+130 pp.
\bibitem{CRSZ} Carey A., Rennie A., Sukochev F., Zanin D. {\it Universal measurability and the Hochschild class of the Chern character.} J. Spectr. Theory {\bf 6} (2016), no. 1, 1--41.
\bibitem{ChakrabortyPal} Chakraborty P., Pal A. {\it Equivariant spectral triples on the quantum $SU(2)$ group.} K-Theory {\bf 28} (2003), no. 2, 107--126;  Chakraborty P., Pal A. {\it Spectral triples and associated Connes-de Rham complex for the quantum $SU(2)$ and the quantum sphere.} Comm. Math. Phys. {\bf 240} (2003), no. 3, 447--456.
\bibitem{Connes-suq2} Connes A. {\it Cyclic cohomology, quantum group symmetries and the local index formula for $SU_q(2).$} J. Inst. Math. Jussieu {\bf 3} (2004), no. 1, 17--68.
\bibitem{Connes-original-spectral-1995} Connes A. {\it Geometry from the spectral point of view.} Lett. Math. Phys., {\bf 34} (3) 203--238, 1995.
\bibitem{Connes-book}  Connes A. {\it Noncommutative geometry.} Academic Press, Inc., San Diego, CA, 1994.
\bibitem{Connes-reality} Connes A., {\it Noncommutative geometry and reality.} J. Math. Phys. {\bf 36} (1995), 6194–-6231 .
\bibitem{Connes-reconstruction} Connes A. {\it On the spectral characterization of manifolds.} J. Noncommut. Geom. {\bf 7} (2013), no. 1, 1--82.
\bibitem{Connes-Chamseddine-Mukhanov-quanta-of-geometry-2015} Chamseddine, A., Connes, A., Mukhanov, V. {\it Quanta of geometry: noncommutative aspects.} Phys. Rev. Lett. {\bf 114} (2015), no. 9, 091302, 5pp.
\bibitem{ConnesFathizadeh} Connes A., Fathizadeh F. {\it The term $a_4$ in the heat kernel expansion of noncommutative tori.} M\"unster J. Math. {\bf 12} (2019), no.2, 239--410.
\bibitem{Connes_team_symmetric} Connes A., Levitina G., McDonald E., Sukochev F., Zanin D. {\it Noncommutative geometry for symmetric non-self-adjoint operators.} J. Funct. Anal. {\bf 277} (2019), no.3, 889--936.
\bibitem{ConnesMoscovici} Connes A., Moscovici H. {\it The local index formula in noncommutative geometry.} Geom. Funct. Anal. {\bf 5} (1995), no. 2, 174--243.
\bibitem{ConnesMoscovici_curvature} Connes A., Moscovici H. {\it Modular curvature for noncommutative two-tori.} J. Amer. Math. Soc. {\bf 27} (2014), no. 3, 639--684.
\bibitem{Connes_team} Connes A., Sukochev F., Zanin D. {\it Trace theorem for quasi-Fuchsian groups.} Mat. Sb. {\bf 208} (2017); Connes A., McDonald E., Sukochev F., Zanin D {\it Conformal trace theorem for Julia sets of quadratic polynomials.} Ergodic Theory Dynam. Systems {\bf 39} (2019), no. 9, 2481--2506.
\bibitem{ConnesTretkoff} Connes A., Tretkoff P. {\it The Gauss-Bonnet theorem for the noncommutative two torus.} Noncommutative geometry, arithmetic, and related topics, 141--158, Johns Hopkins Univ. Press, Baltimore, MD, 2011.
\bibitem{MasF} Fathizadeh F., Khalkhali M. {\it Twisted spectral triples and Connes' character formula.} Perspectives on noncommutative geometry, 79--101, Fields Inst. Commun., {\bf 61}, Amer. Math. Soc., Providence, RI, 2011.
\bibitem{FathizadehKhalkhali} Fathizadeh F., Khalkhali M. {\it Scalar curvature for the noncommutative two torus.} J. Noncommut. Geom. {\bf 7} (2013), no. 4, 1145--1183; Fathizadeh F., Khalkhali M. {\it Scalar curvature for noncommutative four-tori.} J. Noncommut. Geom. {\bf 9} (2015), no. 2, 473--503.
\bibitem{c2dft} di Francesco P., Mathieu P., Senechal D. {\it Conformal Field Theory.} Springer-Verlag, New York, 1997.
\bibitem{gayral-moyal} Gayral V., Gracia-Bondia J., Iochum B., Sch\"ucker T., Varilly J. {\it Moyal planes are spectral triples.} Comm. Math. Phys. {\bf 246} (2004), no. 3, 569--623.
\bibitem{GVF} Gracia-Bondia J., Varilly J., Figueroa H. {\it Elements of noncommutative geometry.} Birkh\"auser Advanced Texts: Basler Lehrb\"ucher. Birkh\"auser Boston, Inc., Boston, MA, 2001.
\bibitem{helgason} Helgason S., {\it Differential geometry, Lie groups, and symmetric spaces.} Graduate Studies in Mathematics, {\bf 34}. American Mathematical Society, Providence, RI, 2001.
\bibitem{JSZ_advances} Junge M., Sukochev F., Zanin D. {\it Embeddings of operator ideals into $L_p$-spaces on finite von Neumann algebras.} Adv. Math. {\bf 312} (2017), 473--546.
\bibitem{Lesch} Lesch M. {\it Divided differences in noncommutative geometry: rearrangement lemma, functional calculus and expansional formula.} J. Noncommut. Geom. {\bf 11} (2017), no. 1, 193--223.
\bibitem{book} Lord S., Sukochev F., Zanin D. {\it Singular traces. Theory and applications.} De Gruyter Studies in Mathematics, {\bf 46}. De Gruyter, Berlin, 2013.
\bibitem{LoreauxWeiss2018} Loreaux J., Weiss G. {\it Traces on ideals and the commutator property}, Operator theory: themes and variations: conference proceedings, Timisoara, June 27 - July 2, 2016 (Hari Bercovici, ed.), Theta series in advanced mathematics, vol.~20, Theta Society, Bucharest, 2018, p.~145--153.
\bibitem{NeshTus} Neshveyev S., Tuset L. {\it The Dirac operator on compact quantum groups.} J. Reine Angew. Math. {\bf 641} (2010), 1--20; Dabrowski L., Landi G., Sitarz A., van Suijlekom W., Varilly J. {\it The Dirac operator on $SU_q(2).$} Comm. Math. Phys. {\bf 259} (2005), no. 3, 729--759.
\bibitem{PotapovSukochev} Potapov D., Sukochev F. {\it Operator-Lipschitz functions in Schatten-von Neumann classes.} Acta Math. {\bf 207} (2011), no. 2, 375--389; Potapov D., Sukochev F. {\it Unbounded Fredholm modules and double operator integrals.} J. Reine Angew. Math. {\bf 626} (2009), 159--185.
\bibitem{Rennie} Rennie A. {\it Smoothness and locality for nonunital spectral triples.} K-Theory {\bf 28} (2003), no. 2, 127-–165; Rennie A. {\it Summability for nonunital spectral triples. } K-Theory {\bf 31} (2004), no. 1, 71--100.
\bibitem{Rosenberg} Rosenberg S. {\it The Laplacian on a Riemannian manifold. An introduction to analysis on manifolds.} London Mathematical Society Student Texts, {\bf 31}. Cambridge University Press, Cambridge, 1997.
\bibitem{SUZ-indiana} Sukochev F., Usachev A., Zanin D. {\it Singular traces and residues of the $\zeta$-function.} Indiana U. Math. J., {\bf 66} (2017), no. 4, 1107--1144.
\bibitem{SZspectral} Sukochev F., Zanin D. {\it Which traces are spectral?}, Adv. Math. {\bf 252} (2014), 406--428.
\bibitem{SZ-cmp} Sukochev F., Zanin D. {\it Connes integration formula for the noncommutative plane.} Comm. Math. Phys. {\bf 359} (2018), no. 2, 449--466.
\bibitem{Weyl} Weyl H. {\it Das asymptotische Verteilungsgesetz der Eigenwerte linearer partieller Differentialgleichungen (mit einer Anwendung auf die Theorie der Hohlraumstrahlung).} Math. Ann. {\bf 71} (1912), no. 4, 441--479. 
\end{thebibliography}


\end{document}

