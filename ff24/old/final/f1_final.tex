 \documentclass[12pt]{article}
\usepackage{amsmath,amsfonts,cite,color}
\usepackage[a4paper,margin=0.5cm]{geometry}

\newtheorem{thm}{Theorem}

\renewcommand{\baselinestretch}{0.95}
\newcommand{\archeading}[1]{\vspace{.3cm} \noindent{\bfseries #1} \vspace{.1cm}   }
\renewcommand{\refname}{{\normalsize References}}

\newcommand{\hl}{\color{blue}}
\newcommand{\edcomment}{\color{red}}


\begin{document}

\parindent=20pt
\pagestyle{empty}
\newpage

\archeading{F1. Justification of Future Fellowship non-salary funding requested from the ARC.} 

\bigskip\archeading{Travel}

Collaboration is a central part of the development, conduct and impact of Mathematics research. The ideas in Section C1 were developed as a result of conversations with overseas colleagues, the technical expertise of colleagues will accelerate the achievement of results, and it is through joint publication and networks that the impact of the Fellowship is maximised.

Below we describe and justify funds for travel and collaboration requested from the ARC. We indicate where our collaborators are contributing external funding. The UNSW School of Mathematics and Statistics will provide the Fellow with additional travel funds. Section F2 describes the contribution of UNSW funds to collaboration during the Fellowship.

Year 1: 15800 AUD in total. Costs are estimated.

(a) Professor Nigel Higson from Penn State University, USA, is a leading expert in non-commutative geometry and group representations. The Fellow discussed with him Aims 6 and 7 during a conference in Chengdu, China (May 2018) and during his short-term visit to UNSW (Dec 2018). Collaboration with Professor Higson is important for accelerating progress on Aims 6 and 7 of the project and will greatly increase the impact of achieved results. Professor Higson invited the Fellow to his home institution for the duration of up to 1 month. The Fellow and one of the PhD students plan to visit him in January 2020. It is an excellent opportunity for a PhD student in operator algebras and non-commutative analysis; the techniques in group representations he will learn from Professor Higson will allow him to contribute to Aim 6 of the project. Funding requested: return economy airfare Sydney-New York 2x2500 AUD; per diem 30 days x 100 AUD/day + 30 days x 80 AUD/day; 10400 AUD in total. Accommodation for the Fellow and for the PhD student and local transportation will be provided by Professor Higson.

(b) Professor Kordyukov from Ufa University, Russia, is known for his work in the geometry of the manifolds and foliation theory. The Fellow has continuous e-mail contact with him regarding Aims 1, 2 and 3 and his involvement is required for the success of these aims. For example, building a spectral triple from compact manifold is simple, but in the non-compact case this is a substantial result communicated to us by Kordyukov. Professor Kordyukov invited the Fellow to Ufa for the duration up to 1 month. The Fellow plans to visit him in May or June 2020. Funding requested: return economy airfare Sydney-Ufa and local transportation 3000 AUD; accommodation 30 days x 50 AUD/day; per diem 30 days x 30 AUD/day; 5400 AUD in total.


Year 2: 15900 AUD in total. Details are provided below. Costs are estimated.

(a) Reciprocal visit of Professor Higson to UNSW will ensure the success of Aims 6 and 7. The Fellow invited him to visit UNSW in 2021 (precise dates to be determined at a later stage) for the duration up to 1 month. Funding requested: accommodation 30 days x 150 AUD/day; per diem 30 days x 100 AUD/day; 7500 AUD in total. Airfare will be paid by Professor Higson.

(b) Professor Ponge from Seoul National University, South Korea, was a PhD student of Alain Connes and is the top specialist in non-commutative conformal geometry. The computations on the Minakshisundaram-Plejel theorem mentioned in Section C1 will be investigated in collaboration with Professor Ponge. His contribution and expertise are central for Aims 4 and 5 of the project. Professor Ponge invited the Fellow to visit him at Seoul for the duration up to 1 month. The Fellow and one of the PhD students will visit him in 2021 (precise dates to be determined at a later stage). The PhD student will learn Professor Ponge's developments on the relation between heat expansion and conformal geometry and work intesively on calculations for the noncommutative torus and noncommutative plane examples. Funding requested: return economy airfare Sydney-Seoul 2x1500 AUD; per diem 30 days x 100 AUD/day + 30 days x 80 AUD/day; 8400 AUD in total. Accommodation for the Fellow and for the PhD student will be paid by Professor Ponge.

Year 3: 13000 AUD in total. Details are provided below. Costs are estimated.

(a) Reciprocal visit of Professor Ponge to UNSW should accelerate the finalisation of Aims 4 and 5. The Fellow invited him to visit UNSW in 2022 (precise dates to be determined at a later stage) for the duration up to 1 month. Funding requested: accommodation 30 days x 150 AUD/day; per diem 30 days x 100 AUD/day; 7500 AUD in total. Airfare will be paid by Professor Ponge.

(b) Dr Caspers from Delft University of Technology, Netherlands, is a top specialist in both Double Operator Integrals and in quantum groups and also a long term collaborator of the Fellow. His expertise in Double Operator Integrals will be beneficial for Aim 5 as the transference principle invented by Dr Caspers may yield new integral formulae similar to the ones in the Approach to Aim 5. On the other hand, Dr Caspers knowledge of quantum groups is extremely relevant for Aim 6 (his unpublished note shows some progress on spectral triples for certain quantum groups). Dr Caspers invited the Fellow to visit him in Delft for the duration up to 1 month. The Fellow plans to visit him in 2022 (precise dates to be determined at a later stage). Funding requested: return economy airfare Sydney-Amsterdam  and local transportation 2500 AUD; per diem 30 days x 100 AUD/day; 5500 AUD in total. Accommodation for the Fellow will be paid by Dr Caspers.

Year 4: 14300 AUD in total. Details are provided below. Costs are estimated.

(a) Professor Dykema from Texas A\&M University, USA, is one of the creators of free Probability Theory and a world leading expert in Operator Theory. He is a long-term collaborator of the Fellow and conversations with him were important for the foundation of this project. Collaboration with Professor Dykema is invaluable in the later stages of the project to maximise impact and look for wider applications. He invited the Fellow to visit his home institution for the duration up to 1 month. The Fellow will visit him in 2023 (precise dates to be determined at a later stage). Funding requested: return economy Sydney-College Station 2500 AUD; per diem 30 days x 100 AUD/day; 5500 AUD in total. Accommodation for the Fellow will be paid by Professor Dykema.

(b) Reciprocal visit of Dr Caspers to UNSW serves the purpose of developing future applications (mostly, to representations of quantum groups) and strategy. Dr Caspers will visit the Fellow in 2023 (precise dates to be determined at a later stage). Funding requested: 50\% of the return economy airfare Amsterdam-Sydney 1300 AUD; accommodation 30 days x 150 AUD/day; per diem 30 days x 100 AUD/day; 8800 AUD in total. Airfare will be partially paid by Dr Caspers.


\end{document}

