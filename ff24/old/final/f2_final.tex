\documentclass[12pt]{article}
\usepackage{amsmath,amsfonts,cite,color}
\usepackage[a4paper,margin=0.5cm]{geometry}

\newtheorem{thm}{Theorem}

\renewcommand{\baselinestretch}{0.95}
\newcommand{\archeading}[1]{\vspace{.3cm} \noindent{\bfseries #1} \vspace{.1cm}   }
\renewcommand{\refname}{{\normalsize References}}

\newcommand{\hl}{\color{blue}}
\newcommand{\edcomment}{\color{red}}


\begin{document}

\parindent=20pt
\pagestyle{empty}
\newpage

\archeading{F2. Details of Administering Organisation contributions.} 

\bigskip\archeading{Travel}

UNSW School of Mathematics and Statistics will provide the Fellow with travel funds (10000 AUD annually). Using these funds, the Fellow and PhD students will attend international and domestic conferences which reduces funds requested from the ARC and increases collaboration and impact. The travel plan below does not interfere with the research visits specified in Section F1. All costs are estimated.

\smallskip Year 1: International. 6000 AUD in total.

\smallskip (a) Conference "Interactions of non-commutative analysis and quantum information theory" at Harbin Institute of Technology, China, June 2020. The Fellow will deliver preliminary results on Aims 1 and 2. Funding utilised: return economy airfare Sydney-Harbin 1500 AUD; accommodation 5 days x 80 AUD/day; per diem 5 days x 80 AUD/day; 2300 AUD in total.

(b) International Workshop on Operator Theory and Applications at Lancaster University, UK, July 2020. The Fellow will deliver preliminary results on Aim 3. Funding utilised: return economy airfare Sydney-London and local transportation 2500 AUD; accommodation 5 days x 120 AUD/day; per diem 5 days x 120 AUD/day; 3700 AUD in total. 

\smallskip Year 2: International. 10000 AUD in total.

\smallskip (a) Great Plains Operator Theory Symposium (USA, location to be determined). Attendees: Fellow and one of the PhD students. The Fellow will deliver results on Aims 1, 2 and 3. Funding utilised: return economy airfare 2x2500 AUD; accommodation 2 x 5 days x 80 AUD/day; per diem 5 days x 100 AUD/day + 5 days x 80 AUD/day; 6700 AUD in total.

(b) Conference "Non-commutative harmonic analysis and non-commutative probability theory" at Bedlewo, Poland. Attendees: Fellow. The Fellow will deliver preliminary results on Aim 6. Funding utilised: return economy airfare Sydney-Warsaw and local transportation 2500 AUD; conference fee 300 AUD; per diem 7 days x 70 AUD/day; 3290 AUD in total.

\smallskip Year 3: International. 10000 AUD in total.

\smallskip (a) Canadian Operator Symposium (Canada, location to be determined). Attendees: Fellow and one of the PhD students. The Fellow will deliver results on Aims 4 and 5. Funding utilised: return economy airfare and local transportation 2x2500 AUD; accommodation 2 x 5 days x 100 AUD/day; per diem 5 days x 100 AUD/day + 5 days x 80 AUD/day; 6900 AUD in total.

(b) Conference "Quantum groups in non-commutative geometry" in Oberwolfach. Attendees: Fellow. The Fellow will deliver preliminary results on Aim 7. Funding utilised: return economy airfare Sydney-Frankfurt 2500 AUD; per diem 6 days x 100 AUD/day; 3100 AUD in total.

\smallskip Year 4: International. 10000 AUD in total.

\smallskip  (a) Conference "Operator Theory 33" (Romania, location to be determined). Attendees: Fellow and both PhD students. The Fellow will deliver the final results on Aims 6 and 7. Funding utilised: return economy airfare Sydney-Bucharest and local transportation 3x2500 AUD; accommodation 3 x 5 days x 100 AUD/day; per diem 5 days x 80 AUD/day + 2 x 5 days x 60 AUD/day; 10000 AUD in total.

(b) Conference travel at this stage of the project should target international conferences that would maximise impact of the results. This may result from invitations or conferences not yet scheduled four years in advance. Administering Organisation funds will be utilised for these opportunities and may adjust conference plans as above.

\smallskip Years 1-4: Domestic. 4000 AUD in total.

\smallskip The outcomes of this project will be of interest in Australia, and broaden existing research strengths in new directions with the development of asymptotics for noncommutative Laplacians as mentioned under "Benefit and Collaboration" in Section C1. UNSW Funding to an estimated amount of 4000 AUD will be utilised as appropriate during the Fellowship to allow the Fellow and PhD students to communicate the progress and results of the Fellowship at Australian meetings and interact with other research groups within Australia (e.g. ANU, Adelaide, Sydney, Wollongong).

\end{document}

