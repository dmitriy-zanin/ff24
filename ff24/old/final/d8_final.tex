\documentclass[12pt]{article}
\usepackage{amsmath,amsfonts,cite,color}
\usepackage[a4paper,margin=0.5cm]{geometry}
\usepackage{mathptmx}

\newtheorem{thm}{Theorem}

\renewcommand{\baselinestretch}{0.95}
\newcommand{\archeading}[1]{\vspace{.3cm} \noindent{\bfseries #1} \vspace{.1cm}   }
\renewcommand{\refname}{{\normalsize References}}

\begin{document}

\archeading{D8. Research Opportunity and Performance Evidence (ROPE) - Details of the Future Fellowship candidate’s academic career and opportunities for research, evidence of research impact and contributions to the field, including those most relevant to this application}

\archeading{Amount of time as an active researcher}

I was awarded my PhD in Mathematics 7 years ago in 2011 at Flinders University of South Australia. During those 7 years, I have had no interruptions in research opportunity.

\archeading{Research opportunities}

Currently I am employed at UNSW as Scientia Fellow in the School of Mathematics and Statistics. It is a research intensive position that began July 2018 and will last 4 years. My workload is 75\% research and 25\% teaching including high-level courses for honours students. At the end of the Scientia Fellowship, my position becomes permanent.

Previously, I was employed as an ARC DECRA Fellow at UNSW (July 2015 -- June 2018). My DECRA project titled "On the new concept of independence in non-commutative probability theory" was ranked highly and produced 29 publications in high ranked international journals (see the list of publications in Section D9 below).

From 2011 until June 2015 I was employed as a researcher under DP grants DP110100064 and DP120103263 awarded to Prof. Sukochev. I contributed subsequently to both DP110100064 and DP120103263. The positions under Prof. Sukochev afforded me the opportunity to extend my areas of interest and expertise in noncommutative probability and functional analysis to noncommutative geometry, and engage internationally (Germany, USA, Russia, China, Poland) through invited visits and attendance at workshops such as Oberwolfach and Bedlewo. Without a permanent position I have been unable to apply for funding under the ARC Discover Project scheme directly.

A number of brilliant mathematicians have made, and continue to make, a significant impact on my research career. While completing my undergraduate study in Tashkent, Uzbekistan, I was impressed by Professor Chilin's attitude to Mathematics. It would be fair to say that following his example I decided to be a mathematician. My PhD supervisor Professor Sukochev taught me the foundations of non-commutative integration theory, but also an approach to Mathematics involving strict work ethics, enthusiasm and perseverance. From 2009, I was introduced to the problems on singular traces and was fortunate to work with Professor Nigel Kalton, a giant of functional analysis and Banach Medal winner. I was positively shocked by the depth and speed of this great thinker, whose insight and technical ability dwarfed everyone else I had encountered. The impression is with me still of the modesty of his genius, he was gentle and patient with those who were not grasping the details as quickly as himself. The technical devices introduced by Professor Kalton subsequently inspired my own inventions in this area. Since 2013, I had an opportunity to collaborate with Professor Dykema during my visits to Texas A\&M University and his reciprocal visits to UNSW. Professor Dykema introduced me to the advanced techniques of non-commutative probability which helped me a lot during my DECRA project. In 2017, I was invited by Professor Connes to talk at his 70th anniversary conference and this developed into a collaboration. Professor Connes is a Fields Medal and Crafoord Prize winner, both are equivalents to the Nobel Prize in Mathematics, so needless to say this collaboration has impressed and inspired me deeply. % Work with Professor Connes taught me how to merge a different, seemingly unrelated, areas of Mathematics into a single project.

\archeading{Research achievements and contributions}

I started my research in non-commutative functional analysis at Flinders University as a PhD student. Since then my contribution to this research field include my PhD Thesis and over 50 peer-reviewed publications including a monograph \lq\lq Singular traces: Theory and Applications\rq\rq~in the De Gruyter series ``Studies in Mathematics''. My research articles are published in high ranked journals such as Journal f\"ur die Reine und Angewandte Mathematik (Crelle's Journal), Advances in Mathematics, Journal of Functional Analysis, Journal of Spectral Theory, Pacific Journal of Mathematics as described in Section D9. All numbered references in this Section refer to the numbered publications in Section D9. My PhD Thesis was highly regarded by the referees, one of them wrote \lq\lq this is the best PhD thesis I ever refereed during my whole career\rq\rq. My subsequent work spans non-commutative analysis, non-commutative geometry and non-commutative probability theory.

Some of the principal achievements of my work now follow.  The primary objective of my PhD study was two-fold. One direction was studying orbits in symmetric function and sequence spaces. The particular question of interest was whether an orbit of a semi-group of bicontractions is a norm-closed convex hull of its extreme points. While answering this question, I established a collaboration with Nigel Kalton. I published a number of research papers on orbits; one of them with Professor Kalton. The other direction concerned Khinchine-type inequalities. The principal tool in the studies of orbits involves Khinchine-type inequalities; a deep extension of the Birkhoff theorem on doubly stochastic matrices. By employing this technique in my PhD Thesis I was able to (i) develop a unified approach to orbits on symmetric function and sequence spaces; (ii) significantly broaden the setting of applicability of earlier results; (iii) to determine the conditions that are necessary and sufficient for the affirmative answer to the above question. Central to the study of probabilistic inequalities (such as Khinchine inequalities) is the so-called Kruglov operator (an integration over a Poisson process). Using the Kruglov operator, I radically extended the area where Khinchine-type inequalities hold true.

Based on the advances in the theory of symmetric function spaces made in my PhD Thesis I obtained several profound results in non-commutative analysis. One of them resolves a long standing question of existence of continuous traces on Banach ideals in $\mathcal{L}(H)$ (the $*-$algebra of all bounded operators on a separable Hilbert space $H$). It was already known to experts that traces (even without the continuity assumption) do not exist on an ideal which can be obtained from the couple $(\mathcal{L}_p,\mathcal{L}_{\infty}),$ $p>1,$ by interpolation. Using ideas from my PhD thesis, I proposed to study a weaker condition:
\begin{equation}\label{my condition}
\frac1n\|A^{\oplus n}\|_{\mathcal{I}}\to0\mbox{ as }n\to\infty\mbox{ for every operator }A\mbox{ in the ideal }\mathcal{I}.
\end{equation}
The condition \eqref{my condition} holds for every interpolation space as above and prevents the existence of a continuous trace on the Banach ideal $\mathcal{I}.$ In my paper [10] (joint with my former supervisor F. Sukochev, published in Crelle's Journal), the converse assertion is proved; namely, condition \eqref{my condition} holds for every Banach ideal $\mathcal{I}$ without continuous traces. In the same paper, I constructed (under the assumption that \eqref{my condition} fails) a positive continuous trace which respects the Hardy-Littlewood submajorization (and also the one that does not). 

It is a standard result in linear algebra that a trace of a matrix depends only on its eigenvalues. A corresponding result for the standard trace on the trace class ideal $\mathcal{L}_1$ is radically harder (it was resolved in the affirmative by V. Lidskii only in 1959). More precisely, the following formula holds.
$${\rm Tr}(A)={\rm Tr}({\rm diag}(\lambda(A)))\mbox{ for every operator }A\mbox{ in the ideal }\mathcal{L}_1.$$
Here, ${\rm diag}(\lambda(A))$ is a diagonal operator with eigenvalues of $A$ on the diagonal (repeated with multiplicities). In his famous 1990 paper, Pietsch asked  whether the same is true for an arbitrary trace on an arbitrary%\footnote{An ideal is assumed to be a purely algebraic object, without any topology on it.} ideal in $\mathcal{L}(H).$
He characterized this question as \lq\lq extremely difficult\rq\rq. A partial answer for countably generated ideals was given by Kalton in his 1998 paper. In my paper [9] (joint with F. Sukochev, published in Advances in Mathematics), a class of ideals closed with respect to the logarithmic submajorization is introduced. In this paper, it is proved that if an ideal $\mathcal{I}$ is closed with respect to the logarithmic submajorization, then
$$\varphi(A)=\varphi({\rm diag}(\lambda(A)))$$
for every operator $A\in\mathcal{I}$ and for every trace $\varphi$ on $\mathcal{I}.$ Hence, for the mentioned class of ideals Pietsch's question is answered in the affirmative. Conversely, if the ideal $\mathcal{I}$ is not closed with respect to the logarithmic submajorization, then there exists an operator $A\in\mathcal{I}$ such that ${\rm diag}(\lambda(A))\notin\mathcal{I}$ and the question is answered in the negative.

A classical result due to Schur states that every matrix is unitarily conjugate to an upper-triangular one. Thus, it is a sum of a normal matrix (diagonal part) and a nilpotent matrix (strictly upper-triangular part). A corresponding result for compact operators in $\mathcal{L}(H)$ is proved by Ringrose in his 1962 paper. Namely, he proved that every compact operator is a sum of a normal one and a quasi-nilpotent one. It is natural to ask whether a similar result holds in the setting of type II von Neumann algebras. In 2013, I worked with Ken Dykema and we managed to solve this question by using techniques from Free Probability Theory. In my paper [8] (joint with K. Dykema and F. Sukochev, published in Crelle's Journal), it is proved that every operator $T$ in a type II$_1$ von Neumann algebra can be written as $T=N+Q,$ where $N$ is normal and $Q$ is \lq\lq almost\rq\rq nilpotent in the following sense
$$|Q^n|^{\frac1n}\to0\mbox{ as }n\to\infty\mbox{ in the strong operator topology.}$$

Connes Character Theorem is of paramount importance in non-commutative geometry. It expresses a certain cocycle (called the Chern character) in terms of Dixmier traces. This cocycle is then used to explicitly compute indices, Euler characteristics and other topological invariants via smooth expressions. In my paper [7] (joint with A. Rennie, A. Carey and F. Sukochev, published in Journal of Spectral Theory), a completely new approach to Connes Character Theorem is proposed. In a holistic way, motivation came from (Kalton et al., Adv. in Math., 2013). The result is now proved for an arbitrary trace (not only a Dixmier trace as in Connes' original approach). 

A 50-year old problem due to Krein asks whether Lipschitz functions are operator Lipschitz on a given Banach ideal. For Schatten ideals $\mathcal{L}_p,$ $1<p<\infty,$ the assertion is known to be true. Namely, Potapov and Sukochev (Acta Math., 2011) proved that
$$\|f(A)-f(B)\|_p\leq c_p\|f'\|_{\infty}\|A-B\|_p$$
for every couple of self-adjoint operators $(A,B)$ such that $A-B\in\mathcal{L}_p.$ For $p=1$ (or for $p=\infty$), a similar result is false. Kato and Davies proved that even absolute value fails to be operator Lipschitz in $\mathcal{L}_1.$ Nazarov and Peller asked whether the following replacement
\begin{equation}\label{np conj}
\|f(A)-f(B)\|_{1,\infty}\leq c_{abs}\|f'\|_{\infty}\|A-B\|_1,
\end{equation}
holds, where $\mathcal{L}_{1,\infty}$ is the principal ideal generated by the operator ${\rm diag}(\{1,\frac12,\frac13,\cdots\})$ equipped with a natural quasi-norm. 

To solve this problem, I co-operated with Martijn Caspers and we jointly created a transference method. This method reduces boundedness of certain Double Operator Integrals (which naturally arise in the setting above) to complete boundedness of certain Fourier multipliers. Using this method, we proved (in collaboration with D. Potapov and F. Sukochev) that, indeed, \eqref{np conj} holds true. The manuscript [1] is accepted for publication in the American Journal of Mathematics. The result has been verified by leading experts G.~Pisier, Q.~Xu, V.~Peller, M.~Junge and others and has now become accepted as an outstanding achievement in operator calculus. Some of the key machinery of our proof was developed in our earlier publications.

Inequalities proved by Johnson and Schechtman relate the norm of the sum of independent random variables to that of their disjoint copies. Namely, if $E\supset L_p,$ $p<\infty,$ is a symmetric function space on $(0,1)$ and if $x_k\in E$ are mean zero independent random variables, then
$$\|\sum_{k\geq0}x_k\|_E\approx_E\|\bigoplus_{k\geq0}x_k\|_{Z_E^2},$$
where $Z_E^2$ is a certain symmetric function space on $(0,\infty).$ Astashkin and Sukochev (Isr. J. Math., 2005) proposed another method of proving Johnson-Schechtman inequalities by means of the so-called Kruglov operator (which is an integration with respect to the Poisson process). In my paper (joint with S. Astashkin and F. Sukochev, published in Pacific Journal of Mathematics), it is proved that Johnson-Schechtman inequalities hold in a symmetric (quasi-Banach) function space if and only if the Kruglov operator is bounded on that space. It is tempting to extend the technique related to the Kruglov operator to the domain of the non-commutative probability. In my paper (joint with F. Sukochev, published in Journal of Functional Analysis), I constructed an analogue of the Kruglov operator in the Free Probability Theory. It appears that Kruglov operator behaves better in the setting of free probability than in the setting of a classical probability; and that Johnson-Schechtman inequalities are always true.

This breakthrough inspired a question: what sort of independence suffices to guarantee the Johnson-Schechtman inequalities for a sufficiently rich class of symmetric spaces?  I proposed this investigation as the subject for my DECRA project approved by the ARC in November 2014. We shared this question with a well-known Chinese probabilist Yong Jiao. Professor Jiao than came to UNSW for a long-term research visit, during which we obtained a significant achievement: if a symmetric space $E$ is an interpolation space for the couple $(L_p,L_q),$ $1<p<q<\infty,$ then Johnson-Schechtman inequalities hold true in $E$ for an arbitrary sequence of mean zero random variables independent in the sense of Junge and Xu. This paper [6] is published in the Journal of the London Mathematical Society. In 2016, it was presented on the Symposium on Modern Analysis and Applications in Harbin.

In 2016, Alain Connes suggested to me the conjecture from his book "Noncommutative geometry". This conjecture due to Connes and Sullivan stood for more than 20 years. There are 2 versions of the conjecture (for Julia sets as stated below) and for Kleinian groups. Let $f_c$ be a quadratic polynomial $f:z\to z^2+c.$ Julia set of $f$ is a Jordan curve if and only if $c$ is in the main cardioid of the Mandelbrot set. For such $c,$ there exists an analytic mapping $Z$ with the property
$$Z\circ f_0=f_c\circ Z.$$
The conjecture asserts that
$$\varphi(M_h|[F,M_Z]|^p)=\int_{J(f_c)}h(z)d\nu(z)$$
for every singular trace $\varphi$ on $\mathcal{L}_{1,\infty}.$ Here, $F$ is the Hilbert transform and $\nu$ is the $p-$dimensional geometric measure on the Julia set $J(f_c).$ In other words, one can recover the geometric measure via singular traces.

Jointly with Alain Connes, Edward McDonald and Fedor Sukochev, we solved this conjecture (see [3] and [13]). I delivered a keynote talk about resolution of this conjecture in the international conference "Noncommutative Geometry: State of the Art and Future Prospects" celebrating Connes's 70'th birthday March 29-April 2 2017 at the Fudan Insitute for Advanced Study in Shanghai. I was among two-dozen invited speakers including Sir Michael Atiyah (Fields Medal, De Morgan Medal, Abel Prize), Pierre Cartier (Ampere Prize), Alain Connes (Fields Medal, Grafoord Prize), Joachim Cuntz (Gottfried Wilhelm Leibniz Prize), Sorin Popa (Ostrowski Prize), Graeme Segal (Sylvester Prize), Dennis Sullivan (Wolf Prize, Oswald Veblen Prize, National Medal of Science), and Dan-Virgil Voiculescu (NAS Award in Mathematics).

Many of these research achievements underpin the research proposed for this Future Fellowship, and demonstrate my insight and ability in solving hard problems in mathematics using novel methods that often extend to new results in considerably more general contexts.

\end{document}
