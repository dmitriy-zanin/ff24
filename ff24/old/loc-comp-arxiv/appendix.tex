\chapter{Appendix}

\section{Properties of the algebra $\mathcal{B}$}\label{b prop app}
    For this section, $(\mathcal{A},H,D)$ is a smooth spectral triple.
    Recall from Definition \ref{smoothness definition} that $\mathcal{B}$ is the $*-$algebra generated by all elements of the form $\delta^k(a)$ or $\partial(\delta^k(a)),$ $k\geq0,$ $a\in\mathcal{A}.$
    Recall that we define $H_\infty := \bigcap_{k\geq 1} \mathrm{dom}(D^k)$, and that for all $T \in \mathcal{B}$, we have $T:H_\infty\to H_\infty$, and for all $k\geq 0$ we have $D^k,|D|^k:H_\infty\to H_\infty$.
    
    The following should be compared with \cite[Lemma 6.2]{CPRS2}. See also the discussion following \cite[Lemma 10.22]{GVF}.
    \begin{lem}\label{left to right lemma}
        For every $x\in\mathcal{B}$ and for every $m\geq0,$ we have the following equalities
        of linear (potentially unbounded) operators on $H_\infty$:
        \begin{align*}
            |D|^mx &= \sum_{k=0}^m \binom{m}{k}\delta^{m-k}(x)|D|^k\text{ and, }\\
            x|D|^m &= \sum_{k=0}^m (-1)^{m-k}\binom{m}{k}|D|^k\delta^{m-k}(x).
        \end{align*}
    \end{lem}
    \begin{proof}
        We prove only the first equality as the proof of the second one follows by an identical argument.

        This formula can be seen by induction on $m.$ Indeed, for $m=1$, this is simply the claim
        that since $\mathcal{B} \subseteq \mathrm{dom}_\infty(\delta)$ we have an equality of operators on $H_\infty$:
        \begin{equation*}
            |D|x = \delta(x)+x|D|.
        \end{equation*}
        
        Now suppose that the claim is true for $m-1$. Then on $H_\infty$ we have
        \begin{align*}
            |D|^mx &= |D|\cdot|D|^{m-1}x \\
                   &= |D|\cdot\sum_{k=0}^{m-1}\binom{m-1}{k}\delta^{m-1-k}(x)|D|^k\\
                   &= \sum_{k=0}^{m-1}\binom{m-1}{k}|D|\delta^{m-1-k}(x)\cdot|D|^k\\
                   &= \sum_{k=0}^{m-1}\binom{m-1}{k}\delta^{m-1-k}(x)|D|^{k+1}+\sum_{k=0}^{m-1}\binom{m-1}{k}\delta^{m-k}(x)|D|^k\\
                   &= \sum_{k=1}^m\binom{m-1}{k-1}\delta^{m-k}(x)|D|^k+\sum_{k=0}^{m-1}\binom{m-1}{k}\delta^{m-k}(x)|D|^k\\
                   &= \sum_{k=0}^m\binom{m}{k}\delta^{m-k}(x)|D|^k.
        \end{align*} 
        and so the statement follows for $m$.
    \end{proof}

    \begin{lem}\label{left to right corollary} 
        Let $(\mathcal{A},H,D)$ be a smooth spectral triple, and assume that $D$ has a spectral gap at $0.$ Then for all $x \in \mathcal{B}$ and $m\geq 0$ we have
        \begin{enumerate}[{\rm (i)}]
            \item\label{left to right 1} The operators $|D|^{-m}x|D|^m,|D|^mx|D|^{-m}:H_\infty\to H_\infty$ have bounded extension
            \item\label{left to right 2} $|D|^{1-m}[|D|^m,x]:H_\infty\to H_\infty$ has bounded extension.
        \end{enumerate}
    \end{lem}
    \begin{proof} 
        By Lemma \ref{left to right lemma}, on $H_\infty$ we have:
        \begin{align*}
            |D|^mx|D|^{-m} &= \sum_{k=0}^m\binom{m}{k}\delta^{m-k}(x)|D|^{k-m},\\
            |D|^{-m}x|D|^m &= \sum_{k=0}^m(-1)^{m-k}\binom{m}{k}|D|^{k-m}\delta^{m-k}(x).  
        \end{align*}
        Clearly, the expressions on the right hand side have bounded extension. This proves the first assertion.

        By Lemma \ref{left to right lemma}, we have
        \begin{align*}
            [|D|^m,x] &= |D|^mx-\sum_{k=0}^m(-1)^{m-k}\binom{m}{k}|D|^k\delta^{m-k}(x)\\
                      &= \sum_{k=0}^{m-1}(-1)^{m-k-1}\binom{m}{k}|D|^k\delta^{m-k}(x).
        \end{align*}
        Therefore,
        $$|D|^{1-m}[|D|^m,x]=\sum_{k=0}^{m-1}(-1)^{m-k-1}\binom{m}{k}|D|^{k+1-m}\delta^{m-k}(x).$$
        Since $x \in \mathrm{dom}_\infty(\delta)$, for each $0 \leq k \leq m-1$ the operator $|D|^{k+1-m}\delta^{m-k}(x)$ has bounded extension. Hence, $|D|^{1-m}[|D|^m,x]$
        has bounded extension.
    \end{proof}
    
    \begin{lem}\label{schwartz lemma} 
        Assume that $(\mathcal{A},H,D)$ satisfies Hypothesis \ref{main assumption}. Let $h$ be a Borel function on $\mathbb{R}$ such that
        \begin{equation*}
            t \mapsto (1+t^2)^{\frac{p+1}{2}}h(t), \quad t \in \mathbb{R}.
        \end{equation*}
        is bounded.
        Then for all $x \in \mathcal{B}$ and $s > 0$ the operator $xh(sD)$ is in $\mathcal{L}_1$, and:
        \begin{equation*}
            \|xh(sD)\|_1=O(s^{-p}),\quad s\downarrow0.
        \end{equation*}
    \end{lem}
    \begin{proof} 
        Let $s > 0$. Clearly,
        $$(1+s^2D^2)^{-\frac{p+1}{2}}=|(1-isD)^{-p-1}|.$$
        Setting $\lambda=\frac1s,$ we obtain from Hypothesis \ref{main assumption} that:
        \begin{align*}
            \|x(1+s^2D^2)^{-\frac{p+1}{2}}\|_1 &= s^{-p-1}\|x(D+\frac{i}{s})^{-p-1}\|_1\\
                                               &= s^{-p-1}\cdot O(s)\\
                                               &= O(s^{-p}),\quad s\downarrow0.
        \end{align*}

        Since the operator $(1+s^2D^2)^{\frac{p+1}{2}}h(sD)$ is bounded, with
        \begin{equation*}
            \|(1+s^2D^2)^{\frac{p+1}{2}}h(sD)\|_\infty \leq \sup_{t \in \mathbb{R}} (1+t^2)^{\frac{p+1}{2}}|h(t)|
        \end{equation*}
        We can conclude that:
        \begin{align*}
            \|xh(sD)\|_1 &\leq \|x(1+s^2D^2)^{-\frac{p+1}{2}}\|_1\|(1+s^2D^2)^{\frac{p+1}{2}}h(sD)\|_\infty\\
                         &= O(s^{-p}),\quad s\downarrow 0.
        \end{align*}
    \end{proof}
    We note in particular that the assumption on $h$ in Lemma \ref{schwartz lemma} is satisfied if $h$ is a Schwartz function.

    \begin{lem}\label{first decay lemma} 
        Let $(\mathcal{A},H,D)$ satisfy Hypothesis \ref{main assumption}, assume that $D$ has a spectral gap at $0$ and let $x\in\mathcal{B}$. For all non-negative integers $m_1,m_2 \geq 0$ with $m_1+m_2<p,$ we have
        \begin{equation*}
            \||D|^{-m_1}x|D|^{-m_2}e^{-s^2D^2}\|_1 = O(s^{m_1+m_2-p}),\quad s\downarrow0.
        \end{equation*}
    \end{lem}
    \begin{proof} 
        Suppose first that $m_1>0.$ Using the triangle inequality:
        \begin{align*}
            \||D|^{-m_1}x|D|^{-m_2}e^{-s^2D^2}\|_1 &\leq \sum_{k,l=0}^{\infty}\|\chi_{[2^k,2^{k+1}]}(|D|)|D|^{-m_1}x|D|^{-m_2}e^{-s^2D^2}\chi_{[2^l,2^{l+1}]}(|D|)\|_1\\
                                                   &\leq \sum_{k,l=0}^{\infty}2^{-km_1-lm_2}\cdot e^{-2^{2l}s^2}\cdot\|\chi_{[2^k,2^{k+1}]}(|D|)x\chi_{[2^l,2^{l+1}]}(|D|)\|_1\\
                                                   &\leq \sum_{k,l=0}^{\infty}2^{-km_1-lm_2}\cdot e^{-2^{2l}s^2}\cdot\|\chi_{[0,2^{k+1}]}(|D|)x\chi_{[0,2^{l+1}]}(|D|)\|_1.
        \end{align*}

        If $m=\min\{k,l\},$ then
        \begin{align*}
            \|\chi_{[0,2^{k+1}]}(|D|)x\chi_{[0,2^{l+1}]}(|D|)\Big\|_1 &\leq \min\Big\{\Big\|x\chi_{[0,2^{m+1}]}(|D|)\Big\|_1,\Big\|\chi_{[0,2^{m+1}]}(|D|)x\Big\|_1\Big\}\\
                                                                      &\leq e\min\Big\{\Big\|xe^{-2^{-2(m+1)}|D|^2}\Big\|_1,\Big\|x^*e^{-2^{-2(m+1)}|D|^2}\Big\|_1\Big\}\\
                                                                      &= O(2^{mp}).
        \end{align*}

        Thus,
        \begin{equation*}
            \||D|^{-m_1}x|D|^{-m_2}e^{-s^2D^2}\|_1 \leq \Big(\sum_{k,l=0}^{\infty}2^{-km_1-lm_2}\cdot e^{-2^{2l}s^2}\cdot 2^{p\cdot\min\{k,l\}}\Big)\cdot O(1).
        \end{equation*}

        Since $p-m_1>0,$ it follows that
        \begin{align*}
            \sum_{\substack{k,l\geq0\\ k\leq l}}2^{-km_1-lm_2}\cdot e^{-2^{2l}s^2}\cdot 2^{p\cdot\min\{k,l\}} &= \sum_{\substack{k,l\geq0\\ k\leq l}}2^{(p-m_1)k}\cdot 2^{-lm_2}\cdot e^{-2^{2l}s^2}\\
                                                                                                              &\leq 2\cdot \sum_{l=0}^{\infty}2^{(p-m_1)l}\cdot 2^{-lm_2}\cdot e^{-2^{2l}s^2}.
        \end{align*}
        Now due to our assumption that $m_1>0,$ it follows that
        \begin{align*}
            \sum_{\substack{k,l\geq0\\ k\geq l}}2^{-km_1-lm_2}\cdot e^{-2^{2l}s^2}\cdot 2^{p\cdot\min\{k,l\}} &= \sum_{\substack{k,l\geq0\\ k\geq l}}2^{-km_1}\cdot 2^{(p-m_2)l}\cdot e^{-2^{2l}s^2}\\
                                                                                                              &\leq 2\cdot \sum_{l=0}^{\infty}2^{-lm_1}\cdot 2^{(p-m_2)l}\cdot e^{-2^{2l}s^2}.
        \end{align*}
        Note that for $m > 0$, we have
        \begin{equation*}
            \sum_{l=0}^\infty 2^{lm}e^{-2^{2l}s^2} = O(s^{-m}), \quad s\downarrow 0.
        \end{equation*}

        Therefore,
        \begin{align*}
            \||D|^{-m_1}x|D|^{-m_2}e^{-s^2D^2}\|_1 &\leq \Big(\sum_{l=0}^{\infty}2^{l(p-m_1-m_2)}\cdot e^{-2^{2l}s^2}\Big)\cdot O(1)\\
                                                   &= O(s^{m_1+m_2-p}).
        \end{align*}
        This completes the proof for the case $m_1>0.$

        To complete the proof we now deal with the case where $m_1=0.$ We have
        \begin{align*}
               \||D|^{-m_1}x|D|^{-m_2}e^{-s^2D^2}\|_1 &\leq \sum_{l=0}^{\infty}\Big\|x|D|^{-m_2}e^{-s^2D^2}\chi_{[2^l,2^{l+1}]}(|D|)\Big\|_1\\
                                                      &\leq \sum_{l=0}^{\infty}2^{-lm_2}\cdot e^{-2^{2l}s^2}\cdot\Big\|x\chi_{[2^l,2^{l+1}]}(|D|)\Big\|_1\\
                                                      &\leq \sum_{l=0}^{\infty}2^{-lm_2}\cdot e^{-2^{2l}s^2}\cdot\Big\|x\chi_{[0,2^{l+1}]}(|D|)\Big\|_1.
        \end{align*}
        Thus,
        \begin{align*}
            \||D|^{-m_1}x|D|^{m_2}e^{-s^2D^2}\|_1 &\leq \Big(\sum_{l=0}^{\infty}2^{l(p-m_2)}\cdot e^{-2^{2l}s^2}\Big)\cdot O(1)\\
                                                  &= O(s^{m_2-p}).
        \end{align*}
        This completes the proof for the case $m_1=0.$
    \end{proof}

    \begin{lem}\label{left convergence estimate} 
        Let $(\mathcal{A},H,D)$ be a spectral triple satisfying Hypothesis \ref{main assumption} and assume that $D$ has spectral gap at $0.$ For all $x\in\mathcal{B}$ we have
        \begin{equation*}
            \|x|D|^{-p-1}(1-e^{-s^2D^2})\|_1 = O(s),\quad s\downarrow0.
        \end{equation*}
    \end{lem}
    \begin{proof}
        By the triangle inequality, we have
        \begin{align}
            \|x|D|^{-p-1}(1-e^{-s^2D^2})\|_1 &\leq \|x|D|^{-p-1}(1-e^{-s^2D^2})\chi_{(\frac1s,\infty)}(|D|)\|_1\nonumber\\
                                             &\quad +\|x|D|^{-p-1}(1-e^{-s^2D^2})\chi_{[0,\frac1s]}(|D|)\|_1\label{split half line}.
        \end{align}

        Let us estimate the first summand of \eqref{split half line}. Since for $t > 1$ we have:
        $$t^{-p-1}(1-e^{-t^2})\leq t^{-p-1}\leq 2^{\frac{p+1}{2}}\cdot (t^2+1)^{-\frac{p+1}{2}},$$
        it follows that
        \begin{equation*}
            |D|^{-p-1}(1-e^{-s^2D^2})\chi_{(\frac1s,\infty)}(|D|) \leq s^{p+1}\cdot 2^{\frac{p+1}{2}}\cdot (1+s^2D^2)^{-\frac{p+1}{2}}.
        \end{equation*}
        So we may estimate the first summand by:
        \begin{align*}
            \|x|D|^{p-1}(1-e^{-s^2D^2})\chi_{(\frac{1}{s},\infty)}(|D|)\|_1 &\leq 2^{\frac{p+1}{2}}\Big\|x(D^2+s^{-2})^{-\frac{p+1}{2}}\Big\|_1 \\
                                                                            &= 2^{\frac{p+1}{2}}\Big\|x(D+\frac{i}{s})^{-p-1}\Big\|_1\\
                                                                            &= O(s)
        \end{align*}
        Where the final step follows from Hypothesis \ref{main assumption}.\eqref{ass2}.

        Let us estimate the second summand of \eqref{split half line}. We have
        $$t^{-p-1}(1-e^{-t^2})\leq t^{1-p}\leq et^{1-p}\cdot e^{-t^2},\quad t\in[0,1],$$
        so it follows that
        \begin{equation*}
            |D|^{-p-1}(1-e^{-s^2D^2})\chi_{[0,\frac1s]}(|D|) \leq es^{p+1}\cdot (s|D|)^{1-p}\cdot e^{-s^2D^2}.
        \end{equation*}
        So, the second summand in \eqref{split half line} can be estimated by:
        \begin{equation*}
            \|x|D|^{-p-1}(1-e^{-s^2D^2})\chi_{[0,\frac{1}{s})]}(|D|)\|_1 \leq es^2\Big\|x|D|^{1-p}e^{-s^2D^2}\Big\|_1.
        \end{equation*}
        From Lemma \ref{first decay lemma}, this is $O(s)$ as $s\downarrow 0$.
    \end{proof}

    \begin{lem}\label{right convergence estimate}
        Let $(\mathcal{A},H,D)$ be a spectral triple satisfying Hypothesis \ref{main assumption} and assume that $D$ has a spectral gap at $0.$ For every $x\in\mathcal{B},$ we have
        \begin{equation*}
            \||D|^{-p-1}x(1-e^{-s^2D^2})\|_1 = O(s),\quad s\downarrow0.
        \end{equation*}
    \end{lem}
    \begin{proof}
        On the subspace $H_\infty$, we have:
        \begin{align*}
            |D|^{-p-1}x &= x|D|^{-p-1}+[|D|^{-p-1},x]\\
                        &= x|D|^{-p-1}-|D|^{-p-1}[|D|^{p+1},x]|D|^{-p-1}.
        \end{align*}
        By Lemma \ref{left to right lemma}, we have (again on $H_\infty$):
        \begin{align*}
            [|D|^{p+1},x] &= |D|^{p+1}x-x|D|^{p+1}\\
                          &= \left(\sum_{k=0}^{p+1}\binom{p+1}{k}\delta^{p+1-k}(x)|D|^k\right)-x|D|^{p+1}\\
                          &= \sum_{k=0}^p\binom{p+1}{k}\delta^{p+1-k}(x)|D|^k.
        \end{align*}
        Thus on $H_\infty$:
        \begin{equation*}
            |D|^{-p-1}x = x|D|^{-p-1}-\sum_{k=0}^p\binom{p+1}{k}|D|^{-p-1}\delta^{p+1-k}(x)|D|^{k-p-1}.
        \end{equation*}
        Multiplying on the right by $(1-e^{-s^2D^2})$, on $H_\infty$ we have:
        \begin{equation}\label{left to right expansion in trace class}
            |D|^{-p-1}x(1-e^{-s^2D^2}) = x|D|^{-p-1}(1-e^{-s^2D^2})-\sum_{k=0}^p \binom{p+1}{k}|D|^{-p-1}\delta^{p+1-k}(x)|D|^{k-p-1}(1-e^{-s^2D^2})
        \end{equation}
        From Lemma \ref{left convergence estimate}, we have that $x|D|^{-p-1}(1-e^{-s^2D^2})$ has bounded extension to an operator in $\mathcal{L}_1$ with norm bounded by $O(s)$.

        For $0\leq k\leq p,$ we have
        \begin{align*}
            \Big\||D|^{-p-1}\delta^{p+1-k}&(x)|D|^{k-p-1}(1-e^{-s^2D^2})\Big\|_1\\
                                          &\leq\Big\||D|^{-p-1}\delta^{p+1-k}(x)\Big\|_1\cdot\Big\||D|^{k-p}\Big\|_{\infty}\\
                                          &\quad\cdot s\Big\|(s|D|)^{-1}(1-e^{-s^2D^2})\Big\|_{\infty}.
        \end{align*}
        The first factor here is finite by Remark \ref{sp gap fact}. The second factor is finite because $k\leq p.$ The third factor is $O(s)$ by the functional calculus. Thus,
        \begin{equation*}
            \||D|^{-p-1}\delta^{p+1-k}(x)|D|^{k-p-1}(1-e^{-s^2D^2})\|_1=O(s),\quad s\downarrow0.
        \end{equation*}        
        Hence, each summand in \eqref{left to right expansion in trace class} extends to a trace class operator with norm bounded by $O(s)$, $s\downarrow 0$. By the triangle
        inequality, this completes the proof.        
    \end{proof}

\section{Integral formulae for commutators}\label{integral app}
    In this section of the appendix, we collect results concerning formulae for commutators with functions of $D$. Many of the results of this section
    will be known to the expert reader, but since they are scattered around various sources we provide them here with short and self-contained proofs.

    In this section, $(\mathcal{A},H,D)$ is a smooth $p$-dimensional spectral triple. 
    
    The following is essentially a consequence of the classical Duhamel formula.
    \begin{lem}\label{exp commutator lemma} 
        Let $(\mathcal{A},H,D)$ be a smooth spectral triple. If $x\in\mathcal{B}$ then for all $t \in \mathbb{R}$:
        \begin{equation*}
            [e^{it|D|},x] = it\int_0^1 e^{it(1-v)|D|}\delta(x)e^{itv|D|}\,dv.
        \end{equation*}
        Here, the integral is understood in the weak operator topology sense.
    \end{lem}
    \begin{proof}
        For $n \geq 0$, we define the projection
        \begin{equation*}
            p_n := \chi_{[0,n]}(|D|).
        \end{equation*}
        Since $|D|$ is non-negative, as $n\to\infty$ the sequence of projections $\{p_n\}_{n\geq 0}$ converges in the strong operator topology to the identity.
        We define the functions $\xi$ and $\eta$ by:
        \begin{align*}
             \xi(v) &:= p_n\exp(it(1-v)|D|),\\
            \eta(v) &:= x\exp(itv|D|)p_n, \quad v \in \mathbb{R}.
        \end{align*}
        Since $p_n|D| \leq n$, the operator valued functions $\xi$ and $\eta$ are continuous and differentiable in the uniform norm. 
        
        Since $\xi$ and $\eta$ are continuous and differentiable, we have:
        \begin{equation*}
            \xi(1)\eta(1)-\xi(0)\eta(0) = \int_0^1 \xi'(v)\eta(v)+\xi(v)\eta'(v)\,dv.
        \end{equation*}
        where since $\xi,$ $\xi',$ $\eta$ and $\eta'$ are continuous, this may be considered as a Bochner integral. Therefore in particular,
        this is an integral in the weak operator topology.
        
        We can compute the terms in the integrand as:
        \begin{align*}
            \xi'(v)\eta(v) &= -itp_n\exp(it(1-v)|D|)|D|x\exp(itv|D|)p_n,\\
            \xi(v)\eta'(v) &= itp_n\exp(it(1-v)|D|)x|D|\exp(itv|D|)p_n.
        \end{align*}
        Thus, 
        \begin{equation*}
            \xi(1)\eta(1)-\xi(0)\eta(0) = -it\int_0^1 p_n\exp(it(1-v)|D|)\delta(x)\exp(itv|D|)p_n\,dv. 
        \end{equation*}
        Since the operator $\exp(it(1-v)|D|)\delta(x)\exp(itv|D|)$ is a continuous function of $v$ (in weak operator topology), the weak operator
        topology integral
        \begin{equation*}
            \int_{0}^1 \exp(it(1-v)|D|)\delta(x)\exp(itv|D|)\,dv
        \end{equation*}
        exists, and we have:
        \begin{equation*}
            p_n \int_{0}^1 \exp(it(1-v)|D|)\delta(x)\exp(itv|D|)\,dv\cdot p_n = \int_{0}^1 p_n \exp(it(1-v)|D|)\delta(x)\exp(itv|D|)p_n\,dv.
        \end{equation*}
        Therefore:
        \begin{equation*}
            \xi(1)\eta(1)-\xi(0)\eta(0) = p_n\int_{0}^1 \exp(it(1-v)|D|)\delta(x)\exp(itv|D|)\,dv\cdot p_n.
        \end{equation*}
        
        On the other hand, we can compute $\xi(1), \eta(1), \xi(0)$ and $\eta(0)$ directly:
        \begin{align*}
            \xi(1)\eta(1)-\xi(0)\eta(0) &= p_nx\exp(it|D|)p_n-p_n\exp(it|D|)xp_n\\
                                        &= p_n[x,\exp(it|D|)]p_n.
        \end{align*}
        Thus,
        \begin{equation*}
            p_n[\exp(it|D|),x]p_n = itp_n\int_0^1 \exp(it(1-v)|D|)\delta(x)\exp(itv|D|)\,dv\cdot p_n.
        \end{equation*}
        Since $n \geq 0$ is arbitrary, we may take $n\to\infty$ and since $p_n$ converges in the strong operator topology to the identity
        we get the desired equality.
    \end{proof}
    
    Combining Lemma \ref{exp commutator lemma} with the Fourier inversion theorem yields a formula for $[f(s|D|),x]$ for quite general functions $f$. The following formula is well known and appears in many places, for example \cite[Theorem 3.2.32]{Bratteli-Robinson1}.
    \begin{lem}\label{first commutator rep lemma} 
        If $\widehat{f},\widehat{f'}\in L_1(\mathbb{R}),$ then for all $x \in \mathcal{B}$, and $s > 0$,
        \begin{equation*}
            [f(s|D|),x] = s\int_{-\infty}^{\infty}\left(\int_0^1\widehat{f'}(u)e^{ius(1-v)|D|}\delta(x)e^{iusv|D|}dv\right)\,du.
        \end{equation*}
        Here, the integral is understood in a weak operator topology sense.
    \end{lem}
    \begin{proof} 
        Indeed, by the Fourier inversion formula, by functional calculus we have a weak operator topology integral:
        $$f(s|D|)=\int_{\mathbb{R}}\widehat{f}(u)e^{ius|D|}du.$$
        Therefore,
        $$[f(s|D|),x]=\int_{-\infty}^{\infty}\widehat{f}(u)[e^{ius|D|},x]du.$$
        By Lemma \ref{exp commutator lemma}, we have a weak operator topology integral:
        $$[e^{ius|D|},x] = ius\int_0^1e^{ius(1-v)|D|}\delta(x)e^{iusv|D|}dv.$$
        Thus,
        $$[f(s|D|),x]=s\int_{-\infty}^{\infty}iu\widehat{f}(u)\Big(\int_0^1e^{ius(1-v)|D|}\delta(x)e^{iusv|D|}dv\Big)du.$$
        Since $iu\widehat{f}(u)=\widehat{f'}(u),$ the assertion follows.
    \end{proof}
    
    
    \begin{lem}\label{second commutator rep lemma} 
        If $\widehat{f},\widehat{f'},\widehat{f''}\in L_1(\mathbb{R}),$ then for all $x \in \mathcal{B}$ and $s > 0$ we have:
        $$[f(s|D|),x]-sf'(s|D|)\delta(x) = -s^2\int_{-\infty}^{\infty}\left(\int_0^1\widehat{f''}(u)(1-v)e^{ius(1-v)|D|}\delta^2(x)e^{iusv|D|}dv\right)du.$$
        $$[f(s|D|),x]-s\delta(x)f'(s|D|) = -s^2\int_{-\infty}^{\infty}\left(\int_0^1\widehat{f''}(u)(1-v)e^{iusv|D|}\delta^2(x)e^{ius(1-v)|D|}dv\right)du.$$
        Here once again the integrals are understood in the weak operator topology.
    \end{lem}
    \begin{proof} We only prove the first equality as the proof of the second one is similar. 
    
        By the Fourier inversion theorem and functional calculus we have a weak operator topology integral representation:
        $$f'(s|D|)=\int_{\mathbb{R}}\widehat{f'}(u)e^{ius|D|}du.$$
        So multiplying on the left by the bounded operator $\delta(x)$,
        \begin{align*}
            sf'(s|D|)\delta(x)  &= s\int_{\mathbb{R}}\widehat{f'}(u)e^{ius|D|}\delta(x)du\\
                                &= s\int_{-\infty}^{\infty}\left(\int_0^1\widehat{f'}(u)e^{ius|D|}\delta(x)dv\right)du.
        \end{align*}

        Now representing $[f(s|D|),x]$ by the integral representation given by Lemma \ref{first commutator rep lemma}, we infer that
        \begin{align}
            [f(s|D|),x]-sf'(s|D|)\delta(x) &= s\int_{-\infty}^{\infty}\Big(\int_0^1\widehat{f'}(u)\Big(e^{ius(1-v)|D|}\delta(x)e^{iusv|D|}-e^{ius|D|}\delta(x)\Big)dv\Big)du\nonumber\\
                                           &= s\int_{-\infty}^{\infty}\Big(\int_0^1\widehat{f'}(u)\Big(e^{ius(1-v)|D|}[\delta(x),e^{iusv|D|}]\Big)dv\Big)du\label{taylor remainder formula}.
        \end{align}

        Applying Lemma \ref{exp commutator lemma} to $\delta(x) \in \mathcal{B}$, we have:
        \begin{equation}\label{delta(x) commutator formula}
            [\delta(x),e^{iusv|D|}] = -iusv\int_0^1e^{iusv(1-w)|D|}\delta^2(x)e^{iusvw|D|}dw.
        \end{equation}
        Combining \eqref{taylor remainder formula} and \eqref{delta(x) commutator formula}, we obtain
        \begin{equation}\label{taylor expansion triple integral}
            [f(s|D|),x]-sf'(s|D|)\delta(x) =-s^2\int_{-\infty}^{\infty}\Big(\int_0^1\Big(\int_0^1\widehat{f''}(u)ve^{ius(1-vw)|D|}\delta^2(x)e^{iusvw|D|}dw\Big)dv\Big)du.
        \end{equation}
        
        We focus on the inner integral. Performing a linear change of variables, $w_0 = vw$, we get:
        \begin{equation*}
            \int_0^1 ve^{ius(1-vw)|D|}\delta^2(x)e^{iusvw|D|}\,dw = \int_0^v e^{ius(1-w_0)|D|}\delta^2(x)e^{iusw_0|D|}\,dw_0,
        \end{equation*}
        and therefore:
        \begin{equation*}
            \int_0^1 \int_0^1 ve^{ius(1-vw)|D|}\delta^2(x)e^{iusvw|D|}\,dwdv = \int_0^1 \int_{0}^v e^{ius(1-w)|D|}\delta^2(x)e^{iusw|D|}\,dwdv.
        \end{equation*}
        Since the integrand is continuous, we may apply Fubini's theorem to interchange the integrals: 
        \begin{align*}
            \int_0^1 \int_0^1 ve^{ius(1-vw)|D|}\delta^2(x)e^{iusvw|D|}\,dwdv &= \int_0^1 \int_{w}^1 e^{ius(1-w)|D|}\delta^2(x)e^{iusw|D|}\,dvdw\\
                                                                             &= \int_0^1 (1-w)e^{ius(1-w)|D|}\delta^2(x)e^{iusw|D|}\,dw.
        \end{align*}         
        So from \eqref{taylor expansion triple integral}:
        \begin{equation*}
            [f(s|D|),x]-sf'(s|D|)\delta(x) = -s^2\int_{-\infty}^\infty \int_0^1 (1-w)e^{ius(1-w)|D|}\delta^2(x)e^{iusw|D|}\,dwdu.
        \end{equation*}
    \end{proof}

\section{Hochschild coboundary computations}\label{coboundary app}

    In this part of the appendix we include some of the lengthy algebraic computations required for Sections \ref{cohomology section} and \ref{preliminary heat section}.    
    Recall that for a multilinear functional $\theta:\mathcal{A}^{\otimes p}\to\mathbb{C}$ the Hochschild coboundary $b\theta:\mathcal{A}^{\otimes (p+1)}\to\mathbb{C}$ is defined in terms of the Hochschild boundary $b$ by $b\theta(c) = \theta(bc)$.
    
    Explicitly, for an elementary tensor $a_0\otimes\cdots \otimes a_p \in \mathcal{A}^{\otimes (p+1)}$ we have:
    \begin{align*}
        (b\theta)(a_0\otimes\cdots a_p) &= \theta(a_0a_1\otimes a_2\otimes\cdots \otimes a_p)+\sum_{k=1}^{p-1} (-1)^k \theta(a_0\otimes\cdots \otimes a_ka_{k+1}\otimes\cdots\otimes a_p)\\
                                        &\quad + (-1)^p\theta(a_pa_0\otimes a_1\otimes\cdots\otimes a_{p-1}).
    \end{align*}

%     If $\theta:\mathcal{A}^{\otimes p}\to\mathbb{C}$ is a continuous multilinear functional, then the multilinear functional $b\theta:\mathcal{A}^{\otimes (p+1)}\to\mathbb{C}$ can be written as follows:
%     $$(b\theta)(a_0\otimes\cdots\otimes a_p)=\theta(a_0a_1\otimes a_2\otimes\cdots\otimes a_p)+$$
%     $$+\sum_{k=1}^{m-2}(-1)^k\theta(a_0\otimes a_1\otimes\cdots\otimes a_{k-1}\otimes a_{k}a_{k+1}\otimes a_{k+2}\otimes\cdots\otimes a_p)+$$
%     $$+(-1)^{m-1}\theta(a_0\otimes a_1\otimes\cdots\otimes a_{m-2}\otimes a_{m-1}a_m\otimes a_{m+1}\otimes\cdots\otimes a_p)+$$
%     $$+\sum_{k=m}^{p-1}(-1)^k\theta(a_0\otimes a_1\otimes\cdots\otimes a_{k-1}\otimes a_{k}a_{k+1}\otimes a_{k+2}\otimes\cdots\otimes a_p)+$$
%     $$+(-1)^p\theta(a_pa_0\otimes a_1\otimes a_2\otimes\cdots\otimes a_{p-1}).$$

\subsection{Coboundaries in Section \ref{cohomology section}}
    Let $\mathscr{A} \subseteq \{1,\ldots,p\}$. Let $T :=D^{2-|\mathscr{A}|}|D|^{p+1}e^{-s^2D^2}.$
    Following the notation of Definition \ref{W definition}, we define
    for $a \in \mathcal{A}$,
    \begin{equation*}
        b_k(a) := \begin{cases}
                      \delta(a),\quad k \in \mathscr{A},\\
                      [F,a],\quad k\notin \mathscr{A}.
                  \end{cases}
    \end{equation*}
    Fix $1\leq m\leq p-1$. We introduce a pair of multilinear mappings, $\theta_s^1$ and $\theta_s^2$, defined on $a_0\otimes\cdots\otimes a_{p-1} \in \mathcal{A}^{\otimes p}$ by:
    \begin{equation*}
        \theta_s^1(a_0\otimes\cdots\otimes a_{p-1}) := \mathrm{Tr}\left(\Gamma a_0\left(\prod_{k=1}^{m-2}b_k(a_k)\right)\delta^2(a_{m-1})\left(\prod_{k=m}^{p-1}b_{k+1}(a_k)\right)T\right).
    \end{equation*}
    and
    \begin{equation*}
        \theta_s^2(a_0\otimes\cdots\otimes a_{p-1}) := \mathrm{Tr}\left(\Gamma a_0\left(\prod_{k=1}^{m-2}b_k(a_k)\right)[F,\delta(a_{m-1})]\left(\prod_{k=m}^{p-1}b_{k+1}(a_k)\right)T\right).
    \end{equation*}
    
    The mapping that here is denoted $\theta_s^1$ is exactly the multilinear mapping $\theta_s$ introduced in Lemma \ref{kogom2}, and similarly
    the multilinear mapping $\theta_s^2$ is the multilinear mapping $\theta_s$ introduced in Lemma \ref{kogom3}. For $1\leq k < m$ we also introduce $X_k^1$ and $X_k^2$ defined by:
    \begin{align*}
        X_k^1 &:= {\rm Tr}\left(\Gamma a_0\left(\prod_{l=1}^{k-1}b_l(a_l)\right)a_k\left(\prod_{l=k}^{m-2}b_l(a_{l+1})\right)\delta^2(a_m)\left(\prod_{l=m+1}^pb_l(a_l)\right)\cdot T\right),\\
        X_k^2 &:= {\rm Tr}\left(\Gamma a_0\left(\prod_{l=1}^{k-1}b_l(a_l)\right)a_k\left(\prod_{l=k}^{m-2}b_l(a_{l+1})\right) [F,\delta(a_m)]\left(\prod_{l=m+1}^pb_l(a_l)\right)\cdot T\right).
    \end{align*}
    
    Now if $m\leq k \leq p$, we define $Y^1_k$ and $Y^2_k$ by:
    \begin{align*}
          Y_k^1 &:= {\rm Tr}\left(\Gamma a_0\left(\prod_{l=1}^{m-2}b_l(a_l)\right)\delta^2(a_{m-1})\left(\prod_{l=m}^{k-1}b_{l+1}(a_l)\right) a_k\left(\prod_{l=k+1}^pb_l(a_l)\right)\cdot T\right),\\
          Y_k^2 &:= {\rm Tr}\left(\Gamma a_0\left(\prod_{l=1}^{m-2}b_l(a_l)\right)[F,\delta(a_{m-1})]\left(\prod_{l=m}^{k-1}b_{l+1}(a_l)\right)a_k\left(\prod_{l=k+1}^pb_l(a_l)\right)\cdot T\right).
    \end{align*}

% \begin{fact} In both lemmas, we clearly have
% $$\theta_s(a_0a_1\otimes a_2\otimes\cdots\otimes a_p)=X_1.$$
% \end{fact}
    \begin{lem}\label{kogom app 1}
        For $j = 1,2$ and $a_0\otimes\cdots\otimes a_p\in \mathcal{A}^{\otimes (p+1)}$. we have:
        \begin{equation*}
            \theta^j_s(a_0a_1\otimes a_2\otimes\cdots\otimes a_p) = X_1^j
        \end{equation*}
    \end{lem}
    \begin{proof}
        This follows immediately from the definition.
    \end{proof}

% \begin{fact} For $1\leq k<m-1,$ we have (in both lemmas)
% $$\theta_s(a_0\otimes a_1\otimes\cdots\otimes a_{k-1}\otimes a_ka_{k+1}\otimes a_{k+2}\otimes\cdots\otimes a_p)=X_k+X_{k+1}.$$
% \end{fact}
    \begin{lem}\label{kogom app 2}
        For $j = 1,2$, $1 \leq k < m-1$ and $a_0\otimes\cdots\otimes a_p \in \mathcal{A}^{\otimes (p+1)}$ we have
        \begin{equation*}
            \theta_s^j(a_0\otimes \cdots\otimes a_{k}a_{k+1}\otimes\cdots\otimes a_p) = X_k^j+X_{k+1}^j.
        \end{equation*}
    \end{lem}
    \begin{proof} 
        We will only describe the $j=1$ case, since the $j=2$ case is identical. By the definition of $\theta_s^j,$ we have
        \begin{align*}
            \theta_s^1(a_0\otimes &a_1\otimes\cdots\otimes a_{k-1}\otimes a_{k}a_{k+1}\otimes a_{k+2}\otimes\cdots\otimes a_p)\\
                                  &= {\rm Tr}\left(\Gamma a_0\left(\prod_{l=1}^{k-1}b_l(a_l)\right)b_k(a_ka_{k+1})\left(\prod_{l=k+1}^{m-2}b_l(a_{l+1})\right)\delta^2(a_m)\left(\prod_{l=m}^{p-1}b_{l+1}(a_{l+1})\right)\cdot T\right).
        \end{align*}
        Now applying the Leibniz rule to $b_k(a_ka_{k+1})$,
        \begin{align*}
            \theta_s^1(&a_0\otimes a_1\otimes\cdots\otimes a_{k-1}\otimes a_{k}a_{k+1}\otimes a_{k+2}\otimes\cdots\otimes a_p)\\
                                  &= {\rm Tr}\left(\Gamma a_0\left(\prod_{l=1}^{k-1}b_l(a_l)\right) a_k b_k(a_{k+1})\left(\prod_{l=k+1}^{m-2}b_l(a_{l+1})\right)\delta^2(a_m)\left(\prod_{l=m}^{p-1}b_{l+1}(a_{l+1})\right)\cdot T\right)\\
                                  &\quad + {\rm Tr}\left(\Gamma a_0\left(\prod_{l=1}^{k-1}b_l(a_l)\right)b_k(a_k)a_{k+1}\left(\prod_{l=k+1}^{m-2}b_l(a_{l+1})\right)\delta^2(a_m)\left(\prod_{l=m}^{p-1}b_{l+1}(a_{l+1})\right)\cdot T\right)\\
                                  &= {\rm Tr}\left(\Gamma a_0\left(\prod_{l=1}^{k-1}b_l(a_l)\right) a_k\left(\prod_{l=k}^{m-2}b_l(a_{l+1})\right)\delta^2(a_m)\left(\prod_{l=m+1}^p b_l(a_l)\right) T\right)\\
                                  &\quad + {\rm Tr}\left(\Gamma a_0\left(\prod_{l=1}^kb_l(a_l)\right) a_{k+1}\left(\prod_{l=k+1}^{m-2}b_l(a_{l+1})\right)\delta^2(a_m)\left(\prod_{l=m+1}^pb_l(a_l)\right)\cdot T\right)\\
                                  &= X_k^1+X_{k+1}^1
        \end{align*}
        as required.
    \end{proof}

    \begin{lem}\label{kogom app 3}
        For $m\leq k<p,$ $j=1,2$ and $a_0\otimes\cdots\otimes a_p \in \mathcal{A}^{\otimes (p+1)}$ we have
        \begin{equation*}
            \theta_s^j(a_0\otimes a_1\otimes\cdots\otimes a_{k-1}\otimes a_{k}a_{k+1}\otimes a_{k+2}\otimes\cdots\otimes a_p) = Y_k^j+Y_{k+1}^j.
        \end{equation*}
    \end{lem}
    \begin{proof} 
        Again we demonstrate only the $j=1$ case since the $j=2$ case is identical. By definition we have:
        \begin{align*}
            \theta_s^1(&a_0\otimes a_1\otimes\cdots\otimes a_{k-1}\otimes a_{k}a_{k+1}\otimes a_{k+2}\otimes\cdots\otimes a_p)\\
                                &= {\rm Tr}\left(\Gamma a_0\left(\prod_{l=1}^{m-2}b_l(a_l)\right)\delta^2(a_{m-1})\left(\prod_{l=m}^{k-1}b_{l+1}(a_l)\right)b_{k+1}(a_ka_{k+1})\left(\prod_{l=m+1}^{p-1}b_{l+1}(a_{l+1})\right)\cdot T\right).
        \end{align*}
        Applying the Leibniz rule to $b_{k+1}(a_ka_{k+1})$ we have:
        \begin{align*}
        \theta_s^1(&a_0\otimes a_1\otimes\cdots\otimes a_{k-1}\otimes a_{k}a_{k+1}\otimes a_{k+2}\otimes\cdots\otimes a_p)\\
                              &={\rm Tr}\left(\Gamma a_0\left(\prod_{l=1}^{m-2}b_l(a_l)\right)\delta^2(a_{m-1})\left(\prod_{l=m}^{k-1}b_{l+1}(a_l)\right)a_k b_{k+1}(a_{k+1})\left(\prod_{l=k+1}^{p-1}b_{l+1}(a_{l+1})\right)\cdot T\right)\\
                              &\quad+{\rm Tr}\left(\Gamma a_0\left(\prod_{l=1}^{m-2}b_l(a_l)\right)\delta^2(a_{m-1})\left(\prod_{l=m}^{k-1}b_{l+1}(a_l)\right)b_{k+1}(a_k)a_{k+1}\left(\prod_{l=k+1}^{p-1}b_{l+1}(a_{l+1})\right)\cdot T\right)\\
                              &={\rm Tr}\left(\Gamma a_0\left(\prod_{l=1}^{m-2}b_l(a_l)\right)\delta^2(a_{m-1})\left(\prod_{l=m}^{k-1}b_{l+1}(a_l\right) a_k\left(\prod_{l=k+1}^pb_l(a_l)\right)\cdot T\right)\\
                              &\quad+{\rm Tr}\left(\Gamma a_0\left(\prod_{l=1}^{m-2}b_l(a_l)\right)\delta^2(a_{m-1})\left(\prod_{l=m}^kb_{l+1}(a_l)\right) a_{k+1}\left(\prod_{l=k+2}^pb_l(a_l)\right)\cdot T\right)\\
                              &=Y_k^1+Y_{k+1}^1.
        \end{align*}
    \end{proof}
    
    We recall also the multilinear maps $\mathcal{W}_{\mathscr{A}}$ from Definition \ref{W definition}. 

    \begin{lem}\label{kogom app 4}
        Let $c=a_0\otimes a_1\otimes\cdots\otimes a_p \in \mathcal{A}^{\otimes (p+1)}$. If we assume that $m-1,m \in \mathscr{A}$, then we have:
        \begin{align*}
            \theta_s^1(&a_0\otimes a_1\otimes\cdots\otimes a_{m-2}\otimes a_{m-1}a_m\otimes a_{m+1}\otimes\cdots\otimes a_p)\\
                       &=X_{m-1}^1+Y_m^1+2{\rm Tr}(\mathcal{W}_{\mathscr{A}}(c)\cdot T).
        \end{align*}
        Now if $\mathscr{B} \subseteq \{1,\ldots, p\}$ is such that $|\mathscr{B}|=|\mathscr{A}|$ and $\mathscr{A}\Delta \mathscr{B} = \{m-1,m\}$ (where $\Delta$ denotes the symmetric difference) then
        \begin{align*}
            \theta_s^2(&a_0\otimes a_1\otimes\cdots\otimes a_{m-2}\otimes a_{m-1}a_m\otimes a_{m+1}\otimes\cdots\otimes a_p)\\
                       &= X_{m-1}^2+Y_m^2+{\rm Tr}(\mathcal{W}_{\mathscr{A}}(c)\cdot T)+{\rm Tr}(\mathcal{W}_{\mathscr{B}}(c)\cdot T).
        \end{align*}
    \end{lem}
    \begin{proof} 
        Using the repeated Leibniz rule, we obtain
        \begin{align*}
            \delta^2(a_{m-1}a_m)   &= \delta^2(a_{m-1})a_m+2\delta(a_{m-1})\delta(a_m)+a_{m-1}\delta^2(a_m),\\
            [F,\delta(a_{m-1}a_m)] &= [F,\delta(a_{m-1})]a_m+[F,a_{m-1}]\delta(a_m)+\delta(a_{m-1})[F,a_m]+a_{m-1}[F,\delta(a_m)].
        \end{align*}
%         Since all the other computations are identical in both Lemmas, we show the assertion only for the $\theta_s$ in Lemma \ref{kogom2} only.
        Let us focus on proving the assertion relating to $\theta^1_s$, since the other assertion is identical.

        By the definition of $\theta_s^1,$ we have
        \begin{align*}
            \theta_s^1(&a_0\otimes a_1\otimes\cdots\otimes a_{m-2}\otimes a_{m-1}a_m\otimes a_{m+1}\otimes\cdots\otimes a_p)\\
                     &= {\rm Tr}\left(\Gamma a_0\left(\prod_{l=1}^{m-2}b_l(a_l)\right)\delta^2(a_{m-1}a_m)\left(\prod_{l=m}^{p-1}b_{l+1}(a_{l+1})\right)\cdot T\right)\\
                     &= {\rm Tr}\left(\Gamma a_0\left(\prod_{l=1}^{m-2}b_l(a_l)\right)\delta^2(a_{m-1}a_m)\left(\prod_{l=m+1}^pb_l(a_l)\right)\cdot T\right).
        \end{align*}
%         It follows from Leibniz rule that
        Applying the Leibniz rule to the $\delta^2(a_{m-1}a_m)$ term:
        \begin{align*}
            \theta_s^1(&a_0\otimes a_1\otimes\cdots\otimes a_{m-2}\otimes a_{m-1}a_m\otimes a_{m+1}\otimes\cdots\otimes a_p)\\
                     &= {\rm Tr}\left(\Gamma a_0\left(\prod_{l=1}^{m-2}b_l(a_l)\right) a_{m-1}\delta^2(a_m)\left(\prod_{l=m+1}^pb_l(a_l)\right)\cdot T\right)\\
                     &\quad + {\rm Tr}\left(\Gamma a_0\left(\prod_{l=1}^{m-2}b_l(a_l)\right) 2\delta(a_{m-1})\delta(a_m)\left(\prod_{l=m+1}^pb_l(a_l)\right)\cdot T\right)\\
                     &\quad + {\rm Tr}\left(\Gamma a_0\left(\prod_{l=1}^{m-2}b_l(a_l)\right)\delta^2(a_{m-1})a_m\left(\prod_{l=m+1}^pb_l(a_l)\right)\cdot T\right)\\
                     &=X_{m-1}+2{\rm Tr}(\mathcal{W}_{\mathscr{A}}(c)\cdot T)+Y_m.
        \end{align*}
    \end{proof}

    \begin{lem}\label{kogom app 5}
        For $a_0\otimes \cdots\otimes a_p \in \mathcal{A}^{\otimes(p+1)}$, we have
        \begin{align*}
            \theta_s^1(&a_pa_0\otimes a_1\otimes a_2\otimes\cdots\otimes a_{p-1})\\
                     &= {\rm Tr}\left(\Gamma a_0\left(\prod_{l=1}^{m-2}b_l(a_l)\right)\delta^2(a_{m-1})\left(\prod_{l=m}^{p-1}b_{l+1}(a_l)\right)[T,a_p]\right)+Y^1_p.
        \end{align*}
        We also have:
        \begin{align*}
            \theta_s^2(&a_pa_0\otimes a_1\otimes a_2\otimes\cdots\otimes a_{p-1})\\
                       &={\rm Tr}\left(\Gamma a_0\left(\prod_{l=1}^{m-2}b_l(a_l)\right)[F,\delta(a_{m-1})]\left(\prod_{l=m}^{p-1}b_{l+1}(a_l)\right)[T,a_p]\right)+Y_p^2.
        \end{align*}
    \end{lem}
    \begin{proof} 
        We prove only the assertion involving $\theta_s^1$, since one can prove the other result by an identical argument.
%         We show it for the $\theta_s$ in Lemma \ref{kogom2} only. Using definition of $\theta_s$ and the fact that $\Gamma$ commutes with $a_p,$ we obtain
        Since $\Gamma$ commutes with $a_p$, we have:
        \begin{align*}
            \theta_s^1(&a_pa_0\otimes a_1\otimes a_2\otimes\cdots\otimes a_{p-1})\\
                     &= {\rm Tr}\left(\Gamma a_pa_0\left(\prod_{l=1}^{m-2}b_l(a_l)\right)\delta^2(a_{m-1})\left(\prod_{l=m}^{p-1}b_{l+1}(a_l)\right)\cdot T\right)\\
                     &= {\rm Tr}\left(a_p\Gamma a_0\left(\prod_{l=1}^{m-2}b_l(a_l)\right)\delta^2(a_{m-1})\left(\prod_{l=m}^{p-1}b_{l+1}(a_l)\right)\cdot T\right).
        \end{align*}
        Using the cyclicity of the trace, we have
        \begin{align*}
            \theta_s^1(&a_pa_0\otimes a_1\otimes a_2\otimes\cdots\otimes a_{p-1})\\
                     &={\rm Tr}\left(\Gamma a_0\left(\prod_{l=1}^{m-2}b_l(a_l)\right)\delta^2(a_{m-1})\left(\prod_{l=m}^{p-1}b_{l+1}(a_l)\right)\cdot Ta_p\right)\\
                     &={\rm Tr}\left(\Gamma a_0\left(\prod_{l=1}^{m-2}b_l(a_l)\right)\delta^2(a_{m-1})\left(\prod_{l=m}^{p-1}b_{l+1}(a_l)\right)\cdot [T,a_p]\right)+Y_p^1.
        \end{align*}
    \end{proof}

    Note that for $j = 1,2$, the telescopic sum
    \begin{align*}
        X_1^j+&\sum_{k=1}^{m-1}(-1)^k(X_k^j+X_{k+1}^j)+(-1)^{m-1}(X_{m-1}^j+Y_m^j)\\
            &+\sum_{k=m}^{p-1}(-1)^k(Y_k^j+Y_{k+1}^j)+(-1)^pY_p^j.
    \end{align*}
    vanishes.

    Therefore, combining Lemmas \ref{kogom app 1}, \ref{kogom app 2}, \ref{kogom app 3}, \ref{kogom app 4} and \ref{kogom app 5} we have:
    \begin{align*}
        (b\theta_s^1)(a_0\otimes\cdots\otimes a_p) &= 2\cdot(-1)^{m-1}\cdot{\rm Tr}(\mathcal{W}_{\mathscr{A}}(c)\cdot T)\\
                                                   &\quad + (-1)^p\cdot {\rm Tr}\left(\Gamma a_0\left(\prod_{l=1}^{m-2}b_l(a_l)\right)\delta^2(a_{m-1})\left(\prod_{l=m}^{p-1}b_{l+1}(a_l)\right)\cdot[T,a_p]\right).
    \end{align*}
    Similarly, if $\mathscr{B}$ is such that $|\mathscr{A}| = |\mathscr{B}|$ and $\mathscr{A}\Delta \mathscr{B} = \{m-1,m\}$ then:
    \begin{align*}
        (b\theta_s^2)(a_0\otimes\cdots\otimes a_p) &= (-1)^{m-1}\cdot{\rm Tr}(\mathcal{W}_{\mathscr{A}}(c)\cdot T)+(-1)^{m-1}\cdot{\rm Tr}(\mathcal{W}_{\mathscr{B}}(c)\cdot T)\\
                                                   &\quad +(-1)^p\cdot {\rm Tr}\left(\Gamma a_0\left(\prod_{l=1}^{m-2}b_l(a_l)\right)[F,\delta(a_{m-1})]\left(\prod_{l=m}^{p-1}b_{l+1}(a_l)\right)\cdot [T,a_p]\right).
    \end{align*}
    This completes the computation of the coboundaries for $\theta^1_s$ and $\theta^2_s$.

\subsection{Coboundaries in Section \ref{preliminary heat section}}

    Again in this subsection, $(\mathcal{A},H,D)$ is a smooth spectral triple where $D$ has a spectral gap at $0$. Let $T = Fe^{-s^2D^2}$. Note that this is different
    to $T$ in the preceding section.

    We define the multilinear mapping $\mathcal{L}_s:\mathcal{A}^{\otimes p}\to\mathbb{C}$ on $a_0\otimes \cdots\otimes a_{p-1} \in \mathcal{A}^{\otimes p}$ by:
    \begin{equation*}
        \mathcal{L}_s(a_0\otimes\cdots\otimes a_{p-1}) := {\rm Tr}\left(\Gamma a_0\left(\prod_{k=1}^{p-1}[F,a_k]\right)\cdot T\right).
    \end{equation*}
    We also define the multilinear mapping $\mathcal{K}_s:\mathcal{A}^{\otimes (p+1)}\to\mathbb{C}$ by:
    \begin{equation*}
        \mathcal{K}_s(a_0\otimes\cdots\otimes a_p) := {\rm Tr}\left(\Gamma a_0\left(\prod_{k=1}^{p-1}[F,a_k]\right)\cdot [T,a_p]\right).
    \end{equation*}
    By definition, $\mathcal{K}_s$ is exactly the mapping from Theorem \ref{first cycle thm}.


    For $1\leq m\leq p$, we define $X_m$ by:
    \begin{equation*}
        X_m := {\rm Tr}\left(\Gamma a_0\left(\prod_{k=1}^{m-1}[F,a_k]\right)a_m\left(\prod_{k=m+1}^p[F,a_k]\right) T\right).
    \end{equation*}
    We have
    \begin{align*}
        \mathcal{L}_s(a_0a_1\otimes a_2\otimes\cdots\otimes a_p) &= {\rm Tr}(\Gamma a_0a_1\prod_{k=2}^p[F,a_k]\cdot T)\\
                                                         &=X_1.
    \end{align*}
    
    Applying the Leibniz rule to $[F,a_{m-1}a_m]$:
    \begin{align*}
        \mathcal{L}_s(&a_0\otimes a_1\otimes\cdots\otimes a_{m-2}\otimes a_{m-1}a_m\otimes a_{m+1}\otimes\cdots\otimes a_p)\\
              &= {\rm Tr}\left(\Gamma a_0\left(\prod_{k=1}^{m-2}[F,a_k]\right)[F,a_{m-1}a_m]\left(\prod_{k=m+1}^p[F,a_k]\right)\cdot T\right)\\
              &= {\rm Tr}\left(\Gamma a_0\left(\prod_{k=1}^{m-2}[F,a_k]\right) a_{m-1}\left(\prod_{k=m}^p[F,a_k]\right)\cdot T\right)\\
              &\quad +{\rm Tr}\left(\Gamma a_0\left(\prod_{k=1}^{m-1}[F,a_k]\right)a_m\left(\prod_{k=m+1}^p[F,a_k]\right)\cdot T\right)\\
              &= X_m+X_{m+1}.
    \end{align*}
    Finally,
    \begin{align*}
        \mathcal{L}_s(a_pa_0\otimes a_1\otimes\cdots\otimes a_{p-1}) &= {\rm Tr}(\Gamma a_pa_0\prod_{k=1}^{p-1}[F,a_k]\cdot T)\\
                                                             &= {\rm Tr}(\Gamma a_0\prod_{k=1}^{p-1}[F,a_k]\cdot Ta_p)\\
                                                             &= {\rm Tr}(\Gamma a_0\prod_{k=1}^{p-1}[F,a_k]\cdot [T,a_p])+{\rm Tr}(\Gamma a_0\prod_{k=1}^{p-1}[F,a_k]\cdot a_pT)\\
                                                             &= \mathcal{K}_s(a_0\otimes\cdots\otimes a_p)+X_p.
    \end{align*}

    Thus,
    \begin{align*}
        (b\mathcal{L}_s)(a_0\otimes\cdots\otimes a_p) &= \mathcal{K}_s(a_0\otimes\cdots\otimes a_p)\\
                                              &\quad+ X_1 +\Big(\sum_{m=1}^{p-1}(-1)^m(X_m+X_{m+1})\Big)+(-1)^pX_p.
    \end{align*}
    The latter sum telescopes and indeed vanishes, so it follows that $b\mathcal{L}_s=\mathcal{K}_s.$

\subsection{Coboundaries in Section \ref{heat section}}
    In this section, $(\mathcal{A},H,D)$ is a smooth spectral triple where $D$ has a spectral gap at $0$, and $T := |D|^{2-p}e^{-s^2D^2}$.
    We define the multilinear mapping $\theta_s:\mathcal{A}^{\otimes p}\to\mathbb{C}$ on $a_0\otimes\cdots a_{p-1} \in \mathcal{A}^{\otimes p}$ by:
    \begin{equation*}
        \theta_s(a_0\otimes\cdots\otimes a_{p-1}) = {\rm Tr}\left(\left(\prod_{k=0}^{p-1}\partial(a_k)\right) T\right).
    \end{equation*}

    For $0 \leq k \leq p$ we also define:
    \begin{equation*}
        X_k = {\rm Tr}\left(\left(\prod_{l=0}^{k-1}\partial(a_l)\right)a_k\left(\prod_{l=k+1}^p\partial(a_l)\right)T\right).
    \end{equation*}
    So in particular,
    \begin{equation*}
        X_0 = \mathrm{Tr}\left(a_0\left(\prod_{l=1}^{p}\partial(a_l)\right)T\right).
    \end{equation*}
    
    

    Applying the Leibniz rule to $\partial(a_ka_{a+1})$ we get:
    \begin{align*}
        \theta_s(&a_0\otimes\cdots\otimes a_{k-1}\otimes a_ka_{k+1}\otimes a_{k+2}\otimes\cdots\otimes a_p)\\
                 &= {\rm Tr}\left(\left(\prod_{l=0}^{k-1}\partial(a_l)\right)\partial(a_ka_{k+1})\left(\prod_{l=k+2}^p\partial(a_l)\right)\cdot T\right)\\
                 &= {\rm Tr}\left(\left(\prod_{l=0}^{k-1}\partial(a_l)\right) a_k\left(\prod_{l=k+1}^p\partial(a_l)\right)\cdot T\right)\\
                 &\quad +{\rm Tr}\left(\left(\prod_{l=0}^k\partial(a_l)\right) a_{k+1}\left(\prod_{l=k+2}^p\partial(a_l)\right)\cdot T\right)\\
                 &=X_k+X_{k+1}.
    \end{align*}

    We also have
    \begin{align*}
        \theta_s(&a_pa_0\otimes a_1\otimes\cdots\otimes a_{p-1})\\
                 &= {\rm Tr}\left(\left(\prod_{k=0}^{p-1}\partial(a_k)\right)\cdot Ta_p\right)\\
                 &\quad + {\rm Tr}\left(a_0\left(\prod_{l=1}^{p-1}\partial(a_k)\right)\cdot T\partial(a_p)\right)\\
                 &= X_p + {\rm Tr}\left(\left(\prod_{k=0}^{p-1}\partial(a_k)\right)\cdot [T,a_p]\right)\\
                 &\quad + X_0+{\rm Tr}\left(a_0\left(\prod_{k=1}^{p-1}\partial(a_k)\right)\cdot [T,\partial(a_p)]\right).
    \end{align*}

    If we assume that $p$ is even, then:
    \begin{align*}
        (b\theta_s)(&a_0\otimes\cdots\otimes a_p)\\
                    &= \left(\sum_{k=0}^{p-1}(-1)^k(X_k+X_{k+1})\right)+(X_p+X_0)\\
                    &\quad +{\rm Tr}\left(\left(\prod_{k=0}^{p-1}\partial(a_k)\right)\cdot [T,a_p]\right)+{\rm Tr}\left(a_0\left(\prod_{k=1}^{p-1}\partial(a_k)\right)\cdot [T,\partial(a_p)]\right)\\
                    &= 2X_0+{\rm Tr}\left(a_0\left(\prod_{k=1}^{p-1}\partial(a_k)\right) [T,\partial a_p]\right) +{\rm Tr}\left(\left(\prod_{k=0}^{p-1}\partial(a_k)\right)[T,a_p]\right)\\
                    &= 2{\rm Tr}\left(a_0\left(\prod_{k=1}^p\partial(a_k)\right)\cdot T\right)+{\rm Tr}\left(a_0\left(\prod_{k=1}^{p-1}\partial(a_k)\right) [T,\partial a_p]\right)\\
                    &\quad+{\rm Tr}\left(\left(\prod_{k=0}^{p-1}\partial(a_k)\right)[T,a_p]\right).
    \end{align*}
    This completes the computation of the coboundaries.

\section{Technical estimates for Section 4.4.1}\label{app 3 section}

    For this section, $(\mathcal{A},H,D)$ is assumed to be a spectral triple satisfying Hypothesis \ref{main assumption} and we
    assume that $D$ has a spectral gap at $0$.

    The results of this section are very similar to that of Lemma \ref{w bounded lemma}. However additional technicalities make the proofs more involved and therefore are included here in the appendix.
    We make use of the mappings $b_k$ from Definition \ref{W definition}, defined in terms of $m\geq 1$ and a subset $\mathscr{B} \subseteq \{1,\ldots,m\}$ on $a \in \mathcal{A}$
    \begin{equation*}
        b_k(a) := \begin{cases}
                      \delta(a),\quad k \in \mathscr{B},\\
                      [F,a],\quad k\notin \mathscr{B}.
                  \end{cases}
    \end{equation*}
        

    \begin{lem}\label{c induction lemma}    
        Let $m\geq 1$ and Let $n\in \mathbb{Z}$. Then for all $\mathscr{B} \subseteq \{1,\ldots,m\}$ the operator on $H_\infty$ given by:
        \begin{equation*}
            |D|^{-n}\left(\prod_{k=1}^m b_k(a_k)\right)|D|^{n+m-|\mathscr{B}|}
        \end{equation*}
        has bounded extension, where $b_k$ is defined in terms of $\mathscr{B}$.
    \end{lem}
    \begin{proof} 
        We prove the assertion by induction on $m.$ If $m=1$ then there are two possible cases, $\mathscr{B} = \emptyset$ and $\mathscr{B} = \{1\}$. 
        
        If $\mathscr{B} = \emptyset$, then on $H_\infty$ we have:
        \begin{align*}
            |D|^{-n}\left(\prod_{k=1}^m b_k(a_k)\right)|D|^{n+m-|\mathscr{B}|} &= |D|^{-n}[F,a_1]|D|^{n+1}\\
                                                                    &= |D|^{-n}L(a_1)|D|^n\\
                                                                    &= |D|^{-n}\partial(a_1)|D|^n-F|D|^{-n}\delta(a_1)|D|^n.
        \end{align*}
        By Lemma \ref{left to right corollary}, the operators $|D|^{-n}\partial(a_1)|D|^n$ and $|D|^{-n}\delta(a_1)|D|^n$ have bounded extension, so this proves
        the $\mathscr{B} = \emptyset$ case. On the other hand, if $\mathscr{B} = \{1\},$ then on $H_\infty$:
        \begin{equation*}
            |D|^{-n}\left(\prod_{k=1}^m b_k(a_k) \right)|D|^{n+m-|\mathscr{B}|} = |D|^{-n}\delta(a_1)|D|^n.
        \end{equation*}
        Again by Lemma \ref{left to right corollary}, the operator $|D|^{-n}\delta(a_1)|D|^n$ has bounded extension. This completes the proof of the $\mathcal{B} = \{1\}$ case.

        Now assume that $m > 1$ and the assertion is true for $m-1$. Define $\mathscr{C} := \mathscr{B}\setminus \{m\}.$ and let $n_1 = n+(m-1)-|\mathscr{C}|$. Then on $H_\infty$:
        \begin{align} 
            |D|^{-n}&\left(\prod_{k=1}^mb_k(a_k)\right)|D|^{n+m-|\mathscr{B}|}\nonumber\\
                                                &=\left(|D|^{-n}\left(\prod_{k=1}^{m-1}b_k(a_k)\right)|D|^{n+(m-1)-|\mathscr{C}|}\right)\left(|D|^{-n_1}b_m(a_m)|D|^{n_1+1-|\mathscr{B}\cap\{m\}|}\right).\label{inductive expanded product}
        \end{align} 
        By the inductive assumption, the first factor has bounded extension.

        If $m\in\mathscr{B},$ then the second factor in \eqref{inductive expanded product} is:
        \begin{equation*}
            |D|^{-n_1}\delta(a_m)|D|^{n_1}
        \end{equation*}
        which has bounded extension by Lemma \ref{left to right corollary}.\eqref{left to right 1}. On the other hand, if $m\notin\mathscr{B},$ then the second factor in \eqref{inductive expanded product} is
        \begin{equation*}
            |D|^{-n_1}F(a_m)|D|^{n_1+1} = |D|^{-n_1}(\partial(a_m)-F\delta(a_m))|D|^{n_1}
        \end{equation*}
        which also has bounded extension by Lemma \ref{left to right corollary}.\eqref{left to right 1}. In either case, the second factor of \eqref{inductive expanded product} has bounded extension. So the assertion is proved
        for $m$, completing the proof by induction.
    \end{proof}


    \begin{lem}\label{c first alert lemma} 
        Let $\mathscr{A}\subseteq\{1,\cdots,p\}$ Assume that there is $m > 1$ be such that $m-1,m\in\mathscr{A}.$ 
        The operator on $H_\infty$ given by:
        \begin{equation*}
            \Gamma a_0\left(\prod_{k=1}^{m-2}b_k(a_k)\right)\delta^2(a_{m-1})\left(\prod_{k=m}^{p-1}b_{k+1}(a_k)\right)|D|^{p-|\mathscr{A}|}
        \end{equation*}
        has bounded extension.
    \end{lem}
    \begin{proof} 
        Let $n_1=|\{1,\cdots,m-2\}\setminus\mathscr{A}|$ and $n_2=|\{m+1,\cdots,p\}\setminus\mathscr{A}|$ so that immediately $n_1 \leq m-2-|\mathscr{A}|$ and $n_2 \leq p-m-|\mathscr{A}|$. On $H_\infty$ we have
        $$\Gamma a_0\left(\prod_{k=1}^{m-2}b_k(a_k)\right)\delta^2(a_{m-1})\left(\prod_{k=m}^{p-1}b_{k+1}(a_k)\right)|D|^{p-|\mathscr{A}|}= \mathrm{I}\cdot \mathrm{II}\cdot \mathrm{III}.$$
        Here,
        \begin{align*}
            \mathrm{I}   &= \Gamma a_0\left(\prod_{k=1}^{m-2}b_k(a_k)\right)|D|^{n_1},\\
            \mathrm{II}  &= |D|^{-n_1}\delta^2(a_{m-1})|D|^{n_1},\\
            \mathrm{III} &= |D|^{-n_1}\left(\prod_{k=m}^{p-1}b_{k+1}(a_k)\right)|D|^{n_1+n_2}.
        \end{align*}
        The operators $\mathrm{I}$ and $\mathrm{III}$ have bounded extension by Lemma \ref{c induction lemma}. On the other hand, $\mathrm{II}$ has bounded extension due to Lemma \ref{left to right corollary}.
    \end{proof}

    \begin{lem}\label{c second alert lemma} 
        Let $\mathscr{A}\subset\{1,\cdots,p\}$ and assume that there is $m > 1$ be such that $m-1\in\mathscr{A}$ and $m\notin\mathscr{A}.$ The operator on $H_\infty$ given by
        \begin{align*}
            \Gamma a_0\left(\prod_{k=1}^{m-2}b_k(a_k)\right)[F,\delta(a_{m-1})]\left(\prod_{k=m}^{p-1}b_{k+1}(a_k)\right)|D|^{p-|\mathscr{A}|}
        \end{align*}
        has bounded extension.
    \end{lem}
    \begin{proof} 
        Let $n_1=|\{1,\cdots,m-2\}\setminus\mathscr{A}|$ and $n_2=|\{m+1,\cdots,p\}\setminus\mathscr{A}|,$ so that we immediately have $n_1 \leq m-2-|\mathscr{A}|$ and $n_2\leq p-m-|\mathscr{A}|$ as in Lemma \ref{c first alert lemma}. We have
        $$\Gamma a_0\left(\prod_{k=1}^{m-2}b_k(a_k)\right)[F,\delta(a_{m-1})]\left(\prod_{k=m}^{p-1}b_{k+1}(a_k)\right)|D|^{p-|\mathscr{A}|}=\Gamma a_0\cdot \mathrm{I}\cdot \mathrm{II}\cdot \mathrm{III}.$$
        Here,
        \begin{align*}
              \mathrm{I} &= \left(\prod_{k=1}^{m-2}b_k(a_k)\right)|D|^{n_1},\\
             \mathrm{II} &= |D|^{-n_1}[F,\delta(a_{m-1})]|D|^{1+n_1},\\
            \mathrm{III} &= |D|^{-n_1-1}\left(\prod_{k=m}^{p-1}b_{k+1}(a_k)\right)|D|^{1+n_1+n_2}.
        \end{align*}
        The operators $\mathrm{I}$ and $\mathrm{III}$ have bounded extension by Lemma \ref{c induction lemma}. On the other hand, 
        \begin{equation*}
            \mathrm{II} = |D|^{-n_1}(\partial(\delta(a_{m-1}))-F\delta^2(a_{m-1}))|D|^{n_1}
        \end{equation*}
        which has bounded extension by Lemma \ref{left to right corollary}.\eqref{left to right 1}.
    \end{proof}

    \begin{lem}\label{c third alert lemma} 
        For every $m\geq1,$ the operator on $H_\infty$ given by
        \begin{equation*}
            \Big(\prod_{k=0}^m[F,a_k]\Big)|D|^{m+1}
        \end{equation*}
        has bounded extension.
    \end{lem}
    \begin{proof} 
        We prove the assertion by induction on $m.$ For $m=0,$ on $H_\infty$ we have
        $$[F,a_0]|D|=\partial(a_0)-F\delta(a_0)$$
        which has has bounded extension.

        Now let $m>1$ and assume that the assertion is true for $m-1$. On $H_\infty$ we write
        $$\Big(\prod_{k=0}^m[F,a_k]\Big)|D|^{m+1} = \Big(\Big(\prod_{k=0}^{m-1}[F,a_k]\Big)|D|^m\Big)\cdot\Big(|D|^{-m}[F,a_m]|D|^{m+1}\Big).$$
        The first factor has bounded extension by the induction assumption. The second factor is
        $$|D|^{-m}L(a_m)|D|^m=|D|^{-m}\partial(a_m)|D|^m-F\cdot|D|^{-m}\delta(a_m)|D|^m$$
        which has bounded extension by Lemma \ref{left to right corollary}.\eqref{left to right 1}. Hence, the assertion holds for $m$, completing the inductive proof.
    \end{proof}

\section{Subkhankulov's computation}\label{subhankulov app}

    The following assertion is identical to \cite[Lemma 2.1.1]{subhankulov}. However to the best of our knowledge there is no published proof in English,
    and \cite{subhankulov} is not easily accessible. For the convenience of the reader we include a proof here.
    
    \begin{prop}\label{subhankulov compute} 
        For all $u\in (0,1)$ and $v\in\mathbb{R},$ we have
        $$\frac1{2\pi}\int_{-1}^1\frac{(1-t^2)^2}{u+it}e^{(u+it)v}dt=(1+u^2)^2\chi_{[0,\infty]}(v)+e^{uv}\cdot\min\{1,v^{-2}\}\cdot O(1).$$
    \end{prop}
    \begin{proof}
        We will deal separately with the $v \geq 0$ and $v < 0$ cases. First, assume that $v \geq 0$.
    
        Let $\gamma_0$ be a smooth curve in $\mathbb{C}$ without self-intersections such that
        \begin{enumerate}
            \item $\gamma_0$ starts at $i$ and ends at $-i.$
            \item $\gamma_0$ lies in the half-plane $\{\Re(z)\leq0\},$
            \item The distance between $\gamma_0$ and the interval $[-1,0]$ is greater than or equal to $1$,
            \item the length of $\gamma_0$ is at most $10.$
            \item $\gamma_0$ is contained in the disc $\{z\;:\;|z| \leq 10\}$.
        \end{enumerate}
        Let $\gamma_1$ be the interval starting at $-i$ and ending at $i$ and let the contour $\gamma$ be the concatenation of $\gamma_0$ and $\gamma_1.$
        
        Define 
        \begin{equation*}
            f(z) := \frac{(1+z^2)^2}{u+z},\quad z\in\mathbb{C}\setminus \{-u\}.
        \end{equation*}
        So that by definition:
        \begin{equation}\label{wick rotation}
            \frac{1}{2\pi} \int_{-1}^1 \frac{(1-t^2)^2}{u+it}e^{(u+it)v}\,dt = \frac{1}{2\pi i} e^{uv}\int_{\gamma_1} f(z)e^{zv}\,dz.
        \end{equation}
        Since $z\mapsto e^{zv}$ is entire, the function $z\to f(z)e^{zv}$ is holomorphic in the set $\mathbb{C}\setminus \{-u\}$ and has a simple pole at at $z=-u$
        with corresponding residue $(1+u^2)^2e^{-uv}.$ By construction,
        the point $-u$ is in the interior of the curve $\gamma$ and so         
        by the Cauchy integral formula we have:
        \begin{equation*}
            \frac{1}{2\pi i}\int_{\gamma}f(z)e^{zv}dz = (1+u^2)^2e^{-uv}.
        \end{equation*}
        Since $\gamma = \gamma_0\cup\gamma_1$:
        \begin{equation}\label{v>0 eq1}
            \frac1{2\pi i}\int_{\gamma_0}f(z)e^{zv}dz+\frac1{2\pi i}\int_{\gamma_1}f(z)e^{zv}dz = (1+u^2)^2e^{-uv}.
        \end{equation}
        
        By definition $\gamma_0$ has length at most $10$, so by the triangle inequality we have the bound:
        \begin{equation*}
            \left|\int_{\gamma_0}f(z)e^{zv}dz\right| \leq 10\sup_{z\in\gamma_0} |f(z)||e^{zv}|.
        \end{equation*}
        
        Since $\gamma_0$ is contained in the half-plane $\{z\;:\;\Re(z) \leq 0\}$ we also have that $\sup_{z \in \gamma_0} |e^{zv}| \leq 1$
        and therefore:
        \begin{equation*}
            \left|\int_{\gamma_0}f(z)e^{zv}dz\right| \leq 10\sup_{z\in\gamma_0} |f(z)|.
        \end{equation*}
        
        For $z \in \gamma$ we have that $|z|\leq 10$ and $|u+z|\geq1$ so it follows that
        $$\sup_{z\in\gamma_0}|f(z)|\leq (1+10^2)^2\leq 10^5.$$
        Therefore, we have
        \begin{equation}\label{v>0 eq2}
            \int_{\gamma_0} f(z)e^{zv}dz =O(1).
        \end{equation}

        Using integration by parts twice and taking into account that $f(\pm i)=f'(\pm i)=0,$ we obtain
        $$\int_{\gamma_0}f(z)e^{zv}dz=v^{-2}\int_{\gamma_0}f''(z)e^{vz}dz.$$
        Thus,
        $$v^2\left|\int_{\gamma_0}f(z)e^{zv}dz\right| \leq 10\sup_{z\in\gamma_0}|f''(z)|.$$
        We may compute $f''(z)$ directly as:
        $$f''(z) = (4+12z)\frac1{u+z}-8z(1+z^2)\frac1{(u+z)^2}+(1+z^2)^2\frac2{(u+z)^3}.$$
        Since $|z|\leq 10$ and $|u+z|\geq1$ for every $z\in\gamma_0,$ it follows that
        \begin{align*}
            \sup_{z\in\gamma_0}|f''(z)| &\leq (4+12\cdot 10)+8\cdot 10\cdot (1+10^2)+2\cdot(1+10^2)^2\\
                                        &\leq 10^5.
        \end{align*}
        Therefore, we have:
        \begin{equation}\label{v>0 eq3}
            \int_{\gamma_0}f(z)e^{zv}dz=O(v^{-2}),
        \end{equation}
        Hence,
        \begin{equation}\label{v>0 eq4}
            \int_{\gamma_0}f(z)e^{zv}dz = \min\{1,v^{-2}\}\cdot O(1).
        \end{equation}
        
        Combining \eqref{v>0 eq1} and \eqref{v>0 eq4}, we obtain
        $$\frac1{2\pi i}\int_{\gamma_1}f(z)e^{zv}dz=(1+u^2)^2e^{-uv}+\min\{1,v^{-2}\}\cdot O(1).$$
                
        So using \eqref{wick rotation} and \eqref{v>0 eq1}:
        $$\frac1{2\pi}\int_{-1}^1\frac{(1-t^2)^2}{u+it}e^{(u+it)v}dt=(1+u^2)^2+e^{uv}\cdot\min\{1,v^{-2}\}\cdot O(1).$$
        This completes the proof of the $v \geq 0$.
        
        Now assume that $v < 0$.
        This proof is similar, but instead we consider a contour in the half plane $\{z\;:\;\Re(z)\geq 0\}$. 
        Let $\gamma_2$ be a smooth curve without self-intersections such that
        \begin{enumerate}
            \item $\gamma_2$ starts at $-i$ and ends at $i.$
            \item $\gamma_2$ lies in the half-plane $\{\Re(z)\geq0\}.$
            \item the distance between $\gamma_2$ and $[-1,0]$ is greater than or equal to $1$.
            \item the length of $\gamma_2$ is at most $10.$
            \item $\gamma_2$ is contained in the disc $\{z\;:\;|z| \leq 10\}$.
        \end{enumerate}
        As in the $v \geq 0$ case, $\gamma_1$ denotes the interval joining $-i$ and $i$, and now write $\gamma'$ for the concatenation of $\gamma_1$ and $\gamma_2$.

        Since $f$ is holomorphic in the half-plane $\Re(z) \geq 0$, we have:
        $$\int_{\gamma'}f(z)e^{zv}dz=0.$$
        Since $\gamma' = \gamma_1\cup\gamma_2$ we have:
        \begin{equation}\label{v<0 eq1}
            \frac1{2\pi i}\int_{\gamma_1}f(z)e^{zv}dz+\frac1{2\pi i}\int_{\gamma_2}f(z)e^{zv}dz=0.
        \end{equation}
        
        Since by definition $\gamma_2$ has length at most $10$, and we are assuming $v < 0$ we have:,
        \begin{align*}
            \left|\int_{\gamma_2}f(z)e^{zv}dz\right| &\leq 10\sup_{z\in\gamma_2}|f(z)e^{zv}|\\
                                                     &\leq 10\sup_{z \in \gamma_2}|f(z)|\\
                                                     &\leq 10(1+10^2)^2.\\
                                                     &=O(1).
        \end{align*}

        Using integration by parts and taking into account once again that $f(\pm i)=f'(\pm i)=0,$ we obtain in also in the $v < 0$ case that:
        $$\int_{\gamma_2}f(z)e^{zv}dz=v^{-2}\int_{\gamma_2}f''(z)e^{vz}dz.$$
        Thus,
        \begin{align*}
            v^2\left|\int_{\gamma_2}f(z)e^{zv}dz\right| &\leq 10\sup_{z\in\gamma_2}|f''(z)||e^{zv}|\\
                                                        &\leq 10^6\\
                                                        &= O(1).
        \end{align*}
        Therefore,
        \begin{equation}\label{v<0 eq2}
            \int_{\gamma_2}f(z)e^{zv}dz=\min\{1,v^{-2}\}\cdot O(1).
        \end{equation}

        Combining \eqref{v<0 eq1} and \eqref{v<0 eq2}, we obtain
        \begin{equation*}
            \frac1{2\pi i}\int_{\gamma_1}f(z)e^{zv}dz=\min\{1,v^{-2}\}\cdot O(1).
        \end{equation*}
        Hence, by \eqref{wick rotation} we conclude the proof for the $v < 0$ case.
    \end{proof}

