\chapter{Preliminaries}\label{preliminaries chapter}


\section{Operators, ideals and traces}

\subsection{General notation}\label{general notations subsection}
    Fix throughout a separable, infinite dimensional complex Hilbert space $H$. We denote by $\mathcal{L}_{\infty}$ the algebra of all bounded operators on $H$, with operator norm denoted $\|\cdot\|_\infty.$ For a compact operator $T$ on $H$,
    let $\lambda(T) := \{\lambda(k,T)\}_{k=0}^\infty$ denote the sequence of eigenvalues of $T$ arranged in order of non-increasing magnitude and with multiplicities. Similarly, let $\mu(T) := \{\mu(k,T)\}_{k=0}^\infty$ denote the sequence of singular
    values of $T$, also arranged in non-increasing order with multiplicities. The $k$th singular value may be described equivalently as either $\mu(k,T) := \lambda(k,|T|)$ or 
    \begin{equation*}
        \mu(k,T) = \inf\{\|T-R\|_\infty\;:\;\mathrm{rank}(R)\leq k\}.
    \end{equation*}
    
    The standard trace on $\mathcal{L}_\infty$ (more precisely on the trace-class ideal) is denoted $\mathrm{Tr}$.


    Fix an orthonormal basis $\{e_k\}_{k=0}^\infty$ on $H$ (the particular choice of basis is inessential). We identify the algebra $\ell_\infty$ of all bounded sequences with the subalgebra of diagonal operators on $H$ with respect
    to the chosen basis. For a given $\alpha \in \ell_\infty$, we denote the corresponding diagonal operator by $\mathrm{diag}(\alpha)$.

    For $A,B \in \mathcal{L}_{\infty}$, we say that $B$ is submajorized by $A$ in the sense of Hardy-Littlewood, written as $B \prec\prec A$, if
    \begin{equation*}
        \sum_{k=0}^n \mu(k,B) \leq \sum_{k=0}^n \mu(k,A), \quad n\geq 0.
    \end{equation*}
    
    We say that $B$ is logarithmically submajorized by $A$, written as $B \prec\prec_{\log} A$ if 
    \begin{equation*}
        \prod_{k=0}^n \mu(k,B)\leq \prod_{k=0}^n \mu(k,A),\quad n\geq 0.
    \end{equation*}
    
    An important result concerning logarithmic submajorisation is the Araki-Lieb-Thirring inequality \cite[Theorem 2]{Kosaki-alt-1992}, which states that
    for all positive bounded operators $A$ and $B$ and all $r \geq 1$,
    \begin{equation}\label{ALT inequality}
        |AB|^r \prec\prec_{\log} A^rB^r.
    \end{equation}
    
    We make frequent use of the following commutator identity: if $A$ and $B$ are operators with $B$ invertible, then
    \begin{equation}\label{favourite commutator identity}
        [B^{-1},A] = -B^{-1}[B,A]B^{-1}.
    \end{equation}
    We must take care to ensure that \eqref{favourite commutator identity} remains valid when $A$ and $B$ are potentially unbounded. If $A$ is bounded, then it is enough
    that $A:\mathrm{dom}(B)\to \mathrm{dom}(B)$.

\subsection{Ideals in $\mathcal{L}_{\infty}$ and related inequalities}
    For $p \in (0,\infty)$, we let $\mathcal{L}_{p}$ denote the Schatten-von Neumann ideal of $\mathcal{L}_{\infty}$,
    \begin{equation*}
        \mathcal{L}_p := \{T \in \mathcal{L}_{\infty}\;:\; \mu(T) \in \ell_p\}
    \end{equation*}
    where $\ell_p$ is the space of $p$-summable sequences. As usual, for $p \geq 1$ the ideal $\mathcal{L}_{p}$ is equipped
    with the norm
    \begin{equation*}
        \|T\|_{p} := \left(\sum_{k=0}^\infty \mu(k,T)^p\right)^{1/p}.
    \end{equation*}
    
    Similarly, given $0 < p < \infty$, we denote by $\mathcal{L}_{p,\infty}$ the ideal in $\mathcal{L}_{\infty}$ defined by
    \begin{equation*}
        \mathcal{L}_{p,\infty} := \{T \in \mathcal{L}_{\infty}\;:\; \sup_{k\geq 0}\, (1+k)^{1/p}\mu(k,T) < \infty\}.
    \end{equation*}
    Equivalently, 
    \begin{equation*}
        \mathcal{L}_{p,\infty} := \{T \in \mathcal{L}_{\infty}\;:\; \sup_{n\geq 0}\, n^{-p}\mathrm{Tr}(\chi_{(\frac{1}{n},\infty)}(|T|)) < \infty\}.
    \end{equation*}
    It is well known that the ideal $\mathcal{L}_{p,\infty}$ may be equipped with a quasi-norm given by the formula
    \begin{equation*}
        \|T\|_{p,\infty} := \sup_{k\geq 0} (k+1)^{1/p}\mu(k,T), \quad T \in \mathcal{L}_{p,\infty}.
    \end{equation*}
    As is conventional, $\mathcal{L}_{\infty,\infty} := \mathcal{L}_{\infty}$. 
    
    We make use of the following H\"older inequality: let $p,p_1,p_2,\ldots,p_n \in (0,\infty]$ satisfy $\frac{1}{p} = \sum_{k=1}^n \frac{1}{p_k}$. If $A_k \in \mathcal{L}_{p_k,\infty}$ for all $k = 1,\ldots, n$, then
    $A_1A_2\cdots A_n \in \mathcal{L}_{p,\infty}$, with an inequality of norms:
    \begin{equation}\label{weak-type Holder}
        \|A_1A_2\cdots A_n\|_{p,\infty} \leq c_{p_1,p_2,\ldots,p_n}\|A_1\|_{p_{1},\infty}\|A_2\|_{p_2,\infty}\cdots \|A_n\|_{p_n,\infty}
    \end{equation}
    where $c_{p_1,p_2,\ldots,p_n} > 0$.
    
    The quasi-norm $\|\cdot\|_{1,\infty}$ is not monotone with respect to Hardy-Littlewood submajorisation. It is, however, monotone under
    logarithmic submajorisation. To be precise, we have that for all $A,B \in \mathcal{L}_{1,\infty}$ if $B \prec\prec_{\log} A$
    then
    \begin{equation}\label{log majorization monotone}
        \|B\|_{1,\infty} \leq e\|A\|_{1,\infty}.
    \end{equation}
    
    Indeed, since the sequence $\{\mu(k,B)\}_{k=0}^\infty$ is nonincreasing, for all $n \geq 0$ we have:
    \begin{equation*}
        \mu(n,B)^{n+1} \leq \prod_{k=0}^n \mu(k,B).
    \end{equation*}
    So if $B \prec\prec_{\log} A$,
    \begin{equation}\label{geometric mean bound}
        \mu(n,B)^{n+1} \leq \prod_{k=0}^n \mu(k,A).
    \end{equation}
    However by definition, $\mu(k,A) \leq \frac{\|A\|_{1,\infty}}{k+1}$ for all $k$, so
    \begin{equation}\label{factorial bound}
        \prod_{k=0}^n \mu(k,A) \leq \frac{\|A\|_{1,\infty}^{n+1}}{(n+1)!}.
    \end{equation}
    Now combining \eqref{geometric mean bound} and \eqref{factorial bound}, 
    \begin{equation*}
        \mu(n,B)^{n+1} \leq \frac{\|A\|_{1,\infty}^{n+1}}{(n+1)!}.
    \end{equation*}
    Now using the Stirling approximation
    $$(n+1)!\geq\big(\frac{n+1}{e}\big)^{n+1},$$
    we arrive at
    \begin{equation*}
        \mu(n,B)^{n+1} \leq \Big(\frac{e\|A\|_{1,\infty}}{n+1}\Big)^{n+1}.
    \end{equation*}
    Hence, for all $n\geq 0$,
    \begin{equation*}
        \mu(n,B) \leq\frac{e\|A\|_{1,\infty}}{n+1}.
    \end{equation*}
    Multiplying by $n+1$, and then taking the supremum over $n$ yields $\|B\|_{1,\infty} \leq e\|A\|_{1,\infty}$ as desired.
    
    Another ideal to which we will refer is the Schatten-Lorentz ideal $\mathcal{L}_{q,1}$ for $q > 1$, defined by
    \begin{equation*}
        \mathcal{L}_{q,1} := \{T \in \mathcal{L}_\infty\;:\;\sum_{k=0}^\infty \mu(k,T)(k+1)^{\frac{1}{q}-1} < \infty\}
    \end{equation*}
    and equipped with the quasi-norm
    \begin{equation*}
        \|A\|_{q,1} := \sum_{k=0}^\infty \mu(k,A)(1+k)^{\frac{1}{q}-1}.
    \end{equation*}

    If $\frac{1}{p}+\frac{1}{q} = 1$, then we have the following H\"older-type inequality:
    \begin{equation}\label{another holder}
        \|AB\|_1 \leq \|A\|_{p,\infty}\|B\|_{q,1},\quad A \in \mathcal{L}_{p,\infty}, B \in \mathcal{L}_{q,1}.
    \end{equation}


\subsection{Traces on $\mathcal{L}_{1,\infty}$}\label{trace subsection}

    \begin{defi}\label{trace def} 
        If $\mathcal{I}$ is an ideal in $\mathcal{L}_{\infty},$ then a unitarily invariant
        linear functional $\varphi:\mathcal{I}\to\mathbb{C}$ is said to be a trace.
    \end{defi}
    Here $\varphi$ being ``unitarily invariant" means that $\varphi(U^*TU) = \varphi(T)$ for all $T \in \mathcal{I}$ and unitary operators $U$. 
    Equivalently, $\varphi(UT) = \varphi(TU)$ for all unitary operators $U$ and $T \in \mathcal{I}$. Since every bounded linear operator can be written as a linear combination
    of at most four unitary operators \cite[Page 209]{Reed-Simon-I-1980}, one may equivalently say that $\varphi(AT) = \varphi(TA)$ for all $A \in \mathcal{L}_{\infty}$
    and $T \in \mathcal{I}$.
    
    The most well-known example of a trace is the classical trace $\mathrm{Tr}$ on the ideal $\mathcal{L}_1$, however we will be primarily concerned with traces on the ideal $\mathcal{L}_{1,\infty}$. 
    There exist many traces on $\mathcal{L}_{1,\infty}$, of which the earliest discovered class of examples are the Dixmier traces which we now describe.
    
    Recall that an extended limit is a continuous linear functional $\omega \in \ell_\infty^*$ from the set of bounded sequences $\ell_\infty$
    which extends the limit functional on the subspace $c$ of convergent sequences. Readers who are more familiar with ultrafilters may consider
    the special case where $\omega$ is the limit along a non-principal (free) ultrafilter on $\mathbb{Z}_+$.
    
    \begin{ex} 
        Let $\omega$ be an extended limit. Then the functional $\mathrm{Tr}_\omega$ is defined on a positive operator $T \in \mathcal{L}_{1,\infty}$ by
        \begin{equation*}
            \mathrm{Tr}_\omega(T) := \omega\left(\left\{\frac{1}{\log(2+n)}\sum_{k=0}^n \mu(k,T)\right\}_{n=0}^\infty\right).
        \end{equation*}
        The functional $\mathrm{Tr}_\omega$ is additive on the cone of positive elements of $\mathcal{L}_{1,\infty}$, and therefore extends by linearity to a a functional on $\mathcal{L}_{1,\infty}$. The thus defined functional $\mathrm{Tr}_\omega:\mathcal{L}_{1,\infty}\to \mathbb{C}$
        is a trace, and we call such traces Dixmier traces.
    \end{ex}
    \begin{proof}
        Let $A$ and $B$ be positive operators. Combining \cite[Theorem 3.3.3, Theorem 3.3.4]{LSZ}, for all $n \geq 0$ we have:
        \begin{equation*}
            \sum_{k=0}^n \mu(k,A+B) \leq \sum_{k=0}^n \mu(k,A)+\mu(k,B) \leq \sum_{k=0}^{2n+1} \mu(k,A+B).
        \end{equation*}
        Hence,
        \begin{equation*}
            0 \leq \sum_{k=0}^n \mu(k,A)+\mu(k,B)-\mu(k,A+B) \leq \sum_{k=n+1}^{2n+1} \mu(k,A+B).
        \end{equation*}
        However $A+B \in \mathcal{L}_{1,\infty}$, so there is a constant $C > 0$ such that for all $k \geq 0$ we have $\mu(k,A+B) \leq \frac{C}{k+1}$ and therefore
        \begin{equation*}
            0 \leq \sum_{k=0}^n \mu(k,A)+\mu(k,B)-\mu(k,A+B) \leq C,\quad n\geq 0.
        \end{equation*}
        Dividing by $\log(2+n)$:
        \begin{align*}
            0 &\leq \frac{1}{\log(2+n)}\sum_{k=0}^n \mu(k,A) + \frac{1}{\log(2+n)}\sum_{k=0}^n \mu(k,B) - \frac{1}{\log(2+n)}\sum_{k=0}^n \mu(k,A+B)\\
              &\leq O(\frac{1}{\log(2+n)}), \quad n\to\infty.
        \end{align*} 
        Then applying $\omega$:
        \begin{equation*}
            0 \leq \mathrm{Tr}_\omega(A)+\mathrm{Tr}_\omega(B)-\mathrm{Tr}_\omega(A+B) \leq 0.
        \end{equation*}
        So indeed $\mathrm{Tr}_\omega(A+B) = \mathrm{Tr}_\omega(A)+\mathrm{Tr}_\omega(B)$ for any two positive operators $A$ and $B$.
    \end{proof}
    
    \begin{rem}
        Dixmier traces were first defined by J. Dixmier in \cite{Dixmier}, albeit with some important differences to $\mathrm{Tr}_\omega$ as given in the above example.
        \begin{enumerate}[{\rm (i)}]
            \item Originally Dixmier traces were defined on the ideal $\mathcal{M}_{1,\infty}$ which is strictly larger than $\mathcal{L}_{1,\infty}.$
            \item $\mathrm{Tr}_\omega$ was originally shown to be additive only for those extended limits which are translation and dilation invariant.
        \end{enumerate}
        For more technical details and historical information we refer the reader to \cite[Chapter 6]{LSZ}.
                
        As the preceding example shows, $\mathrm{Tr}_\omega$ is additive on $\mathcal{L}_{1,\infty}$ for an arbitrary extended limit.
    \end{rem}    

    An extensive discussion of traces, and more recent developments in the theory, may be found in \cite{LSZ} including a discussion of the following facts:
    \begin{enumerate}
        \item All Dixmier traces on $\mathcal{L}_{1,\infty}$ are positive.
        \item All positive traces on $\mathcal{L}_{1,\infty}$ are continuous in the quasi-norm topology.
        \item Every continuous trace is a linear combination of positive traces.
        \item There exist positive traces on $\mathcal{L}_{1,\infty}$ which are not Dixmier traces (see \cite{SSUZ-pietsch}).
        \item There exist traces on $\mathcal{L}_{1,\infty}$ which fail to be continuous (see \cite{DFWW}).
        \item Every trace on $\mathcal{L}_{1,\infty}$ vanishes on $\mathcal{L}_1$ (see \cite{DFWW}).
    \end{enumerate}

    We are mostly interested in {\it normalised traces} $\varphi:\mathcal{L}_{1,\infty}\to\mathbb{C},$ that is, satisfying $\varphi({\rm diag}(\{\frac1{k+1}\}_{k\geq0}))=1.$

    The following definition, extending that originally introduced in \cite[Definition 2.$\beta$.7]{NCG-book}, plays an important role here.
    \begin{defi}\label{def:uni-meas}
        An operator $T\in\mathcal{L}_{1,\infty}$ is called universally measurable if all normalised traces take the same value on $T.$
    \end{defi}
    
    The following result characterises universally measurable operators in terms of their eigenvalues, and a detailed proof may be found in \cite[Theorem 10.1.3(g)]{LSZ}
    \begin{thm}\label{universal measurability criterion} 
        An operator $T\in\mathcal{L}_{1,\infty}$ is universally measurable if and only if there exists a constant $c$ such that
        $$\sum_{k=0}^n\lambda(k,T)=c\cdot\log(n)+O(1), \quad n\to\infty$$
        In this case, we have $\varphi(T)=c$ for every normalised trace $\varphi$ on $\mathcal{L}_{1,\infty}.$
    \end{thm}

\section{Spectral triples}\label{spectral triple subsection}
    A spectral triple is an algebraic model for a Riemannian manifold, defined as follows:
    \begin{defi}\label{spectral triple definition}
        A spectral triple $(\mathcal{A},H,D)$ consists of the following data:
        \begin{enumerate}[{\rm (a)}]
            \item{} a separable Hilbert space $H$.
            \item{} a (possibly unbounded) self-adjoint operator $D$ on $H$ with a dense domain $\mathrm{dom}(D)\subseteq H$.
            \item{} a $*$-subalgebra $\mathcal{A}$ of the algebra of bounded linear operators on $H$.
        \end{enumerate}
        Such that for all $a \in \mathcal{A}$ we have:
        \begin{enumerate}
            \item{} $a\cdot \mathrm{dom}(D) \subseteq \mathrm{dom}(D)$,
            \item{} The commutator $[D,a]:\mathrm{dom}(D)\to H$ extends to a bounded linear operator on $H$, which we denote $\partial(a)$,
            \item{} $a(D+i)^{-1}$ is a compact operator.
        \end{enumerate}
    \end{defi}
        
    \begin{rem}
        Definition \ref{spectral triple definition} should be compared to \cite[Definition 3.1]{CGRS2}, of which it is a special case (when the underlying von Neumann algebra is $\mathcal{L}_{\infty}(H)$). 
        Within the literature there is some variation in the definition of a spectral triple. In many sources (such as \cite[Definition 9.16]{GVF}) it is assumed that the resolvent $(D+i)^{-1}$ is compact.
%         In many sources (such as \cite[Definition 9.16]{GVF}) it is assumed that the resolvent $(D+i)^{-1}$ is compact.
        We will refer to spectral triples where $(D+i)^{-1}$ is compact as compact spectral triples. In particular a spectral triple where $\mathcal{A}$ contains the identity operator is compact. 
        If $(\mathcal{A},H,D)$ is not necessarily compact, we will say that is it locally compact.
    \end{rem}
    
    \begin{defi}
    Given a spectral triple $(\mathcal{A},H,D)$ let $F_D$ denote the partial isometry defined via functional calculus as
    $$F_D := \chi_{(0,\infty)}(D)-\chi_{(-\infty,0)}(D).$$
    Where there is no ambiguity, we will frequently denote $F_D$ as $F$.        
    \end{defi}
    If $D$ has trivial kernel, then $F_D^2 = 1.$
    
    We may define the operator $|D|:\mathrm{dom}(D)\to H$ by functional calculus.
    Since $D$ is self-adjoint, for all $n\geq 1$ we have $|D|^n = |D^n|$, and so $\mathrm{dom}(|D|^n) = \mathrm{dom}(D^n)$.
    We have $F|D| = D$ as an equality of operators on $\mathrm{dom}(D)$, and on $\mathrm{dom}(D^2)$:
    \begin{equation*}
        |D|D = D|D|.
    \end{equation*}
    We also have $|D|=FD$.
    
    Note that $F_D^* = F_{D^*} = F_D$. Hence, we also have $D = D^*=|D|^*F^* = |D|F$.
    Since $|D|F = D$, it follows that $F:\mathrm{dom}(D)\to\mathrm{dom}(D)$.
    
    By similar reasoning, we also have that for all $n\geq 1$ that $D^{n}F = FD^{n}$ and hence that $F:\mathrm{dom}(D^n)\to \mathrm{dom}(D^n)$.
    
    
    Consequently, for $n,m \geq 1$, the operators $F$, $D^n$ and $|D|^m$ all mutually commute on $\mathrm{dom}(D^{n+m})$.
    
\subsection{Properties of spectral triples}
    Smoothness of a spectral triple is defined in terms of boundedness of commutators with $|D|$ (see Subsection \ref{smoothness discussion} for discussion of this issue).
    The following results will be known to the expert reader. The notion of smoothness defined in terms of domains of commutators with $|D|$ originates with Connes \cite[Section 1]{Connes-original-spectral-1995} and
    is also used in \cite{Connes-Moscovici} and \cite[Section 1.3]{CGRS2}. We provide detailed proofs here for convenience.
    
    If $T$ is a bounded operator with $T:\mathrm{dom}(D)\to \mathrm{dom}(D)$, then the commutator
    $|D|T-T|D|:\mathrm{dom}(D)\to H$ is meaningful. More generally, if there is some $n$ such that for all $0 \leq k \leq n$ we
    have $T:\mathrm{dom}(D^k)\to \mathrm{dom}(D^k)$ then we may consider the higher iterated commutator:
    \begin{equation}\label{nth iterated commutator}
        [|D|,[|D|,[\cdots [|D|,T]\cdots]]] = \sum_{k=0}^n (-1)^k \binom{n}{k}|D|^ka|D|^{n-k}.
    \end{equation}
    This is a well defined operator on $\mathrm{dom}(D^n)$ in the following sense: for each $k$ we have $|D|^{n-k}:\mathrm{dom}(D^n)\to\mathrm{dom}(D^k)$,
    $a:\mathrm{dom}(D^k)\to\mathrm{dom}(D^k)$ and $|D|^k:\mathrm{dom}(D^k)\to H$.
    
    We wish to define $\delta^n(T)$ as the bounded extension of the $n$th iterated commutator $[|D|,[|D|,[\cdots,[|D|,T]\cdots]]]$
    when such an extension exists. This motivates the following definition:
    \begin{defi}
        For $n\geq 1$, we define $\mathrm{dom}(\delta^n)$ to be the set of bounded linear operators $T$ such that for all $0 < k \leq n$ we
        have $T:\mathrm{dom}(D^k)\to \mathrm{dom}(D^k)$ and the $n$th iterated commutator in \eqref{nth iterated commutator} has bounded extension.
        
        For $T \in \mathrm{dom}(\delta^n)$, we let $\delta^n(T)$ be the bounded extension of the $n$th iterated commutator \eqref{nth iterated commutator}.
        
        The $n=0$ case is defined by $\mathrm{dom}(\delta^0) := \mathcal{L}_\infty(H)$ and $\delta^0(T) := T$.
        
        We define 
        \begin{equation*}
            \mathrm{dom}_\infty(\delta) := \bigcap_{k\geq 0} \mathrm{dom}(\delta^k).
        \end{equation*}
    \end{defi}
    
    \begin{lem}\label{dom delta infty is an algebra}
        The set $\mathrm{dom}_\infty(\delta)$ is closed under multiplication.
    \end{lem}
    \begin{proof}
        Let $T,S \in \mathrm{dom}_\infty(\delta)$. Then by definition, for all $k \geq 1$ we have that $T,S:\mathrm{dom}(D^k)\to \mathrm{dom}(D^k)$,
        and hence $TS:\mathrm{dom}(D^k)\to \mathrm{dom}(D^k)$. The $k$th iterated commutator $\delta^k(TS)|\mathrm{dom}(D^k)$ is given by:
        \begin{equation*}
            \delta^k(TS) = \sum_{j=0}^k \binom{k}{j} \delta^{k-j}(T)\delta^j(S).
        \end{equation*}
        Since for all $j$ we have $\delta^j(S) \in \mathrm{dom}_\infty(\delta)$ and $\delta^{k-j}(T) \in \mathrm{dom}_\infty(\delta)$, the above expression
        is well defined as an operator on $\mathrm{dom}(D^k)$ and has bounded extension.
    \end{proof}
    
    
    It is clear that if $k\geq 0$ and $T \in \mathrm{dom}(\delta^{k+1})$, then $\delta(T) \in \mathrm{dom}(\delta^k)$ and $\delta^k(T) \in \mathrm{dom}(\delta)$. Moreover $\delta^{k+1}(T) = \delta^k(\delta(T)) = \delta(\delta^k(T))$.
    
    We may also define $\mathrm{dom}(\partial)$ to be the set of bounded operators $T$ such that $T:\mathrm{dom}(D)\to \mathrm{dom}(D)$
    and $[D,T]:\mathrm{dom}(D)\to H$ has a bounded extension, which we denote $\partial(T)$. 
    
    The relevance of $\mathrm{dom}(\partial)$ is the following:
    \begin{lem}\label{delta and partial commute}
        Suppose that $T \in \mathrm{dom}(\delta)\cap\mathrm{dom}(\partial)$ is such that $\partial(T) \in \mathrm{dom}(\delta)$ and $\delta(T):\mathrm{dom}(D)\to \mathrm{dom}(D)$. Then $\delta(T) \in \mathrm{dom}(\partial)$ and
        \begin{equation*}
            \partial(\delta(T)) = \delta(\partial(T)).
        \end{equation*}
    \end{lem}
    \begin{proof}
        Since $T \in \mathrm{dom}(\delta)\cap\mathrm{dom}(\partial)$, we have in particular that $T:\mathrm{dom}(D)\to\mathrm{dom}(D)$. Since $\partial(T) \in \mathrm{dom}(\delta)$,
        we also have that $\partial(T):\mathrm{dom}(D)\to \mathrm{dom}(D)$. Let $\xi \in \mathrm{dom}(D^2)$. Then,
        \begin{equation*}
            DT\xi = \partial(T)\xi + TD\xi.
        \end{equation*}
        Since $T:\mathrm{dom}(D)\to \mathrm{dom}(D)$ and $\partial(T):\mathrm{dom}(D^2)\to\mathrm{dom}(D)$, it follows that $DT\xi \in \mathrm{dom}(D)$ and therefore $T:\mathrm{dom}(D^2)\to \mathrm{dom}(D^2)$. 
        
        Now since the operators $D$ and $|D|$ commute on $\mathrm{dom}(D^2)$, we have that for all $\xi \in \mathrm{dom}(D^2)$:
        \begin{equation*}
            [D,[|D],T]]\xi = [|D|,[D,T]]\xi.
        \end{equation*}
        Since by assumption $T \in \mathrm{dom}(\delta)$, $\partial T \in \mathrm{dom}(\delta)$ and $\delta(T):\mathrm{dom}(D)\to\mathrm{dom}(D)$, we may further write:
        \begin{equation*}
            [D,\delta(T)]\xi = \delta(\partial(T))\xi.
        \end{equation*}
        Since the operator on the right hand side by assumption has bounded extension, and using the fact that $\mathrm{dom}(D^2)$ is dense { in $H$} it follows that $[D,\delta(T)]$ has bounded extension and therefore
        $\delta(T) \in \mathrm{dom}(\partial)$. Thus, $\partial(\delta(T)) = \delta(\partial(T))$.
    \end{proof}
    
    
    Next we define the notion of a smooth spectral triple. Some sources (such as \cite[Definition 3.18]{CGRS2}) use the term "$QC^{\infty}$ spectral triple", and others (such as \cite[Definition 4.25]{higson} use the term "regular" spectral triple).
    \begin{defi}\label{smoothness definition}
        A spectral triple $(\mathcal{A},H,D)$ is called smooth if for all $a \in \mathcal{A}$, we have
        \begin{equation*}
            a, \partial(a) \in \mathrm{dom}_\infty(\delta).
        \end{equation*}
        If $(\mathcal{A},H,D)$ is smooth, we let $\mathcal{B}$ be the $*-$subalgebra of $\mathcal{L}_\infty(H)$ generated by all elements of the form $\delta^k(a)$ or $\delta^k(\partial(a)),$ $k\geq0,$ $a\in\mathcal{A}.$
    \end{defi}
    By Lemma \ref{dom delta infty is an algebra} and since $\delta^k(a)^* = (-1)^k\delta^k(a^*)$, we automatically have that $\mathcal{B} \subseteq \mathrm{dom}_\infty(\delta)$.
    
    
    \begin{cor}
        Let $(\mathcal{A},H,D)$ be smooth, and $a \in \mathcal{A}$. Then for all $k\geq 1$ we have $\delta^k(a) \in \mathrm{dom}(\partial)$ and
        \begin{equation*}
            \partial(\delta^k(a)) = \delta^k(\partial(a)).
        \end{equation*}
    \end{cor}
    \begin{proof}
        This proof proceeds by induction on $k$. 
        For $k=1$, by the definition of smoothness we have $\partial(a) \in \mathrm{dom}(\delta)$ and $a \in \mathrm{dom}(\delta)\cap\mathrm{dom}(\partial)$, and by definition $a:\mathrm{dom}(D)\to\mathrm{dom}(D)$. So by Lemma \ref{delta and partial commute} it follows that
        $\delta(a) \in \mathrm{dom}(\partial)$ and 
        \begin{equation*}
            \partial(\delta(a)) = \delta(\partial(a)).
        \end{equation*}
        
        Now we suppose that the claim is proved for $k-1$, $k\geq 2$ and we prove the claim for $k$. Since $(\mathcal{A},H,D)$ is smooth, $\delta^{k-1}(a):\mathrm{dom}(D)\to \mathrm{dom}(D)$ and by the inductive hypothesis, $\delta^{k-1}(a) \in \mathrm{dom}(\partial)$ and
        \begin{equation*}
            \delta^{k-1}(\partial(a)) = \partial(\delta^{k-1}(a)).
        \end{equation*}
        However since $\delta^{k-1}(a) \in \mathrm{dom}(\delta)\cap\mathrm{dom}(\partial)$ and $\delta^{k-1}(a):\mathrm{dom}(D)\to\mathrm{dom}(D)$ we may apply Lemma \ref{delta and partial commute} with $T = \delta^{k-1}(a)$ to conclude
        that $\delta^{k}(a) \in \mathrm{dom}(\partial)$ and
        \begin{align*}
            \delta(\partial(\delta^{k-1}(a))) &= \partial(\delta(\delta^{k-1}(a)))\\
                                              &= \partial(\delta^k(a)).
        \end{align*}
        By the inductive hypothesis, $\delta(\partial(\delta^{k-1}(a))) = \delta(\delta^{k-1}(\partial(a))) = \delta^k(\partial(a))$, and so this proves the result for $k$.
    \end{proof}
    
    
    \begin{defi}
        Let $T \in \mathrm{dom}(\partial)\cap \mathrm{dom}(\delta)$. Define
        \begin{equation*}
            L(T) := \partial(T)-F\delta(T).
        \end{equation*}
    \end{defi}
    Note that by definition $L(T)$ is bounded. On $\mathrm{dom}(D)$ we have:
    \begin{equation*}
        L(T) = [F,T]|D|.
    \end{equation*}
    The boundedness of $L(T)$ on $\mathrm{dom}(D)$ was already implicitly noted in the proof of \cite[Lemma 2]{CPRS1}.
    
    Our computations are greatly simplified by introducing a common dense subspace $H_\infty \subseteq H$ on which all powers $D^k$
    are defined:
    \begin{defi}\label{H_infty definition}
        Let $H_\infty := \bigcap_{n\geq 0} \mathrm{dom}(D^n)$.
    \end{defi}
    The subspace $H_\infty$ is a well known object in noncommutative geometry, appearing in \cite[Section 1]{Connes-original-spectral-1995} and more recently in \cite[Equation 10.64]{GVF} and \cite[Definition 1.20]{CGRS2}.
    One way to see that $H_\infty$ is dense in $H$ (and in particular non-zero) is to note that $\mathrm{dom}(e^{D^2}) \subseteq H_\infty$.
    If $T \in \mathrm{dom}_\infty(\delta)$, then $T:H_\infty\to H_\infty$ since by definition if $T \in \mathrm{dom}(\delta^n)$ then $T:\mathrm{dom}(D^k)\to\mathrm{dom}(D^k)$ for all $0 \leq k \leq n$. Moreover since $F:\mathrm{dom}(D^n)\to \mathrm{dom}(D^n)$ for all $n$, we also
    have $F:H_\infty\to H_\infty$. It is useful to note that for each $k$ the unbounded operators $D^k$ and $|D|^k$ map $H_\infty$ to $H_\infty$. This observation is justified in the next lemma.
    
    \begin{lem}
        Let $f:\mathbb{R}\to \mathbb{R}$ be a Borel function which has "polynomial growth at infinity" in the sense that there exists $n \geq 0$ such that
        $t\mapsto (1+t^2)^{-n/2}f(t)$ is bounded on $\mathbb{R}$. Then $f(D)$ (defined by Borel functional calculus) maps $H_\infty$ to $H_\infty$.
    \end{lem}
    \begin{proof}
        Let $k > n$, and $\xi \in \mathrm{dom}(D^k)$. By assumption $(1+D^2)^{-n/2}f(D)$ defines a bounded operator,
        \begin{equation*}
            (1+D^2)^{-\frac{n}{2}}f(D)D^k:\mathrm{dom}(D^k)\to H.
        \end{equation*}
        However $(1+D^2)^{-n/2}f(D)D^k = D^k(1+D^2)^{-n/2}f(D)$ on a dense domain. Since $D^n(1+D^2)^{-n/2}$ defines a bounded operator, we get that $D^{k-n}f(D):\mathrm{dom}(D^k)\to H$. Therefore
        $f(D):\mathrm{dom}(D^k)\to \mathrm{dom}(D^{k-n})$. Since $k > n$ is arbitrary, we have that $f(D):H_\infty\to H_\infty$.
    \end{proof}
    
    \begin{lem}
        If $T \in \mathrm{dom}(\delta)\cap \mathrm{dom}(\partial)$ is such that $T:H_\infty\to H_\infty$, then $L(T):H_\infty \to H_\infty$.
    \end{lem}
    \begin{proof}
        For $\xi \in H_\infty$,
        \begin{equation*}
            L(T)\xi = [F,T]|D|\xi.
        \end{equation*}
        However $|D|\xi \in H_\infty$, and $F:H_\infty\to H_\infty$. Thus $[F,T]|D|\xi \in H_\infty$.
    \end{proof}
    
    \begin{lem}
        Let $T\in \mathrm{dom}(\delta^2)\cap \mathrm{dom}(\partial)$ be such that $\partial(T) \in \mathrm{dom}(\delta)$. Then $L(T) \in \mathrm{dom}(\delta)$ and
        \begin{equation*}
            \delta(L(T)) = L(\delta(T)).
        \end{equation*}
    \end{lem}
    \begin{proof}
        Since $\partial(T) \in \mathrm{dom}(\delta)$, we have from Lemma \ref{delta and partial commute} that $\delta(T) \in \mathrm{dom}(\partial)\cap \mathrm{dom}(\delta)$ and hence $L(\delta(T))$ is defined and bounded.
    
        The required identity can be checked on $\mathrm{dom}(D^2)$, since $T:\mathrm{dom}(D^2)\to \mathrm{dom}(D^2)$. 
        For $\xi \in \mathrm{dom}(D^2)$, we have:
        \begin{align*}
            \delta(L(T))\xi &= [|D|,[F,T]|D|]\xi\\
                            &= [F,[|D|,T]]|D|\xi\\
                            &= L(\delta(T))\xi.
        \end{align*}
    \end{proof}
    
    Spectral triples are often classed as \emph{even} or \emph{odd}:
    \begin{defi}\label{oddeven} 
        A spectral triple $(\mathcal{A},H,D)$ is said to be
        \begin{enumerate}[{\rm (a)}]
        \item even if equipped with $\Gamma\in\mathcal{L}_{\infty}$ such that $\Gamma=\Gamma^*,$ $\Gamma^2=1$ and such that $[\Gamma,a]=0$ for all $a\in\mathcal{A},$ $\{D,\Gamma\}=0.$ Here $\{\cdot,\cdot\}$ denotes anticommutator.
        \item odd if not equipped with such $\Gamma.$ In this case, we set $\Gamma=1.$
        \item $p-$dimensional if for all $a \in \mathcal{A}$ we have $a(D+i)^{-p} \in \mathcal{L}_{1,\infty}$ and $\partial(a)(D+i)^{-p}\in\mathcal{L}_{1,\infty}$, and for all $q < p$ there exists $a_0\in \mathcal{A}$ such that $a_0(D+i)^{-q}\notin\mathcal{L}_{1,\infty}$.
        \end{enumerate}
    \end{defi}
    For an even spectral triple, we have $D^2\Gamma=\Gamma D^2.$ Therefore, $|D|\Gamma=\Gamma|D|$. We furthermore have that $F\Gamma+\Gamma F = 0$.
    
    We follow the convention of \cite{CGRS2}, where we write $\Gamma$ in all formulae referring to spectral triples, with the understanding that if the spectral triple is odd then $\Gamma=1$ and the assumption that $\{D,\Gamma\}=0$ is dropped.
    
    For an arbitrary spectral triple, we have $|D|^k\Gamma = \Gamma|D|^k$ for all $k$, and therefore $\Gamma:\mathrm{dom}(D^k)\to \mathrm{dom}(D^k)$ for all $k$. Hence $\Gamma:H_\infty\to H_\infty$.
    
    The following assertion is well-known in the compact case (see e.g. \cite{CPRS1} and \cite{PS-crelle}). To the best of our knowledge, no proof has been published in the locally compact case. 
    We supply a proof in Section \ref{fredholm section}, Proposition \ref{f der def}.
    \begin{prop*}
        {Let $p \in \mathbb{N}$.} If $(\mathcal{A},H,D)$ is a $p-$dimensional spectral triple satisfying Hypothesis \ref{main assumption}, then $[F,a]\in\mathcal{L}_{p,\infty}$ for all $a\in\mathcal{A}.$
    \end{prop*}
        
    Let $\mathcal{A}^{\otimes(p+1)}$ denote the $(p+1)$-fold algebraic tensor power of $\mathcal{A}$. We now define the two important mappings $\mathrm{ch}$ and $\Omega$.
    \begin{defi}\label{ch omega def} 
        Suppose that $D$ has a spectral gap at $0.$
        Define the multilinear mappings ${\rm ch}:\mathcal{A}^{\otimes (p+1)}\to\mathcal{L}_{\infty}$ and $\Omega:\mathcal{A}^{\otimes (p+1)}\to\mathcal{L}_{\infty}$ 
        on elementary tensors $a_0\otimes a_1\otimes\cdots\otimes a_p \in \mathcal{A}^{\otimes(p+1)}$ by
        \begin{equation*}
            \mathrm{ch}(a_0\otimes a_1\otimes\cdots \otimes a_p) = \Gamma F\prod_{k=0}^p[F,a_k]
        \end{equation*}
        and
        \begin{equation*}
            \Omega(a_0\otimes a_1\otimes \cdots \otimes a_p) = \Gamma a_0\prod_{k=1}^p \partial (a_k).
        \end{equation*}
    \end{defi}

    If $(\mathcal{A},H,D)$ is $p$-dimensional, then it follows from Proposition \ref{f der def} and the H\"older inequality \eqref{weak-type Holder} that ${\rm ch}(c)\in\mathcal{L}_{\frac{p}{p+1},\infty}\subset\mathcal{L}_1$
    for all $c\in\mathcal{A}^{\otimes (p+1)}.$ This permits the following definition:

    \begin{defi}\label{chern character zero kernel def}
        If $(\mathcal{A},H,D)$ is $p-$dimensional spectral triple satisfying Hypothesis \ref{main assumption} and if kernel of $D$ is trivial, then Connes-Chern character $\mathrm{Ch}:\mathcal{A}^{\otimes (p+1)}\to\mathbb{C}$ is defined by setting
        \begin{equation*}
            {\rm Ch}(c)=\frac12{\rm Tr}({\rm ch}(c)),\quad c\in\mathcal{A}^{\otimes (p+1)}.
        \end{equation*}
    \end{defi}
    
    In general, however, Chern character cannot be defined in terms of $F$ because $F^2\neq 1$ when $D$ has non-trivial kernel. 
    In order to ensure that ${\rm Ch}$ is a cyclic cocycle (in the sense of \cite[2.1.4]{Loday-cyclic-homology}), we require that $F^2 = 1$.
%     It is desirable for the applications in cyclic homology to
    Hence we define the Chern character of a general spectral triple in terms of another $F_0$ such that $F_0=F_0^*=F_0^2.$ For this purpose, we use a doubling trick.
    
    \begin{defi}\label{doubling definition}
        Let $(\mathcal{A},H,D)$ be a spectral triple with grading $\Gamma$, and let $P$ be the projection onto $\ker(D)$. Consider the following unitary self-adjoint operators on the Hilbert space $H_0=\mathbb{C}^2\otimes H$ defined by:
        \begin{align*}
                F_0 := \begin{pmatrix} F & P \\ P & -F\end{pmatrix}\\
            \Gamma_0 := \begin{pmatrix} \Gamma & 0 \\ 0 & (-1)^{\rm deg}\Gamma\end{pmatrix}.
        \end{align*}
        Here, ${\rm deg}=1$ for even triples and ${\rm deg}=0$ for odd triples. The algebra $\mathcal{A}$ is represented on $H_0$ by:
        \begin{equation*}
            \pi(a) = \begin{pmatrix}
                        a & 0 \\ 0 & 0
                     \end{pmatrix}.
        \end{equation*}
        For an elementary tensor $a_0\otimes\cdots a_p \in \mathcal{A}^{\otimes (p+1)}$ we set:
        \begin{equation*}
            {\rm ch}_0(a_0\otimes \cdots\otimes a_p) = \Gamma_0F_0\prod_{k=0}^p [F_0,\pi(a_k)].
        \end{equation*}
    \end{defi}
    

    
    
%     This allows a definition of Chern character in general situation. In the case when $D$ has trivial kernel, both definitions coincide.
    
    \begin{defi}\label{chern character def}
    If $(\mathcal{A},H,D)$ is $p-$dimensional spectral triple satisfying Hypothesis \ref{main assumption}, then Connes-Chern character $\mathrm{Ch}:\mathcal{A}^{\otimes (p+1)}\to\mathbb{C}$ is defined by setting
        \begin{equation*}
            {\rm Ch}(c)=\frac12({\rm Tr_2}\otimes{\rm Tr})({\rm ch}_0(c)),\quad c\in\mathcal{A}^{\otimes (p+1)}.
        \end{equation*}
        Here, $\mathrm{Tr}_2$ denotes the $2\times 2$ trace on matrices.
    \end{defi}
    
    Note that Definition \ref{doubling definition} does not conflict with Definition \ref{ch omega def}: if $\ker(D)$ is trivial (i.e., $P=0$) then both definitions of $\mathrm{Ch}$ coincide.

    Strictly speaking, the Connes-Chern character is conventionally defined to be the class of $\mathrm{Ch}$ in periodic cyclic cohomology. This distinction is not relevant to the results of this paper, so in the sequel
    we will consider $\mathrm{Ch}$ merely as a multilinear functional as above.
   
    \begin{rem}
        We have opted to define the Chern character of a spectral triple $(\mathcal{A},H,D)$ in terms of the "doubled" operator $F_0$ in Definition \ref{doubling definition}. This definition is different
	from earlier work such as \cite[Definition 6]{CPRS1} and \cite[Definition 2.23]{CGRS2}. In those papers the Chern character of a spectral triple is defined to be the Chern character of any Fredholm module
        equivalent to the pre-Fredholm module $(\mathcal{A},H,F)$. It is known that the class in periodic cyclic cohomology of the chern character defined in that way is independent of the choice of Fredholm module equivalent to $(\mathcal{A},H,F)$ \cite[Section 5, Lemma 1]{Connes-differential-geometry}.
        
        In order to avoid technicalities, we have defined the Chern character in terms of a specific Fredholm module $(\pi(\mathcal{A}),H_0,F_0)$. This has the advantage of simplicity of presentation, and makes
        no difference in regards to the character formula. A reader interested in a more refined definition of the Chern character in periodic cyclic cohomology may wish to consult \cite{Connes-differential-geometry, CPRS1, CGRS2}.
    \end{rem}
    
\subsection{Discussion of smoothness}\label{smoothness discussion}
    It is tempting to define smoothness only in terms of $\partial$, without reference to $\delta$. One might naively suggest that $(\mathcal{A},H,D)$ is smooth if for all $n \geq 0$
    we have $a\cdot \mathrm{dom}(D^n)\to \mathrm{dom}(D^n)$ and the $n$th iterated commutator $[D,[D,[\cdots,[D,a]\cdots]$ extends to a bounded operator on $H$.
    
    However this condition does not hold for even the simplest spectral triples. A standard spectral triple associated to the $2$-torus $\mathbb{T}^2$ is
    $(1\otimes C^\infty(\mathbb{T}^2), L_2(\mathbb{T}^2,\mathbb{C}^2),D)$, where the $L_2$ space is defined with respect to the Haar measure, the algebra $C^\infty(\mathbb{T}^2)$
    {  of smooth complex valued functions on $\mathbb{T}^2$}
    acts on $L_2(\mathbb{T}^2,\mathbb{C}^2)$ by pointwise multiplication, and the Dirac operator $D$ is defined by:
    \begin{equation*}
        D = -i\gamma_1\otimes \partial_1-i\gamma_2\otimes \partial_2,
    \end{equation*}
    where $\partial_1$ and $\partial_2$ are differentiation with respect to the first and second coordinates on $\mathbb{T}^2$
    and $\gamma_1,\gamma_2$ are $2\times 2$ complex matrices satisfying $\gamma_j\gamma_k+\gamma_k\gamma_j = 2\delta_{j,k}1$, $j,k = 1,2$.
    
    Then if $f \in C^\infty(\mathbb{T}^2)$,
    \begin{align*}
        [D,[D,1\otimes M_f]] &= -[\gamma_1\otimes\partial_1+\gamma_2\otimes \partial_2,\gamma_1\otimes M_{\partial_1 f}+\gamma_2\otimes M_{\partial_2 f}]\\
                             &= -1\otimes M_{\partial_1^2f+\partial_2^2f}+2\gamma_1\gamma_2\otimes (M_{\partial_2f} \partial_1+M_{\partial_1f} \partial_2)
    \end{align*}
    However this operator is typically unbounded: if we choose $f(z_1,z_2) = z_1$, then $[D,[D,1\otimes M_f]] = 2\gamma_1\gamma_2\otimes(\partial_2)$ which is unbounded.

    This example breaks the implication: ``if $f \in C^\infty(\mathbb{T}^2)$, then $[D,[D,1\otimes M_f]]$ extends to a bounded linear operator".
    
 
\subsection{Discussion of dimension}\label{dimension discussion}
    As we have defined it, we say that a spectral triple $(\mathcal{A},H,D)$ is $p$-dimensional if for all $a \in \mathcal{A}$ the operators $a(D+i)^{-p}$ and $\partial(a)(D+i)^{-p}$ are in $\mathcal{L}_{1,\infty}$.
    
    An alternative definition, also used in the literature, is to say that $(\mathcal{A},H,D)$ is $p$-dimensional if $a(D+i)^{-1}$ and $\partial(a)(D+i)^{-1}$ are in $\mathcal{L}_{p,\infty}$. 
    An example of a definition along these lines is \cite[Definition 3.1]{gayral-moyal}.
    Clearly in the case where $\mathcal{A}$ is unital these definitions are equivalent, since $(D+i)^{-1} \in \mathcal{L}_{p,\infty}$ if and only if $(D+i)^{-p} \in \mathcal{L}_{1,\infty}$. 
    However in the non-unital case, the distinction may be important. 
    
%     To see this, consider the example of a spectral triple on $\mathbb{R}^2$ [this example needs to be properly explained].


\subsection{Hochschild (co)homology}\label{hochschild subsection}
    Hochschild homology and cohomology provide noncommutative generalisations of the notion of differential forms and de Rham currents
    respectively. A detailed exposition of the theory of Hochschild (co)homology and its relationship with noncommutative geometry may be found in \cite{Quillen,Loday-cyclic-homology}.
    
    Let $A$ be a (possibly non-unital) algebra. The Hochschild complex is a chain complex:
    \begin{equation*}
        \cdots \xrightarrow{b} A\otimes A\otimes A\otimes A \xrightarrow{b} A\otimes A\otimes A \xrightarrow{b} A \otimes A\xrightarrow{b} A.
    \end{equation*}
    For $n \geq 1$, the $n$th entry in the Hochschild chain complex is the $n$th tensor power $A^{\otimes n}$. The Hochschild boundary operator $b:A^{\otimes (n+1)}\to A^{\otimes n}$
    is defined on elementary tensors $a_0\otimes a_1\otimes\cdots \otimes a_n$ by:
    \begin{align*}
        b(a_0\otimes a_1\otimes \cdots \otimes a_n) &= a_0a_1\otimes a_2\otimes \cdots\otimes a_n + \sum_{k=1}^{n-1} (-1)^k a_0\otimes \cdots \otimes a_ka_{k+1}\otimes\cdots \otimes a_n\\
                                                    &+ (-1)^na_na_0\otimes a_1\cdots\otimes a_{n-1}.
    \end{align*}
    It can be verified that $b^2 = 0$, so the Hochschild complex is indeed a chain complex. An element $c \in A^{\otimes (n+1)}$ such that $bc = 0$
    is called a Hochschild cycle. For example, when $n = 1$, an elementary tensor $a_0\otimes a_1$ is a Hochschild cycle if and only if $b(a_0\otimes a_1) = a_0a_1-a_1a_0 = 0$, i.e. if $a_0$ and
    $a_1$ commute.
    
    The Hochschild cochain complex is defined in a similar way: Let $C_n(A)$ denote the space of continuous multilinear functionals from $A^{\otimes n} \to \mathbb{C}$. The Hochschild cochain complex is,
    \begin{equation*}
        C_1(A)\xrightarrow{b} C_2(A)\xrightarrow{b} C_3(A) \xrightarrow{b} \cdots
    \end{equation*}
    where the Hochschild coboundary operator $b$ is defined as follows: if $\theta:A^{\otimes n}\to \mathbb{C}$, then $b\theta:A^{\otimes (n+1)}\to \mathbb{C}$
    is given on an elementary tensor $a_0\otimes a_1\otimes\cdots \otimes a_n$ by
    \begin{align*}
        (b\theta)(a_0\otimes\cdots\otimes a_n)&=\theta(a_0a_1\otimes a_2\otimes\cdots\otimes a_n)\\
        &+\sum_{k=1}^{n-1}(-1)^k\theta(a_0\otimes a_1\otimes\cdots \otimes a_{k-1}\otimes a_ka_{k+1}\otimes a_{k+2}\otimes\cdots\otimes a_n)\\
        &+(-1)^n\theta(a_na_0\otimes a_1\otimes a_2\otimes\cdots\otimes a_{n-1}).
    \end{align*}
    Put simply, for $c \in A^{\otimes (n+1)}$ and $\theta \in C_n(A)$, the Hochschild boundary and coboundary operators are linked by,
    \begin{equation}\label{stokes formula}
        (b\theta)(c) = \theta(bc).
    \end{equation}
    A cochain $\theta \in C^n(A)$ is called a Hochschild cocycle if $b\theta = 0$. Due to \eqref{stokes formula}, a Hochschild coboundary
    vanishes on every Hochschild cycle.    
%     
%     If $\mathcal{A}$ is unital, $p$ is even and $e \in \mathcal{A}$ is a projection (i.e., $e = e^* = e^2$) then an example of a Hochschild cycle is the Connes-Chern character
%     on $K$-theory:
%     \begin{equation*}
%         (e-\frac{1}{2})\otimes e^{\otimes p}.
%     \end{equation*}
%     
%     Another example is available when $\mathcal{A}$ is unital and $p$ is odd. Then if $u \in \mathcal{A}$ is unitary (i.e, $uu^* = u^*u = 1$) then 
%     \begin{equation*}
%         u^*\otimes u\otimes u^*\otimes \cdots \otimes u \in \mathcal{A}^{\otimes(p+1)}
%     \end{equation*}
%     is a Hochschild cycle.
%     
    \begin{rem}
        One must distinguish between Hochschild (co)homology as we have just defined it, and the analogous continuous Hochschild homology \cite[Section 8.5]{GVF}, \cite{Pflaum-continuous-hh-1998}. Continuous Hochschild (co)homology is defined with topological tensor products in place of algebraic tensor products. In this text we are only concerned with algebraic tensor products.
    \end{rem}
    
\section{Weak integrals and double operator integrals}
    
\subsection{Weak integration in $\mathcal{L}_{\infty}$}\label{weak int}
    This section concerns the theory of ``weak operator topology integrals" of operator valued functions. The following definitions, and the subsequent construction of weak integrals, are folklore. We provide suitable references whenever they
    exist, otherwise we supply a proof. For example, one can look at \cite[Definition 3.26]{rudin}, and consider the example where the topological vector space $X$ there is $\mathcal{L}_{\infty}$ equipped with the strong operator topology. Every
    continuous linear functional on $X$ can be written as a linear combination of $x\to\langle x\xi,\eta\rangle,$ $\xi,\eta\in H.$

    \begin{defi}\label{weak meas def} 
        A function $f:\ \mathbb{R}\to \mathcal{L}_{\infty}$ is measurable in the weak operator topology if, for every pair of vectors $\xi,\eta\in H,$ the function
        $$s\to\langle f(s)\xi,\eta\rangle,\quad s\in\mathbb{R},$$
        is (Lebesgue) measurable.
    \end{defi}
    
    For a function $f$, measurable in the weak operator topology, there is a notion of ``pointwise norm". 
    Namely, the scalar-valued mapping
    \begin{equation*}
        s\mapsto \|f(s)\|_\infty := \sup_{\|\xi\|,\|\eta\| \leq 1} |\langle f(s)\xi,\eta\rangle|, \quad s \in \mathbb{R}
    \end{equation*}
    is Lebesgue measurable. { Here it is crucial that we work with separable Hilbert spaces, as otherwise it is not clear whether the function $s\mapsto \|f(s)\|_\infty$ is measurable.}

    Suppose that a function $f:\mathbb{R}\to\mathcal{L}_{\infty}$ is measurable in the weak operator topology. We say that $f$ is integrable in the weak operator topology if
    \begin{equation}\label{necessary-condition}
        \int_{\mathbb{R}}\|f(s)\|_{\infty}ds < \infty.
    \end{equation}
    In particular, for all $\xi,\eta \in H$, we have
    \begin{equation*}
        \int_{\mathbb{R}} |\langle f(s)\xi,\eta\rangle|\,ds < \infty.
    \end{equation*}
    
    Hence for a function $f$ satisfying \eqref{necessary-condition}, we may therefore define the sesquilinear form
    \begin{equation*}
        (\xi,\eta)_f := \int_{\mathbb{R}} \langle f(s)\xi,\eta\rangle\,ds,\quad \xi,\eta \in H.
    \end{equation*}
    It then follows that:
    \begin{align*}
        |(\xi,\eta)_f|  &\leq \int_{\mathbb{R}} |\langle f(s)\xi,\eta\rangle|\,ds\\
                        &\leq \int_{\mathbb{R}} \|f(s)\|_{\infty}\|\xi\|\|\eta\|\,ds\\
                        &= \left(\int_{\mathbb{R}} \|f(s)\|_{\infty}\,ds\right)\|\xi\|\|\eta\|.
    \end{align*}
    Thus for a fixed $\xi \in H$, the mapping $\eta \mapsto (\xi,\eta)_f$ defines a bounded linear functional on $H$. Hence
    there is a unique $x_\xi \in H$ such that $(\xi,\eta)_f = \langle x_\xi,\eta\rangle $ for all $\eta \in H$.
    
    One can easily verify that the map $\xi\mapsto x_\xi$ is linear, and furthermore
    \begin{align*}
        \|x_\xi\|^2 &= \langle x_\xi,x_\xi\rangle\\
                    &= |(\xi,x_\xi)_f|\\
                    &\leq \left(\int_{\mathbb{R}} \|f(s)\|_{\infty}\,ds\right)\|\xi\|\|x_\xi\|.
    \end{align*}
    So the mapping $\xi\to x_\xi$ is bounded. Let $T$ be the unique bounded linear operator such that $x_\xi = T\xi$, we now define
    \begin{equation}\label{wot integral definition}
        \int_{\mathbb{R}} f(s)\,ds := T.
    \end{equation}
    
    Due to the above computation, we have that
    \begin{equation*}
        \left\|\int_{\mathbb{R}} f(s)\,ds\right\|_\infty\leq \int_{\mathbb{R}} \|f(s)\|_{\infty}\,ds.
    \end{equation*}
    Furthermore, we have that if $A \in \mathcal{L}_{\infty}$, and $f$ is integrable in the weak operator topology, then 
    $s\mapsto Af(s)$ is also integrable in the weak operator topology, and
    \begin{equation*}
        \int_{\mathbb{R}} Af(s)\,ds = A\int_{\mathbb{R}} f(s)\,ds.
    \end{equation*}
    
    Closely related to the weak integral is the Bochner integral: indeed, if $(\mathcal{E},\|\cdot\|_{\mathcal{E}})$ is a normed ideal in $\mathcal{L}_\infty$
    and $f:\mathbb{R}\to \mathcal{E}$ is Bochner integrable, then it is integrable in the weak operator topology and the weak integrals
    and Bochner integrals coincide, since if $f$ is weakly $\mathcal{E}$-valued measurable, then it is weak operator topology measurable, and if $\|f\|_{\mathcal{E}}$
    is integrable then $\|f\|_\infty$ is integrable.
    
\subsection{Properties of the weak integral}\label{peter subsection}

    The authors thank Professor Peter Dodds for his assistance with the arguments in this subsection.

    \begin{lem}\label{peter lemma} Let $s\to a(s),$ $s\in\mathbb{R},$ be continuous in the weak operator topology. If $a(s)\in\mathcal{L}_1$ for every $s\in\mathbb{R}$ and if
        $$\int_{\mathbb{R}}\|a(s)\|_1ds<\infty,$$
        then $a(s)$ is integrable in the weak operator topology, $\int_{\mathbb{R}}a(s)ds \in \mathcal{L}_1$ and
        $$\Big\|\int_{\mathbb{R}}a(s)ds\Big\|_1\leq\int_{\mathbb{R}}\|a(s)\|_1ds,\quad {\rm Tr}\Big(\int_{\mathbb{R}}a(s)ds\Big)=\int_{\mathbb{R}}{\rm Tr}(a(s))ds.$$
    \end{lem}
    \begin{proof}
        Since $\|a(s)\|_\infty \leq \|a(s)\|_{1}$ for all $s \in \mathbb{R}$, we have
        \begin{align*}
            \int_{\mathbb{R}} \|a(s)\|_\infty \,ds &\leq \int_{\mathbb{R}} \|a(s)\|_{1}\,ds\\
                                            &< \infty
        \end{align*}
        so that condition \eqref{necessary-condition} holds, so $s\mapsto a(s)$ is integrable in the weak operator topology. Thus, { let $A$ be the bounded linear
        operator on $H$ given by
        $$A : =\int_{\mathbb{R}}a(s)ds$$
        in the sense of \eqref{wot integral definition}. Next, we shall show that $A \in \mathcal{L}_1$.}

        Let $A=U|A|$ be a polar decomposition of $A$. 
        For an arbitrary finite rank projection $p,$ we have
        $$p|A|p=\int_{\mathbb{R}}pU^*a(s)pds$$
        Since $p$ is finite rank, the algebra $p\mathcal{L}_{\infty}p,$ is finite dimensional, and so here the weak operator topology coincides with the norm topology. 
        Hence, the mapping $s\mapsto pU^*a(s)p$ is continuous in the norm topology. Since on the algebra $p\mathcal{L}_{\infty}p$ the classical trace is a continuous functional with respect to the uniform norm, it follows that
        $${\rm Tr}(p|A|p)=\int_{\mathbb{R}}{\rm Tr}(pU^*a(s)p)ds.$$
        Thus,
        \begin{align*}
            {\rm Tr}(p|A|p) &\leq \int_{\mathbb{R}}|{\rm Tr}(pU^*a(s)p)|ds\\
                            &\leq \int_{\mathbb{R}}\|a(s)\|_1ds.
        \end{align*}
        Taking the supremum over all finite rank projections $p,$ we arrive at
        $$\|A\|_1\leq\int_{\mathbb{R}}\|a(s)\|_1ds.$$
        This proves the first assertion.

        Choose now a sequence $\{p_n\}_{n\geq1}$ of finite rank projections such that $p_n\uparrow 1.$ 
        We have
        $${\rm Tr}(p_nAp_n)=\int_{\mathbb{R}}{\rm Tr}(p_na(s)p_n)ds.$$
        Clearly, ${\rm Tr}(p_na(s)p_n)\to{\rm Tr}(a(s))$ as $n\to\infty$ for every $s\in\mathbb{R}.$ Since the function $s \mapsto \mathrm{Tr}(a(s))$ is integrable, we can apply the dominated convergence theorem to obtain
        $$\int_{\mathbb{R}}{\rm Tr}(p_na(s)p_n)ds\to\int_{\mathbb{R}}{\rm Tr}(a(s))ds.$$
        On the other hand, we have ${\rm Tr}(p_nAp_n)\to{\rm Tr}(A)$ as $n\to\infty.$ This proves the second assertion.
    \end{proof}
    
    According to the preceding lemma, if $a:\mathbb{R}\to \mathcal{L}_1$ is continuous and $\int_{\mathbb{R}} \|a(s)\|_1 \,ds < \infty$, then we
    have that $\int_{\mathbb{R}} a(s)\,ds \in \mathcal{L}_1$. The following lemma shows that the same implication holds when $\mathcal{L}_1$ is replaced
    by $\mathcal{L}_{r,\infty}$ for any $r > 1$.
    
    \begin{prop}\label{peter norm lemma} 
        Let $s\to a(s),$ $s\in\mathbb{R},$ be continuous in the weak operator topology. Fix $r > 1$, and suppose that for all $s$ we have $a(s)\in\mathcal{L}_{r,\infty}$.
        If $\int_{\mathbb{R}} \|a(s)\|_{r,\infty}\,ds < \infty$ then $\int_{\mathbb{R}} a(s)\,ds \in \mathcal{L}_{r,\infty}$, where the integral is understood in a weak sense, and we have a bound:
        \begin{equation*}
            \left\|\int_{\mathbb{R}} a(s)\,ds\right\|_{r,\infty} \leq \frac{r}{r-1}\int_{\mathbb{R}} \|a(s)\|_{r,\infty}\,ds.
        \end{equation*}
    \end{prop}
    \begin{proof}
        Similar to the $\mathcal{L}_{1}$ case, since $\|a(s)\|_{\infty} \leq \|a(s)\|_{r,\infty}$, we have that $\int_{\mathbb{R}} \|a(s)\|_{\infty}\,ds < \infty$,
        and so condition \eqref{necessary-condition} holds. Hence, $s\mapsto a(s)$ is integrable in the weak operator topology. Let $A := \int_{\mathbb{R}} a(s)\,ds$ { in the sense of \eqref{wot integral definition}}. Let $A = U|A|$ be a polar decomposition of $A$, and let $p$ be a rank $n$ projection, $n\geq 1$. Then,
        \begin{equation*}
            p|A|p = \int_{\mathbb{R}} pU^*a(s)p\,ds.
        \end{equation*}
        Thus,
        \begin{align*}
            \mathrm{Tr}(p|A|p) &\leq \int_{\mathbb{R}} |\mathrm{Tr}(pU^*a(s)p)|\,ds\\
                       &\leq \int_{\mathbb{R}} \|pU^*a(s)p\|_1\,ds.
        \end{align*}
        The latter integral converges because $\|pU^*a(s)p\|_1 \leq n\|a(s)\|_\infty$. Now,
        \begin{align*}
            \|pU^*a(s)p\|_1 &\leq \sum_{k=0}^{n-1} \mu(k,a(s))\\
                            &\leq \|a(s)\|_{r,\infty} \sum_{k=0}^{n-1} (k+1)^{-1/r}.
        \end{align*}
        The right hand side depends only on $n$ and not $p$, so we may take the supremum
        over all projections of rank $n$ to obtain:
        \begin{equation*}
            \sum_{k=0}^{n-1} \mu(k,A) \leq \int_{\mathbb{R}} \|a(s)\|_{r,\infty}\,ds\cdot \sum_{k=0}^{n-1} (k+1)^{-1/r}.
        \end{equation*}
        We can bound the latter sum as:
        \begin{align*}
            \sum_{k=0}^{n-1} (k+1)^{-1/r} &\leq 1+\int_1^n t^{-1/r}\,dt\\
                                          &\leq \frac{r}{r-1} n^{1-\frac{1}{r}}.
        \end{align*}
        Therefore:
        \begin{equation*}
            \sum_{k=0}^{n-1} \mu(k,A) \leq \int_{\mathbb{R}} \|a(s)\|_{r,\infty}\,ds \frac{r}{r-1}n^{1-\frac{1}{r}}.
        \end{equation*}
        Hence,
        \begin{equation*}
            n\mu(n-1,A) \leq \frac{r}{r-1}n^{1-\frac{1}{r}}\int_{\mathbb{R}} \|a(s)\|_{r,\infty}\,ds.
        \end{equation*}
        Multiplying through by $n^{\frac{1}{r}-1}$, and taking the supremum over $n$, it follows that
        \begin{equation*}
            \sup_{n\geq 1} n^{\frac{1}{r}}\mu(n-1,A) \leq \frac{r}{r-1}\int_{\mathbb{R}} \|a(s)\|_{r,\infty}\,ds.
        \end{equation*}
        So by the definition of the quasinorm on $\mathcal{L}_{r,\infty}$, the assertion follows.
\end{proof}


\section{Double operator integrals}\label{doi} 
    Here, we state the definition and basic properties of double operator integrals. This theory was initiated by the work
    of Birman and Solomyak \cite{Birman-Solomyak-I,Birman-Solomyak-II,Birman-Solomyak-III}, and more recent summaries
    of the theory may be found in \cite{Birman-Solomyak-2003, Peller-doi-2016}.
    
%     We refer the reader to \cite{PS-crelle} {  [additional citation please: Birman-Solomyak, etc]} for the proofs and for more advanced properties.

    Heuristically, given self-adjoint operators $X$ and $Y$ with spectra $\sigma(X)$ and $\sigma(Y)$, spectral resolutions $E_X$ and $E_Y$ and a bounded measurable function $\phi$ on $\sigma(X)\times \sigma(Y)$, the double operator integral $T^{X,Y}_{\phi}$ 
    applied to an operator $A \in \mathcal{L}_{\infty}$ is given by the formula:
    $$T_{\phi}^{X,Y}(A)=\iint_{\sigma(X)\times \sigma(Y)} \phi(\lambda,\mu)dE_X(\lambda)AdE_Y(\mu).$$
    The formal expression for $T_{\phi}^{X,Y}$ is well defined as a bounded operator on the Hilbert-Schmidt class $\mathcal{L}_2$. The theory of double operator integrals is primarily concerned
    with defining $T_\phi^{X,Y}$ on other ideals.
    This is not possible for arbitrary bounded measurable functions $\phi$, so we must restrict attention to the following class of ``good" functions. 
    
    That is, we assume that $\phi$ admits a representation
    \begin{equation}\label{integral tensor product}
        \phi(\lambda,\mu) = \int_{\Omega} a(\lambda,s)b(\mu,s)\,d\kappa(s),\quad \lambda \in \sigma(X), \mu \in \sigma(Y)
    \end{equation}
    where $(\Omega,\kappa)$ is a measure space, and where
    \begin{equation}\label{doi sufficient condition}
        \int_{\Omega} \sup_{\lambda \in \sigma(X)}|a(\lambda,s)|\sup_{\mu \in \sigma(Y)} |b(\mu,s)|\,d\kappa(s) < \infty.
    \end{equation}
    For such functions $\phi$, we may define
    \begin{equation}\label{doi definition}
        T_{\phi}^{X,Y}(A) := \int_{\Omega} a(X,s)Ab(Y,s)\,d\kappa(s)
    \end{equation}
    where the operators $a(X,s)$ and $b(Y,s)$ are defined by Borel functional calculus, and the integral can be understood in the weak operator topology.
    
    The following is proved in \cite[Theorem 4]{PS-crelle}:
    \begin{thm}
        If $\phi$ admits a decomposition as in \eqref{integral tensor product}, then the operator $T_{\phi}^{X,Y}$ is a bounded linear map from:
        \begin{enumerate}[{\rm (a)}]
            \item{} $\mathcal{L}_{\infty}$ to $\mathcal{L}_{\infty}$;
            \item{} $\mathcal{L}_{1}$ to $\mathcal{L}_{1}$;
            \item{} $\mathcal{L}_{r}$ to $\mathcal{L}_r$, for all $r \in (1,\infty)$;
            \item{} $\mathcal{L}_{r,\infty}$ to $\mathcal{L}_{r,\infty}$ for all $r \in (1,\infty)$.
        \end{enumerate}
    \end{thm}

    One of the key properties of double operator integrals is that they respect algebraic operations (see e.g.\cite[Proposition 2.8]{PSW}). Namely,
    \begin{equation}\label{doi algebraic}
        T_{\phi_1+\phi_2}^{X,Y}=T_{\phi_1}^{X,Y}+T_{\phi_2}^{X,Y},\quad T_{\phi_1\cdot \phi_2}^{X,Y}=T_{\phi_1}^{X,Y}\circ T_{\phi_2}^{X,Y}.
    \end{equation}

    If, in \eqref{integral tensor product} we take $\Omega$ to be a one-point set, then $\phi(\lambda,\mu) = a(\lambda)b(\mu)$ and
    \begin{equation}\label{separated variables in doi}
        T_{\phi}^{X,Y}(A)=a(X)Ab(Y).
    \end{equation}

\section{Fourier transform conventions}

    We follow the convention that the Fourier transform of a function $g \in L_1(\mathbb{R})$ is defined by the formula
    \begin{equation*}
        \mathcal{F}(g)(t) := (2\pi)^{-1/2}\int_{\mathbb{R}} g(s)e^{-its}\,ds
    \end{equation*}
    So that the inverse Fourier transform is given for $h \in L_1(\mathbb{R})$ by,
    \begin{equation*}
        \mathcal{F}^{-1}(h)(s) := (2\pi)^{-1/2}\int_{\mathbb{R}} h(t)e^{its}\,dt
    \end{equation*}
    and so that $\mathcal{F}$ extends to a unitary operator on $L_2(\mathbb{R})$.
    
    We often make use of the fact that if $g \in L_1(\mathbb{R})$ satisfies $g,g' \in L_2(\mathbb{R})$ then $\mathcal{F}g \in L_1(\mathbb{R})$ \cite[Lemma 7]{PS-crelle}.
    Here, the derivative $g'$ may be defined in a distributional sense. 
    
    Everywhere in the text, the symbol $\hat{g}$ denotes $(2\pi)^{-1/2}\mathcal{F}(g)$. This allows us to write for $g \in L_1(\mathbb{R})$ with $\mathcal{F}(g) \in L_1(\mathbb{R})$:
    \begin{equation*}
        g(t) = \int_{\mathbb{R}} \hat{g}(s)e^{its}\,ds.
    \end{equation*}
    We caution the reader that $\hat{g}$ does not denote the Fourier transform, but its rescaling by $(2\pi)^{-1/2}$.

    
