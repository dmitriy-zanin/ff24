\chapter{Asymptotic of the heat trace}\label{heat chapter}
    
    In this chapter we complete the proof of Theorem \ref{heat thm}. This will require some delicate computations 
    exploiting Hochschild homology.
    
    For the remainder of this chapter, we assume that $(\mathcal{A},H,D)$ is a spectral triple satisfying Hypothesis \ref{main assumption}.
    Furthermore we will need the following auxiliary assumption:
    \begin{hyp}\label{auxiliary assumption}
        The spectral triple $(\mathcal{A},H,D)$ satisfies the following:
        \begin{enumerate}[{\rm (i)}]
            \item{}\label{aux ass0} $(\mathcal{A},H,D)$ has the same parity as $p$: this means that $(\mathcal{A},H,D)$ is even when $p$ is even (with grading $\Gamma$) and odd when $p$ is odd.
            \item{}\label{aux ass1} $D$ has a spectral gap at $0$.
        \end{enumerate}
    \end{hyp}
    We will show at the end of this chapter how Hypothesis \ref{auxiliary assumption} can be removed.
    
    Recall from Definition \ref{ch omega def} the fundamental mappings $\mathrm{ch}$ and $\Omega$, given by:
    \begin{align*}
        \mathrm{ch}(a_0\otimes a_1\otimes\cdots\otimes a_p) &:= \Gamma F[F,a_0][F,a_1]\cdots[F,a_p],\\
        \Omega(a_0\otimes a_1\otimes\cdots\otimes a_p) &:= \Gamma a_0\partial(a_1)\partial(a_2)\cdots \partial(a_p)
    \end{align*}
    and that Theorem \ref{heat thm} states that for all Hochschild cycles $c \in \mathcal{A}^{\otimes(p+1)}$,
    \begin{equation*}
        \mathrm{Tr}(\Omega(c)(1+D^2)^{1-\frac{p}{2}}e^{-s^2D^2}) = \frac{p}{2}\mathrm{Tr}(\mathrm{ch}(c))s^{-2}+O(s^{-1}),\quad s\to 0.
    \end{equation*}
    
    The computations in Sections \ref{combinatorial section} and \ref{commutator section} are {inspired by those in \cite{CPRS1,CRSZ}. Since
    we do not assume that the algebra $\mathcal{A}$ is unital, the computations are more delicate than those of \cite{CPRS1,CRSZ}.}
%     albeit more delicate since we do not assume that the algebra
%     $\mathcal{A}$ is unital.

\section[Combinatorial expression]{Combinatorial expression for $\mathrm{Tr}(\Omega(c)|D|^{2-p}e^{-s^2D^2})$}\label{combinatorial section}
    We begin this section with the introduction of a new set of multilinear maps:
    \begin{defi}\label{W definition}
        Let $\mathscr{A} \subseteq \{1,2,\ldots,p\}$. We define the multilinear map $\mathcal{W}_\mathscr{A}:\mathcal{A}^{\otimes(p+1)}\to \mathcal{L}_{\infty}$ by:
        \begin{equation*}
            \mathcal{W}_{\mathscr{A}}(a_0\otimes a_1\otimes\cdots\otimes a_p) = \Gamma a_0\prod_{k=1}^p b_k(a_k)
        \end{equation*}
        where for $a \in \mathcal{A}$ and each $k$ we define:
        \begin{equation*}
            b_k(a) = \begin{cases}
                            \delta(a),\,k \in \mathscr{A},\\
                            [F,a],\,k \notin \mathscr{A}.
                       \end{cases}
        \end{equation*} 
        In the case where $\mathscr{A} = \{m\}$, a single number, $1 \leq m \leq p$, then write $\mathcal{W}_m$ for $\mathcal{W}_{\{m\}}$.
    \end{defi}
    Since by assumption $(\mathcal{A},H,D)$ is smooth, the operators $\delta(a_k)$ are defined and bounded, so that $\mathcal{W}_{\mathscr{A}}$ is well defined as a bounded operator.
    
    The two extreme cases, $\mathcal{W}_{\emptyset}$ and $\mathcal{W}_{\{1,2,\ldots,p\}}$ are easily described as:
    \begin{equation*}
        \mathcal{W}_{\emptyset}(a_0\otimes a_1\otimes\cdots\otimes a_p) = \Gamma a_0\prod_{k=1}^p [F,a_k]
    \end{equation*}
    It will be important to observe that if $(\mathcal{A},H,D)$ has the same parity as $p$, then $\mathrm{ch}(c) = \mathcal{W}_{\emptyset}(c)+F\mathcal{W}_{\emptyset}(c)F$ (this is Lemma \ref{W and ch link}).
    
    On the other extreme.
    \begin{equation*}
        \mathcal{W}_{\{1,2,\ldots,p\}}(a_0\otimes a_1\otimes\cdots\otimes a_p) = \Gamma a_0\prod_{k=1}^p\delta(a_k).
    \end{equation*}
        
    Associated to a subset $\mathscr{A} \subseteq \{1,2,\ldots,p\}$ we have the number,
    \begin{equation*}
        n_\mathscr{A} = |\{(j,k) \in \{1,2,\ldots,p\}^2\;:\; j < k\text{ and }j \in \mathscr{A}, k\neq \mathscr{A}\}|,
    \end{equation*}
    where $|\cdot|$ denotes the cardinality of a set.

    Theorem \ref{combinatorial theorem} (to be stated below) is the main result of this section. Roughly speaking, it shows that one can replace $\Omega(c)$ in $\mathrm{Tr}(\Omega(c)|D|^{2-p}e^{-s^2D^2})$ with a sum of $\mathcal{W}_\mathscr{A}(c)$
    over all subsets $\mathscr{A} \subseteq \{1,\ldots,p\}$.
    However first we need a Lemma which constitutes the core of the proof of Theorem \ref{combinatorial theorem}. Most of this section is devoted to the proof of the following lemma, which is split into various parts.
    \begin{lem}\label{combinatorial lemma} 
        Let $(\mathcal{A},H,D)$ be a smooth spectral triple where $D$ has a spectral gap at $0.$ For all $c\in(\mathcal{A}+\mathbb{C})^{\otimes (p+1)},$ the operator
        \begin{equation*}
            \left(\Omega(c)|D|^{2-p}-\sum_{\mathscr{A}\subseteq \{1,\cdots,p\}}(-1)^{n_{\mathscr{A}}}\mathcal{W}_{\mathscr{A}}(c)D^{2-|\mathscr{A}|}\right)\cdot|D|^{p-1}.
        \end{equation*}
        has bounded extension.
    \end{lem}
        
    Before proving Lemma \ref{combinatorial lemma} above we initially need:

    \begin{lem}\label{w bounded lemma} 
        Let $(\mathcal{A},H,D)$ be a smooth spectral triple, where $D$ has a spectral gap at $0$. For all $c\in(\mathcal{A}+\mathbb{C})^{\otimes (p+1)},$ the operator
        $$\mathcal{W}_{\mathscr{A}}(c)\cdot |D|^{p-|\mathscr{A}|}$$
        has bounded extension.
    \end{lem}
    \begin{proof} 
        By linearity it suffices to prove the assertion for elementary tensors. Let $c = a_0\otimes a_1\otimes\cdots\otimes a_p\in(\mathcal{A}+\mathbb{C})^{\otimes (p+1)}$ 
        and let $c'=a_0\otimes a_1\otimes\cdots\otimes a_{p-1}\in(\mathcal{A}+\mathbb{C})^{\otimes p}.$

        We will prove this assertion by induction on $p.$ If $p=1$ and $\mathscr{A}=\emptyset,$ then
        $$\mathcal{W}_{\mathscr{A}}(c)|D|^{p-|\mathscr{A}|}=\Gamma a_0[F,a_p]|D|$$
        and recall that $[F,a_1]|D|$ has a bounded extension $L(a_p)$.
        On the other hand, if $p=1$ and $\mathscr{A}=\{1\},$ then
        $$\mathcal{W}_{\mathscr{A}}(c)|D|^{p-|\mathscr{A}|}=\Gamma a_0\delta(a_1)\in\mathcal{L}_{\infty}.$$
        This proves the base of induction. 
        
        Next, let $p \geq 2$ and assume that the statement is true for $p-1$. To this end, 
        let $\mathscr{B}\subseteq\{1,\cdots,p-1\}$ be defined by the formula $\mathscr{B}=\mathscr{A}\setminus\{p\}.$ There are two distinct cases: when $p \in \mathscr{A}$ and $p \notin \mathscr{A}$.

        First suppose $p\in\mathscr{A}.$ Here we have $|\mathscr{B}| = |\mathscr{A}|-1$, and
        \begin{align*}
            \mathcal{W}_{\mathscr{A}}(c)|D|^{p-|\mathscr{A}|} &= \mathcal{W}_{\mathscr{B}}(c')\delta(a_p)|D|^{p-1-|\mathscr{B}|}\\
                                      &= \left(\mathcal{W}_{\mathscr{B}}(c')|D|^{p-1-|\mathscr{B}|}\right)\left(|D|^{-p+1+|\mathscr{B}|}\delta(a_p)|D|^{p-1-|\mathscr{B}|}\right)
        \end{align*}
        The first factor in the right hand side has bounded extension by the inductive assumption. Moreover, the second factor on the right hand side has bounded extension by Lemma \ref{left to right corollary}. This proves step of induction for the case $p\in\mathscr{A}.$
        
        Now we deal with the case where $p\notin \mathscr{A}$. Then $\mathscr{B} = \mathscr{A}$, and
        \begin{align*}
            \mathcal{W}_{\mathscr{A}}(c)|D|^{p-|\mathscr{A}|} &= \mathcal{W}_{\mathscr{B}}(c')L(a_p)|D|^{p-1-|\mathscr{B}|}\\
                                      &= \left(\mathcal{W}_{\mathscr{B}}(c')|D|^{p-1-|\mathscr{B}|}\right)\left(|D|^{-p+1+|\mathscr{B}|}L(a_p)|D|^{p-1-|\mathscr{B}|}\right)
        \end{align*}
        The first factor in the above right hand side is bounded by the inductive assumption. For the second factor, we use the expression $L(a_p) = \partial(a_p)-F\delta(a_p)$.
        Since $\partial(a_p), \delta(a_p) \in \mathcal{B}$ then it follows from Lemma \ref{left to right corollary} that the second factor on the right hand side has bounded extension. Hence, $\mathcal{W}_{\mathscr{A}}(c)|D|^{p-|\mathscr{A}|}$
        extends to a bounded linear operator. 
        
    \end{proof}

    In order to prove Lemma \ref{combinatorial lemma}, we introduce two more classes of multilinear functionals.
    \begin{defi}
        Let $\mathscr{A} \subseteq \{1,2,\ldots,p\}$. We define the multilinear map $\mathcal{R}_{\mathscr{A}}:\mathcal{A}^{\otimes(p+1)}\to \mathcal{L}_{\infty}$ by
        \begin{equation*}
            \mathcal{R}_{\mathscr{A}}(a_0\otimes\cdots\otimes a_p) := \Gamma a_0\prod_{k=1}^px_k(a_k),    
        \end{equation*}
        where for each $1\leq k \leq p$ and $a \in \mathcal{A}$,
        \begin{equation*}
            x_k(a) := \begin{cases}
                        F\delta(a),\,k \in \mathscr{A},\\
                        L(a), k \notin \mathscr{A}.
                   \end{cases}
        \end{equation*}
        
        We also define the multilinear map $\mathcal{P}_{\mathscr{A}}:\mathcal{A}^{\otimes(p+1)}\to \mathcal{L}_{\infty}$ by
        \begin{equation*}
            \mathcal{P}_{\mathscr{A}}(a_0\otimes \cdots\otimes a_p) := \Gamma a_0\prod_{k=1}^p y_k(a_k)
        \end{equation*}
        where for each $1\leq k \leq p$ and $a \in \mathcal{A}$,
        \begin{equation*}
            y_k(a) := \begin{cases}
                        \delta(a),\,k \in \mathscr{A},\\
                        L(a),\, k\notin \mathscr{A}.
                   \end{cases}
        \end{equation*}
    \end{defi}


    \begin{lem}\label{first combinatorial lemma} 
        Let $(\mathcal{A},H,D)$ be a smooth spectral triple, where $D$ has a spectral gap at $0$.
        For all $c \in(\mathcal{A}+\mathbb{C})^{\otimes(p+1)}$ and all $\mathscr{A} \subseteq \{1,2,\ldots,p\}$, the operator
        \begin{equation*}
            \left(\mathcal{R}_{\mathscr{A}}(c)-(-1)^{n_{\mathscr{A}}}\mathcal{P}_{\mathscr{A}}(c)\cdot F^{|\mathscr{A}|}\right)\cdot |D|
        \end{equation*}
        has bounded extension.
    \end{lem}
    \begin{proof} 
        This proof is similar to that of \ref{w bounded lemma}. Once again it suffices to prove the result for an elementary tensor $c = a_0\otimes a_1\otimes\cdots \otimes a_p$,
        and we prove the statement by induction on $p$. Denote, for brevity, $c'=a_0\otimes a_1\otimes\cdots\otimes a_{p-1}\in(\mathcal{A}+\mathbb{C})^{\otimes p}.$

        For the base of induction, when $p = 1$, we deal with the two possibilities $\mathscr{A} = \{1\}$ and $\mathscr{A} = \emptyset$. 
        If $\mathscr{A} = \{1\}$, then
        \begin{align*}
            \mathcal{R}_{\mathscr{A}}(c) &= \Gamma a_0F\delta(a_1)\\
            \mathcal{P}_{\mathscr{A}}(c)F^{|\mathscr{A}|} &= \Gamma a_0\delta(a_1)F.
        \end{align*}
        So,
        \begin{align*}
            \left(\mathcal{R}_{\mathscr{A}}(c)-(-1)^{n_{\mathscr{A}}}\mathcal{P}_{\mathscr{A}}(c)\cdot F^{|\mathscr{A}|}\right)\cdot |D| &= \Gamma a_0(F\delta(a_1)-\delta(a_1)F)|D|\\
                                                                                         &= \Gamma a_0[F,\delta(a_1)]|D|\\
                                                                                         &= \Gamma a_0L(\delta(a_1)).
        \end{align*}
        since $L(\delta(a_1))$ has bounded extension, so does the above left hand side.
        
        Now in the case that $p = 1$ and $\mathscr{A} = \emptyset$,
        \begin{align*}
            \mathcal{R}_{\mathscr{A}}(c) &= \Gamma a_0 L(a_1)\\
            \mathcal{P}_{\mathscr{A}}(c) &= \Gamma a_0 L(a_1)
        \end{align*}
        and $|\mathscr{A}| = 0$, so $\mathcal{R}_{\mathscr{A}}(c) - \mathcal{P}_{\mathscr{A}}(c)F^{|\mathscr{A}|} = 0$. This proves the $p = 1$ case.
        
        Now suppose that $p > 1$ and the assertion is proved for $p-1$.
        For this purpose, let $\mathscr{B}\subset\{1,\cdots,p-1\}$ be defined by 
        $\mathscr{B}=\mathscr{A}\backslash\{p\}.$ 
        
        If $p \in \mathscr{A}$, then $n_{\mathscr{A}} = n_{\mathscr{B}}$, and if $p \notin \mathscr{A}$ then $n_{\mathscr{A}} = n_{\mathscr{B}} + |\mathscr{B}|$.
        
%         Note that $n_{\mathscr{A}}=n_{\mathscr{B}}$ for $p\in\mathscr{A}$ and $n_{\mathscr{A}} = n_{\mathscr{B}}+|\mathscr{B}|$ 
%         for $p\notin\mathscr{A}.$
        
        Now we consider separately the cases $p \in \mathscr{A}$ and $p\notin \mathscr{A}$.
        
        First, if $p \in \mathscr{A}$, then,
        \begin{align*}
            \mathcal{R}_{\mathscr{A}}(c) &= \mathcal{R}_{\mathscr{B}}(c')F\delta(a_p),\\
            \mathcal{P}_{\mathscr{A}}(c) &= \mathcal{P}_{\mathscr{B}}(c')\delta(a_p).
        \end{align*}   
        Hence, since $|\mathscr{A}| = |\mathscr{B}|+1$ and $n_{\mathscr{A}} = n_{\mathscr{B}}$ in this case:
        $$(\mathcal{R}_{\mathscr{A}}(c) - (-1)^{n_{\mathscr{A}}}\mathcal{P}_{\mathscr{A}}(c)F^{|\mathscr{A}|})|D| =(\mathcal{R}_{\mathscr{B}}(c')F\delta(a_p)-(-1)^{n_{\mathscr{A}}}\mathcal{P}_{\mathscr{B}}(c')\delta(a_p)F^{|\mathscr{B}|+1})|D|=$$
        $$= \mathcal{R}_{\mathscr{B}}(c')[F,\delta(a_p)]|D|+(\mathcal{R}_{\mathscr{B}}(c')\delta(a_p)F-(-1)^{n_{\mathscr{B}}}\mathcal{P}_{\mathscr{B}}(c')\delta(a_p)F\cdot F^{|\mathscr{B}|})|D|.$$
        $$= \mathcal{R}_{\mathscr{B}}(c')L(a_p)+(\mathcal{R}_{\mathscr{B}}(c')\delta(a_p)-(-1)^{n_{\mathscr{B}}}\mathcal{P}_{\mathscr{B}}(c')\delta(a_p)\cdot F^{|\mathscr{B}|})D.$$

        In the case that $|\mathscr{B}|$ is even, we have $F^{|\mathscr{B}|} = 1$ and so:
        $$(\mathcal{R}_{\mathscr{A}}(c) - (-1)^{n_{\mathscr{A}}}\mathcal{P}_{\mathscr{A}}(c)F^{|\mathscr{A}|})|D|=$$
        $$=\mathcal{R}_{\mathscr{B}}(c')L(a_p)+(\mathcal{R}_{\mathscr{B}}(c')-(-1)^{n_{\mathscr{B}}}\mathcal{P}_{\mathscr{B}}(c')\cdot F^{|\mathscr{B}|})\cdot \delta(a_p)D.$$
        $$=\mathcal{R}_{\mathscr{B}}(c')L(a_p)+(\mathcal{R}_{\mathscr{B}}(c')-(-1)^{n_{\mathscr{B}}}\mathcal{P}_{\mathscr{B}}(c')\cdot F^{|\mathscr{B}|})D\cdot \delta(a_p)-$$
        $$-(\mathcal{R}_{\mathscr{B}}(c')-(-1)^{n_{\mathscr{B}}}\mathcal{P}_{\mathscr{B}}(c')\cdot F^{|\mathscr{B}|})\cdot\partial(\delta(a_p)).$$
        Since $\delta(a_p),$ $\partial(\delta(a_p))$ and $L(a_p)$ are bounded, by the inductive hypothesis the above has bounded extension, completing the proof of the case where $p \in \mathscr{A}$ and $|\mathscr{B}|$ is even.
        
        On the other hand, if $p \in \mathscr{A}$ and $|\mathscr{B}|$ is odd, then:
        $$(\mathcal{R}_{\mathscr{A}}(c) - (-1)^{n_{\mathscr{A}}}\mathcal{P}_{\mathscr{A}}(c)F^{|\mathscr{A}|})|D|=$$
        $$= \mathcal{R}_{\mathscr{B}}(c')L(a_p)+\mathcal{R}_{\mathscr{B}}(c')\delta(a_p)D-(-1)^{n_{\mathscr{B}}}\mathcal{P}_{\mathscr{B}}(c')\delta(a_p)|D|=$$
        $$= \mathcal{R}_{\mathscr{B}}(c')L(a_p)+(\mathcal{R}_{\mathscr{B}}(c')-(-1)^{n_{\mathscr{B}}}\mathcal{P}_{\mathscr{B}}(c')F^{|\mathscr{B}|})D\cdot\delta(a_p)-$$
        $$-\mathcal{R}_{\mathscr{B}}(c')\delta(a_p)D+(-1)^{n_{\mathscr{B}}}\mathcal{P}_{\mathscr{B}}(c')\delta^2(a_p).$$
        Since $\delta(a_p),$ $\delta^2(a_p)$ and $L(a_p)$ are bounded, by the inductive hypothesis the above has bounded extension, completing the proof of the case where $p \in \mathscr{A}$ and $|\mathscr{B}|$ is odd.
        
        Now assume that $p\notin \mathscr{A}$. Then,
        \begin{align*}
            \mathcal{R}_{\mathscr{A}}(c) &= \mathcal{R}_{\mathscr{B}}(c')L(a_p)\\
            \mathcal{P}_{\mathscr{A}}(c) &= \mathcal{P}_{\mathscr{B}}(c')L(a_p).
        \end{align*}
        Focusing on $\mathcal{P}_{\mathscr{A}}(c)$:
        \begin{align*}
            \mathcal{P}_{\mathscr{A}}(c)F^{|\mathscr{A}|} &= \mathcal{P}_{\mathscr{B}}(c')L(a_p)F^{|\mathscr{B}|}.
        \end{align*}
        So:
        \begin{equation*}
             (\mathcal{R}_{\mathscr{A}}(c)-(-1)^{n_{\mathscr{A}}}\mathcal{P}_{\mathscr{A}}(c)F^{|\mathscr{A}|})|D| = (\mathcal{R}_{\mathscr{B}}(c')L(a_p)-(-1)^{n_{\mathscr{B}}+|\mathscr{B}|}\mathcal{P}_{\mathscr{B}}(c')L(a_p)F^{|\mathscr{B}|})|D|.
        \end{equation*}
        
        Note that since $F$ anticommutes with $[F,a_p]$, $F$ also anticommutes with $L(a_p)$. Hence,
        \begin{equation*}
            L(a_p)F^{|\mathscr{B}|} = (-1)^{|\mathscr{B}|}F^{|\mathscr{B}|}L(a_p)
        \end{equation*}
        and so
        \begin{align*}
            (\mathcal{R}_{\mathscr{A}}(c)-(-1)^{n_{\mathscr{A}}}\mathcal{P}_{\mathscr{A}}(c)F^{|\mathscr{A}|})|D| &= (\mathcal{R}_{\mathscr{B}}(c')-(-1)^{n_{\mathscr{B}}}\mathcal{P}_{\mathscr{B}}(c')F^{|\mathscr{B}|})L(a_p)|D|\\
                                                                  &= (\mathcal{R}_{\mathscr{B}}(c')-(-1)^{n_{\mathscr{B}}}\mathcal{P}_{\mathscr{B}}(c')F^{|\mathscr{B}|})(|D|L(a_p)+L(\delta(a_p))).
        \end{align*}
        By the inductive assumption, the operator $(\mathcal{R}_{\mathscr{B}}(c')-(-1)^{n_{\mathscr{B}}}\mathcal{P}_{\mathscr{B}}(c')F^{|\mathscr{B}|})|D|$ has bounded extension. This completes the proof of the $p \notin \mathscr{A}$ case.
        
        Hence, the statement is true for $p$ and this completes the induction.
    \end{proof}

    \begin{lem}\label{second combinatorial lemma} 
        Let $(\mathcal{A},H,D)$ be a smooth spectral triple. Suppose $D$ has a spectral gap at $0.$ For all $c\in(\mathcal{A}+\mathbb{C})^{\otimes (p+1)}$ and for all $\mathscr{A}\subseteq \{1,\ldots,p\}$ the operator
        \begin{equation*}
            \left(\mathcal{P}_{\mathscr{A}}(c)-\mathcal{W}_{\mathscr{A}}(c)\cdot |D|^{p-|\mathscr{A}|}\right)\cdot|D|.
        \end{equation*}
        has bounded extension.
    \end{lem}
    \begin{proof} 
        This proof is again similar to the proofs of Lemmas \ref{w bounded lemma} and \ref{first combinatorial lemma}.
        
        Once more, it suffices to prove the assertion for an elementary tensor $c = a_0\otimes a_1\otimes \cdots \otimes a_p \in (\mathcal{A}+\mathbb{C})^{\otimes(p+1)}$. Let $c' := a_0\otimes\cdots\otimes a_{p-1} \in (\mathcal{A}+\mathbb{C})^{\otimes p}$.
        The proof proceeds by induction on $p$.
        
        First, if $p = 1$, then either $\mathscr{A} = \{1\}$ or $\mathscr{A} = \emptyset$. If $\mathscr{A} = \{1\}$, then
        \begin{align*}
                         \mathcal{P}_{\mathscr{A}}(c) &= \Gamma a_0\delta(a_1),\\
            \mathcal{W}_{\mathscr{A}}(c)|D|^{p-|\mathscr{A}|} &= \Gamma a_0\delta(a_1).
        \end{align*}
        and if $\mathscr{A} = \emptyset$, then
        \begin{align*}
                         \mathcal{P}_{\mathscr{A}}(c) &= \Gamma a_0L(a_1),\\
            \mathcal{W}_{\mathscr{A}}(c)|D|^{p-|\mathscr{A}|} &= \Gamma a_0[F,a_1]|D|.
        \end{align*}
        Since $L(a_1) = [F,a_1]|D|$ on the dense subspace $H_\infty$, it follows that in all cases with $p = 1$ the difference $\mathcal{P}_{\mathscr{A}}(c)-\mathcal{W}_{\mathscr{A}}(c)|D|^{p-|\mathscr{A}|}$ is identically
        zero on $H_\infty$ and therefore has trivial bounded extension to $H$. This establishes the $p=1$ case.
        
        Now suppose that $p > 1$ and the assertion has been proved for $p-1$. Let $\mathscr{B} = \mathscr{A} \setminus \{p\}$, and we consider
        the two cases of $p \in \mathscr{A}$ and $p \notin \mathscr{A}$. 
        
        Suppose that $p \in \mathscr{A}$. Then,
        \begin{align*}
                         \mathcal{P}_{\mathscr{A}}(c) &= \mathcal{P}_{\mathscr{B}}(c')\delta(a_p),\\
            \mathcal{W}_{\mathscr{A}}(c)|D|^{p-|\mathscr{A}|} &= \mathcal{W}_{\mathscr{B}}(c')\delta(a_p)|D|^{p-1-|\mathscr{B}|}.
        \end{align*}
        
        So,
        \begin{align}\label{simplified pc expression} 
            \mathcal{P}_{\mathscr{A}}(c)|D| &= \mathcal{P}_{\mathscr{B}}(c')\delta(a_p)|D|\nonumber\\
                            &= \mathcal{P}_{\mathscr{B}}(c')|D|-\mathcal{P}_{\mathscr{B}}(c')\delta^2(a_p).
        \end{align}
        and
        \begin{align}\label{simplified wc expression}
            \mathcal{W}_{\mathscr{A}}(c)|D|^{p-|\mathscr{A}|+1} &= \mathcal{W}_{\mathscr{B}}(c')\delta(a_p)|D|^{p-|\mathscr{B}|}\nonumber\\
                                        &= \mathcal{W}_{\mathscr{B}}(c')|D|^{p-|\mathscr{B}|}\delta(a_p)+\mathcal{W}_{\mathscr{B}}(c')[\delta(a_p),|D|^{p-|\mathscr{B}|}]\nonumber\\
                                        &= \mathcal{W}_{\mathscr{B}}(c')|D|^{p-|\mathscr{B}|}\delta(a_p)-\mathcal{W}_{\mathscr{B}}(c')|D|^{p-|\mathscr{B}|-1}|D|^{|\mathscr{B}|-p+1}[|D|^{p-|\mathscr{B}|},\delta(a_p)].
        \end{align}
        Applying Lemma \ref{left to right corollary}, the operator $|D|^{|\mathscr{B}|-p+1}[|D|^{p-|\mathscr{B}|},\delta(a_p)]$ has bounded extension. 
        So combining \eqref{simplified pc expression} and \eqref{simplified wc expression}:
        \begin{align*}
            (\mathcal{P}_{\mathscr{A}}(c)-\mathcal{W}_{\mathscr{A}}(c)|D|^{p-|\mathscr{A}|})|D| - (\mathcal{P}_{\mathscr{B}}(c')-\mathcal{W}_{\mathscr{B}}(c')|D|^{p-1-|\mathscr{B}|})|D|
        \end{align*}
        has bounded extension. So by the inductive hypothesis, $(\mathcal{P}_{\mathscr{A}}(c)-\mathcal{W}_{\mathscr{A}}(c)|D|^{p-|\mathscr{A}|})|D|$ has bounded extension in the case that $p \in \mathscr{A}$.
        
        Now suppose that $p\notin \mathscr{A}$. In this case, we have:
        \begin{align*}
            \mathcal{P}_{\mathscr{A}}(c) &= \mathcal{P}_{\mathscr{B}}(c')L(a_p)\\
            \mathcal{W}_{\mathscr{A}}(c) &= \mathcal{W}_{\mathscr{B}}(c')[F,a_p].
        \end{align*}
        
        Multiplying by $|D|$, we have
        \begin{equation}\label{simplified pc expression 2}
            \mathcal{P}_{\mathscr{A}}(c)|D| = \mathcal{P}_{\mathscr{B}}(c')|D|L(a_p)-\mathcal{P}_{\mathscr{B}}(c')L(\delta(a_p)).
        \end{equation}
        Note that $\mathcal{P}_{\mathscr{B}}(c')L(\delta(a_p))$ is bounded.
        
        Also,
        \begin{align}\label{simplified wc expression 2}
            \mathcal{W}_{\mathscr{A}}(c)|D|^{p+1-|\mathscr{A}|} &= \mathcal{W}_{\mathscr{B}}(c')L(a_p)|D|^{p-|\mathscr{A}|}\nonumber\\
                                        &= \mathcal{W}_{\mathscr{B}}(c')|D|^{p-|\mathscr{A}|}L(a_p)-\mathcal{W}_{\mathscr{B}}(c')[|D|^{p-|\mathscr{A}|},L(a_p)]\nonumber\\
                                        &= \mathcal{W}_{\mathscr{B}}(c')|D|^{p-|\mathscr{A}|}L(a_p)-\mathcal{W}_{\mathscr{B}}(c')|D|^{p-1-|\mathscr{A}|}|D|^{-p+1+|\mathscr{A}|}[|D|^{p-|\mathscr{A}|},L(a_p)].
        \end{align}
        By Lemma \ref{left to right corollary}, the operator $|D|^{-p+1+|\mathscr{A}|}[|D|^{p-|\mathscr{A}|},L(a_p)]$ has bounded extension. So combining \eqref{simplified pc expression 2}
        and \eqref{simplified wc expression 2}, it follows that
        \begin{equation*}
            (\mathcal{P}_{\mathscr{A}}(c)-\mathcal{W}_{\mathscr{A}}(c)|D|^{p-|\mathscr{A}|})|D|-(\mathcal{P}_{\mathscr{B}}(c')-\mathcal{W}_{\mathscr{B}}(c')|D|^{p-1-|\mathscr{B}|})|D|L(a_p)
        \end{equation*} 
        has bounded extension. So by the inductive hypothesis, it follows that $(\mathcal{P}_{\mathscr{A}}(c)-\mathcal{W}_{\mathscr{A}}(c)|D|^{p-|\mathscr{A}|})|D|$ has bounded extension
        in the case $p \notin \mathscr{A}$.
    \end{proof}
    
    The main idea used in the proof of Lemma \ref{combinatorial lemma} is the algebraic identity,
    \begin{equation}\label{combinatorial fact}
        \prod_{k=1}^p (x_k+y_k) = \sum_{\mathscr{A} \subseteq \{1,\cdots,p\}} z_{\mathscr{\mathscr{A}}}
    \end{equation}
    where $z_{\mathscr{\mathscr{A}}}$ is given by the product $z_1z_2\cdots z_p$, where $z_k = x_k$ for $k \in \mathscr{A}$
    and $z_k = y_k$ for $k\notin \mathscr{A}$.
    
    Now we are ready to complete the proof of Lemma \ref{combinatorial lemma}:

    \begin{proof}[Proof of Lemma \ref{combinatorial lemma}] 
        Since
        \begin{equation*}
            [D,a] = F[|D|,a]+[F,a]|D|,
        \end{equation*}
        as an equality of operators on $H_\infty$, it follows that we have $\partial(a) = F\delta(a)+L(a)$.
        Now using \eqref{combinatorial fact}:
        \begin{align*}
            \Omega(c) &= \Gamma a_0\prod_{k=1}^p (F\delta(a_k)+L(a_k))\\
                      &= \sum_{\mathscr{A}\subseteq \{1,\cdots,p\}} \mathcal{R}_{\mathscr{A}}(c).
        \end{align*}
        So on $H_\infty$:
        \begin{equation*}
            \Omega(c)|D| = \sum_{\mathscr{A}\subseteq \{1,\cdots,p\}} \mathcal{R}_{\mathscr{A}}(c)|D|.
        \end{equation*}
        We may now apply Lemma \ref{first combinatorial lemma} to each summand to conclude that
        \begin{equation*}
            \Omega(c)|D| - \sum_{\mathscr{A}\subseteq \{1,\cdots,p\}} (-1)^{n_{\mathscr{A}}}\mathcal{P}_{\mathscr{A}}(c)F^{|\mathscr{A}|}|D|
        \end{equation*}
        has bounded extension.
        Now applying Lemma \ref{second combinatorial lemma} to each summand, we have that the operator
        \begin{equation*}
            \Omega(c)|D|-\sum_{\mathscr{A}\subseteq \{1,\cdots,p\}}(-1)^{n_{\mathscr{A}}}\mathcal{W}_{\mathscr{A}}(c)|D|^{p-|\mathscr{A}|+1}F^{|\mathscr{A}|}
        \end{equation*}
        has bounded extension.
        
        Equivalently,
        \begin{equation*}
            \left(\Omega(c)|D|^{2-p}-\sum_{\mathscr{A}\subseteq \{1,\ldots,p\}}(-1)^{n_{\mathscr{A}}}\mathcal{W}_{\mathscr{A}}(c)D^{2-|\mathscr{A}|}\right)|D|^{p-1}
        \end{equation*}
        has bounded extension.
    \end{proof}
    
    We now prove the main result of this section.
    \begin{thm}\label{combinatorial theorem} 
        Let $(\mathcal{A},H,D)$ be a spectral triple satisfying Hypothesis \ref{main assumption} and Hypothesis \ref{auxiliary assumption}. 
        For all $c\in\mathcal{A}^{\otimes (p+1)},$ we have
        \begin{align*}
            \mathrm{Tr}(\Omega(c)|D|^{2-p}e^{-s^2D^2}) = \sum_{\mathscr{A} \subseteq \{1,2,\ldots,p\}} (-1)^{n_{\mathscr{A}}}\mathrm{Tr}(\mathcal{W}_{\mathscr{A}}(c)D^{2-|\mathscr{A}|}e^{-s^2D^2})+O(s^{-1})
        \end{align*}
        as $s\to 0$.
    \end{thm}
    \begin{proof}
        As in the preceding lemmas it suffices to prove the result for an elementary tensor $c = a_0\otimes\cdots \otimes a_p \in \mathcal{A}^{\otimes(p+1)}$. Let $c' = 1\otimes a_1\otimes\cdots\otimes a_p$. 
        By the cyclicity of the trace and the fact that $\mathcal{A}$ commutes with $\Gamma$, we have:
        \begin{equation*}
            \mathrm{Tr}(\Omega(c)|D|^{2-p}e^{-s^2D^2}) = \mathrm{Tr}(\Omega(c')|D|^{2-p}e^{-s^2D^2}a_0)
        \end{equation*}
        and for all $\mathscr{A} \subseteq \{1,2,\ldots,p\}$:
        \begin{equation*}
            \mathrm{Tr}(\mathcal{W}_{\mathscr{A}}(c)D^{2-|\mathscr{A}|}e^{-s^2D^2}) = \mathrm{Tr}(\mathcal{W}_{\mathscr{A}}(c')D^{2-|\mathscr{A}|}e^{-s^2D^2}a_0).
        \end{equation*}
        Thus,
        \begin{align*}
            &\mathrm{Tr}(\Omega(c)|D|^{2-p}e^{-s^2D^2})-\sum_{\mathscr{A}\subseteq \{1,\ldots,p\}}(-1)^{n_{\mathscr{A}}}\mathrm{Tr}(\mathcal{W}_{\mathscr{A}}(c)D^{2-|\mathscr{A}|}e^{-s^2D^2})\\
                                                    &=\mathrm{Tr}\left(\left(\Omega(c')|D|^{2-p}-\sum_{\mathscr{A}\subseteq \{1,\ldots,p\}}(-1)^{n_{\mathscr{A}}}\mathcal{W}_{\mathscr{A}}(c')D^{2-|\mathscr{A}|}\right)|D|^{p-1}|D|^{1-p}e^{-s^2D^2}a_0\right)
        \end{align*}
            Hence,
        \begin{align*}
            &\left|\mathrm{Tr}(\Omega(c)|D|^{2-p}e^{-s^2D^2}-\sum_{\mathscr{A}\subseteq \{1,\ldots,p\}}(-1)^{n_{\mathscr{A}}}\mathrm{Tr}(\mathcal{W}_{\mathscr{A}}(c)D^{2-|\mathscr{A}|}e^{-s^2D^2})\right|\\
            &\leq \left\|\left(\Omega(c')D^{2-p}-\sum_{\mathscr{A}\subseteq \{1,\ldots,p\}}(-1)^{n_{\mathscr{A}}}\mathcal{W}_{\mathscr{A}}(c')D^{2-|\mathscr{A}|}\right)|D|^{p-1}\right\|_\infty\||D|^{1-p}e^{-s^2D^2}a_0\|_1
        \end{align*}
        The first factor is finite, by Lemma \ref{combinatorial lemma}, and the second factor is $O(s^{-1})$, by Lemma \ref{first decay lemma}.
        This completes the proof.
    \end{proof}


\section{Auxiliary commutator estimates}\label{commutator section}
    This section is a slight detour from the main task of this chapter. Here we establish bounds on the $\mathcal{L}_1$-norm of commutators
    of the form $[f(s|D|),x]$, where $x \in \mathcal{B}$, $s > 0$ and $f$ is the square of a Schwartz class function on $\mathbb{R}.$ These bounds are used everywhere in the subsequent sections of this chapter.
    
    Recall that the algebra $\mathcal{B}$ is defined in Definition \ref{smoothness definition}.
    
    The following lemma serves the same purpose as \cite[Lemma 18]{CPRS1}, but the right hand sides are different here due to the fact that we deal with non-unital spectral triples.
    \begin{lem}\label{first commutator lemma} 
        Let $(\mathcal{A}, H, D)$ be a smooth spectral triple. Let $h$ be a Schwartz class function on $\mathbb{R}$ and let $f = h^2$. Then for all $x \in \mathcal{B}$, we have
        \begin{equation*}
            \left\|[f(s|D|),x]-\frac{s}{2}\{f'(s|D|),\delta(x)\}\right\|_1 \leq \frac{1}{2}s^2\|\widehat{h''}\|_1\cdot \left(\|\delta^2(x)h(s|D|)\|_1+\|h(s|D|)\delta^2(x)\|_1\right)
        \end{equation*}
        Here, $\{\cdot,\cdot\}$ denotes the anti-commutator.
    \end{lem}
    \begin{proof}
        Since $f'(s|D|) = 2h'(s|D|)h(s|D|)$, by the Leibniz rule we have:
        $$[f(s|D|),x]-\frac{s}{2}\{f'(s|D|),\delta(x)\} = [h(s|D|)^2,x]-s\{h'(s|D|)h(s|D|),\delta(x)\}=$$
        $$= h(s|D|)\big([h(s|D|),x]-sh'(s|D|)\delta(x)\big)+\big([h(s|D|),x]-s\delta(x)h'(s|D|)\big)h(s|D|).$$
        Applying Lemma \ref{second commutator rep lemma}, we have
        $$[h(s|D|),x]-sh'(s|D|)\delta(x) = -s^2\int_{-\infty}^\infty \int_0^1 \widehat{h''}(u)(1-v)e^{ius(1-v)|D|}\delta^2(x)e^{iusv|D|}\,dvdu$$
        $$[h(s|D|),x]-s\delta(x)h'(s|D|) = -s^2\int_{-\infty}^\infty \int_0^1 \widehat{h''}(u)(1-v)e^{iusv|D|}\delta^2(x)e^{ius(1-v)|D|}\,dvdu.$$
        
        Therefore,
        \begin{align*}
            [f(s|D|),x] - \frac{s}{2}&\{f'(s|D|),\delta(x)\}= \\
            & -s^2\int_{-\infty}^\infty \int_0^1 \widehat{h''}(u)(1-v)e^{iu(1-v)s|D|}h(s|D|)\delta^2(x)e^{iuvs|D|}\,dvdu\\
            & -s^2\int_{-\infty}^\infty \int_0^1 \widehat{h''}(u)(1-v)e^{iuvs|D|}\delta^2(x)h(s|D|)e^{iu(1-v)s|D|}\,dvdu.
        \end{align*}
        Now applying Lemma \ref{peter lemma} to each integral, we have
        \begin{align*} 
             \|[f(s|D|),x]&-\frac{s}{2}\{f'(s|D|),\delta(x)\}\|_1 \\ 
                               &\leq s^2\int_{-\infty}^\infty \int_0^1 \left\|\widehat{h''}(u)(1-v)e^{iu(1-v)s|D|}h(s|D|)\delta^2(x)e^{iuvs|D|}\right\|_1dvdu\\
                               &+s^2\int_{-\infty}^\infty \int_0^1 \left\|\widehat{h''}(u)(1-v)e^{iuvs|D|}\delta^2(x)h(s|D|)e^{iu(1-v)s|D|}\right\|_1dvdu\\ 
                               &= s^2\|\widehat{h''}\|_1(\|\delta^2(x)h(s|D|)\|_1+\|h(s|D|)\delta^2(x)\|_1)\int_0^1(1-v)dv\\ 
                               &= \frac{1}{2}s^2\|\widehat{h''}\|_1(\|\delta^2(x)h(s|D|)\|_1+\|h(s|D|)\delta^2(x)\|_1). 
        \end{align*}
% 
%         By Lemma \ref{peter lemma}, we have
%         $$\Big\|[f(s|D|,x)]-\frac{s}2\{f'(s|D|),\delta(x)\}\Big\|_1\leq $$
%         $$\leq s^2\int_{-\infty}^{\infty}\Big(\int_0^1\Big\|\hat{h''}(u)(1-v)e^{iu(1-v)s|D|}h(s|D|)\delta^2(x)e^{iuvs|D|}\Big\|_1dv\Big)du+$$
%         $$+s^2\int_{-\infty}^{\infty}\Big(\int_0^1\Big\|\hat{h''}(u)(1-v)e^{iu(1-v)s|D|}\delta^2(x)h(s|D|)e^{iuvs|D|}\Big\|_1dv\Big)du=$$
%         $$=\frac12s^2\|\hat{h''}\|_1\cdot\Big(\|\delta^2(x)h(s|D|)\|_1+\|h(s|D|)\delta^2(x)\|_1\Big).$$
    \end{proof}

    \begin{lem}\label{second commutator lemma} 
        Let $(\mathcal{A},H,D)$ be a smooth spectral triple and assume that $D$ has a spectral gap at $0$. Let $h$ be a Schwartz function on $\mathbb{R}$ and let $f=h^2.$ Then for every $x\in\mathcal{B},$ we have
        \begin{equation*}
            \left\||D|^m[f(s|D|),x]\right\|_1 \leq s\|\widehat{h'}\|_1\left(\left\||D|^mh(s|D|)\delta(x)\right\|_1+\left\||D|^m\delta(x)h(s|D|)\right\|_1\right).
        \end{equation*}
%         $$\Big\||D|^m[f(s|D|),x]\Big\|_1\leq s\|\hat{h'}\|_1\cdot\Big(\Big\||D|^mh(s|D|)\delta(x)\Big\|_1+\Big\||D|^m\delta(x)h(s|D|)\Big\|_1\Big).$$
    \end{lem}
    \begin{proof} 
        Since $f = h^2$, by the Leibniz rule, we have
        \begin{equation*}
            [f(s|D|),x] = h(s|D|)[h(s|D|),x]+[h(s|D|),x]h(s|D|).
        \end{equation*}
        Using Lemma \ref{first commutator rep lemma}, we have
        \begin{equation*}
            [h(s|D|),x] = s\int_{\mathbb{R}} \int_0^1 \widehat{h'}(u)e^{ius(1-v)|D|}\delta(x)e^{iusv|D|}dvdu.
        \end{equation*}
        
        Thus,
        \begin{align*}
            |D|^m[f(s|D|),x] &= s\int_{\mathbb{R}} \int_0^1 \widehat{h'}(u)e^{ius(1-v)|D|}|D|^mh(s|D|)\delta(x)e^{iusv|D|}\,dvdu\\
                             &\quad\quad +s\int_{\mathbb{R}} \int_0^1 \widehat{h'}(u)e^{ius(1-v)|D|}|D|^m\delta(x)h(s|D|)e^{iusv|D|}\,dvdu.
        \end{align*}
        
        Bounding the $\mathcal{L}_1$ norm using Lemma \ref{peter lemma}, we have
        \begin{align*}
            \left\||D|^m[f(s|D|),x]\right\|_1 &\leq s\int_{\mathbb{R}}\int_0^1 |\widehat{h'}(u)|\||D|^mh(s|D|)\delta(x)\|_1\,dvdu\\
                                              &\quad\quad + s\int_\mathbb{R} \int_0^1 |\widehat{h'}(u)|\||D|^m\delta(x)h(s|D|)\|_1\,dvdu\\
                                              &= s\|\widehat{h'}\|_{L_1(\mathbb{R})}(\||D|^mh(s|D|)\delta(x)\|_1+\||D|^m\delta(x)h(s|D|)\|_1).
        \end{align*}
    \end{proof}
    
    \begin{lem}\label{commutator 6} 
        Let $(\mathcal{A},H,D)$ be a smooth spectral triple satisfying Hypothesis \ref{main assumption} and Hypothesis \ref{auxiliary assumption}. For all $x\in\mathcal{B}$ and for all integers $m>-p,$ we have
        \begin{equation*}
            \||D|^m[e^{-s^2D^2},x]\|_1=O(s^{1-p-m}),\quad s\downarrow0.
        \end{equation*}
    \end{lem}
    \begin{proof} 
        Let $h(t)=e^{-\frac{1}{2}t^2},$ $t\in\mathbb{R}.$ By Lemma \ref{second commutator lemma}, we have
        \begin{equation*}
            \||D|^m[e^{-s^2D^2},x]\|_1 \leq s\|\widehat{h'}\|_1(\||D|^me^{-\frac12s^2D^2}\delta(x)\|_1+\||D|^m\delta(x)e^{-\frac12s^2D^2}\|_1).
        \end{equation*}
        
        If $m \leq 0$, then the assertion now follows from applying Lemma \ref{first decay lemma} to the terms $\||D|^me^{-\frac{1}{2}s^2D^2}\delta(x)\|_1$ and $\||D|^m\delta(x)e^{-\frac{1}{2}s^2D^2}\|_1$.        
        Assume now $m > 0$. Using Lemma \ref{schwartz lemma} with the Schwartz function $t\mapsto t^me^{-\frac{1}{2}t^2}$, we obtain
        \begin{equation*}
            s\left\||D|^me^{-\frac{1}{2}s^2D^2}\delta(x)\right\|_1 = O(s^{1-p-m}),\quad s\to 0.
        \end{equation*}
        By Lemma \ref{left to right lemma}, we have
        \begin{equation*}
            |D|^m\delta(x)e^{-\frac{1}{2}s^2D^2} = \sum_{k=0}^m \binom{m}{k}\delta^{m+1-k}(x)|D|^ke^{-\frac{1}{2}s^2D^2}.
        \end{equation*}
        Now we apply Lemma \ref{schwartz lemma} to each summand, using the function $t\mapsto t^ke^{-\frac{1}{2}s^2D^2}$ for the $k$th summand.
        So,
        \begin{align*}
            s\||D|^m\delta(x)e^{-\frac{1}{2}s^2D^2}\|_1 &\leq s\sum_{k=0}^m \binom{m}{k}O(s^{1-p-k})\\
                                                        &= O(s^{1-p-m}), \quad s\to 0.
        \end{align*}
    \end{proof}
    
    The following lemma is used in the proof of Theorem \ref{first cycle thm}.
    \begin{lem}\label{third commutator lemma} 
        Let $(\mathcal{A},H,D)$ be a spectral triple satisfying Hypothesis \ref{main assumption} and Hypothesis \ref{auxiliary assumption}. Let $f(t)=e^{-t^2},$ $t\in\mathbb{R}.$
        \begin{enumerate}[{\rm (i)}]
            \item\label{3com1} for every $a\in\mathcal{A},$ we have
                $$\Big\|[f(s|D|),a]-s\delta(a)f'(s|D|)\Big\|_{\infty}=O(s^2),\quad s\downarrow0.$$
            \item\label{3com2} for every $a\in\mathcal{A},$ we have
                $$\Big\|[f(s|D|),a]-s\delta(a)f'(s|D|)\Big\|_1=O(s^{2-p}),\quad s\downarrow0.$$
            \item\label{3com3} for every $a\in\mathcal{A},$ we have
                $$\Big\|[f(s|D|),a]-s\delta(a)f'(s|D|)\Big\|_{p,1}=O(s),\quad s\downarrow0.$$
        \end{enumerate}
    \end{lem}
    \begin{proof} 
        First we prove \eqref{3com1}: this is a simple combination of Lemma \ref{second commutator rep lemma} and the triangle inequality:
        \begin{equation*}
            \|[f(s|D|),a]-sf'(s|D|)\delta(a)\|_\infty \leq s^2\|\widehat{f''}\|_1\|\delta^2(a)\|_\infty.
        \end{equation*}
        
        Now we prove \eqref{3com2}. Let $h(t) = e^{-t^2/2}$, $t \in \mathbb{R}$, so that $f = h^2$. By Lemma \ref{first commutator lemma}, for all $a \in \mathcal{A}$ we have
        \begin{equation}\label{big commutator difference bound}
            \left\|[f(s|D|),a]-\frac{s}{2}\{f'(s|D|),\delta(a)\}\right\|_1 \leq \frac{1}{2}s^2\|\hat{h''}\|_1(\|\delta^2(a)h(s|D|)\|_1+\|h(s|D|)\delta^2(a)\|_1).
        \end{equation} 
        Using Lemma \ref{schwartz lemma}, we have
        \begin{align}\label{schwartz type bounds}
            \|\delta^2(a)h(s|D|)\|_1 &= O(s^{-p})\text{ and, }\nonumber\\
            \|h(s|D|)\delta^2(a)\|_1 &= O(s^{-p}).
        \end{align}
        Combining \eqref{big commutator difference bound} and \eqref{schwartz type bounds}, we arrive at:
        \begin{equation}\label{com31}
            \|[f(s|D|),a]-\frac{s}{2}\{f'(s|D|),\delta(a)\}\|_1 = O(s^{2-p}).
        \end{equation}
        On the other hand,
        \begin{align*}
            \|[f'(s|D|),\delta(a)]\|_1 &= 2s\|[|D|e^{-s^2D^2},\delta(a)]\|_1\\
                                       &\leq 2s\|\delta^2(a)e^{-s^2D^2}\|_1+2s\||D|[e^{-s^2D^2},\delta(a)]\|_1.
        \end{align*}
        Due to Lemma \ref{schwartz lemma}, we have $2s\|\delta^2(a)e^{-s^2D^2}\|_1 = O(s^{1-p})$, and by Lemma \ref{commutator 6}
        we also have $2s\||D|[e^{-s^2D^2},\delta(a)]\|_1 = O(s^{1-p})$. Therefore,
        \begin{equation}\label{com32}
            \|[f'(s|D|),\delta(a)]\|_1 = O(s^{1-p}),\quad s \downarrow 0.
        \end{equation}
        
        By combining \eqref{com31} and \eqref{com32}, we obtain \eqref{3com2}.
        
        Finally, to prove \eqref{3com3}, we use the inequality
        $$\|T\|_{p,1}\leq\|T\|_1^{\frac1p}\|T\|_{\infty}^{1-\frac1p}$$
        and write
        $$\|[f(s|D|),a]-s\delta(a)f'(s|D|)\|_{p,1} \leq$$
        $$\leq\|[f(s|D|),a]-s\delta(a)f'(s|D|)\|_1^{\frac1p}\|[f(s|D|,a)]-s\delta(a)f'(s|D|)\|_{\infty}^{1-\frac1p}=$$
        $$= O(s^{2-p})^{\frac1p}\cdot O(s^{2})^{1-\frac1p}=O(s).$$
        \end{proof} 

    The following Lemma is used in Lemma \ref{kogom2}, Lemma \ref{kogom3} and Lemma \ref{second opposite lemma}.
    \begin{lem}\label{commutator 7} 
        Let $(\mathcal{A},H,D)$ be a smooth $p-$dimensional spectral triple satisfying Hypothesis \ref{main assumption} and Hypothesis \ref{auxiliary assumption}. For every $0\leq m\leq p,$ and $x \in \mathcal{B}$ we have
        \begin{equation*}
            \|D^{m-p}[D^{2-m}e^{-s^2D^2},x]\|_1 = O(s^{-1}),\quad s\downarrow0.
        \end{equation*}
    \end{lem}
    \begin{proof} 
        By the Leibniz rule,
        \begin{equation*}
            [D^{2-m}e^{-s^2D^2},x] = [D^{2-m},a]e^{-s^2D^2}+D^{2-m}[e^{-s^2D^2},x].
        \end{equation*}
        Thus,
        \begin{equation*}
            \|D^{m-p}[D^{2-m}e^{-s^2D^2},x]\|_1 \leq \|D^{2-p}[e^{-s^2D^2},x]\|_1+\|D^{m-p}[D^{2-m},x]e^{-s^2D^2}\|_1.
        \end{equation*}
        
        By Lemma \ref{commutator 6}, we have $\|D^{2-p}[e^{-s^2D^2},x]\|_1 = O(s^{-1})$, so we now focus on the second summand.
        First, for the case when $m > 2$ we apply the Leibniz rule
        \begin{align*}
            [D^{2-m},x] &= -D^{2-m}[D^{m-2},x]D^{2-m}\\
                        &= -\sum_{k+l=m-3} D^{k+2-m}\partial(x)D^{l+2-m}.
        \end{align*}
        Now using the triangle inequality:
        \begin{equation*}
            \|D^{m-p}[D^{2-m},x]e^{-s^2D^2}\|_1 \leq \sum_{k+l=m-3}\||D|^{k+2-p}\partial(x)|D|^{l+2-m}e^{-s^2|D|^2}\|_1.
        \end{equation*}
        Applying Lemma \ref{first decay lemma} to each summand, we then conclude that $${\|D^{m-p}[D^{2-m},x]e^{-s^2D^2}\|_1 = O(s^{-1})},$$
        thus proving the claim for $m > 2$.
        
        We now deal with the remaining cases $m = 0,1,2$ individually.        
        In the case $m=2$, we have $D^{m-p}[D^{2-m},x]e^{-s^2D^2} = 0$, and so the claim follows trivially in this case.
        
        For $m = 1$, we have
        \begin{equation*}
            \|D^{m-p}[D^{2-m},x]e^{-s^2D^2}\|_1 = \||D|^{1-p}\partial(x)e^{-s^2D^2}\|_1
        \end{equation*}
        So by Lemma \ref{first decay lemma}, we also have in this case that the above is $O(s^{-1})$.
        
        Finally, for $m=0$,
        \begin{align*}
            [D^2,x] &= [|D|^2,x]\\ 
                    &= |D|\delta(x)+\delta(x)|D|\\
                    &= 2|D|\delta(x)-\delta^2(x).
        \end{align*}
        So by the triangle inequality:
        \begin{equation*}
            \|D^{m-p}[D^{2-m},x]e^{-s^2D^2}\|_1 \leq 2\||D|^{1-p}\delta(x)e^{-s^2D^2}\|_1+\||D|^{-p}\delta^2(x)e^{-s^2D^2}\|_1
        \end{equation*}
        so an application of Lemma \ref{first decay lemma} to each of the above summands yields the result.
    \end{proof}

\section{Exploiting Hochschild homology}\label{cohomology section}
    
    Recall the multilinear mapping $\mathcal{W}_p$ from Definition \ref{W definition}. In this section, we prove the following:
    \begin{thm}\label{reduction} 
        Let $(\mathcal{A},H,D)$ be a spectral triple satisfying Hypothesis \ref{main assumption} and Hypothesis \ref{auxiliary assumption}. For every Hochschild cycle $c\in\mathcal{A}^{\otimes (p+1)}$ we have:
        \begin{equation*}
            \mathrm{Tr}(\Omega(c)|D|^{2-p}e^{-s^2D^2})-p\mathrm{Tr}(\mathcal{W}_p(c)De^{-s^2D^2}) = O(s^{-1}),\quad s\downarrow 0.
        \end{equation*}
    \end{thm}
    We achieve this using the commutator estimates of the preceding section.
    
    Our strategy to prove Theorem \ref{reduction} is to start from Theorem \ref{combinatorial theorem} and then show that:
    \begin{enumerate}
        \item{} all of the terms with $|\mathscr{A}| \geq 2$ are $O(s^{-1})$ (see Lemma \ref{kogom4})
        \item{} the $\mathscr{A} = \emptyset$ term is $O(s^{-1})$ (see Lemma \ref{kogom5}
        \item{} finally we complete the proof by showing that the terms with $|\mathscr{A}|=1$ are all equal to the $\mathscr{A}=\{p\}$ term up to terms of size $O(s^{-1})$.
    \end{enumerate}
    The proofs in this section rely crucially on the assumption that $c$ is a Hochschild cycle.
    
    First, we show that terms in Theorem \ref{combinatorial theorem} such that there is some $m$ with $m-1,m \in \mathscr{A}$ are $O(s^{-1})$.
    \begin{lem}\label{kogom2} 
        Let $(\mathcal{A},H,D)$ be a spectral triple satisfying Hypothesis \ref{main assumption} and Hypothesis \ref{auxiliary assumption}. Let $m \in \{1,\ldots,p\}$ and suppose that $m-1,m\in\mathscr{A}$ (so necessarily we have $|\mathscr{A}|\geq 2$). 
        For every Hochschild cycle $c\in\mathscr{A}^{\otimes (p+1)},$ we have
        \begin{equation*}
            \mathrm{Tr}(\mathcal{W}_{\mathscr{A}}(c)D^{2-|\mathscr{A}|}e^{-s^2D^2}) = O(s^{-1}),\quad s\downarrow 0.
        \end{equation*}
    \end{lem}
    \begin{proof} 
        For $s > 0$, consider the multilinear mapping $\theta_s:\mathcal{A}^{\otimes p}\to \mathbb{C}$ defined by:
        \begin{equation*}
            \theta_s(a_0\otimes \cdots \otimes a_{p-1}) = \mathrm{Tr}\left(\Gamma a_0\left(\prod_{k=1}^{m-2}b_k(a_k)\right)\delta^2(a_{m-1})\left(\prod_{k=m}^{p-1}b_{k+1}(a_k)\right)D^{2-|\mathscr{A}|}e^{-s^2D^2}\right)
        \end{equation*}
        where $b_k$ is as in Definition \ref{W definition}. Then, from the computation in Appendix \ref{coboundary app}, we have that the Hochschild coboundary is:
        \begin{align*}
            (b\theta_s)(a_0\otimes &\cdots \otimes a_p)\\
                                   &= (-1)^p\mathrm{Tr}\left(\Gamma a_0\left(\prod_{k=1}^{m-2}[b_k,a_k]\right)\delta^2(a_{m-1})\left(\prod_{k=m}^{p-1}[b_{k+1},a_k]\right)[D^{2-|\mathscr{A}|}e^{-s^2D^2},a_p]\right)\\
                                   &+ 2(-1)^{m-1}\mathrm{Tr}(\mathcal{W}_{\mathscr{A}}(a_0\otimes \cdots \otimes a_p)D^{2-|\mathscr{A}|}e^{-s^2D^2}). 
        \end{align*}
        
        We now claim that the first summand is $O(s^{-1})$ as $s\downarrow 0$. Indeed, dividing and multiplying by $D^{|\mathscr{A}|-p}$:
        \begin{align*}
            \Big\|\Gamma a_0\left(\prod_{k=1}^{m-2}[b_k,a_k]\right)&\delta^2(a_{m-1})\left(\prod_{k=m}^{p-1}[b_{k+1},a_k]\right)[D^{2-|\mathscr{A}|}e^{-s^2D^2},a_p]\Big\|_1\\
                                                                   &\leq \left\|D^{|\mathscr{A}|-p}[D^{2-|\mathscr{A}|}e^{-s^2D^2},a_p]\right\|_1\\
                                                                   &\times \left\|\Gamma a_0\left(\prod_{k=1}^{m-2}[b_k,a_k]\right)\delta^2(a_{m-1})\left(\prod_{k=m}^{p-1}[b_{k+1},a_k]\right)|D|^{p-|\mathscr{A}|}\right\|_\infty 
        \end{align*}
        The first factor is $O(s^{-1})$ by Lemma \ref{commutator 7}, and the second factor is finite by Lemma \ref{c first alert lemma} and has no dependence on $s$.
        
        To summarise, so far we that if $c \in \mathcal{A}^{\otimes (p+1)}$: 
        \begin{equation*}
            (b\theta_s)(c) = 2(-1)^{m-1}\mathrm{Tr}(\mathcal{W}_{\mathscr{A}}(c)D^{2-|\mathscr{A}|}e^{-s^2D^2})+O(s^{-1}),\quad s\downarrow 0.
        \end{equation*}
        If $c$ is a Hochschild cycle, then $(b\theta_s)(c) = \theta_s(bc) = 0$, and so
        \begin{equation*}
            2(-1)^{m-1}\mathrm{Tr}(\mathcal{W}_{\mathscr{A}}(c)D^{2-|\mathscr{A}|}e^{-s^2D^2}) = O(s^{-1})
        \end{equation*}
        as required.
    \end{proof}

    \begin{lem}\label{kogom3} 
        Let $(\mathcal{A},H,D)$ be a spectral triple satisfying Hypothesis \ref{main assumption} and Hypothesis \ref{auxiliary assumption}. 
        Let $\mathscr{A}_1,\mathscr{A}_2 \subseteq \{1,\ldots,p\}$, with $|\mathscr{A}_1|=|\mathscr{A}_2|$ and that the symmetric difference $\mathscr{A}_1\Delta\mathscr{A}_2=\{m-1,m\}$ for some $m.$ 
        Then for every Hochschild cycle $c\in\mathcal{A}^{\otimes (p+1)},$ we have
        \begin{equation*}
            \mathrm{Tr}(\mathcal{W}_{\mathscr{A}_1}(c)D^{2-|\mathscr{A}_1|}e^{-s^2D^2})+\mathrm{Tr}(\mathcal{W}_{\mathscr{A}_2}(c)D^{2-|\mathscr{A}_2|}e^{-s^2D^2}) = O(s^{-1}),\quad s\downarrow0.
        \end{equation*}
    \end{lem}
    \begin{proof} 
        This proof is similar to that of Lemma \ref{kogom2}. For $s > 0$ we consider the multilinear mapping $\theta_s:\mathcal{A}^{\otimes p}\to \mathbb{C}$ given by
        \begin{equation*}
            \theta_s(a_0\otimes\cdots\otimes a_{p-1}) = \mathrm{Tr}\left(\Gamma a_0\left(\prod_{k=1}^{m-2}b_k(a_k)\right)[F,\delta(a_{m-1})]\left(\prod_{k=m}^{p-1}b_{k+1}(a_k)\right)D^{2-|\mathscr{A}_1|}e^{-s^2D^2}\right).
        \end{equation*}
        Here, as in Lemma \ref{kogom2}, the operators $b_k$ are defined as in Definition \ref{W definition}, relative to the set $\mathscr{A} = \mathscr{A}_1$. From the computation in Appendix \ref{coboundary app},
        \begin{align*}
            (b\theta_s)(a_0&\otimes\cdots\otimes a_p)\\
                           &= (-1)^p\mathrm{Tr}\left(\Gamma a_0\left(\prod_{k=1}^{m-2}[b_k,a_k]\right)[F,\delta(a_{m-1})]\left(\prod_{k=m}^{p-1}[b_{k+1},a_k]\right)[D^{2-|\mathscr{A}_1|}e^{-s^2D^2},a_p]\right)\\
                           &+ (-1)^{m-1}\mathrm{Tr}(\mathcal{W}_{\mathscr{A}_1}(a_0\otimes\cdots\otimes a_p)D^{2-|\mathscr{A}_1|}e^{-s^2D^2})\\
                           &+ (-1)^{m-1}\mathrm{Tr}(\mathcal{W}_{\mathscr{A}_2}(a_0\otimes\cdots\otimes a_p)D^{2-|\mathscr{A}_2|}e^{-s^2D^2}).
        \end{align*}
        We first show that the first summand above is $O(s^{-1})$. Indeed,
        \begin{align*}       
            \Big\|\Gamma a_0\left(\prod_{k=1}^{m-2}[b_k,a_k]\right)&[F,\delta(a_{m-1})]\left(\prod_{k=m}^{p-1}[b_{k+1},a_k]\right)[D^{2-|\mathscr{A}_1|}e^{-s^2D^2},a_p]\Big\|_1\\
                                                                   &\leq \left\|D^{|\mathscr{A}_1|-p}[D^{2-|\mathscr{A}_1|}e^{-s^2D^2},a_p]\right\|_1\\
                                                                   &\times\left\|\Gamma a_0\prod_{k=1}^{m-2}[b_k,a_k][F,\delta(a_{m-1})]\prod_{k=m}^{p-1}[b_{k+1},a_k]\cdot|D|^{p-|\mathscr{A}_1|}\right\|_{\infty}.
        \end{align*}
        The first factor above is $O(s^{-1})$ due to Lemma \ref{commutator 7}, and the second factor is finite by Lemma \ref{c second alert lemma} and has no dependence on $s$.
        
        Summarising the above, if $c \in \mathcal{A}^{\otimes (p+1)}$ we have
        \begin{align*}
            (b\theta_s)(c) &= (-1)^{m-1}\mathrm{Tr}(\mathcal{W}_{\mathscr{A}_1}(c)D^{2-|\mathscr{A}_1|}e^{-s^2D^2})\\
                           &+(-1)^{m-1}\mathrm{Tr}(\mathcal{W}_{\mathscr{A}_2}(c)D^{2-|\mathscr{A}_2|}e^{-s^2D^2})+ O(s^{-1})
        \end{align*}
        as $s\downarrow 0$. Hence, if $c$ is a Hochschild cycle then $(b\theta_s)(c) = \theta_s(bc) = 0$, and this completes the proof.
    \end{proof}

    \begin{lem}\label{kogom4} 
        Let $(\mathcal{A},H,D)$ be a spectral triple satisfying Hypothesis \ref{main assumption} and Hypothesis \ref{auxiliary assumption}. 
        For every Hochschild cycle $c\in\mathcal{A}^{\otimes (p+1)}$ and for every $\mathscr{A}\subset\{1,\cdots,p\}$ with $|\mathscr{A}|\geq 2,$ we have
        \begin{equation*}
            \mathrm{Tr}(\mathcal{W}_{\mathscr{A}}(c)D^{2-|\mathscr{A}|}e^{-s^2D^2}) = O(s^{-1}),\quad s\downarrow0.
        \end{equation*}
    \end{lem}
    \begin{proof} 
        Let $m$ be the maximum element in $\mathscr{A}$ and let $n$ be the maximal element in $\mathscr{A}\backslash\{m\}.$ 
        If $n=m-1,$ then the assertion is already proved in Lemma \ref{kogom2}. If $n < m-1,$ then $m-n > 1$, and hence $n+j \notin \mathscr{A}$
        for all $1 \leq j < m-n$. Now for each $0 \leq j < m-n$ we define $\mathscr{A}_j$ to be $\mathscr{A}$ with $n$ replaced with $n+j$. That is:
        \begin{equation*}
            \mathscr{A}_j := (\mathscr{A}\backslash\{n\})\cup\{j+n\},\quad 0 \leq j < m-n.
        \end{equation*}
        Then by construction:
        \begin{enumerate}
            \item $|\mathscr{A}_j|=|\mathscr{A}|$ and $\mathscr{A}_j\Delta\mathscr{A}_{j-1} = \{n+j,n+j-1\}$ for all $1 \leq j < m-n.$
            \item $\mathscr{A}_0=\mathscr{A}$ and $m-1,m\in\mathscr{A}_{m-n-1}.$
        \end{enumerate}
        Hence if $1 \leq j < m-n$ the subsets $\mathscr{A}_j$ and $\mathscr{A}_{j-1}$ satisfy the conditions of Lemma \ref{kogom3}. So for all Hochschild cycles $c \in \mathcal{A}^{\otimes (p+1)}$:
        \begin{equation*}
            \mathrm{Tr}(\mathcal{W}_{\mathscr{A}_{j-1}}(c)D^{2-|\mathscr{A}_{i-1}|}e^{-s^2D^2})=-\mathrm{Tr}(\mathcal{W}_{\mathscr{A}_j}(c)D^{2-|\mathscr{A}_{j}|}e^{-s^2D^2}) + O(s^{-1}),\quad s \downarrow 0.
        \end{equation*}
        So by induction, we have:
        \begin{align}\label{beginning and end}
            \mathrm{Tr}(\mathcal{W}_{\mathscr{A}_0}(c)D^{2-|\mathscr{A}_0|}e^{-s^2D^2}) &= (-1)^{m-n-1}\mathrm{Tr}(\mathcal{W}_{\mathscr{A}_{m-n-1}}(c)D^{2-|\mathscr{A}_{m-n-1}|}e^{-s^2D^2})\nonumber\\
                                                        &\quad+O(s^{-1}),\quad s\downarrow0.
        \end{align}
        On the other hand, since $m-1,m \in \mathscr{A}_{m-n-1}$, we may apply Lemma \ref{kogom2} to $\mathscr{A}_{m-n-1}$ to obtain:
        \begin{equation*}\label{terminal}
            \mathrm{Tr}(\mathcal{W}_{\mathscr{A}_{m-n-1}}(c)D^{2-|\mathscr{A}_{m-n-1}|}e^{-s^2D^2}) = O(s^{-1}),\quad s\downarrow0.
        \end{equation*}
        Combining \eqref{beginning and end} and \eqref{terminal}, we get
        \begin{equation*}
            \mathrm{Tr}(\mathcal{W}_{\mathscr{A}_0}(c)D^{2-|\mathscr{A}_0|}e^{-s^2D^2}) = O(s^{-1}).
        \end{equation*}
        since $\mathscr{A}_0 = \mathscr{A}$, the proof is complete.
    \end{proof}
    
    Recall the mapping $\mathrm{ch}$ from Definition \ref{ch omega def}.
    \begin{lem}\label{kogom pre 5} 
        Let $(\mathcal{A},H,D)$ be a spectral triple satisfying Hypothesis \ref{main assumption} and Hypothesis \ref{auxiliary assumption}. For every $c\in\mathcal{A}^{\otimes (p+1)},$ we have
        \begin{align*}
            \|\mathrm{ch}(c)D^2e^{-s^2D^2}\|_1 = O(s^{-1}),\quad s \downarrow 0.
        \end{align*}
    \end{lem}
    \begin{proof} 
        Recall that on $H_\infty$ we have $[F,a_p]|D| = L(a_p)$.
        So on $H_\infty$:
        \begin{align*}
            [F,a_{p-1}][F,a_p]|D|^2 &= [F,a_{p-1}]\cdot L(a_p)\cdot|D|\\
                                    &= [F,a_{p-1}]\cdot |D|L(a_p)-[F,a_{p-1}]\cdot [|D|,L(a_p)]\\
                                    &= [F,a_{p-1}]|D|\cdot L(a_p)-[F,a_{p-1}]\cdot\delta(L(a_p))\\
                                    &= L(a_{p-1})\cdot L(a_p)-[F,a_{p-1}]\cdot L(\delta(a_p)).
        \end{align*}
        So for $c=a_0\otimes\cdots\otimes a_p \in \mathcal{A}^{\otimes(p+1)}$ we have
        \begin{align*}
            \mathrm{ch}(c)\cdot D^2e^{-s^2D^2} &= \Gamma\left(\prod_{k=0}^{p-2}[F,a_k]\right) |D|^{p-1}\cdot |D|^{1-p}L(a_{p-1})L(a_p)e^{-s^2D^2}\\
                                       &\quad-\Gamma\left(\prod_{k=0}^{p-1}[F,a_k]\right)|D|^{p-1}\cdot|D|^{1-p}L(\delta(a_p))e^{-s^2D^2}.
        \end{align*}
        Using Lemma \ref{c third alert lemma}, the operators $\left(\prod_{k=0}^{p-2}[F,a_k]\right)|D|^{p-1}$ and $\left(\prod_{k=0}^{p-1}[F,a_k]\right)|D|^{p-1}$ both
        have bounded extension and no dependence on $s$.
        From Lemma \ref{first decay lemma}, we have that:
        \begin{align*}
            \||D|^{1-p}L(a_{p-1})L(a_p)e^{-s^2D^2}\|_1 &= O(s^{-1})\\
              \||D|^{1-p}L(\delta(a_p))e^{-s^2D^2}\|_1 &= O(s^{-1}).
        \end{align*}
        So by the triangle inequality: $\|\mathrm{ch}(c)D^2e^{-s^2D^2}\|_1 = O(s^{-1})$ as $s\downarrow 0$.
    \end{proof}
            
    \begin{lem}\label{W and ch link}
        Let $(\mathcal{A},H,D)$ be a spectral triple of dimension $p$, where $p$ has the same parity as $(\mathcal{A},H,D)$. If $c \in \mathcal{A}^{\otimes(p+1)}$ then
        \begin{equation*}
            \mathrm{ch}(c) = \mathcal{W}_{\emptyset}(c)+F\mathcal{W}_{\emptyset}(c)F.
        \end{equation*}
    \end{lem}
    \begin{proof}
        Let $c = a_0\otimes \cdots \otimes a_p \in \mathcal{A}^{\otimes (p+1)}$. Recall that
        \begin{equation*}
            \mathcal{W}_{\emptyset}(c) = \Gamma a_0\prod_{k=1}^p [F,a_k].
        \end{equation*}
        Using the fact that $F$ anticommutes with $[F,a_k]$ for all $k$, we have:
        \begin{align*}
            \mathrm{ch}(c) &= \Gamma F[F,a_0]\prod_{k=1}^p [F,a_k]\\
                   &= \Gamma a_0\prod_{k=1}^p [F,a_k]-\Gamma Fa_0F\prod_{k=1}^p [F,a_k].\\
                   &= \mathcal{W}_{\emptyset}(c)+(-1)^{p+1}\Gamma Fa_0\left(\prod_{k=1}^p [F,a_k]\right)F.
        \end{align*}
        Since $\Gamma^2 = 1$,
        \begin{equation*}
            \mathrm{ch}(c) = \mathcal{W}_{\emptyset}(c) + (-1)^{p+1}\Gamma F\Gamma \mathcal{W}_{\emptyset}(c)F.
        \end{equation*}
        Since the parity of $p$ matches the parity of $\Gamma$, we have $\Gamma F = (-1)^{p+1}F\Gamma$. This completes the proof.
    \end{proof}

    \begin{lem}\label{kogom5}
        Let $(\mathcal{A},H,D)$ be a spectral triple satisfying Hypothesis \ref{main assumption} and Hypothesis \ref{auxiliary assumption}. For every $c \in \mathcal{A}^{\otimes(p+1)}$, we have
        \begin{equation*}
            \mathrm{Tr}(\mathcal{W}_{\emptyset}(c)D^2e^{-s^2D^2}) = O(s^{-1}),\quad s\downarrow 0.
        \end{equation*}
    \end{lem}
    \begin{proof} 
        By Lemma \ref{W and ch link}, we have:
        \begin{equation*}
            \mathrm{Tr}(\mathrm{ch}(c)D^2e^{-s^2D^2}) = \mathrm{Tr}(\mathcal{W}_{\emptyset}(c)D^2e^{-s^2D^2})+\mathrm{Tr}(F\mathcal{W}_{\emptyset}(c)FD^2e^{-s^2D^2}).
        \end{equation*}
        However since $F$ commutes with $D^2e^{-s^2D^2}$ and $F^2=1$ we have:
        \begin{equation*}
            2\mathrm{Tr}(\mathcal{W}_{\emptyset}(c)D^2e^{-s^2D^2}) = \mathrm{Tr}(\mathrm{ch}(c)D^2e^{-s^2D^2}).
        \end{equation*}
        However by Lemma \ref{kogom pre 5},
        \begin{equation*}
            |\mathrm{Tr}(\mathrm{ch}(c)D^2e^{-s^2D^2})| = O(s^{-1}).
        \end{equation*}
        Hence $\mathrm{Tr}(\mathcal{W}_{\emptyset}(c)D^2e^{-s^2D^2}) = O(s^{-1})$.
    \end{proof}

    We are now ready to prove the main result of this section.

    \begin{proof}[Proof of Theorem \ref{reduction}] 
        Let $c  \in \mathcal{A}^{\otimes (p+1)}$. Then using Theorem \ref{combinatorial theorem}:
        \begin{equation*}
            \mathrm{Tr}(\Omega(c)|D|^{2-p}e^{-s^2D^2}) = \sum_{\mathscr{A} \subseteq \{1,\ldots,p\}}(-1)^{n_{\mathscr{A}}}\mathrm{Tr}(\mathcal{W}_{\mathscr{A}}(c)D^{2-|\mathscr{A}|}e^{-s^2D^2})+O(s^{-1}),\quad s\downarrow 0.
        \end{equation*}
        
        Applying Lemma \ref{kogom4} to every summand with $|\mathscr{A}| \geq 2$, and Lemma \ref{kogom5} to the summand $\mathscr{A} = \emptyset$, it follows that:
        \begin{equation*}
            \mathrm{Tr}(\Omega(c)|D|^{2-p}e^{-s^2D^2}) = \sum_{k=1}^p (-1)^{n_{\{k\}}} \mathrm{Tr}(\mathcal{W}_k(c)De^{-s^2D^2})+O(s^{-1}),\quad s\downarrow 0.
        \end{equation*}
        Recall that $n_{\mathscr{A}} = |\{(j,k)\in \{1,\ldots,p\}^2\;:\;j \in \mathscr{A}, k\notin \mathscr{A}\}|$.
        So in particular, $n_{\{k\}} = p-k$. Hence:
        \begin{equation}\label{single element expansion}
            \mathrm{Tr}(\Omega(c)|D|^{2-p}e^{-s^2D^2}) = \sum_{k=1}^p (-1)^{p-k}\mathrm{Tr}(\mathcal{W}_{k}(c)De^{-s^2D^2})+O(s^{-1}),\quad s\downarrow 0.
        \end{equation}
        For any $1 \leq k \leq p-1$, the sets $\mathscr{A}_1 = \{k\}$ and $\mathscr{A}_2 = \{k+1\}$ satisfy the conditions of Lemma \ref{kogom3} with $m = k+1$. So we have
        \begin{equation*}
            \mathrm{Tr}(\mathcal{W}_k(c)De^{-s^2D^2}) = -\mathrm{Tr}(\mathcal{W}_{k+1}(c)De^{-s^2D^2}) + O(s^{-1}),\quad s\downarrow0,
        \end{equation*}
        Hence by induction, for any $1 \leq k \leq p$:
        \begin{equation}\label{all singletons are p}
            \mathrm{Tr}(\mathcal{W}_k(c)De^{-s^2D^2}) = (-1)^{p-k}\mathrm{Tr}(\mathcal{W}_{p}De^{-s^2D^2}) + O(s^{-1}),\quad s\downarrow 0.
        \end{equation}        
        Substituting \eqref{all singletons are p} into each summand of \eqref{single element expansion} we finally get:
        \begin{equation*}
            \mathrm{Tr}(\Omega(c)|D|^{2-p}e^{-s^2D^2}) = p\mathrm{Tr}(\mathcal{W}_p(c)De^{-s^2D^2})+O(s^{-1}).
        \end{equation*}
    \end{proof}

\section{Preliminary heat semigroup asymptotic}\label{preliminary heat section}

    In this section, we move closer to proving Theorem \ref{heat thm}.
    We will show that if $(\mathcal{A},H,D)$ satisfies Hypothesis \ref{main assumption} and Hypothesis \ref{auxiliary assumption} {(in particular, $D$ has a spectral gap at $0$)}, then for a Hochschild cycle $c \in \mathcal{A}^{\otimes(p+1)}$
    we have 
    \begin{equation}\label{prelim heat asymptotic}
        \mathrm{Tr}(\mathcal{W}_p(c)|D|^{2-p}e^{-s^2D^2}) = \frac14\mathrm{Tr}(\mathrm{ch}(c))s^{-2}+O(s^{-1}), s\downarrow 0.
    \end{equation}
    By the Theorem \ref{reduction}, this is a formula very close to Theorem \ref{heat thm}: the only difference is the assumption of Hypothesis \ref{auxiliary assumption} and that the occurance of $|D|$ should be replaced with $(1+D^2)^{1/2}.$ 
    
    We start with the following asymptotic result.
    \begin{lem}\label{initial convergence theorem} 
        Let $(\mathcal{A},H,D)$ be a spectral triple satisfying Hypothesis \ref{main assumption} and Hypothesis \ref{auxiliary assumption}. For all $c\in\mathcal{A}^{\otimes (p+1)},$ we have
        \begin{equation*}
            \mathrm{Tr}(\mathrm{ch}(c)e^{-s^2D^2}) = \mathrm{Tr}(\mathrm{ch}(c)) + O(s),\quad s\downarrow0.
        \end{equation*}
    \end{lem}
    \begin{proof} 
        We wish to show that
        \begin{equation*}
            \mathrm{Tr}(\mathrm{ch}(c)(1-e^{-s^2D^2})) = O(s)
        \end{equation*}
        so it suffices to prove
        \begin{equation*}
            \|\mathrm{ch}(c)(1-e^{-s^2D^2})\|_1 = O(s).
        \end{equation*}
        
        Let $c = a_0\otimes\cdots \otimes a_p \in \mathcal{A}^{\otimes (p+1)}$. Then using $[F,a_p] = |D|^{-1}\partial(a_p)-|D|^{-1}\delta(a_p)F$, we have (on $H_\infty$)
        \begin{align*}
            \mathrm{ch}(c)(1-e^{-s^2D^2}) &= \Gamma F \left(\prod_{k=0}^{p-1}[F,a_k]\right)(|D|^p|D|^{-p})[F,a_p](1-e^{-s^2D^2})\\
                                  &= \Gamma F \left(\prod_{k=0}^{p-1}[F,a_k]\right)|D|^p\cdot |D|^{-p-1}\partial(a_p)(1-e^{-s^2D^2})\\
                                  &\quad-\Gamma F \left(\prod_{k=0}^{p-1}[F,a_k]\right)|D|^p\cdot |D|^{-p-1}\delta(a_p)(1-e^{-s^2D^2})F.
        \end{align*}
        
        Thus,
        \begin{align*}
                 \|\mathrm{ch}(c)(1-e^{-s^2D^2})\|_1 &\leq  \left\|\left(\prod_{k=0}^{p-1}[F,a_k]\right)|D|^p\right\|_{\infty}\cdot\left\||D|^{-p-1}\partial(a_p)(1-e^{-s^2D^2})\right\|_1\\
                                             &\quad+\left\|\left(\prod_{k=0}^{p-1}[F,a_k]\right)|D|^p\right\|_{\infty}\cdot\left\||D|^{-p-1}\delta(a_p)(1-e^{-s^2D^2})\right\|_1.
        \end{align*}
        In both summands, the first factor is finite by Lemma \ref{c third alert lemma}, the second factor is $O(s)$ by Lemma \ref{right convergence estimate}.
    \end{proof}


    \begin{thm}\label{first cycle thm} 
        Let $(\mathcal{A},H,D)$ be a spectral triple satisfying Hypothesis \ref{main assumption} and Hypothesis \ref{auxiliary assumption}. For every Hochschild cycle $c\in\mathcal{A}^{\otimes (p+1)},$ we have
        \begin{equation}\label{wp heat estimate}
            \mathrm{Tr}(\mathcal{W}_p(c)De^{-s^2D^2}) = \frac{\mathrm{Ch}(c)}{2}s^{-2}+O(s^{-1}),\quad s\downarrow0.
        \end{equation}
    \end{thm}
    \begin{proof}
        Let $c \in \mathcal{A}^{\otimes (p+1)}$ be a Hochschild cycle.
        By Lemma \ref{initial convergence theorem}, we have
        \begin{equation*}
            \mathrm{Tr}(\mathrm{ch}(c)e^{-s^2D^2}) = \mathrm{Tr}(\mathrm{ch}(c))+O(s),\quad s\downarrow0.
        \end{equation*}
        Since the spectral triple and $p$ both have the same parity, we may apply Lemma \ref{W and ch link} to get:
        \begin{equation}\label{restated initial convergence}
            2\mathrm{Tr}(\mathcal{W}_{\emptyset}(c)e^{-s^2D^2}) = \mathrm{Tr}(\mathrm{ch}(c))+O(s),\quad s\downarrow0,
        \end{equation}
        for all $c \in \mathcal{A}^{\otimes (p+1)}$

        Define the multilinear mappings $\mathcal{K}_s,\,\mathcal{H}_s:\mathcal{A}^{\otimes (p+1)}\to\mathbb{C}$ by setting
        \begin{align*}
            \mathcal{K}_s(a_0\otimes\cdots\otimes a_p) &= \mathrm{Tr}(\Gamma a_0\left(\prod_{k=1}^{p-1}[F,a_k]\right)[Fe^{-s^2D^2},a_p]),\\
            \mathcal{H}_s(a_0\otimes\cdots\otimes a_p) &= \mathrm{Tr}(\Gamma a_0\left(\prod_{k=1}^{p-1}[F,a_k]\right)F[e^{-s^2D^2},a_p]).
        \end{align*}

        By the Leibniz rule, we have
        \begin{equation*}
            [F,a_p]e^{-s^2D^2} = [Fe^{-s^2D^2},a_p]-F[e^{-s^2D^2},a_p].
        \end{equation*}
        Therefore:
        \begin{equation}\label{k minus h}
            \mathrm{Tr}(\mathcal{W}_{\emptyset}(c)e^{-s^2D^2}) = \mathcal{K}_s(c)-\mathcal{H}_s(c).
        \end{equation}

        Now combining \eqref{restated initial convergence} and \eqref{k minus h}, we arrive at
        \begin{equation}\label{k minus h convergence}
            2\mathcal{K}_s(c)-2\mathcal{H}_s(c) = \mathrm{Tr}(\mathrm{ch}(c))+O(s),\quad s\downarrow0,
        \end{equation}

%         We define another multilinear mapping $\mathcal{L}_s:\mathcal{A}^{\otimes p}\to\mathbb{C}$ by:
%         \begin{equation*}
%             \mathcal{L}_s(a_0\otimes\cdots\otimes a_{p-1}) = \mathrm{Tr}(\Gamma a_0\left(\prod_{k=1}^{p-1}[F,a_k]\right)Fe^{-s^2D^2}).
%         \end{equation*}
        However it is shown in Appendix \ref{coboundary app} that $\mathcal{K}_s$ is a Hochschild coboundary, and thus since $c$ is a Hochschild cycle we have $\mathcal{K}_s(c) = 0$.
        
        Using \eqref{k minus h convergence}, we obtain
        \begin{equation}\label{minus h convergence}
            -2\mathcal{H}_s(c) = \mathrm{Tr}(\mathrm{ch}(c))+O(s),\quad s\downarrow0,
        \end{equation}

        Define the multilinear mapping $\mathcal{V}_s:\mathcal{A}^{\otimes (p+1)}\to\mathbb{C}$ by setting
        \begin{equation*}
            \mathcal{V}_s(a_0\otimes\cdots\otimes a_p) = \mathrm{Tr}(\Gamma a_0\left(\prod_{k=1}^{p-1}[F,a_k]\right)F\delta(a_p)|D|e^{-s^2D^2}).
        \end{equation*}

        Let $\frac{1}{q} = 1-\frac{1}{p}$. By the H\"older inequality in the form \eqref{another holder}:
        \begin{align*}
            |(\mathcal{H}_s+2s^2\mathcal{V}_s)&(a_0\otimes\cdots\otimes a_p)| \\
                    &= \left|\mathrm{Tr}\Big(\Gamma a_0\Big(\prod_{k=1}^{p-1}[F,a_k]\Big)F\cdot\Big([e^{-s^2D^2},a_p]+2s^2\delta(a_p)|D|e^{-s^2D^2}\Big)\Big)\right|\\
                    &\leq \Big\|\Gamma a_0\prod_{k=1}^{p-1}[F,a_k]F\Big\|_{q,\infty}\Big\|[e^{-s^2D^2},a_p]+2s^2\delta(a_p)|D|e^{-s^2D^2}\Big\|_{p,1}.
        \end{align*}

        The first factor on the above right hand side is finite by Proposition \ref{f der def}, and the second factor above is $O(s)$ by Lemma \ref{third commutator lemma}.\eqref{3com3}. Therefore, we have
        \begin{equation}\label{h minus v}
            (\mathcal{H}_s+2s^2\mathcal{V}_s)(c) = O(s),\quad s\downarrow0,
        \end{equation}

        Combining \eqref{h minus v} and \eqref{minus h convergence}, we arrive at
        \begin{equation}\label{minus v convergence}
            4s^2\mathcal{V}_s(c) = \mathrm{Tr}(\mathrm{ch}(c))+O(s),\quad s\downarrow0,
        \end{equation}
        for all Hochschild cycles $c\in\mathcal{A}^{\otimes (p+1)}.$

        From the definition of $\mathcal{W}_p$, if $a_0\otimes \cdots \otimes a_p \in \mathcal{A}^{\otimes (p+1)}$:
%         We have (see the definition of $\mathcal{W}_p$ in the beginning of Section \ref{combinatorial section})
        \begin{align*}
            \mathrm{Tr}(\mathcal{W}_p(a_0\otimes\cdots\otimes a_p)De^{-s^2D^2}) &= \mathrm{Tr}(\Gamma a_0\left(\prod_{k=1}^{p-1}[F,a_k]\right)\delta(a_p)De^{-s^2D^2})\\
                                                                &= \mathrm{Tr}(\Gamma a_0\left(\prod_{k=1}^{p-1}[F,a_k]\right)\delta(a_p)F|D|e^{-s^2D^2}).
        \end{align*}
        Thus,
        \begin{align*}
            \Big|\mathcal{V}_s(a_0\otimes\cdots\otimes a_p)&-\mathrm{Tr}(\mathcal{W}_p(a_0\otimes\cdots\otimes a_p)De^{-s^2D^2})\Big| \\
                                                   &= \Big|\mathrm{Tr}(\Gamma a_0\left(\prod_{k=1}^{p-1}[F,a_k]\right)[F,\delta(a_p)]|D|e^{-s^2D^2})\Big|\\
                                                   &= \Big|\mathrm{Tr}(\Gamma \left(\prod_{k=1}^{p-1}[F,a_k]\right)[F,\delta(a_p)]|D|e^{-s^2D^2}a_0)\Big|\\
                                                   &\leq \Big\|\Gamma \left(\prod_{k=1}^{p-1}[F,a_k]\right)[F,\delta(a_p)]|D|^p\Big\|_{\infty}\cdot\Big\||D|^{1-p}e^{-s^2D^2}a_0\Big\|_1.
        \end{align*}
        By Lemma \ref{first decay lemma}, we have that this is $O(s^{-1})$ and therefore, we have
        \begin{equation}\label{v minus answer}
        \mathcal{V}_s(c)=\mathrm{Tr}(\mathcal{W}_p(c)De^{-s^2D^2})+O(s^{-1}),\quad s\downarrow0,
        \end{equation}
        for all $c\in\mathcal{A}^{\otimes (p+1)}.$

        Combining \eqref{minus v convergence} and \eqref{v minus answer}, we arrive at
        \begin{equation*}
            4s^2\mathrm{Tr}(\mathcal{W}_p(c)De^{-s^2D^2})=\mathrm{Tr}(\mathrm{ch}(c))+O(s),\quad s\downarrow0
        \end{equation*}
        for all Hochschild cycles $c\in\mathcal{A}^{\otimes (p+1)}.$ Dividing by $4s^2$,
        \begin{equation*}
            \mathrm{Tr}(\mathcal{W}_p(c)De^{-s^2D^2}) = \frac{1}{4}s^{-2}\mathrm{Tr}(\mathrm{ch}(c))+O(s^{-1}).
        \end{equation*}        
        Since $\mathrm{Ch}(c) = \frac{1}{2}\mathrm{Tr}(\mathrm{ch}(c))$, this formula coincides with \eqref{wp heat estimate}.
    \end{proof}
    
    We remark that \eqref{prelim heat asymptotic} follows as a simple combination of Theorems \ref{reduction} and \ref{first cycle thm}.

\section{Heat semigroup asymptotic: the proof of the first main result}\label{heat section}

    In this section, we finally complete the proof of Theorem \ref{heat thm}. We start by removing the assumption of Hypothesis \ref{auxiliary assumption}.
    
    The following two lemmas show that if the parity of $p$ does not match $(\mathcal{A},H,D)$, then the statement of \eqref{prelim heat asymptotic} becomes trivial.
    \begin{lem}\label{first opposite lemma}
        Let $(\mathcal{A},H,D)$ be a spectral triple satisfying Hypothesis \ref{main assumption}. Suppose that $D$ has a spectral gap at $0$. Suppose that the dimension $p$ is odd but $(\mathcal{A},H,D)$ is even. Then:
        \begin{enumerate}[{\rm (i)}]
            \item\label{first opposite 1} for every $c\in\mathcal{A}^{\otimes (p+1)},$ we have
                \begin{equation*}
                    \mathrm{Tr}(\Omega(c)|D|^{2-p}e^{-s^2D^2})=0,\quad s>0.
                \end{equation*}
            \item\label{first opposite 2} for every $c\in\mathcal{A}^{\otimes (p+1)},$ we have
                \begin{equation*}
                    \mathrm{Tr}(\mathrm{ch}(c))=0.
                \end{equation*}
        \end{enumerate}
    \end{lem}
    \begin{proof}
        Let us prove \eqref{first opposite 1}. Since $\Gamma D = -D\Gamma$ and $\Gamma$ commutes with $a \in \mathcal{A}$, we have $\Gamma [D,a]=-[D,a]\Gamma$ on $H_\infty$. Hence $\Gamma \partial(a) = -\partial(a)\Gamma$
        for all $a \in \mathcal{A}$.
        Thus for $c = a_0\otimes \cdots\otimes a_p \in \mathcal{A}^{\otimes (p+1)}$ since $p$ is odd we have:
        $$\Omega(c)=\Gamma a_0\prod_{k=1}^p\partial(a_k)=-a_0\left(\prod_{k=1}^p\partial(a_k)\right)\Gamma.$$
        However $\Gamma D^2 = D^2\Gamma$, so by the spectral theorem we have $\Gamma e^{-s^2D^2} = e^{-s^2D^2}$. So multiplying by $e^{-s^2D^2}$ on the right and taking the trace,
        \begin{align*}
        \mathrm{Tr}(\Omega(c)|D|^{2-p}e^{-s^2D^2}) &= -\mathrm{Tr}(a_0\left(\prod_{k=1}^p\partial(a_k)\right)|D|^{2-p}e^{-s^2D^2}\Gamma)\\
                                           &= -\mathrm{Tr}(\Gamma a_0\left(\prod_{k=1}^p\partial(a_k)\right)|D|^{2-p}e^{-s^2D^2}).
        \end{align*}
        This proves \eqref{first opposite 1}

        The argument for \eqref{first opposite 2} is similar. Note that we have $\Gamma[F,a]=-[F,a]\Gamma$ for every $a\in\mathcal{A}.$ Thus since $p+1$ is even:
        \begin{align*}
            \mathrm{ch}(c) &= \Gamma F\prod_{k=0}^p[F,a_k]\\
                   &= -F\Gamma \prod_{k=0}^p[F,a_k]\\
                   &= -F\cdot\prod_{k=0}^p[F,a_k]\Gamma.
        \end{align*}
        Thus,
        \begin{align*}
            \mathrm{Tr}(\mathrm{ch}(c)) &= -\mathrm{Tr}(\Gamma F\prod_{k=0}^p[F,a_k])\\
                        &= -\mathrm{Tr}(\mathrm{ch}(c)).
        \end{align*}
        This proves the second assertion.
    \end{proof}
    
    Now, we deal with the other case where the parity of $(\mathcal{A},H,D)$ does not match $p$.
    \begin{lem}\label{second opposite lemma} 
        Let $(\mathcal{A},H,D)$ be a spectral triple satisfying Hypothesis \ref{main assumption}. Suppose $D$ has a spectral gap at $0$ and $(\mathcal{A},H,D)$ is odd but $p$ is even.
        \begin{enumerate}[{\rm (i)}]
            \item\label{second opposite 1} for every Hochschild cycle $c\in\mathcal{A}^{\otimes (p+1)},$ we have
                \begin{equation*}
                    \mathrm{Tr}(\Omega(c)|D|^{2-p}e^{-s^2D^2}) = O(s^{-1}),\quad s\downarrow0.
                \end{equation*}
            \item\label{second opposite 2} for every $c\in\mathcal{A}^{\otimes (p+1)},$ we have
                \begin{equation*}
                    \mathrm{Tr}(\mathrm{ch}(c))=0.
                \end{equation*}
        \end{enumerate}
    \end{lem}    
    \begin{proof} 
        First, we prove \eqref{second opposite 1}. Consider the multilinear mapping $\theta_s:\mathcal{A}^{\otimes p}\to\mathbb{C}$ defined by the formula
        \begin{equation*}
            \theta_s(a_0\otimes\cdots\otimes a_{p-1}) = \mathrm{Tr}(\left(\prod_{k=0}^{p-1}\partial(a_k)\right) |D|^{2-p}e^{-s^2D^2}).
        \end{equation*}
        The Hochschild coboundary $b\theta_s$ is computed in Section \ref{coboundary app} by the formula:
        \begin{align*}
            (b\theta_s)(a_0\otimes\cdots\otimes a_p) &= 2\mathrm{Tr}(a_0\left(\prod_{k=1}^p\partial(a_k)\right) |D|^{2-p}e^{-s^2D^2})\\
                                                     &\quad +\mathrm{Tr}(a_0\left(\prod_{k=1}^{p-1}\partial(a_k)\right)[|D|^{2-p}e^{-s^2D^2},\partial(a_p)])\\
                                                     &\quad +\mathrm{Tr}(\left(\prod_{k=0}^{p-1}\partial(a_k)\right) [|D|^{2-p}e^{-s^2D^2},a_p]).
        \end{align*}

        Using the H\"older inequality, we have
        \begin{align*}
            \Big|\mathrm{Tr}(a_0\left(\prod_{k=1}^{p-1}\partial(a_k)\right)&[|D|^{2-p}e^{-s^2D^2},\partial(a_p)])\Big|\\
                                                                   &\leq\|a_0\|_{\infty}\prod_{k=1}^{p-1}\|\partial(a_k)\|_{\infty}\cdot\Big\|[|D|^{2-p}e^{-s^2D^2},\partial a_p])\Big\|_1.
        \end{align*}
        By Lemma \ref{commutator 7}, we have
        \begin{equation}\label{sec op eq1}
            \mathrm{Tr}(a_0\left(\prod_{k=1}^{p-1}\partial(a_k)\right) [|D|^{2-p}e^{-s^2D^2},\partial(a_p)]) = O(s^{-1}),\quad s\downarrow0.
        \end{equation}
        Similarly, we have
        \begin{equation*}
            \Big|\mathrm{Tr}(\left(\prod_{k=0}^{p-1}\partial(a_k)\right) [|D|^{2-p}e^{-s^2D^2},a_p])\Big| \leq \prod_{k=0}^{p-1}\|\partial(a_k)\|_{\infty}\cdot\Big\|[|D|^{2-p}e^{-s^2D^2},a_p])\Big\|_1.
        \end{equation*}
        By Lemma \ref{commutator 7}, we have
        \begin{equation}\label{sec op eq2}
            \mathrm{Tr}(\left(\prod_{k=0}^{p-1}\partial(a_k)\right) [|D|^{2-p}e^{-s^2D^2},a_p])=O(s^{-1}),\quad s\downarrow0.
        \end{equation}

        For every $c\in\mathcal{A}^{\otimes (p+1)},$ it follows from \eqref{sec op eq1} and \eqref{sec op eq2} that
        \begin{equation*}
            (b\theta_s)(c) = 2\mathrm{Tr}(\Omega(c)|D|^{2-p}e^{-s^2D^2}) + O(s^{-1}),\quad s\downarrow0.
        \end{equation*}
        If $c$ is a Hochschild cycle, then $(b\theta_s)(c)=0.$ Thus,
        \begin{equation*}
            \mathrm{Tr}(\Omega(c)|D|^{2-p}e^{-s^2D^2}) = O(s^{-1}),\quad s\downarrow0.
        \end{equation*}
        for every Hochschild cycle $c.$ 
        This completes the proof of \eqref{second opposite 1}.

        The proof of \eqref{second opposite 2} is similar to Lemma \ref{first opposite lemma}.\eqref{first opposite 2}. 
        For all $a \in \mathcal{A}$, we have $F[F,a]=-[F,a]F.$ Since $p+1$ is odd,
        \begin{equation*}
            F\cdot \prod_{k=0}^p[F,a_k]=-\left(\prod_{k=0}^p[F,a_k]\right) F.
        \end{equation*}
        Hence,
        \begin{equation*}
            \mathrm{Tr}(F\prod_{k=0}^p[F,a_k])=-\mathrm{Tr}(F\prod_{k=0}^p[F,a_k]).
        \end{equation*}
        This proves \eqref{second opposite 2}.
    \end{proof}
    
    
    The preceding two lemmas show how to remove the assumption that the parities of $p$ and $(\mathcal{A},H,D)$ match. It remains to remove the assumption that $D$ has a spectral gap at $0.$ For this purpose, we use the doubling trick.
    The "doubling trick" in this form follows \cite[Definition 6]{CPRS1}.
    
    Let $\mu >0.$ We define another spectral triple $(\pi(\mathcal{A}),H_0,D_{\mu}),$ where
    \begin{align*}
          H_0 &= \mathbb{C}^2\otimes H,\\
        D_\mu &= \begin{pmatrix} D & \mu \\ \mu & -D\end{pmatrix}.
    \end{align*}
    and $\pi$ is the same representation of $\mathcal{A}$ as in Definition \ref{doubling definition}. That is:
    \begin{equation*}
        \pi(a) := \begin{pmatrix} a & 0 \\ 0 & 0\end{pmatrix}.
    \end{equation*}
    For a tensor $c \in \mathcal{A}^{\otimes (p+1)}$, we denote $\pi(c)$ for the corresponding element of $(\pi(\mathcal{A}))^{\otimes (p+1)}$ obtained by applying the map $\pi^{\otimes (p+1)}$ to $c$.
    The spectral triple $(\pi(\mathcal{A}),H_0,D_{\mu})$ is equipped with grading operator
    $$\Gamma_0=\begin{pmatrix} \Gamma & 0 \\ 0 & (-1)^{\rm deg}\Gamma\end{pmatrix}$$
    where $\Gamma$ is the (possibly trivial) grading of $(\mathcal{A},H,D)$ (see Definition \ref{doubling definition}).
        
    Let $\Omega_{\mu}$ and ${\rm ch}_{\mu}$ be the multilinear mappings $\Omega$ and ${\rm ch}$ (just as in Definition \ref{ch omega def}) as applied to the spectral triple $(\pi(\mathcal{A}),H_0,D_{\mu}).$
    
    \begin{lem}\label{first doubling lemma} 
        Let $(\mathcal{A},H,D)$ be a spectral triple satisfying Hypothesis \ref{main assumption}. Let $F_0$ be as in Definition \ref{doubling definition}. If $a\in\mathcal{A},$ then
        as $\mu\downarrow 0$ we have:
        $$[{\rm sgn}(D_{\mu}),\pi(a)]\to[F_0,\pi(a)]$$
        in $\mathcal{L}_{p+1}$.
    \end{lem}
    \begin{proof} 
        We have
        \begin{equation*}
            \mathrm{sgn}(D_{\mu}) = \begin{pmatrix} 
                                            \frac{D}{(D^2+\mu^2)^{1/2}}   & \frac{\mu}{(D^2+\mu^2)^{1/2}} \\ 
                                            \frac{\mu}{(D^2+\mu^2)^{1/2}} & -\frac{D}{(D^2+\mu^2)^{1/2}}
                                    \end{pmatrix}.
        \end{equation*}
%         $${\rm sgn}(D_{\mu})=(e_{11}-e_{22})\otimes\frac{D}{(D^2+\mu^2)^{\frac12}}+(e_{12}+e_{21})\otimes\frac{\mu}{(D^2+\mu^2)^{\frac12}},$$
    Hence,
    \begin{equation*}
        [\mathrm{sgn}(D_\mu),\pi(a)]  = \begin{pmatrix}
                                            \left[\frac{D}{(D^2+\mu^2)^{1/2}},a\right] & -a\frac{\mu}{(D^2+\mu^2)^{1/2}}\\ \frac{\mu}{(D^2+\mu^2)^{1/2}}a & 0.
                                        \end{pmatrix}
    \end{equation*}
    On the other hand, we have
    \begin{equation*}
        [F_0,\pi(a)] = \begin{pmatrix} 
                          [\mathrm{sgn}(D),a] & -aP\\ Pa & 0
                       \end{pmatrix}.
    \end{equation*}
%     $$[F_0,e_{11}\otimes a]=e_{11}\otimes[{\rm sgn}(D),a]+e_{21}\otimes Pa-e_{12}\otimes aP.$$
    Therefore:
    \begin{align*}
        \Big\|[{\rm sgn}(D_{\mu}),\pi(a)&]-[F_0,\pi(a)]\Big\|_{p+1}\\
                                        &\leq \Big\|\Big({\rm sgn}(D)-\frac{D}{(D^2+\mu^2)^{\frac12}}\Big)a\Big\|_{p+1}+\Big\|a\Big({\rm sgn}(D)-\frac{D}{(D^2+\mu^2)^{\frac12}}\Big)\Big\|_{p+1}\\
                                        &\quad + \Big\|\Big(P-\frac{\mu}{(D^2+\mu^2)^{\frac12}}\Big)a\Big\|_{p+1}+\Big\|a\Big(P-\frac{\mu}{(D^2+\mu^2)^{\frac12}}\Big)\Big\|_{p+1}\\
                                        &= \Big\|a^*\Big({\rm sgn}(D)-\frac{D}{(D^2+\mu^2)^{\frac12}}\Big)^2a\Big\|_{\frac{p+1}{2}}^{\frac12}+\Big\|a\Big({\rm sgn}(D)-\frac{D}{(D^2+\mu^2)^{\frac12}}\Big)^2a^*\Big\|_{\frac{p+1}{2}}^{\frac12}\\
                                        &\quad + \Big\|a^*\Big(P-\frac{\mu}{(D^2+\mu^2)^{\frac12}}\Big)^2a\Big\|_{\frac{p+1}{2}}^{\frac12}+\Big\|a\Big(P-\frac{\mu}{(D^2+\mu^2)^{\frac12}}\Big)^2a^*\Big\|_{\frac{p+1}{2}}^{\frac12}.
    \end{align*}
    By functional calculus, as $\mu\downarrow  0$ we have:
    \begin{align*}
        \Big({\rm sgn}(D)-\frac{D}{(D^2+\mu^2)^{\frac12}}\Big)^2 &\downarrow 0,\\
                 \Big(P-\frac{\mu}{(D^2+\mu^2)^{\frac12}}\Big)^2 &\downarrow 0
    \end{align*}
    in the weak operator topology.
    Hence,
    $$a^*\Big({\rm sgn}(D)-\frac{D}{(D^2+\mu^2)^{\frac12}}\Big)^2a\downarrow 0,\quad a\Big({\rm sgn}(D)-\frac{D}{(D^2+\mu^2)^{\frac12}}\Big)^2a^*\downarrow 0,\quad\mu\downarrow0,$$
    $$a^*\Big(P-\frac{\mu}{(D^2+\mu^2)^{\frac12}}\Big)^2a\downarrow 0,\quad a\Big(P-\frac{\mu}{(D^2+\mu^2)^{\frac12}}\Big)^2a^*\downarrow0,\quad\mu\downarrow0.$$
    also in the weak operator topology.
    The assertion follows now from the order continuity of the $\mathcal{L}_{\frac{p+1}{2}}$ norm.
    \end{proof}

   
    \begin{lem}\label{second doubling lemma} 
        Let $(\mathcal{A},H,D)$ be a spectral triple satisfying Hypothesis \ref{main assumption}. If $c\in\mathcal{A}^{\otimes (p+1)},$ then
        \begin{equation*}
            \lim_{\mu\to 0} {\rm ch}_{\mu}(\pi(c)) = {\rm ch}_0(c)
        \end{equation*}
        in the $\mathcal{L}_1$-norm.
    \end{lem}
    \begin{proof} 
        It suffices to prove the assertion for $c=a_0\otimes\cdots\otimes a_p.$ Then we have
        \begin{align*}
                 {\rm ch}_{0}(c)-{\rm ch}_\mu(\pi(c)) &= \Gamma_0(F_0-{\rm sgn}(D_{\mu}))\prod_{k=0}^p[F_0,\pi(a_k)]\\
                                                           & - \Gamma_0\mathrm{sgn}(D_\mu)\Big(\prod_{k=0}^p[{\rm sgn}(D_{\mu}),\pi(a_k)]-\prod_{k=0}^p[F_0,\pi(a_k)]\Big).
        \end{align*}
        
        Next we use the fact that if $q \geq 1$ and if $A_\mu\to A$ in the strong operator topology, $B_\mu\to B\in \mathcal{L}_q$ in the $\mathcal{L}_q$ norm, and $\sup_{\mu\downarrow 0}\|A_\mu\|_\infty < \infty$ then $A_\mu B_\mu\to AB$
        in $\mathcal{L}_1$ (see \cite[Chapter 2, Example 3]{Simon}).
        
        We have that $\mathrm{sgn}(D_\mu)- F_0\to 0$ in the strong operator topology, and since $\prod_{k=0}^p [F_0,\pi(a_k)] \in \mathcal{L}_1$, it follows that the first
        summand converges to $0$ in the $\mathcal{L}_1$ norm.
                
        For the second summand, we apply Lemma \ref{first doubling lemma}. Since for each $k$ we have that $[\mathrm{sgn}(D_\mu),\pi(a_k)] \to [F_0,\pi(a_k)]$
        in $\mathcal{L}_{p+1}$, it follows that the second summand converges to $0$ in $\mathcal{L}_1$.
    \end{proof}
    
    \begin{lem}\label{third doubling lemma} 
        Let $(\mathcal{A},H,D)$ be a spectral triple satisfying Hypothesis \ref{main assumption}. 
        For every $c\in\mathcal{A}^{\otimes (p+1)},$ we have
        $$\Big\|\Omega_{\mu}(\pi(c))(1\otimes (1+D^2)^{-\frac{p}{2}}e^{-s^2D^2})\Big\|_1=O(s^{-1}),\quad s\downarrow0.$$
    \end{lem}
    \begin{proof} 
        Once more it suffices to prove the assertion for an elementary tensor $c=a_0\otimes\cdots\otimes a_p.$ We have
        \begin{align*}
              \Big\|\Omega_{\mu}&(\pi(c))(1\otimes (1+D^2)^{-\frac{p}{2}}e^{-s^2D^2})\Big\|_1\\
                                &\leq \Big\|\Gamma_0 \pi(a_0)\prod_{k=1}^{p-1}[D_{\mu},\pi(a_k)]\Big\|_{\infty}\Big\|[D_{\mu},\pi(a_p)](1\otimes (1+D^2)^{-\frac{p}{2}}e^{-s^2D^2})\Big\|_1\\
                                &\leq \|a_0\|_{\infty}\prod_{k=1}^{p-1}\|[D_{\mu},\pi(a_k)]\|_{\infty}\Big\|[D,a_p](1+D^2)^{-\frac{p}{2}}e^{-s^2D^2}\Big\|_1\\
                                &\quad +2\mu\|a_0\|_{\infty}\prod_{k=1}^{p-1}\|[D_{\mu},\pi(a_k)]\|_{\infty}\Big\|a_p(1+D^2)^{-\frac{p}{2}}e^{-s^2D^2}\Big\|_1.
        \end{align*}
        The assertion follows by applying Lemma \ref{first decay lemma} (with $m_1=0$ and $m_2=p-1$) to the odd spectral triple $(\mathcal{A},H,F(1+D^2)^{\frac12}).$    
    \end{proof}

    We note that since $\pi$ is an algebra homomorphism, if $c \in \mathcal{A}^{\otimes (p+1)}$ is a Hochschild cycle then so is $\pi(c)$.
    \begin{lem}\label{fourth doubling lemma} 
        Let $(\mathcal{A},H,D)$ be a spectral triple satisfying Hypothesis \ref{main assumption}. Let $c\in\mathcal{A}^{\otimes (p+1)}$ be a Hochschild cycle. We have
        $$(\mathrm{Tr}_2\otimes\mathrm{Tr})(\Omega_{\mu}(\pi(c))(1\otimes (1+D^2)^{1-\frac{p}{2}}e^{-s^2D^2}))= \frac{p}{2}\mathrm{Ch}_{\mu}(\pi(c))s^{-2} + O(s^{-1}),\quad s\downarrow 0.$$
    \end{lem}
    \begin{proof}     
        For $\mu>0,$ the spectral triple $(\pi(\mathcal{A}),H_0,D_{\mu})$ satisfies Hypothesis \ref{main assumption} and the spectral gap assumption. This allows us to apply Theorems \ref{reduction} and \ref{first cycle thm} (or Lemmas \ref{first opposite lemma} and \ref{second opposite lemma}) to the Hochschild cycle $\pi(c)\in (\pi(\mathcal{A}))^{\otimes (p+1)}.$
    
        A combination of Theorems \ref{reduction} and \ref{first cycle thm} (if the parities of $p$ and $(\pi(\mathcal{A}),H_0,D_{\mu})$ match) or one of Lemmas \ref{first opposite lemma} and \ref{second opposite lemma} (if parities of $p$ and 
        $(\pi(A),H_0,D_{\mu})$ do not match) yields
        $$(\mathrm{Tr}_2\otimes\mathrm{Tr})(\Omega_{\mu}(\pi(c))|D_{\mu}|^{2-p}e^{-s^2D_{\mu}^2}) = \frac{p}{2}\mathrm{Ch}_{\mu}(\pi(c))s^{-2} + O(s^{-1}),\quad s\downarrow 0.$$
        Noting that $D_{\mu}^2=D_0^2+\mu^2,$ and $e^{-s^2\mu^2} = O(1)$ we obtain
        $$(\mathrm{Tr}_2\otimes\mathrm{Tr})(\Omega_{\mu}(\pi(c))|D_{\mu}|^{2-p}e^{-s^2D_0^2}) = \frac{p}{2}\mathrm{Ch}_{\mu}(\pi(c))s^{-2} + O(s^{-1}),\quad s\downarrow 0.$$
        By Lemma \ref{third doubling lemma}, we have
        \begin{align*}
            \Big\|\Omega_{\mu}(\pi(c))&|D_{\mu}|^{2-p}e^{-s^2D_0^2}-\Omega_{\mu}(\pi(c))|D_1|^{2-p}e^{-s^2D_0^2}\Big\|_1\\
                                      &\leq\Big\|\Omega_{\mu}(\pi(c))|D_1|^{-p}e^{-s^2D_0^2}\Big\|_1\Big\|(|D_{\mu}|^{2-p}-|D_1|^{2-p})\cdot |D_1|^{p-2}\Big\|_{\infty}\\
                                      &=O(s^{-1}).
        \end{align*}
    \end{proof}


    We are now ready to prove the main result of this chapter. 

    \begin{proof}[Proof of Theorem \ref{heat thm}] 
        For $\mathscr{A}\subseteq\{1,\cdots,p\},$ we define the multilinear functional on $a_0\otimes\cdots a_p\in \mathcal{A}^{\otimes(p+1)}$ by:
        $$\mathcal{T}_{\mathscr{A}}(a_0\otimes\cdots\otimes a_p)=\Gamma_0\pi(a_0)\prod_{k=1}^p y_k(a_k).$$
        Here,
        \begin{equation*}
            y_k(a) := \begin{cases}
                        \begin{pmatrix} \partial(a) & 0 \\ 0 & 0\end{pmatrix}, k\notin\mathscr{A}\\
                        \begin{pmatrix} 0 & -a \\ a& 0\end{pmatrix}         , k\in\mathscr{A}.
                      \end{cases}
        \end{equation*}
        In particular,
        \begin{equation*}
            \mathcal{T}_{\emptyset}(a_0\otimes\cdots a_p) = \begin{pmatrix} 
                                                            \Omega(a_0\otimes\cdots\otimes a_p) & 0\\
                                                                0 & 0
                                                            \end{pmatrix}.
        \end{equation*}       
        For $c\in\mathcal{A}^{\otimes(p+1)},$ we apply \eqref{combinatorial fact} to get
        $$\Omega_{\mu}(\pi(c)) = \sum_{\mathscr{A}\subseteq\{1,\cdots,p\}}\mu^{|\mathscr{A}|}\mathcal{T}_{\mathscr{A}}(c).$$
        For each $0\leq k\leq p,$ we set
        $$f_k(s)=\sum_{|\mathscr{A}|=k}(\mathrm{Tr}_2\otimes\mathrm{Tr})(\mathcal{T}_{\mathscr{A}}(c)\cdot (1\otimes (1+D^2)^{1-\frac{p}{2}}e^{-s^2D^2})).$$
        That is, $f_k(s)$ is the coefficient of $\mu^k$ in $(\mathrm{Tr}_2\otimes \mathrm{\mathrm{Tr}})(\Omega_\mu(\pi(c))(1\otimes (1+D^2)^{1-\frac{p}{2}}e^{-s^2D^2}))$.
        
        Now if $c\in\mathcal{A}^{\otimes (p+1)}$ is a Hochschild cycle, then Lemma \ref{fourth doubling lemma} yields
        \begin{equation}\label{heat main epsilon}
            \sum_{k=0}^p\mu^kf_k(s) = \frac{p}{2}\mathrm{Ch}_{\mu}(\pi(c))s^{-2} + O(s^{-1}),\quad s\downarrow 0.
        \end{equation}
        
        Select a set $\{\mu_0,\ldots,\mu_p\}$ of distinct positive numbers, and for each $0 \leq l \leq p$ we may take $\mu = \mu_l$ in \eqref{heat main epsilon}
        to arrive at:
        $$\sum_{k=0}^p\mu_l^kf_k(s) = \frac{p}{2}\mathrm{Ch}_{\mu_l}(\pi(c))s^{-2} + O(s^{-1}),\quad s\downarrow 0,\quad 0\leq l\leq p.$$
        Since the Vandermonde matrix $\{\mu_l^k\}_{0\leq l,k\leq p}$ is invertible, it follows that there exist $\{\alpha_0,\ldots, \alpha_k\}$
        such that:
        $$f_k(s)=\frac{\alpha_k}{s^2}+O(s^{-1}),\quad s\downarrow0,\quad 0\leq k\leq p.$$
        Substituting this back to \eqref{heat main epsilon}, we obtain
        $$\sum_{k=0}^p\mu^k\alpha_k=\frac{p}{2}\mathrm{Ch}_{\mu}(\pi(c)).$$
        In particular
        $$\alpha_0=\frac{p}{2}\lim_{\mu\to0}\mathrm{Ch}_{\mu}(\pi(c))$$
        So by Lemma \ref{second doubling lemma},
        \begin{equation*}
            \alpha_0 = \frac{p}{2}{\rm Ch}(c).
        \end{equation*}
        Hence,
        $$f_0(s)=\frac{p}{2}\mathrm{Ch}(c)s^{-2} + O(s^{-1}),\quad s\downarrow 0.$$
        The assertion follows now from the definition of $f_0.$
    \end{proof}

    