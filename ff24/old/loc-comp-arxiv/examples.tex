\chapter{Spectral Triples: Basic properties and examples}\label{examples chapter}
    
    This chapter is primarily concerned with Hypothesis \ref{main assumption}. We study the consequences of this hypothesis, and also show that it
    is satisfied for two important classes of examples.
    
    We begin with the proof of Proposition \ref{f der def}, an important prerequisite to the definition
    of the Chern character (Definition \ref{chern character def}).     
    Next, we show that Hypothesis \ref{main assumption} is equivalent to a modified set of assumptions, Hypothesis \ref{replacement assumption}.
    { Hypothesis \ref{replacement assumption} is stated in terms of an operator $\Lambda$ (given in Definition \ref{lambda def}) rather than $\delta$. This has
    the advantage of making Hypothesis \ref{replacement assumption} more easily verified in the classes of examples studied in this chapter.}
    
    The remainder of the chapter is devoted to demonstrating that the assumptions made in Hypothesis \ref{main assumption} are satisfied for spectral triples associated to the following classes of examples:
    \begin{enumerate}[{\rm (a)}]
        \item{}\label{manifolds} Complete Riemannian manifolds.
        \item{}\label{moyal planes} Noncommutative Euclidean spaces (also known as Moyal planes or Moyal-Groenwald planes in the $2$-dimensional case).
    \end{enumerate}
    
    We re-emphasise that Hypothesis \ref{main assumption} is automatically satisfied for smooth $p$-dimensional unital spectral triples, and therefore 
    we concern ourselves with showing that it is satisfied for non-unital algebras.
    
    The first class of examples \eqref{manifolds} is purely commutative. For the Dirac operator in these examples, we use the Hodge-Dirac operator (see \cite{krym} or \cite{rosenberg}).
    In \cite{CGRS2}, spectral triples for noncompact Riemannian manifolds were studied under the significant restriction that they have bounded geometry: this is a global
    geometric property which we are able to avoid by working in local coordinates. Earlier, Rennie had studied noncompact Riemannian spin manifolds which are not necessarily of bounded geometry by similar methods \cite[Section 5]{Rennie-2004}. It is hoped that by including such a wide class of manifolds we may demonstrate the applicability of noncommutative methods in "classical" (commutative) geometry.
    
    The second example \eqref{moyal planes} is one of the most heavily studied classes of non-unital and strictly noncommutative spectral triples. A detailed exposition
    of the noncommutative Euclidean spaces may be found in \cite{gayral-moyal}.
    
\section{A spectral triple defines a Fredholm module}\label{fredholm section}

    This section is devoted to the proof of Proposition \ref{f der def}. We prove this in several steps, initially working with the assumption that $D$ has a spectral
    gap at $0$ (i.e., that $D$ has bounded inverse). We later show how this assumption can be removed. 
    
    Note that if $D$ has a spectral gap at $0$, then $F = D|D|^{-1} = |D|^{-1}D$.

    \begin{rem}\label{sp gap fact} 
        Let $(\mathcal{A},H,D)$ be a spectral triple satisfying Hypothesis \ref{main assumption}. Suppose $D$ has a spectral gap at $0.$ For every $a\in\mathcal{A},$ and all $k\geq 1$ we have
        the following four inclusions:
        \begin{align*}
            a|D|^{-p}\in \mathcal{L}_{1,\infty},&\quad \partial(a)|D|^{-p} \in \mathcal{L}_{1,\infty},\\
            \delta^k(a)|D|^{-p-1} \in \mathcal{L}_{1},&\quad \partial(\delta^k(a))|D|^{-p-1}\in \mathcal{L}_1.
        \end{align*}
    \end{rem}
    \begin{proof} 
        All four inclusions follow from the observation that since by assumption $|D|$ is invertible, the operator $\frac{D+i}{|D|}:\mathrm{dom}(D)\to H$ has bounded extension. 
        Since $(\mathcal{A},H,D)$ is $p$-dimensional, we have
        \begin{equation*}
            a(D+i)^{-p},\;\partial(a)(D+i)^{-p} \in \mathcal{L}_{1,\infty}
        \end{equation*}
        and multiplying by (the bounded extension of) $\left(\frac{D+i}{|D|}\right)^p$ yields the first two inclusions.
        
        The second pair of inclusions follow from Hypothesis \ref{main assumption}.\eqref{ass2}: we have $\delta^k(a)(D+i)^{-p-1} \in \mathcal{L}_1$ and $\partial(\delta^k(a))(D+i)^{-p-1} \in \mathcal{L}_1$.
        Then simply multiplying by $\left(\frac{D+i}{|D|}\right)^{p+1}$ again yields the result.
    \end{proof}

    Note that the preceding lemma showed that $\partial(a)|D|^{-p} \in \mathcal{L}_{1,\infty}$. We require a little more effort to show that $\delta(a)|D|^{-p} \in \mathcal{L}_{1,\infty}$.
    \begin{lem}\label{lambda lemma} 
        Let $(\mathcal{A},H,D)$ be a spectral triple satisfying Hypothesis \ref{main assumption}. Suppose $D$ has a spectral gap at $0.$ Then for all $a\in\mathcal{A},$ we have $\delta(a)|D|^{-p}\in\mathcal{L}_{1,\infty}.$
    \end{lem}
    \begin{proof} 
        Using \eqref{favourite commutator identity}, we have the following equality of operators on $H_\infty$:
        \begin{align}\label{lambda 11}
            [|D|^{-1},\partial(a)] &= -|D|^{-1}[|D|,\partial(a)]|D|^{-1}\nonumber\\
                                   &= -|D|^{-1}\partial(\delta(a))|D|^{-1}.
        \end{align}
        Where the last equality uses the fact that $\delta$ and $\partial$ commute (from Lemma \ref{delta and partial commute}). Similarly,
        \begin{align}\label{lambda 12}
            [|D|^{-1},\partial(\delta(a))] &= -|D|^{-1}[|D|,\partial(\delta(a))]|D|^{-1} \nonumber\\
                                           &= -|D|^{-1}\partial(\delta^2(a))|D|^{-1}.
        \end{align}
        Additionally, working with operators on $H_\infty$:
        \begin{align*}
            |D|^{-1}[D^2,a] &= |D|^{-1}\cdot(D\partial(a)+\partial(a)D)\\
                            &= F\partial(a)+|D|^{-1}\partial(a)D\\
                            &= F\partial(a)+\partial(a)|D|^{-1}D+[|D|^{-1},\partial(a)]D.
        \end{align*}
        Now applying \eqref{lambda 11}:    
        \begin{align*}
            |D|^{-1}[D^2,a] &= F\partial(a)+\partial(a)F-|D|^{-1}\partial(\delta(a))F\\
                            &= F\partial(a)+\partial(a)F-\partial(\delta(a))|D|^{-1}F-[|D|^{-1},\partial(\delta(a))]F
        \end{align*}
        then using \eqref{lambda 12}:
        \begin{equation*}
            |D|^{-1}[D^2,a] = F\partial(a)+\partial(a)F-\partial(\delta(a))D^{-1}+|D|^{-1}\partial(\delta^2(a))D^{-1}.
        \end{equation*}
        So multiplying on the right by $|D|^{-p}$:
        \begin{align*}
            |D|^{-1}[D^2,a]|D|^{-p} &= \Big(F\cdot\partial(a)|D|^{-p}+\partial(a)|D|^{-p}\cdot F\Big)\\
                                    &+ \Big(-\partial(\delta(a))|D|^{-p-1}\cdot F+|D|^{-1}\cdot\partial(\delta^2(a))|D|^{-p-1}\cdot F\Big).
        \end{align*}
        From Remark \ref{sp gap fact}, the first summand extends to an operator in $\mathcal{L}_{1,\infty}$ and the second summand extends to an operator in $\mathcal{L}_1.$
        Hence, the operator $|D|^{-1}[D^2,a]|D|^{-p}$ has extension to an operator in $\mathcal{L}_{1,\infty}$.

        On the other hand since $|D|^2 = D^2$, we have (again, as operators on $H_\infty$)
        \begin{align*}
            |D|^{-1}[D^2,a] &= |D|^{-1}[|D|^2,a]\\
                            &= |D|^{-1}\cdot(|D|\delta(a)+\delta(a)|D|)\\
                            &= \delta(a)+|D|^{-1}\delta(a)|D|\\
                            &= \delta(a)+\delta(a)|D|^{-1}|D|+[|D|^{-1},\delta(a)]|D|\\
                            &= 2\delta(a)-|D|^{-1}\delta^2(a)\\
                            &= 2\delta(a)-\delta^2(a)|D|^{-1}-[|D|^{-1},\delta^2(a)]\\
                            &= 2\delta(a)-\delta^2(a)|D|^{-1}+|D|^{-1}\delta^3(a)|D|^{-1}.
        \end{align*}
        
        So multiplying by $|D|^{-p}$:
        \begin{equation*}
            |D|^{-1}[D^2,a]|D|^{-p} = 2\delta(a)|D|^{-p}-\delta^2(a)|D|^{-p-1}+|D|^{-1}\delta^3(a)|D|^{-p-1}.
        \end{equation*}
        By Remark \ref{sp gap fact}, the operators $\delta^2(a)|D|^{-p-1}$ and $\delta^3(a)|D|^{-p-1}$ are in $\mathcal{L}_1$. 
        Since $|D|^{-1}[D^2,a]|D|^{-p}$ has extension to an operator in $\mathcal{L}_{1,\infty}$, it follows that $2\delta(a)|D|^{-p} \in \mathcal{L}_{1,\infty}$.
    \end{proof}

    Still working with the assumption that $D$ has a spectral gap at $0$, the following lemma is a refinement of the $\mathcal{L}_{1,\infty}$ inclusions in Remark \ref{sp gap fact} and the result of Lemma \ref{lambda lemma}. The following result should be compared with \cite[Lemma 1.37]{CGRS2}, which is of a similar nature but is stated in terms of Schatten ideals rather than weak Schatten ideals. There is a substantial difference between Schatten ideals and weak Schatten ideals, necessitating the introduction of new tools: here we use logarithmic submajorisation and the Araki-Lieb-Thirring inequality.
    
    A related antecedent to the following lemma is also \cite[Proposition 10]{Rennie-2004}, which proved a result similar to the first assertion in the setting of local spectral triples.
    
    \begin{lem}\label{z<p lemma} 
        Let $(\mathcal{A},H,D)$ be a smooth $p-$dimensional spectral triple satisfying Hypothesis \ref{main assumption}. 
        Suppose $D$ has a spectral gap at $0.$ For every $a\in\mathcal{A}$ and for every $0 < s\leq p,$ we have 
        $a|D|^{-s}\in\mathcal{L}_{\frac{p}{s},\infty},$ $\partial(a)|D|^{-s}\in\mathcal{L}_{\frac{p}{s},\infty},$ and $\delta(a)|D|^{-s}\in\mathcal{L}_{\frac{p}{s},\infty}.$
    \end{lem}
    \begin{proof} 
        We prove here only the third statement: that $\delta(a)|D|^{-s} \in \mathcal{L}_{p/s,\infty}$, the other results can be proved similarly.
        
        Let $r = \frac{p}{s} \geq 1$. By the Araki-Lieb-Thirring inequality \eqref{ALT inequality},
        \begin{equation*}
            |\delta(a)|D|^{-s}|^{r} \prec\prec_{\log} |\delta(a)|^r|D|^{-p}.
        \end{equation*}
        Due to \eqref{log majorization monotone} (the $\mathcal{L}_{1,\infty}$ quasi-norm is monotone with respect to logarithmic submajorisation)
        \begin{align*}
            \||\delta(a)|D|^{-s}|^r\|_{1,\infty} &\leq e\||\delta(a)|^r|D|^{-p}\|_{1,\infty}\\
                                                 &\leq e\|\delta(a)\|_\infty^{r-1}\|\delta(a)|D|^{-p}\|_{1,\infty}.
        \end{align*}
        Hence,
        \begin{equation*}
            \|\delta(a)|D|^{-s}\|_{r,\infty}^r \leq e\|\delta(a)\|_{\infty}^{r-1}\|\delta(a)|D|^{-p}\|_{1,\infty}.
        \end{equation*}
        By Lemma \ref{lambda lemma}, the right hand side is finite, and so $\delta(a)|D|^{-s} \in \mathcal{L}_{\frac{p}{s},\infty}$. 
        
        To prove the first two statements, one applies the same proof but with Remark \ref{sp gap fact} in place of Lemma \ref{lambda lemma}.
    \end{proof}
    
%     The next proposition provides the technique for passing between the cases where $D$ has a spectral gap and when it does not.
    So far the results of this section have been stated with the assumption that $D$ is invertible. The following proposition shows how we can apply these results to a spectral triple
    where $D$ may not have a spectral gap at zero, by finding a spectral triple with very similar properties but where the corresponding operator $D$ is invertible.
    A similar proposition appeared in the Remark following Definition 2.2 of \cite{CPRS2}. 
%     To the best of our knowledge, no published proof exists in the non-unital setting.
    \begin{prop}\label{pass to spectral gap 1} 
        Let $(\mathcal{A},H,D)$ be a spectral triple, and define $D_0 := F(1+D^2)^{1/2}$ with $\mathrm{dom}(D_0) = \mathrm{dom}(D)$. Then $(\mathcal{A},H,D_0)$ is spectral triple, and:
        \begin{enumerate}[{\rm (i)}]
            \item{}\label{iff p dim} $(\mathcal{A},H,D_0)$ is $p$-dimensional if and only if $(\mathcal{A},H,D)$ is $p$-dimensional;
            \item{}\label{iff dom} Let $\delta_0$ denote the bounded extension of $[|D_0|,T]$, and define $\mathrm{dom}_\infty(\delta_0)$ identically to $\mathrm{dom}_\infty(\delta)$ with $D_0$ in place of $D$. Then we have $\mathrm{dom}_\infty(\delta_0) = \mathrm{dom}_\infty(\delta)$.
            \item{}\label{iff smooth} $(\mathcal{A},H,D_0)$ is smooth if and only if $(\mathcal{A},H,D)$ is smooth;
            \item{}\label{iff main assumption} $(\mathcal{A},H,D_0)$ satisfies Hypothesis \ref{main assumption} if and only if $(\mathcal{A},H,D)$ does.
        \end{enumerate}
        Moreover, $D_0$ has a spectral gap at $0$.
    \end{prop}
    \begin{proof} 
        First, note that $\mathrm{dom}(D_0^n) = \mathrm{dom}(D^n)$ for all $n\geq 1$ and therefore that the space $H_\infty$ is identical for $D_0$ and for $D$. It is clear that $D_0$ has a spectral gap at $0$, since $|D_0| = (1+D^2)^{1/2} \geq 1$.
        Since $D_0^2 = 1+D^2$, we have $|D_0| = (1+D^2)^{1/2}$, and $F|D_0| = D_0$.
        As $|D_0| \geq 1$, the operator $|D_0|+|D|$ is invertible. Furthermore, since $|D_0|^2 = |D|^2+1$, we have:
        \begin{equation*}
            \frac{1}{|D_0|+|D|} = |D_0|-|D|.
        \end{equation*}
        Multiplying by $F$ we obtain
        \begin{equation*}
        \frac{F}{|D_0|+|D|} = D_0-D.
        \end{equation*}
        So both $|D_0|-|D|$ and $D_0-D$ extend to bounded operators. Moreover, since for all $k\geq 1$,
        \begin{equation*}
            \frac{1}{|D_0|+|D|}:\mathrm{dom}(D^k)\to \mathrm{dom}(D^{k+1})\subseteq \mathrm{dom}(D^k)
        \end{equation*}
        we have that the bounded extensions of $D_0-D$ and $|D_0|-|D|$ map $\mathrm{dom}(D^k)$ to $\mathrm{dom}(D^k)$ for all $k\geq 1$.
        
        For $T \in \mathcal{L}_\infty(H)$, let $\partial_1(T)$ denote the commutator of the bounded extension of $D_0-D$ with $T$, $\partial_1(T) := [\frac{F}{|D_0|+|D|},T]$.
        Similarly, let $\delta_1(T)$ denote the commutator of the bounded extension of $|D_0|-|D|$ with $T$, $\delta_1(T) := [\frac{1}{|D_0|+|D|},T]$.
        
        Then we have the following identity on $H_\infty$:
        \begin{equation*}
            [D_0,a] = \partial_1(a)+\partial(a).
        \end{equation*}
        Since $\partial(a)$ and $\partial_1(a)$ are bounded, it follows that $[D_0,a]$ extends to a bounded linear operator, which we denote $\partial_0(a)$.
        
        Since $D_0^2 = D^2+1$, we have that the operator $(D+i)(D_0+i)^{-1}$ has bounded extension. Hence, for all $a \in \mathcal{A}$ we have that $a(D_0+i)^{-1}$ is compact.
        This completes the proof that $(\mathcal{A},H,D_0)$ is a spectral triple. 
        
        One may similarly prove \eqref{iff p dim}: to see that $(\mathcal{A},H,D_0)$ is $p$-dimensional if $(\mathcal{A},H,D)$ is $p$-dimensional,
        we write:
        \begin{equation*}
            a(D_0+i)^{-p} = a(D+i)^{-p}\cdot \left(\frac{D+i}{D_0+i}\right)^p
        \end{equation*}
        which is in $\mathcal{L}_{1,\infty}$, because $(D+i)(D_0+i)^{-1}$ has bounded extension. Next,
        \begin{equation*}
            \partial_0(a)(D_0+i)^{-p} = (\partial(a)(D+i)^{-p}+\partial_1(a)(D+i)^{-p})\left(\frac{D+i}{D_0+i}\right)^p.
        \end{equation*}
        and $\partial_1(a)(D+i)^{-p} = (D_0-D)a(D+i)^{-p}-a(D+i)^{-p}(D_0-D)$, hence $\partial_0(a)(D_0+i)^{-p} \in \mathcal{L}_{1,\infty}$ and so $(\mathcal{A},H,D_0)$ is $p$-dimensional. The reverse
        implication may be established by an identical argument { using the fact that $(D_0+i)(D+i)^{-1}$ has bounded extension.}
        
        Next we prove \eqref{iff dom}. We have already shown that $|D_0|-|D|$ is an operator with bounded extension and which maps $\mathrm{dom}(D^k)$ to $\mathrm{dom}(D^k)$, for all $k\geq 1$. 
        By verifying the identity on $H_\infty$, we have:
        \begin{equation*}
            \delta^k(\delta_1(T)) = \delta_1(\delta^k(T)).
        \end{equation*}
        Hence if $T \in \mathrm{dom}(\delta^k)$, then $\delta_1(T) \in \mathrm{dom}(\delta^k)$.
        
        If $T \in \mathrm{dom}(\delta)$, then $[|D_0|,T] = \delta_1(T)+\delta(T)$ on $H_\infty$. So if $T \in \mathrm{dom}_\infty(\delta)$ we can compute the $k$th iterated commutator of $T$ with $|D_0|$ as:
        \begin{align}
            [|D_0|,[|D_0|,[\cdots,[|D_0|,T]\cdots]]] &= (\delta+\delta_1)^k(T)\nonumber\\
                                                     &= \sum_{j=0}^k \binom{k}{j}\delta_1^{k-j}(\delta^{j}(T))\label{binomial formula}
        \end{align}
        Thus the $k$th iterated commutator of $|D_0|$ and $T$ has bounded extension, so $T \in \mathrm{dom}_\infty(\delta_0)$. Repeating the proof using the identity $[|D|,T] = \delta_0(T)-\delta_1(T)$, 
        we also have that $\mathrm{dom}_\infty(\delta_0) \subseteq \mathrm{dom}_\infty(\delta)$. This completes the proof of \eqref{iff dom}.
        
        Now we prove \eqref{iff smooth}. Note that if $T \in \mathrm{dom}_\infty(\delta_0)$, then $\partial_1(T) \in \mathrm{dom}_\infty(\delta_0)$. Hence, if
        \begin{equation*}
            a,\, \partial(a) \in \mathrm{dom}_\infty(\delta) = \mathrm{dom}_\infty(\delta_0)
        \end{equation*}
        then
        \begin{equation*}
            a,\, \partial(a)+\partial_1(a) \in \mathrm{dom}_\infty(\delta_0).
        \end{equation*}
        Since $\partial_0(a) = \partial(a)+\partial_1(a)$, this completes the proof that if $(\mathcal{A},H,D)$ is smooth then $(\mathcal{A},H,D_0)$ is smooth.
        For the converse, we use $\partial_0(a) = \partial(a)-\partial_1(a)$.
        

        It now remains to show \eqref{iff main assumption}. Assume that $(\mathcal{A},H,D)$ satisfies
        Hypothesis \ref{main assumption}. From \eqref{binomial formula}, we have that
        \begin{align*}
            \delta_0^k(a)(D_0+i\lambda)^{-p-1} &= \delta_0^k(a)(D+i\lambda)^{-p-1}\left(\frac{D+i\lambda}{D_0+i\lambda}\right)^{p+1}\\
                                               &= \left(\sum_{l=0}^k \binom{k}{l} \delta_1^{k-l}(\delta^l(a))(D+i\lambda)^{-p-1}\right)\left(\frac{D+i\lambda}{D_0+i\lambda}\right)^{p+1}.
        \end{align*}
        However since $|D_0|-|D|$ commutes with functions of $D$,
        \begin{equation*}
            \delta_0^k(a)(D_0+i\lambda)^{-p-1} = \left(\sum_{l=0}^k \binom{k}{l} \delta_1^{k-l}(\delta^l(a)(D+i\lambda)^{-p-1})\right)\left(\frac{D+i\lambda}{D_0+i\lambda}\right)^{p+1}.
        \end{equation*}
        
        Now since the operator $\frac{D+i\lambda}{D_0+i\lambda}$ is bounded, and $\delta_1$ is a commutator with a bounded operator, and since $(\mathcal{A},H,D)$ satisfies Hypothesis \ref{main assumption},
        \begin{equation*}
            \|\delta^l(a)(D+i\lambda)^{-p-1}\|_1 = O(\lambda^{-1}), \quad\lambda > 0
        \end{equation*}
        it follows that
        \begin{equation*}
            \|\delta_0^l(a)(D_0+i\lambda)^{-p-1}\|_1 = O(\lambda^{-1}),\quad \lambda > 0.
        \end{equation*}
        Similarly, by writing $\partial_0 = \partial_1+\partial$, we also obtain:
        \begin{equation*}
            \|\partial_0(\delta_0^k(a))(D_0+i\lambda)^{-p-1}\|_1 = O(\lambda^{-1}).
        \end{equation*}
        To prove the converse, we write $\delta(a) = \delta_0-\delta_1$ and repeat the same argument.
    \end{proof}
    
    We are now able to prove Proposition \ref{f der def} -- without any assumptions on the invertibility of $D.$ Similar results are well
    known in the unital case (see e.g. \cite[Lemma 1]{CPRS1}, \cite[Lemma 10.18]{GVF} and \cite[Lemma 5]{BF}). In the non-unital setting, a related result is \cite[Proposition 2.14]{CGRS2} which instead proves that $[F,a] \in \mathcal{L}_{p+1}$. To the best of our knowledge, no complete proof of the following result has been published in the non-unital setting.
    \begin{prop}\label{f der def}
        If $(\mathcal{A},H,D)$ is a $p-$dimensional spectral triple satisfying Hypothesis \ref{main assumption}, then $[F,a]\in\mathcal{L}_{p,\infty}$ for all $a\in\mathcal{A}.$
    \end{prop}
    \begin{proof} 
        Let $D_0 = F(1+D^2)^{1/2}$, so that by Proposition \ref{pass to spectral gap 1}, the spectral triple $(\mathcal{A},H,D_0)$ satisfies
        Hypothesis \ref{main assumption}. As an equality of operators on $H_\infty$, we have:
        \begin{align*}
            [F,a] &= [D_0|D_0|^{-1},a]\\
                  &= [D_0,a]|D_0|^{-1}+D_0[|D_0|^{-1},a].
        \end{align*}
        Using \eqref{favourite commutator identity},
        \begin{equation*}
            [F,a] = [D_0,a]|D_0|^{-1}-F[|D_0|,a]|D_0|^{-1}.
        \end{equation*}
        Since the spectral triple $(\mathcal{A},H,D_0)$ satisfies Hypothesis \ref{main assumption} and has a spectral gap at $0$,
        we may apply Lemma \ref{z<p lemma} with $s = p$ to conclude that the operators $[D_0,a]|D_0|^{-1}$ and $[|D_0|,a]|D_0|^{-1}$ have
        extension to operators in $\mathcal{L}_{p,\infty}$. Thus, $[F,a] \in \mathcal{L}_{p,\infty}$.
    \end{proof}

\section{Restatement of Hypothesis 1.2.1}\label{replacement section}

    In this section, we introduce the operator $\Lambda$, formally defined by:
    \begin{equation*}
        \Lambda(T) = (1+D^2)^{-\frac12}[D^2,T].
    \end{equation*}
    Strictly speaking, $\Lambda(T)$ will be defined to be the bounded extension of the above operator. What is here denoted $\Lambda$ appeared in the unital settings of \cite[Appendix B]{Connes-Moscovici} 
    (there denoted $L$), \cite[Definition 6.5]{CPRS2} (there denoted $L_1$) and \cite[Equation 10.66]{GVF} (there denoted $L$).
    The mapping $\Lambda$ was also used in the non-unital setting of \cite[Definition 1.20]{CGRS2} (there called $L$).
    We undertake a self-contained development of these ideas, since our assumptions are different to those used in previous work.
    There is not any substantial conceptual difference between the proofs for the unital and nonunital cases, however there are small technical obstacles which require care to be taken when computing repeated integrals of operator valued functions (see the proof of Lemma \ref{delta to lambda power}). An expert reader familiar with this theory could skip to Hypothesis \ref{replacement assumption}.
    
    We must take care to ensure that $\Lambda(T)$ is well defined, as well as that higher powers $\Lambda^k(T)$ are defined. For this purpose we introduce the spaces $\mathrm{dom}(\Lambda^k)$.    
    \begin{defi}\label{lambda def}
        Let $k \geq 1$. We define $\mathrm{dom}(\Lambda^k)$ to be the set of bounded linear operators $T$ such that for all $1 \leq j \leq k$, we have $T:\mathrm{dom}(D^{2j})\to \mathrm{dom}(D^{2j})$ and 
        such that the $k$th iterated commutator,
        \begin{equation*}
            (1+D^2)^{-1/2}[D^2,(1+D^2)^{-1/2}[D^2,\cdots,T]]:\mathrm{dom}(D^{2k})\to H
        \end{equation*}
        has bounded extension, which we denote $\Lambda^k(T)$.
        
        Define 
        \begin{equation*}
            \mathrm{dom}_\infty(\Lambda) := \bigcap_{k\geq 0} \mathrm{dom}(\Lambda^k).
        \end{equation*}
    \end{defi}

    The mapping $\Lambda$ can be thought of as a replacement for $\delta$, and we introduce it since it is easier to work with $\Lambda$ rather than $\delta$
    in the examples covered in this chapter.
    \begin{defi}
        A spectral triple $(\mathcal{A},H,D)$ is called $\Lambda$-smooth if for all $a \in \mathcal{A}$ we have,
        \begin{equation*}
            a,\,\partial(a) \in \mathrm{dom}_\infty(\Lambda).
        \end{equation*}
    \end{defi}
    We will show that $\mathrm{dom}_\infty(\Lambda) = \mathrm{dom}_\infty(\delta_0)$, and so in view of Theorem \ref{pass to spectral gap 1}.\eqref{iff dom} the notion of
    $\Lambda$-smoothness is identical to smoothness. {  This fact is well known in the unital setting, similar results having appeared in \cite[Appendix B]{Connes-Moscovici} and \cite[Proposition 6.5]{CPRS2}. We
    provide a full proof here since to the best of our knowledge no published proof exists in the non-unital setting.}
    
    The easiest direction to establish is that $\mathrm{dom}_\infty(\delta_0)\subseteq \mathrm{dom}_\infty(\Lambda)$, as the following Lemma shows:
    \begin{lem}\label{delta smooth implies lambda smooth}
        We have $\mathrm{dom}_\infty(\delta_0) \subseteq \mathrm{dom}_\infty(\Lambda)$.
    \end{lem}
    \begin{proof}
        Let $T \in \mathrm{dom}_\infty(\delta_0)$. We have $T:\mathrm{dom}(D^k)\to \mathrm{dom}(D^k)$ for all $k\geq 1$, and so working on $H_\infty$, we can write,
        \begin{align*}
            (1+D^2)^{-1/2}[D^2,T]  &= |D_0|^{-1}[|D_0|^2,T]\\
                                   &= 2[|D_0|,T]-|D_0|^{-1}[|D_0|,[|D_0|,T]]\\
                                   &= 2\delta_0(T)-|D_0|^{-1}\delta_0^2(T).
        \end{align*}
        By assumption $T \in \mathrm{dom}_{\infty}(\delta)$, hence, $\Lambda(T)$ has bounded extension
        and so $T \in \mathrm{dom}(\Lambda)$, and on all $H$ we have:
        \begin{equation*}
            \Lambda(T) = 2\delta_0(T)-|D_0|^{-1}\delta_0^2(T).
        \end{equation*}
        However since $\delta_0(T), \delta_0^2(T)$ and $|D_0|^{-1}$ are in $\mathrm{dom}_\infty(\delta_0)$, it follows that $\Lambda(T) \in \mathrm{dom}_\infty(\delta_0)$.
        
        Hence, $\Lambda(T) \in \mathrm{dom}(\Lambda)$, and continuing by induction we get that $T \in \mathrm{dom}(\Lambda^k)$ for all $k\geq 1$.
    \end{proof}
    
    It takes some more work to prove that $\mathrm{dom}_\infty(\Lambda)\subseteq \mathrm{dom}_\infty(\delta_0)$. We achieve this by an integral representation of $\delta_0(T)$
    in terms of $\Lambda(T)$ and $\Lambda^2(T)$. We make use of the dense subspace $H_\infty$ from Definition \ref{H_infty definition}. 
    The following Lemma should be compared with the proof of \cite[Lemma B2]{Connes-Moscovici}.
    \begin{lem}\label{delta to lambda}
        Let $T\in \mathrm{dom}_\infty(\Lambda)$. Then for all $\xi \in H_\infty$ we have:
        \begin{equation*}
            [|D_0|,T]\xi = \frac{1}{2}\Lambda(T)\xi + \frac{1}{\pi}\int_0^\infty \lambda^{1/2} \frac{D_0^2}{(\lambda+D_0^2)^2}\Lambda^2(T)\frac{1}{\lambda+D_0^2}\xi\,d\lambda.
        \end{equation*}
        The integral above may be understood as a weak operator topology integral.
    \end{lem}
    \begin{proof} 
        This is essentially a combination of the following two { well known} integral formulae:
        \begin{equation}\label{little integral 1}
            (1+D^2)^{-1/2} = \frac{1}{\pi} \int_0^\infty \frac{1}{1+\lambda+D^2}\frac{d\lambda}{\lambda^{1/2}}
        \end{equation}
        and
        \begin{equation}\label{little integral 2}
            (1+D^2)^{-1/2} = \frac{2}{\pi} \int_0^\infty \frac{\lambda^{1/2}}{(1+\lambda+D^2)^2}d\lambda
        \end{equation}
        which can both be understood as integrals in the weak operator topology, since
        \begin{equation*}
            \left\|\frac{1}{1+\lambda+D^2}\right\|_\infty \leq \frac{1}{1+\lambda},\quad \lambda > 0
        \end{equation*}
        and
        \begin{equation*}
            \left\|\frac{\lambda^{1/2}}{(1+\lambda+D^2)^2}\right\|_\infty \leq \frac{\lambda^{1/2}}{(1+\lambda)^2},\quad\lambda > 0.
        \end{equation*}
        
        Let $\xi \in H_\infty$. Multiplying \eqref{little integral 1} by $(1+D^2)\xi$, we get:
        \begin{equation}\label{little integral 3}
            (1+D^2)^{1/2}\xi = \frac{1}{\pi}\int_0^\infty \frac{1+D^2}{1+\lambda+D^2}\xi\frac{d\lambda}{\lambda^{1/2}}.
        \end{equation}
        The above is a convergent Bochner integral in $H$, since
        \begin{equation*}
            \left\|\frac{1+D^2}{1+\lambda+D^2}\xi\right\|_H \leq \frac{1}{1+\lambda}\|(1+D^2)\xi\|_H.
        \end{equation*}
        Now, replacing $1+D^2 = D_0^2$ by \eqref{favourite commutator identity}:
        \begin{align}\label{adding lambda to Lambda}
            \left[\frac{1}{\lambda+D_0^2},T\right] &= -(\lambda+D_0^2)^{-1}[D^2,T](\lambda+D_0^2)^{-1}\nonumber\\
                                                   &= -\frac{D_0^2}{\lambda+D_0^2}\Lambda(T)(\lambda+D_0^2)^{-1}.
        \end{align}
        Hence,
        \begin{align*}
            \left[\frac{D_0^2}{\lambda+D_0^2},T\right] &= \left[1-\frac{\lambda}{\lambda+D_0^2},T\right]\\
                                                       &= \frac{\lambda(D_0^2)}{\lambda+D_0^2}\Lambda(T)(\lambda+D_0^2)^{-1}\\
                                                       &= \frac{\lambda(D_0^2)}{\lambda+D_0^2}([\Lambda(T),(\lambda+D_0^2)^{-1}]+(\lambda+D_0^2)^{-1}\Lambda(T)).
        \end{align*}
        Applying \eqref{adding lambda to Lambda} a second time:
        \begin{align*}
            \left[\frac{D_0^2}{\lambda+D_0^2},T\right] &= \frac{\lambda D_0^2}{\lambda+D_0^2}\Big(\frac{D_0^2}{\lambda+D_0^2}\Lambda^2(T)(\lambda+D_0^2)^{-1} +(\lambda+D_0^2)^{-1}\Lambda(T)\Big)\\
                                                       &= \frac{\lambda|D_0|}{(\lambda+D_0^2)^2}\Lambda(T)+ \frac{\lambda D_0^2}{(\lambda+D_0^2)^2}\Lambda^2(T)(\lambda+D_0^2)^{-1}.
        \end{align*}
        Now we apply the integral formula \eqref{little integral 3} to obtain:
        \begin{align*}
            [|D_0|,T]         &= \frac{1}{\pi}\int_0^\infty \left[\frac{D_0^2}{\lambda+D_0^2},T\right]\frac{d\lambda}{\lambda^{1/2}}\\
                              &= \frac{1}{\pi}\int_0^\infty \frac{\lambda|D_0|}{(\lambda+D_0^2)^2}\Lambda(T)\frac{d\lambda}{\lambda^{1/2}}\\
                              &\quad+\frac{1}{\pi}\int_0^\infty \frac{\lambda D_0^2}{(\lambda+D_0^2)^2}\Lambda^2(T)(\lambda+D_0^2)^{-1}\frac{d\lambda}{\lambda^{1/2}}.
        \end{align*}
        Now applying \eqref{little integral 2} again, we have:
        \begin{equation*}
            \int_0^\infty \lambda^{1/2}\frac{|D_0|}{(\lambda+D_0^2)^2}d\lambda = \frac{\pi}{2}.
        \end{equation*}
        Hence,
        \begin{equation*}
            [|D_0|,T]\xi = \frac{1}{2}\Lambda(T) + \frac{1}{\pi}\int_0^\infty \lambda^{1/2}\frac{D_0^2}{(\lambda+D_0^2)^2}\Lambda^2(T)\frac{1}{\lambda+D_0^2}\xi\,d\lambda.
        \end{equation*}
    \end{proof}

    The following lemma provides an integral representation of the $n$th iterated commutator $\delta_0^n(T)$. This will allow
    us to relate $\mathrm{dom}_\infty(\delta_0)$ to $\mathrm{dom}_\infty(\Lambda)$. 
    We need to take care to ensure that the relevant version of a Fubini's theorem applies.
    \begin{lem}\label{delta to lambda power} 
        For all $m\geq1,$ and $T \in \mathrm{dom}_\infty(\Lambda)$. Then for all $\xi \in H_\infty$ the $m$th iterated commutator of $|D_0|$ 
        \begin{align*}
            [|D_0|,[|D_0|,[\cdots[|D_0|,T]\cdots]]]\xi &= 2^{-m}\sum_{k=0}^m\binom{m}{k}\Big(\frac{2}{\pi}\Big)^k\int_{\mathbb{R}^k_+}\prod_{l=1}^k\frac{\lambda_l^{1/2}D_0^2}{(\lambda_l+D_0^2)^2}\\
                                                       &\quad \cdot \Lambda^{m+k}(T)\prod_{l=1}^k\frac{1}{\lambda_l+D_0^2}\xi d\lambda_1 d\lambda_2\cdots d\lambda_k.
        \end{align*}
    \end{lem}
    \begin{proof} 
        Let
        \begin{equation*}
            \Theta(T) := \int_0^\infty \lambda^{1/2} \frac{D_0^2}{(\lambda+D_0^2)^2} \Lambda^2(T)\frac{1}{\lambda+D_0^2}d\lambda
        \end{equation*}
        so that Lemma \ref{delta to lambda} states that $\delta_0 = \frac{1}{2}\Lambda+\frac{1}{\pi}\Theta$. 
        
        Since $\Lambda$ commutes with $\frac{D_0^2}{(\lambda+D_0^2)^2}$ and $\frac{1}{\lambda+D_0^2}$, we have $\Theta\circ\Lambda = \Lambda\circ \Theta$. Hence,
        \begin{equation}\label{binomial expression}
            \delta_0^m = \frac{1}{2^m}\sum_{k=0}^m \binom{m}{k} \left(\frac{2}{\pi}\right)^k\Theta^k\circ \Lambda^{m-k}.
        \end{equation}
        By the Fubini theorem { for Hilbert space valued functions (see \cite[Theorem III.11.13]{Dunford-Schwartz-1}), for all $\xi \in H_\infty$} we have:
        \begin{equation*}
            \Theta^k(T)\xi = \int_{[0,\infty)^k} \prod_{l=1}^k \frac{\lambda_l^{1/2}D_0^2}{(\lambda_l+D_0^2)^2}\Lambda^{2k}(T)\prod_{l=1}^k \frac{1}{\lambda_l+D_0^2}\xi d\lambda_1d\lambda_2\cdots d\lambda_k.
        \end{equation*}
        Therefore,
        \begin{equation*}
            \Theta^k(\Lambda^{m-k}(T))\xi = \int_{[0,\infty)^k} \prod_{l=1}^k \frac{\lambda_l^{1/2}D_0^2}{(\lambda_l+D_0^2)^2}\Lambda^{m+k}(T)\prod_{l=1}^k \frac{1}{\lambda_l+D_0^2}\xi d\lambda_1d\lambda_2\cdots d\lambda_k.
        \end{equation*}
        Substituting into \eqref{binomial expression} yields the result.
    \end{proof}
    
    The following corollary was already noted in the unital settings of \cite[Appendix B]{Connes-Moscovici}, \cite[Proposition 6.5]{CPRS2} 
    and in the non-unital setting of \cite[Equation 1.12]{CGRS2}.
    \begin{cor}
        We have $\mathrm{dom}_\infty(\Lambda) = \mathrm{dom}_\infty(\delta_0)$, and $\Lambda$-smoothness { of a spectral triple} is equivalent to smoothness { as stated in Definition \ref{smoothness definition}}
    \end{cor}
    \begin{proof}
        From Lemma \ref{delta smooth implies lambda smooth} we already know that $\mathrm{dom}_\infty(\delta_0) \subseteq \mathrm{dom}_\infty(\Lambda)$, and so we concentrate 
        on the reverse inclusion.
        
        If $T \in \mathrm{dom}_\infty(\Lambda)$, then for each $k\geq 1$ the operator $\Lambda^k(T)$ on $H_\infty$ has bounded extension. Hence the integral
        in Lemma \ref{delta to lambda power} converges as a Bochner integral, and so the $m$th iterated commutator $\delta_0^m(T)$ is bounded, for all $m\geq 0$.
        Thus $T \in \mathrm{dom}_\infty(\delta_0)$, and this completes the proof.
    \end{proof}
    
    The remainder of this section is devoted to showing that Hypothesis \ref{main assumption} is equivalent to the following:
    \begin{hyp}\label{replacement assumption} 
        The spectral triple $(\mathcal{A},H,D)$ satisfies the following conditions:
        \begin{enumerate}[{\rm (i)}]
            \item\label{rass0} $(\mathcal{A},H,D)$ is a $\Lambda$-smooth spectral triple.
            \item\label{rass1} $(\mathcal{A},H,D)$ is $p-$dimensional, i.e., for every $a\in\mathcal{A},$
                               $$a(D+i)^{-p}\in\mathcal{L}_{1,\infty},\quad \partial(a)(D+i)^{-p}\in\mathcal{L}_{1,\infty}.$$
            \item\label{rass2} For every $a\in\mathcal{A}$ and for all $k\geq0,$ we have
                               $$\Big\|\Lambda^k(a)(D+i\lambda)^{-p-1}\Big\|_1=O(\lambda^{-1}),\quad\lambda\to\infty,$$
                               $$\Big\|\partial(\Lambda^k(a))(D+i\lambda)^{-p-1}\Big\|_1=O(\lambda^{-1}),\quad\lambda\to\infty.$$
            \end{enumerate}
    \end{hyp}
    Hypothesis \ref{replacement assumption} is precisely Hypothesis \ref{main assumption}, but with smoothness replaced by $\Lambda$-smoothness, and the occurances of $\delta$
    replaced with $\Lambda$.    

    For the next two lemmas, we borrow { techniques} that were developed in \cite{CGRS2}. The next Lemma shows that if $(\mathcal{A},H,D)$ satisfies
    Hypothesis \ref{replacement assumption} then $(\mathcal{A},H,D_0)$ satisfies Hypothesis \ref{main assumption}.
    \begin{lem}\label{main replacement lemma} 
        Let $(\mathcal{A},H,D)$ be a spectral triple satisfying Hypothesis \ref{replacement assumption}. For every $a\in\mathcal{A}$ and for all $m\geq0,$ we have 
        $$\Big\|\delta_0^m(a)(D_0+i\lambda)^{-p-1}\Big\|_1=O(\lambda^{-1}),\quad\lambda\to\infty,$$
        $$\Big\|\partial_0(\delta_0^m(a))(D_0+i\lambda)^{-p-1}\Big\|_1=O(\lambda^{-1}),\quad\lambda\to\infty.$$
    \end{lem}
    \begin{proof} 
        We prove only the first assertion. The proof of the second assertion is similar.
        
        By the spectral theorem,
        \begin{align*}
            \left\|\prod_{l=1}^k \frac{\lambda_l^{1/2}(1+D^2)}{(1+\lambda_l+D^2)^2}\right\|_\infty &\leq \prod_{l=1}^k\left\|\frac{\lambda_l^{1/2}(1+D^2)}{(1+\lambda_l+D^2)^2}\right\|_\infty\\
                                                                                                   &\leq \prod_{l=1}^k \sup_{t_l\geq 1} \frac{\lambda_l^{1/2}t_l}{(\lambda_l+t_l)^2}\\
                                                                                                   &\leq \prod_{l=1}^k \lambda_l^{-1/2}
        \end{align*}    
        and also
        \begin{equation*}
            \left\|\prod_{l=1}^k \frac{1}{1+\lambda_l+D^2}\right\|_\infty \leq \prod_{l=1}^k \frac{1}{1+\lambda_l}.
        \end{equation*}
        Hence, for all $a \in \mathcal{A}$, since $(D+i\lambda)^{-1}$ and $(1+\lambda_l+D^2)^{-1}$ commute,
        \begin{align*}
            \Big\|\left(\prod_{l=1}^k \frac{\lambda_l^{1/2}(1+D^2)}{(1+\lambda_l+D^2)^2}\right)\Lambda^{m+k}(a)&\left(\prod_{l=1}^k \frac{1}{1+\lambda_l+D^2}\right)(D+i\lambda)^{-p-1}\Big\|_1\\
                                                            &\leq \|\Lambda^{m+k}(a)(D+i\lambda)^{-p-1}\|_1\prod_{l=1}^k \frac{1}{\lambda_l^{1/2}(1+\lambda_l)}.
        \end{align*}
        Now applying Lemma \ref{delta to lambda power} with Lemma \ref{peter lemma},
        \begin{align*}
            \|&\delta_0^m(a)(D+i\lambda)^{-p-1}\|_1 \\
                            &\leq 2^{-m}\sum_{k=0}^m \binom{m}{k} \left(\frac{2}{\pi}\right)^k \|\Lambda^{m+k}(a)(D+i\lambda)^{-p-1}\|_1\int_{[0,\infty)^k} \prod_{l=1}^k \frac{d\lambda_l}{\lambda_l^{1/2}(1+\lambda_l)}.
        \end{align*}
        Since $\int_0^\infty \frac{1}{\lambda^{1/2}(1+\lambda)}\,d\lambda = \pi$, we arrive at
        \begin{equation*}
            \|\delta_0^m(a)(D+i\lambda)^{-p-1}\|_{1} \leq 2^{-m} \sum_{k=0}^m \binom{m}{k} 2^k \|\Lambda^{m+k}(a)(D+i\lambda)^{-p-1}\|_1\pi^k.
        \end{equation*}
        By Hypothesis \ref{replacement assumption}, each summand above is $O(\lambda^{-1})$. Hence $\|\delta_0^m(a)(D+i\lambda)^{-p-1}\|_1 = O(\lambda^{-1})$.
        
        
        Now using the fact that the operator $\left(\frac{D+i\lambda}{D_0+i\lambda}\right)^{p+1}$ has bounded extension, { and
        \begin{align*}
            \left\|\left(\frac{D+i\lambda}{D_0+i\lambda}\right)^{p+1}\right\|_\infty &\leq \sup_{t \in \mathbb{R}} \left(\frac{t^2+\lambda^2}{1+t^2+\lambda^2}\right)^{\frac{p+1}{2}}\\
                                                                                     &\leq 1,
        \end{align*}
        we get:}
        \begin{align*}
            \|\delta_0^m(a)(D_0+i\lambda)^{-p-1}\|_1 &\leq \|\delta_0(a)(D+i\lambda)^{-p-1}\|_1\left\|\left(\frac{D+i\lambda}{D_0+i\lambda}\right)^{p+1}\right\|_\infty\\
                                                     &= O(\lambda^{-1})
        \end{align*}
        as desired.
        
    \end{proof}
    
    We can now conclude the proof of the main result of this subsection:
    \begin{thm}\label{replacement thm} 
        A spectral triple $(\mathcal{A},H,D)$ satisfies Hypothesis \ref{main assumption} if and only if it satisfies Hypothesis \ref{replacement assumption}.
    \end{thm}
    \begin{proof}
        We have already proved that $(\mathcal{A},H,D)$ satisfies \ref{replacement assumption}.\eqref{rass0} if and only if it satisfies \ref{main assumption}.\eqref{ass0},
        and \eqref{replacement assumption}.\eqref{rass1} is identical to \eqref{main assumption}.\eqref{ass1}. We now focus on \eqref{replacement assumption}.\eqref{rass2}.
    
        Suppose that $(\mathcal{A},H,D)$ satisfies Hypothesis \ref{replacement assumption}. 
        By Lemma \ref{main replacement lemma}, we have that $(\mathcal{A},H,D_0)$ satisfies Hypothesis \ref{main assumption}. Then by Proposition \ref{pass to spectral gap 1} 
        $(\mathcal{A},H,D)$ satisfies Hypothesis \ref{main assumption}.
        
        Now we prove the converse. Suppose that $(\mathcal{A},H,D)$ satisfies Hypothesis \ref{main assumption}. For $T \in \mathrm{dom}_\infty(\delta)$, we define $\alpha(T)$ and $\beta(T)$ by:
        \begin{align*}
            \alpha(T) := \frac{|D|}{(D^2+1)^{1/2}}\delta(T),\\
            \beta(T) := \frac{1}{(D^2+1)^{1/2}}\delta^2(T).
        \end{align*}
        We can express $\Lambda$ in terms of $\alpha$ and $\beta$, by applying the Leibniz rule as follows:
        \begin{align*}
            \Lambda(T) &= (1+D^2)^{-1/2}[|D|^2,T]\\
                       &= \frac{|D|}{(1+D^2)^{1/2}}\delta(T) + (1+D^2)^{-1/2}\delta(T)|D|\\
                       &= 2\alpha(T) - (1+D^2)^{-1/2}\delta^2(T)\\
                       &= 2\alpha(T)-\beta(T).
        \end{align*}
        Since $\alpha\circ\beta = \beta\circ\alpha$,
        \begin{equation*}
            \Lambda^m = \sum_{k=0}^m \binom{m}{k}(-1)^k2^{m-k}\beta^k\circ\alpha^{m-k}.
        \end{equation*}
        
        For every $k = 0,\ldots,m$ and $T \in \mathrm{dom}_\infty(\delta)$, we have:
        \begin{equation*}
            \beta^k(\alpha^{m-k}(T)) = \frac{|D|^{m-k}}{(D^2+1)^{m/2}}\delta^{m+k}(T).
        \end{equation*}
        So for $a \in \mathcal{A}$ and $m \geq 1$,
        \begin{equation*}
            \Lambda^m(a) = \sum_{k=0}^m \binom{m}{k} (-1)^k2^{m-k} \frac{|D|^{m-k}}{(D^2+1)^{m/2}}\delta^{m+k}(a).
        \end{equation*}
        Noting that the operator $\frac{|D|^{m-k}}{(D^2+1)^{m/2}}$ is bounded, there exists a constant $C_m$ such that
        \begin{equation*}
            \|\Lambda^m(a)(D+i\lambda)^{-p-1}\|_1 \leq C_m\sum_{k=0}^m \|\delta^{m+k}(a)(D+i\lambda)^{-p-1}\|_1
        \end{equation*}
        Since we are assuming that $(\mathcal{A},H,D)$ satisfies Hypothesis \ref{main assumption}, it follows
        that $\|\Lambda^m(a)(D+i\lambda)^{-p-1}\|_1 = O(\lambda^{-1})$. 
        
        We may similarly deal with $\|\partial(\Lambda^m(a))(D+i\lambda)^{-p-1}\|_1$: since $\partial$ commutes with functions of $D$:
        \begin{equation*}
            \partial(\Lambda^m(a)) = \sum_{k=0}^m (-1)^k2^{m-k} \frac{|D|^{m-k}}{(D^2+1)^{m/2}}\partial(\delta^{m+k}(a)).
        \end{equation*} 
        Thus by the same argument, we have $\|\partial(\Lambda^m(a))(D+i\lambda)^{-p-1}\|_1 = O(\lambda^{-1})$, and so $(\mathcal{A},H,D)$
        satisfies Hypothesis \ref{replacement assumption}.
    \end{proof}
    
    Thanks to Theorem \ref{replacement thm}, we can work assuming Hypothesis \ref{replacement assumption} rather than Hypothesis \ref{main assumption}.

\section{Example: Noncommutative Euclidean space}\label{nc section}
    We now discuss the most heavily studied example of a non-unital spectral triple: noncommutative Euclidean space. Subsection \ref{ncp definitions subsection} covers
    the definitions of noncommutative Euclidean spaces and their associated spectral triples. Subsection \ref{ncp verification subsection} is devoted to the proof that 
    these spectral triples satisfy Hypothesis \ref{main assumption}.
    
    Noncommutative Euclidean spaces can be found in the literature under various names, such as canonical commutation relation (CCR) algebras (as in \cite[Section 5.2.2.2]{Bratteli-Robinson2}),
    and in the $2$-dimensional case are called Moyal planes or Moyal-Groenwald planes (as in \cite{gayral-moyal}).
        
        
\subsection{Definitions for Noncommutative Euclidean spaces}\label{ncp definitions subsection}
    Our approach to noncommutative Euclidean space is to proceed from the Weyl commutation relations, in line with \cite[Section 5.2.2.2]{Bratteli-Robinson2} and \cite{LeSZ-cwikel}. An alternative
    approach is to use the Moyal product, as in \cite{gayral-moyal} and \cite[Section 5.2]{CGRS2}. We caution the reader that the approach considered here is the "Fourier dual" of the approach in \cite{gayral-moyal}.
    We briefly cite the required facts needed for this section, and refer the reader to \cite{LeSZ-cwikel} for detailed exposition and proofs.
    
    Let $\theta$ be an antisymmetric real $p\times p$ matrix. 
    Abstractly, the von Neumann algebra $L_\infty(\mathbb{R}^p_\theta)$ is generated by a strongly continuous family $\{U(t)\}_{t \in \mathbb{R}^p}$ satisfying
    \begin{equation}\label{weyl CCR}
        U(t+s) = \exp\left(\frac{1}{2}i(t,\theta s)\right)U(t)U(s),\quad t,s \in  \mathbb{R}^p.
    \end{equation}
    Here we avoid technicalities by defining $L_\infty(\mathbb{R}^p_\theta)$ to be a von Neumann algebra generated by a specific family of unitary operators on $L_2(\mathbb{R}^p)$.
    
    \begin{defi}
        Let $\theta$ be an antisymmetric real matrix. For $t \in \mathbb{R}^p$, let $U(t)$ be the linear operator on $L_2(\mathbb{R}^p)$ given by:
        \begin{equation*}
            (U(t)\xi)(r) = \exp\left(-i(t,\theta r)\right)\xi(r-t),\quad r \in \mathbb{R}^p,\,\xi \in L_2(\mathbb{R}^p).
        \end{equation*}
        Define $L_\infty(\mathbb{R}^p_\theta)$ to be the von Neumann subalgebra of $\mathcal{L}_\infty(L_2(\mathbb{R}^p))$ generated by the family $\{U(t)\}_{t \in \mathbb{R}^p}$.
    \end{defi}
    
    \begin{rem}
        It can easily be shown that $U(t)$ satisfies \eqref{weyl CCR}. Since $U(t)$ is a composition of a translation, and pointwise multiplication by $\exp\left(-i\frac{1}{2}(t,\theta s)\right)$, it is clear that each $U(t)$ is unitary, and that $t\mapsto U(t)$ is strongly continuous. Since $\theta$ is antisymmetric, $U(-t) = U(t)^{-1} = U(t)^*$.
        
        The map $t\mapsto U(t)$ is a twisted left-regular representation of $\mathbb{R}^p$ on $L_2(\mathbb{R}^p)$, in the sense of \cite{Echterhoff-2008}.
        
        Note that if $\theta = 0$, then the family $\{U(t)\}_{t \in \mathbb{R}^p}$ is simply the semigroup of translations on $\mathbb{R}^p$, and so generates the von Neumann algebra $L_\infty(\mathbb{R}^p)$. 
    \end{rem}
    
    If $\theta$ is nondegenerate (that is, $\det(\theta) \neq 0$) then $p$ is even and the algebra $L_\infty(\mathbb{R}^p_\theta)$ is isomorphic to $\mathcal{L}_\infty(L_2(\mathbb{R}^{p/2}))$. 
    This is proved in \cite{LeSZ-cwikel}, where a spatial isomorphism is constructed.    
    \begin{thm}\label{spatial}
        If $\det(\theta) \neq 0$, then there is a spatial isomorphism
        \begin{equation*}
            L_\infty(\mathbb{R}^p_\theta)\cong \mathcal{L}_\infty(L_2(\mathbb{R}^{p/2})).
        \end{equation*}
    \end{thm}
    
    We now focus exclusively on the case where $\det(\theta) \neq 0$.
    \begin{defi}
        The semifinite trace $\tau_\theta$ on $L_\infty(\mathbb{R}^p)$ is defined via the isomorphism in Theorem \ref{spatial} to be simply the classical
        trace $\mathrm{Tr}$ on $\mathcal{L}_\infty(L_2(\mathbb{R}^{p/2}))$. For $r \in [1,\infty)$, the space $L_r(\mathbb{R}^p_\theta)$ is defined by:
        \begin{equation*}
            L_r(\mathbb{R}^p_\theta) := \{x \in L_\infty(\mathbb{R}^p_\theta)\;:\;\tau_\theta(|x|^r) < \infty\}.
        \end{equation*}
        The space $L_{r}(\mathbb{R}^p_\theta)$ is equipped with the norm $\|x\|_{L_r} = \tau_\theta(|x|^r)^{1/r}$.
    \end{defi}
    Note that $L_r(\mathbb{R}^p_\theta)$ is identical to the Schatten-von Neumann class $\mathcal{L}_r(L_2(\mathbb{R}^{p/2}))$, since $\tau_\theta$ is simply the classical trace. 
    
    \begin{defi}
        For $k = 1,\ldots,p$, we define the operator $D_k^\theta$ on $L_2(\mathbb{R}^p)$ by
        \begin{equation*}
            (D_k^\theta\xi)(t) = t_k\xi(t),\quad t \in \mathbb{R}^p.
        \end{equation*}
        
        The Dirac operator $D^\theta$ is defined on the Hilbert space $L_2(\mathbb{R}^d,\mathbb{C}^{2^{p/2}})$ by 
        $D^\theta = \gamma_1\otimes D^\theta_1+\cdots +\gamma_p\otimes D^\theta_p$, where $\gamma_1,\gamma_2,\ldots\gamma_p$
        are complex $2^{p/2}\times 2^{p/2}$ matrices satisfying $\gamma_j\gamma_k+\gamma_k\gamma_j = 2\delta_{j,k}$ and $\gamma_j = \gamma_j^*$ for $1\leq j,k \leq p$.
    \end{defi}
    Evidently, the operators $D_j$ are unbounded, but may be initially defined on the dense subspace of compactly supported functions.
    
    It follows readily from the definitions of $D_k^\theta$ and $U(t)$ that
    \begin{equation*}
        [D_k^\theta,U(t)] = t_kU(t),\quad t \in \mathbb{R}^p.
    \end{equation*}
    
    Since the operators $D_1^\theta,\ldots D_p^\theta$ form a family of mutually commuting self-adjoint operators, we may apply functional calculus to define $e^{i(s,\nabla)}$, $s \in \mathbb{R}^p$.
    where $\nabla^\theta = (D_1^\theta,D_2^\theta,\ldots,D_p^\theta)$, given by:
    \begin{equation*}
        (e^{i(s,\nabla^\theta)}\xi)(r) = \exp(i(s,r))\xi(r),\quad r \in \mathbb{R}^p.
    \end{equation*}
    Hence,
    \begin{equation*}
        e^{i(s,\nabla^\theta)}U(s)e^{-i(s,\nabla^\theta)} = e^{i(s,t)}U(s),\quad s,t \in \mathbb{R}^p.
    \end{equation*}
    For convenience we also introduce the notation $\Delta^\theta := \sum_{k=1}^p (D_k^\theta)^2$.
    
    The following is \cite[Proposition 6.12]{LeSZ-cwikel}:
    \begin{lem}\label{nc poincare}
        Let $k = 1,\ldots,p$. If $x \in L_\infty(\mathbb{R}^p_\theta)$, and the operator $[D_k^\theta,x]$ has bounded extension, then its extension is an element
        of $L_\infty(\mathbb{R}^d_\theta)$.
    \end{lem}
    
    \begin{defi}
        If $x \in L_\infty(\mathbb{R}^d_\theta)$ is such that $[D_k^\theta,x]$ has bounded extension, then we denote $\partial_kx$ for the extension.
        
        We denote $\partial_j^0x := x$, for all $x \in L_\infty(\mathbb{R}^p_\theta)$ and $j$.
        
        Generally, let $\alpha = (\alpha_1,\alpha_2,\ldots,\alpha_p)$ be a multi-index. If for all $1 \leq j \leq p$ the operator
        \begin{equation*}
            \partial_j^{\alpha_j}(\partial_{j+1}^{\alpha_{j+1}}(\cdots(\partial_p^{\alpha_p}(x))\cdots))
        \end{equation*}
        has bounded extension, then the mixed partial derivative $\partial^{\alpha}x$ is defined as the operator:
        \begin{equation*}
            \partial_1^{\alpha_1}(\partial_2^{\alpha_2}(\cdots(\partial_p^{\alpha_p}(x))\cdots)).
        \end{equation*}
    \end{defi}
    By Lemma \ref{nc poincare}, we always have $\partial^\alpha x \in L_\infty(\mathbb{R}^p_\theta)$ if it is well defined.
    
    \begin{defi}
        Let $m \geq 0$ and $r \geq 1$. The space $W^{m,r}(\mathbb{R}^p_\theta)$ is defined to be the set of $x \in L_\infty(\mathbb{R}^p_\theta)$ such that 
        $\partial^{\alpha}x \in L_r(\mathbb{R}^p_\theta)$ for every $|\alpha| \leq m$, equipped with the norm:
        \begin{equation*}
            \|x\|_{W^{m,r}} := \sum_{|\alpha|\leq m} \|\partial^{\alpha}x\|_r.
        \end{equation*}
        
        We define $W^{\infty,r}(\mathbb{R}^p_\theta) := \bigcap_{m\geq 0} W^{m,r}(\mathbb{R}^p_\theta)$.
    \end{defi}  
    
    As suggested by the notation, the spaces $W^{m,r}(\mathbb{R}^p_\theta)$ are the analogues of Sobolev spaces for noncommutative Euclidean spaces. The space $W^{\infty,1}(\mathbb{R}^p_\theta)$
    is important because it forms a part of our spectral triple for noncommutative Euclidean space.
    
    The remainder of this section is devoted to showing that the triple
    \begin{equation}\label{ncp spectral triple}
        (1_{2^{p/2}}\otimes W^{\infty,1}(\mathbb{R}^p_\theta),L_2(\mathbb{R}^{p},\mathbb{C}^{2^{p/2}}),D^\theta)
    \end{equation}
    is a spectral triple satisfying Hypothesis \ref{main assumption}.
    

    We wish to verify Hypothesis \ref{main assumption} for noncommutative spaces in order to support our claim that \ref{main assumption} is a reasonable
    assumption to make. However, in the nondegenerate case $\det(\theta) \neq 0$, the Character Theorem \ref{main thm} is trivial for at least a dense
    subalgebra of $W^{\infty,1}(\mathbb{R}^d_\theta$).
    
    The reason for this is that due to \cite[Proposition 2.5]{gayral-moyal},
    there is a dense subalgebra of $W^{\infty,1}(\mathbb{R}^d_\theta)$ isomorphic to the 
    algebra $M_\infty(\mathbb{C})$ of finitely supported infinite matrices. However
    due to \cite[Theorem 1.4.14]{Loday-cyclic-homology}, if $n\geq 0$ then the
    $n$th Hochschild homology of $M_\infty(\mathbb{C})$ is computed by:
    \begin{equation*}
        HH_{n}(M_\infty(\mathbb{C})) = HH_n(\mathbb{C}).
    \end{equation*}
    For $n> 0$, the $n$th Hochschild homology of $\mathbb{C}$ is trivial \cite[Lemma 8.9]{GVF}.
    Hence for $n> 0$, the $n$th Hochschild homology of $M_\infty(\mathbb{C})$ is trivial.
    
    This entails that every degree $p+1$ Hochschild cycle of $M_\infty(\mathbb{C})$ is a Hochschild boundary. However the left and right hand sides of the Character Theorem,
    $c \mapsto \mathrm{Ch}(c)$ and $c\mapsto \varphi(\Omega(c)(1+D^2)^{-p/2})$, are
    both Hochschild cocycles and hence vanish on any Hochschild boundary.       
    
\subsection{Verification of Hypothesis \ref{main assumption} for Noncommutative Euclidean spaces}\label{ncp verification subsection}
    Now we prove that the triple \eqref{ncp spectral triple} is a spectral triple satisfying Hypothesis \ref{main assumption}. In fact
    it is easier to use Hypothesis \ref{replacement assumption}.

%     
    Our main reference for this section is \cite[Section 7]{LeSZ-cwikel}. As in that reference, the spaces $\ell_1(L_\infty)$ and $\ell_{1,\infty}(L_\infty)$ are defined as follows:
    Let $K = [0,1]^p$ be the unit $p$-cube. Then $\ell_1(L_\infty)$ and $\ell_{1,\infty}(L_\infty)$ are the subspaces of $L_\infty(\mathbb{R}^d)$ such that the following norms are finite:
    \begin{align*}
        \|g\|_{\ell_1(L_\infty)} &:= \|\{\|g\|_{L_{\infty}(m+K)}\}_{m \in \mathbb{Z}^p}\|_{\ell_1(\mathbb{Z}^p)},\\
        \|g\|_{\ell_{1,\infty}(L_\infty)} &:= \|\{\|g\|_{L_{\infty}(m+K)}\}_{m \in \mathbb{Z}^p}\|_{\ell_{1,\infty}(\mathbb{Z}^p)}.
    \end{align*}
    
    The following is a special case of \cite[Theorem 7.6, Theorem 7.7]{LeSZ-cwikel}:    
    \begin{thm}\label{nc cwikel theorem}
        Let $p \geq 1$. There are constants $c_p > 0$ and $c_p' > 0$ such that for all $x \in W^{p,1}(\mathbb{R}^p_\theta)$ we have:
        \begin{enumerate}[{\rm (a)}]
            \item{}\label{first nc cwikel} If $g \in \ell_1(L_\infty)$, then $xg(\nabla^\theta) \in \mathcal{L}_1$, and $$\|xg(\nabla^\theta)\|_1 \leq c_p\|x\|_{W^{p,1}}\|g\|_{\ell_1(L_\infty)}.$$
            \item{}\label{second nc cwikel} If $g \in \ell_{1,\infty}(L_\infty)$, then $xg(\nabla^\theta) \in \mathcal{L}_{1,\infty}$ and $$\|xg(\nabla^\theta)\|_{1,\infty} \leq c_p'\|x\|_{W^{p,1}}\|g\|_{\ell_{1,\infty}(L_\infty)}.$$
        \end{enumerate}
    \end{thm}
    
    With Theorem \ref{nc cwikel theorem} at hand we can prove the following:
    \begin{thm}\label{nc dirac cwikel}
        Let $p\geq 1$. Then there exist constants $c_p > 0$ and $c_p' > 0$ such that for all $x \in W^{p,1}(\mathbb{R}^p_\theta)$ we have:
        \begin{enumerate}[{\rm (a)}]
            \item{}\label{first cwikel} $(1\otimes x)(D^\theta+i\lambda)^{-p-1} \in \mathcal{L}_1$ and $$\|(1\otimes x)(D^\theta+i\lambda)^{-p-1}\|_1 \leq c_p\frac{\|x\|_{W^{p,1}}}{\lambda},$$
            \item{}\label{second cwikel} $(1\otimes x)(D^\theta+i)^{-p} \in \mathcal{L}_{1,\infty}$ and $$\|(1\otimes x)(D^\theta+i)^{-p}\|_{1,\infty} \leq c_p'\|x\|_{W^{p,1}}.$$
        \end{enumerate}
    \end{thm}
    \begin{proof}
        Let $g(t) := (\lambda^2+\sum_{k=1}^p t_k^2)^{-(p+1)/2}$. Since $\|ab\|_1 = \|a|b^*|\|_1$,
        \begin{align*}
            \|(1\otimes x)(D^\theta + /i\lambda)^{-p-1}\|_1 &= \|(1\otimes x)|(D^\theta-i\lambda)^{-p-1}|\|_1\\
                                                          &= \|(1\otimes x)((D^\theta)^2+\lambda^2)^{-\frac{p+1}{2}}\|_1.
        \end{align*}
        So we have
        \begin{equation*}
            \|(1\otimes x)(D^\theta+i\lambda)^{-p-1}\|_1 = c_p\|xg(\nabla^\theta)\|_1.
        \end{equation*}
        It can be directly verified that $\|g\|_{\ell_1(L_\infty)} = O(\lambda^{-1})$, we can immediately apply Theorem \ref{nc cwikel theorem} to obtain \eqref{first cwikel}.
        
        To obtain \eqref{second cwikel}, we instead consider the function $g(t) = (1+\sum_{k=1}^p t_k^2)^{-p/2}$ and apply Theorem \ref{nc cwikel theorem}.\eqref{second nc cwikel}.
    \end{proof}
    
    Recall the operator $\Lambda$ from Section \ref{replacement section}, defined formally as $\Lambda(T) = (1+D^2)^{-1/2}[D^2,T]$.

    \begin{lem}\label{ncplane main assumption for lambda} 
    If $x\in W^{\infty,1}(\mathbb{R}^p_{\theta})$, then for all $m\geq 0$:
        \begin{equation*}
            \left\|\Lambda^m(1\otimes x)(D^\theta+i\lambda)^{-p-1}\right\|_1 = O(\lambda^{-1}),\quad\lambda\to\infty.
        \end{equation*}
    \end{lem}
    \begin{proof}
        We prove the assertion by induction on $m.$ Since by definition $\Lambda^0$ is the identity, the $m=0$ case is handled by Theorem \ref{nc dirac cwikel}.\eqref{first cwikel}.

        Now suppose that $m\geq 1$ and the assertion holds for $m-1.$ 
        Since $(D^\theta)^2 = 1\otimes \sum_{k=1}^p (D_k^\theta)^2$, we have
        \begin{equation*}
            \Lambda(1\otimes x) = (1+(D^\theta)^2)^{-\frac{1}{2}}\cdot(\sum_{k=1}^p1\otimes[(D_k^\theta)^2,x]).
        \end{equation*}
        Applying the Leibniz rule,
        \begin{align*} 
            [(D_k^\theta)^2,x] &= [D_k^\theta,x]D_k^\theta+D_k^\theta[D_k^\theta,x]\\
                               &= 2D_k^\theta[D_k^\theta,x]-[D_k^\theta,[D_k^\theta,x]].
        \end{align*}
        By assumption, (the bounded extensions of) $[D_k^\theta,x]$ and $[D_k^\theta,[D_k^\theta,x]]$ are in $W^{m,1}(\mathbb{R}^p_{\theta})$ for all $m\geq0.$

        Hence since $\Lambda$ commutes with $\partial$,
        \begin{align*}
            \Lambda^m(1\otimes x) = \sum_{k=1}^p&\Big(1\otimes \frac{2D_k^\theta}{(1-\Delta^\theta)^{\frac12}}\Big)\cdot \Lambda^{m-1}(1\otimes [D_k^\theta,x])\\
                                  &-\sum_{k=1}^p\Big(1\otimes\frac1{(1-\Delta^\theta)^{\frac12}}\Big)\cdot \Lambda^{m-1}(1\otimes [D_k^\theta,[D_k^\theta,x]]).
        \end{align*}
        So by the triangle inequality, we have
        \begin{align*}
            \left\|\Lambda^m(1\otimes x)(D^\theta+i\lambda)^{-p-1}\right\|_1 &\leq 2\sum_{k=1}^p\left\|\Lambda^{m-1}(1\otimes [D_k^\theta,x])(D^\theta+i\lambda)^{-p-1}\right\|_1\\
                                                                             &\quad +\sum_{k=1}^p\|\Lambda^{m-1}(1\otimes [D_k^\theta,[D_k^\theta,x]])(D^\theta+i\lambda)^{-p-1}\Big\|_1.
        \end{align*}
        The right hand side is $O(\lambda^{-1})$ as $\lambda\to\infty$ by the inductive assumption. Hence, so is the left hand side.
    \end{proof}

    We can now conclude with the main result of this subsection:
    \begin{thm}
        The triple
        \begin{equation*}
            (1\otimes W^{\infty,1}(\mathbb{R}^p,\theta),L_2(\mathbb{R}^d,\mathbb{C}^{2^{p/2}}),D^\theta)
        \end{equation*}
        is a spectral triple satisfying Hypothesis \ref{main assumption}.
    \end{thm}
    \begin{proof}
        We establish Hypothesis \ref{replacement assumption} instead, as permitted by Theorem \ref{replacement thm}. 
        First we prove that we indeed have a spectral triple. 
        
        By the definition of $W^{\infty,1}(\mathbb{R}^p)$, if $x \in W^{\infty,1}(\mathbb{R}^p)$ then $[D^\theta,1\otimes x]$
        has bounded extension, and therefore $1\otimes x:\mathrm{dom}(D^\theta)\to \mathrm{dom}(D^\theta)$.
        
%         {  [it has not been proved that $1\otimes x:\mathrm{dom}(D)\to \mathrm{dom}(D)$]}
        
        If $x \in W^{\infty,1}(\mathbb{R}^p_\theta)$ then:
        \begin{equation*}
            \partial(1\otimes x) = \sum_{j=1}^p \gamma_j\otimes (\partial_jx)
        \end{equation*}
        and this is bounded, by the definition of $W^{\infty,1}$.        
        
        Now we show that Hypothesis \ref{replacement assumption}.\eqref{rass0} holds.
        If $x \in W^{\infty,1}(\mathbb{R}^p_\theta)$, we show that $x,\partial(x) \in \mathrm{dom}(\Lambda^m)$ for all $m\geq 0$ by induction. We have automatically that $x,\partial(x) \in \mathrm{dom}(\Lambda^0)$. Now if we assume that $x,\partial(x) \in \mathrm{dom}(\Lambda^{m-1})$, for $m \geq 1$, we apply the Leibniz rule to obtain:
        \begin{align*}
            \Lambda^m(1\otimes x) &= \sum_{k=1}^p\Big(1\otimes \frac{2D_k^\theta}{(1-\Delta^\theta)^{\frac12}}\Big)\cdot \Lambda^{m-1}(1\otimes [D_k^\theta,x])\\
                                  &\quad-\sum_{k=1}^p\Big(1\otimes\frac1{(1-\Delta^\theta)^{\frac12}}\Big)\cdot \Lambda^{m-1}(1\otimes [D_k^\theta,[D_k^\theta,x]]).
        \end{align*}
        By the definition of $W^{\infty,1}(\mathbb{R}^p_\theta)$, the operators $[D_k^\theta,x]$ and $[D_k^\theta,[D_k^\theta,x]]$ have bounded extension, and by Lemma \ref{nc poincare}, the extensions of $[D_k^\theta,x]$ and $[D_k^\theta,[D_k^\theta,x]]$ are elements of $W^{\infty,1}(\mathbb{R}^p_\theta)$, and therefore by the inductive hypothesis are in $\mathrm{dom}(\Lambda^{m-1})$. Hence, $1\otimes x \in \mathrm{dom}(\Lambda^m)$ and so by induction $1\otimes x \in \mathrm{dom}_\infty(\Lambda)$. Applying an identical argument to $\partial(1\otimes x)$ yields $\partial(1\otimes x) \in \mathrm{dom}_\infty(\Lambda)$, and so $(1\otimes W^{\infty,1}(\mathbb{R}^p_\theta),L_2(\mathbb{R}^d,\mathbb{C}^{2^{p/2}}),D^\theta)$ is $\Lambda$-smooth.        
        
        We now show that Hypothesis \ref{replacement assumption}.\eqref{rass1} holds. 
        Let $x \in W^{\infty,1}(\mathbb{R}^p_\theta)$. By Lemma \ref{nc dirac cwikel}.\eqref{first cwikel}, the first inclusion in Hypothesis \ref{replacement assumption}.\eqref{rass1} follows. 
        
        To see the second inclusion in Hypothesis \ref{replacement assumption}.\eqref{rass1}, write
        \begin{equation*}
            \partial(1\otimes x)(D^\theta+i)^{-p} = \sum_{k=1}^p \left(\gamma_k\otimes 1\right)\cdot\left((1\otimes [D_k^\theta,x])(D^\theta+i)^{-p}\right).
        \end{equation*}
        Using the quasi-triangle inequality for $\mathcal{L}_{1,\infty}$, there is a constant $C_p$ such that
        \begin{equation*}
            \|\partial(1\otimes x)(D^\theta+i)^{-p}\|_{1,\infty} \leq C_p\sum_{k=1}^p \|(1\otimes \partial_k x)(D^\theta+i)^{-p}\|_{1,\infty}.
        \end{equation*}
        By the definition of $W^{\infty,1}(\mathbb{R}^p_\theta)$, for all $1 \leq k \leq p$ we have $\partial_k x \in W^{\infty,1}(\mathbb{R}^p_\theta)$, so we may apply Theorem \ref{nc dirac cwikel}.\eqref{second cwikel} to each summand to deduce the second inclusion in Hypothesis \ref{replacement assumption}.\eqref{rass1}.
        
        Now we discuss Hypothesis \ref{replacement assumption}.\eqref{rass2}. By Lemma \ref{ncplane main assumption for lambda}, the first inequality in Hypothesis \ref{replacement assumption}.\eqref{rass2} holds.
        
        To deduce the second inequality, we may commute $\partial$ with $\Lambda^{m}$ to obtain:
        \begin{equation*}
            \partial(\Lambda^m(1\otimes x))(D^\theta+i\lambda)^{-p-1} = \sum_{k=1}^p \left(\gamma_k\otimes 1\right)\cdot\left(\Lambda^m(1\otimes [D_k^\theta,x]\right)(D^\theta+i\lambda)^{-p-1}.
        \end{equation*}
        Note that here $\partial(T)$ denotes $[D^\theta,T]$. Using the $\mathcal{L}_1$-norm triangle inequality,
        \begin{equation*}
            \|\partial(\Lambda^m(1\otimes x))(D^\theta+i\lambda)^{-p-1}\|_1 \leq C_p\sum_{k=1}^p \|\Lambda^m(1\otimes \partial_k x)(D^\theta+i\lambda)^{-p-1}\|_1.
        \end{equation*}
        By assumption, each $\partial_k x$ is in $W^{\infty,1}(\mathbb{R}^p_\theta)$, and so by Lemma \ref{ncplane main assumption for lambda}, each summand above is $O(\lambda^{-1})$
        as $\lambda \to \infty$.
% To see the second inequality in Hypothesis \ref{replacement assumption} \eqref{rass2}, let us write
% $$\partial(\Lambda^m(1\otimes x))(D+i\lambda)^{-p-1}=\sum_{k=1}^p\Big(\varsigma_k\otimes 1\Big)\cdot\Big(\Lambda^m(1\otimes [D_k,x])(D+i\lambda)^{-p-1}\Big).$$
% By triangle inequality, we have
% $$\Big\|\partial(\Lambda^m(1\otimes x))(D+i\lambda)^{-p-1}\Big\|_1\leq \sum_{k=1}^p\Big\|\Lambda^m(1\otimes [D_k,x])(D+i\lambda)^{-p-1}\Big\|_{1,\infty}.$$
% Since $[D_k,x]\in\mathcal{A},$ it follows from Lemma \ref{ncplane hyp111 for lambda} that the right hand side is $O(\lambda^{-1}).$ Hence, we also have 
% $$\Big\|\partial(\Lambda^m(1\otimes x))(D+i\lambda)^{-p-1}\Big\|_1=O(\lambda^{-1}),\quad\lambda\to\infty.$$
    \end{proof}
    
    \begin{rem}
        We have worked exclusively with the case that $\det(\theta) \neq 0$. Of course this excludes the fundamental $\theta = 0$ case of Euclidean space $\mathbb{R}^d$. One
        may verify directly that the standard spectral triple for $\mathbb{R}^d$ satisfies Hypothesis \ref{main assumption}, by using classical Cwikel theory, or alternatively
        $\mathbb{R}^d$ may be considered as a special case of the complete Riemannian manifolds considered in the following section.
    \end{rem}

% \section[Hypothesis \ref{main assumption} for Euclidean space]{Verification of Hypothesis \ref{main assumption} for Euclidean space}
% 
% Let $\mathcal{A}$ be the algebra of all Schwartz functions. Let $\varsigma_k,$ $1\leq k\leq p,$ be Pauli matrices. Define Dirac operator as
% $$D=\sum_{k=1}^p\varsigma_k\otimes D_k.$$
% Here, $D_k$ is the differentiation operator with respect to the $k-$th coordinate. Our algebra representation is $x\to 1\otimes M_x.$
% 
% \begin{thm}\label{plane verification thm} Spectral triple $(\mathcal{A},H,D)$ for the Noncommutative Euclidean space constructed in Section \ref{ncplane def section} satisfies Hypothesis \ref{main assumption}.
% \end{thm}
% \begin{proof} The proof follows that of Theorem \ref{ncplane verification thm} {\it mutatis mutandi}. The only difference is that we refer to Theorem 4.5 in \cite{LeSZ-cwikel} instead of Theorems 7.6 and 7.7 in \cite{LeSZ-cwikel}.
% \end{proof}

\section{Example: Riemannian manifolds}\label{manifold section}

    The authors wish to thank Professor Yuri Kordyukov for significant contributions to this section, including providing many of the proofs.

\subsection{Basic notions about manifolds}
    
%     {  The wording in this subsection should be checked very carefully}
    We briefly recall the relevant definitions for Riemannian manifolds. The material in this subsection is standard,
    and may be found in for example \cite[Chapter 2]{rosenberg} or \cite{Lawson-Michelsohn-1989}. 
    Let $X$ be a second countable $p$-dimensional complete smooth Riemannian manifold with metric tensor $g$.
    Recall that $g$ defines a canonical measure $\nu_g$ on $X$. The notation $L_r(X,g)$ denotes $L_r(X,\nu_g)$. The assumption that $X$ is second countable
    ensures that $L_2(X,g)$ is separable.
    
    We denote the space of smooth compactly supported differential $k$-forms as $\Omega_c^k(X)$, and define $\Omega_c(X) := \bigoplus_{k=0}^p \Omega_c^k(X)$.
    Associated to the metric $g$ is an inner product $(\cdot,\cdot)_g$ defined on $\Omega_c(X)$.
    Let $H_k$ denote the completion of $\Omega^k_c(X)$ with respect to this inner product, and define $L_2\Omega(X,g) := \bigoplus_{k=0}^p H_k$.
    There is a grading on $L_2\Omega(X,g)$ with grading operator $\Gamma$ defined by $\Gamma|_{H_k} = (-1)^k$.
    
    For $f \in C^\infty_c(X)$, let $M_f$ denote the operator of pointwise multiplication by $f$ on $L_2\Omega(X,g)$.
    
    The exterior differential $d$ is a linear map $d:\Omega_c(X)\to\Omega_c(X)$ such that for all $k = 0,1,\ldots,p-1$
    we have
    $d:\Omega^{k}_c(X)\to \Omega^{k+1}_c(X)$, and $d|\Omega^p_c(X) = 0$. The linear operator $d$ has a formal
    adjoint $d^*$ with respect to the inner product on $\Omega_c(X)$.
    
    The Hodge-Dirac operator $D_g$ is defined by $D_g := d+d^*$. Since $X$ is complete, the operator $D_g$ uniquely extends to a self-adjoint unbounded operator on $L_2\Omega(X,g)$ (see \cite{chernov}). The Hodge-Laplace operator is defined as $\Delta_g := -D_g^2 = -dd^*-d^*d$, and each subspace $H_k$
    is invariant under $\Delta_g$. The restriction of $\Delta_g$ to $H_0 = L_2(X,g)$ coincides with the Laplace-Beltrami operator.
    
    The main focus of this section is the following:
    \begin{thm}\label{manifold theorem} 
        Let $(X,g)$ be a second countable $p$-dimensional complete Riemannian manifold. The algebra $C^\infty_c(X)$ acts on $L_2\Omega(X,g)$ by pointwise multiplication, and $D_g$ denotes
        the Hodge-Dirac operator. Then
        \begin{equation*}
            (C^\infty_c(X),L_2\Omega(X,g),D_g)
        \end{equation*}
        is an even spectral triple satisfying Hypothesis \ref{main assumption}, where the grading $\Gamma$ is defined by $\Gamma|_{H_k} = (-1)^k$.
    \end{thm}
    
    Note that the spectral triple is always even regardless of $p$. The use of the Hodge-Dirac operator to define spectral triples for arbitrary Riemannian manifolds has previously been studied in \cite{Lord-Rennie-Varilly}, and for related work see \cite{Frohlich-Grandjean-Recknagel}.
    
    To the best of our knowledge, the main results of this paper, Theorems \ref{heat thm}, \ref{zeta thm} and \ref{main thm} are new in the setting of the Hodge-Dirac operator on arbitrary complete manifolds. Most previous work on geometric applications of noncommutative geometry, such as \cite{Rennie-2004} and \cite{CGRS2} are applied to a spin Dirac operator. 
        
    The Cwikel-type estimates we establish in this section: Lemma \ref{kord first lemma} and \ref{kord second lemma}, are of interest in their own right. A predecessor to this work may be found in \cite{Rennie-2004}.
    
\subsection{Proof of Theorem \ref{manifold theorem}}

    The proof proceeds by showing the required Cwikel-type estimates for the case of a torus: $X = \mathbb{T}^p$ with the flat metric. We then
    deduce the general case by an argument involving local coordinates.
    
%     We consider $\mathbb{T}^p$ with complex coordinates $\{z_1,\ldots,z_p\}$ such that $|z_j| = 1$ for all $1\leq j \leq p$.
    
     
    We define the $p$-torus as $\mathbb{T}^p := \mathbb{R}^p/\mathbb{Z}^p$. The space $\mathbb{T}^p$ is a smooth $p$-dimensional manifold,
    and we may select local coordinates $x_1,\ldots,x_p \in (0,1]$ defined by considering the image of $(x_1,\ldots,x_d)$ in $\mathbb{R}^p/\mathbb{Z}^p$.
%     We define $\mathbb{T} := \{z \in \mathbb{C}\;:\;|z|=1\}$ and consider the $p$-torus $\mathbb{T}^p$. This is a $p$-dimensional
%     smooth manifold. Near any point may select local coordinates $x_1,\ldots,x_p \in (0,1)$ such that $(e^{2\pi i x_1},e^{2\pi i x_2},\ldots,e^{2\pi i x_p}) \in \mathbb{T}^p$.
%     
    We equip $\mathbb{T}^p$ with the flat metric $g_0$ defined locally by $g_0 = dx_1^2+\cdots+dx_p^2$. 
    
    
    First, we describe the Cwikel-type estimates for $\mathbb{T}^p$.
    \begin{lem}\label{cwikel}
        Let $g_0$ denote the flat metric on $\mathbb{T}^p$, with corresponding Hodge-Dirac operator denoted $D_0$. Then:
        \begin{enumerate}[{\rm (i)}]
            \item{}\label{flat cwikel 1} We have 
                \begin{equation*}
                    (D_0+i)^{-1} \in \mathcal{L}_{p,\infty}(L_2\Omega(\mathbb{T}^p,g_0)).
                \end{equation*}
            \item{}\label{flat cwikel 2} For $\lambda \to\infty$, we have:
                \begin{equation*}
                    \|(D_0+i\lambda)^{-1}\|_{\mathcal{L}_{p+1}(L_2\Omega(\mathbb{T}^p,g_0))} = O(\lambda^{-\frac{1}{p+1}}).
                \end{equation*}
        \end{enumerate}
    \end{lem}
        
    Since the manifold $\mathbb{T}^p$ is compact, there is essentially no difference between the spaces $L_2\Omega(\mathbb{T}^p,g)$ for different metrics $g$ as the following lemma shows:
    \begin{lem}\label{L_2 is invariant under changing metric}
        Let $g_0$ be the flat metric on $\mathbb{T}^p$, and let $g$ be an arbitrary metric on $\mathbb{T}^p$. Then the Hilbert spaces $L_2\Omega(\mathbb{T}^p,g)$ and $L_2\Omega(\mathbb{T}^p,g_0)$
        coincide with an equivalence of norms. To be precise, there exist constants $0 < c_g < C_g < \infty$ with $v \in L_2\Omega(\mathbb{T}^p,g)$ we have:
        \begin{equation*}
            c_g\|v\|_{L_2\Omega(\mathbb{T}^p,g_0)} \leq \|v\|_{L_2\Omega(\mathbb{T}^p,g)} \leq C_g\|v\|_{L_2\Omega(\mathbb{T}^p,g_0)}.
        \end{equation*}
    \end{lem}
    \begin{proof}
        The metric $\nu_g$ corresponding to $g$ has Radon-Nikodym derivative $\sqrt{|\det(g)|}$ with respect to $\nu_{g_0}$. Since $\mathbb{T}^p$
        is compact and $g$ is positive definite, the Radon-Nikodym derivative $\sqrt{|\det(g)|}$ is bounded above and bounded away from zero. Hence,
        the $L_2$-norms corresponding to $\nu_g$ and $\nu_{g_0}$ are equivalent.
    \end{proof}
    
    Sobolev spaces on $\mathbb{T}^p$ {  -- and more generally on a compact Riemannian manifold -- are defined following \cite{Lawson-Michelsohn-1989}:}
    \begin{defi}
        Let $g$ be a metric on $\mathbb{T}^p$, and for $j=1,\ldots,p$ we let $\frac{\partial}{\partial x_j}$ denote the differentiation
        with respect to the $j$th coordinate of $\mathbb{T}^p$. The Sobolev space $H^1\Omega(\mathbb{T}^p)$ is defined to be the set of $v \in L_2\Omega(\mathbb{T}^p,g)$ such that the Sobolev norm:
        \begin{equation*}
            \|v\|_{H^1\Omega(\mathbb{T}^p,g)}^2 := \|v\|_{L_2\Omega(\mathbb{T}^p,g)}^2 + \sum_{j=1}^p \left\|\frac{\partial v}{\partial x_j}\right\|_{L_2\Omega(\mathbb{T}^p,g)}^2
        \end{equation*}
        is finite.
    \end{defi}
    Lemma \ref{L_2 is invariant under changing metric} shows that the space $H^1\Omega(\mathbb{T}^p)$ is independent of the choice of metric used to define the Sobolev norm,
    since different metrics will define equivalent norms.
        
    The following result is the well known G\aa rding's inequality. A proof for general compact manifolds may be found in \cite[Theorem 2.44]{rosenberg} {  and for general elliptic operators on compact manifolds in \cite[Theorem 5.2]{Lawson-Michelsohn-1989}. }
    \begin{lem}\label{gardings inequality}
        Let $g$ be a metric on $\mathbb{T}^p$, and let $D_g$ denote the corresponding Dirac operator. Then there is a constant $C_g >0$ such that for all $v \in L_2\Omega(\mathbb{T}^p,g)$ such
        that $D_gv \in L_2\Omega(\mathbb{T}_p,g)$, we have
        \begin{equation*}
            \|v\|_{H^1\Omega(\mathbb{T}^p,g)} \leq C_g(\|v\|_{L_2\Omega(\mathbb{T}^p)}+\|D_gv\|_{L_2\Omega(\mathbb{T}^p,g)}).
        \end{equation*}
    \end{lem}
    
    The following lemma is the essential technical result allowing us to transfer the Cwikel-type estimates for $\mathbb{T}^p$ with the flat metric
    to $\mathbb{T}^p$ with an arbitrary metric.
    \begin{lem}\label{main transition lemma} 
        Let $g$ be an arbitrary metric on $\mathbb{T}^p$, and let $g_0$ be the flat metric on $\mathbb{T}^p$. Denote by $D_g$ and $D_0$ the Hodge-Dirac
        operators corresponding to $g$ and $g_0$ respectively. Then the operator
        \begin{equation*}
            D_0(D_g+i)^{-1}
        \end{equation*}
        defined initially on $\Omega(\mathbb{T}^p)$ has bounded extension to $L_2\Omega(\mathbb{T}^p,g)$ (or equivalently $L_2\Omega(\mathbb{T}^p,g_0)$).
    \end{lem}
    \begin{proof} 
        By the definition of the Sobolev norm $\|\cdot \|_{H^1\Omega(\mathbb{T}^p,g)}$, for all $v \in H^1\Omega(\mathbb{T}^p,g)$ we have:
        \begin{equation*}
            \|D_0v\|_{L_2\Omega(\mathbb{T}^p,g)}^2 \leq \|v\|_{H^1\Omega(\mathbb{T}^p,g)}^2.
        \end{equation*}
        Thus by Lemma \ref{gardings inequality}, there is a constant $C$ such that,
        \begin{equation*}
            \|D_0v\|_{L_2\Omega(\mathbb{T}^p,g)} \leq C(\|D_gv\|_{L_2\Omega(\mathbb{T}^p,g)}+\|v\|_{L_2\Omega(\mathbb{T}^p,g)}).
        \end{equation*}
        Hence, if $v \in \mathrm{dom}(D_g)$ then $v \in \mathrm{dom}(D_0)$, and so if $u \in L_2\Omega(\mathbb{T}^p,g)$ and $v = (D_g+i)^{-1}u$ then $v \in \mathrm{dom}(D_0)$. 
        So substituting $v = (D_g+i)^{-1}u$ we obtain:
        \begin{equation*}
            \|D_0(D_g+i)^{-1}u\|_{L_2\Omega(\mathbb{T}^p,g)} \leq C(1+\|(D_g+i)^{-1}\|_{\mathcal{L}_{\infty}(L_2\Omega(\mathbb{T}^p,g)})\|u\|_{L_2\Omega(\mathbb{T}^p,g)}.
        \end{equation*}
        However since $(D_g+i)^{-1}$ is bounded, we get:
        \begin{equation*}
            \|D_0(D_g+i)^{-1}u\|_{L_2\Omega(\mathbb{T}^p,g)} \leq C\|u\|_{L_2\Omega(\mathbb{T}^p,g)}.
        \end{equation*}
    \end{proof}
    
    As a consequence of Lemmas \ref{cwikel} and \ref{main transition lemma}, we get:
    \begin{cor}\label{non flat cwikel}
        Let $g$ be an arbitrary metric on $\mathbb{T}^p$ and let $D_g$ be the corresponding Hodge-Dirac operator. Then,
        \begin{enumerate}[{\rm (i)}]
            \item{}\label{non flat cwikel 1} We have,
                \begin{equation*}
                    (D_g+i)^{-1} \in \mathcal{L}_{p,\infty}(L_2\Omega(\mathbb{T}^p,g)).
                \end{equation*}
            \item{}\label{non flat cwikel 2} For $\lambda\to\infty$, we have
                \begin{equation*}
                    \|(D_g+i\lambda)^{-1}\|_{\mathcal{L}_{p+1}(L_2\Omega(\mathbb{T}^p,g))} = O(\lambda^{-\frac{1}{p+1}}).
                \end{equation*}
        \end{enumerate}
    \end{cor}
    \begin{proof}
        Part \eqref{non flat cwikel 1} follows immediately from Lemma \ref{cwikel} and Lemma \ref{main transition lemma}.\ref{flat cwikel 1}.
        
        Now we prove \eqref{non flat cwikel 2}. First we compute $(D_0+i\lambda)(D_g+i\lambda)^{-1}$ (working on the dense domain $\Omega(\mathbb{T}^p)$):
        \begin{align*}
            (D_0+i\lambda)(D_g+i\lambda)^{-1} &= (D_0-D_g+D_g+i\lambda)(D_g+i\lambda)^{-1}\\
                                              &= 1 + (D_0-D_g)(D_g+i\lambda)^{-1}\\
                                              &= 1+D_0(D_g+i)^{-1}\frac{D_g+i}{D_g+i\lambda} - \frac{D_g}{D_g+i\lambda}.
        \end{align*}
        By Lemma \ref{main transition lemma}, the operator $D_0(D_g+i)^{-1}$ has bounded extension. Moreover, by functional calculus
        the operators $\frac{D_g+i}{D_g+i\lambda}$ and $\frac{D_g}{D_g+i\lambda}$ have bounded extension with norm bounded by a constant independent of $\lambda$.
        Thus,
        \begin{equation*}
            \|(D_0+i\lambda)(D_g+i\lambda)^{-1}\|_{\mathcal{L}_\infty(L_2\Omega(\mathbb{T}^p,g)} = O(1).
        \end{equation*}
        Now Lemma \ref{cwikel}.\eqref{flat cwikel 2} yields the result.
    \end{proof}
        
    With the above results in hand we are able to establish our first Cwikel-type result for $(X,g).$ A similar result to Lemma \ref{kord first lemma} can also be found in \cite[Proposition 13]{Rennie-2004}. The result in \cite{Rennie-2004} is very similar in nature (despite applying
    to the somewhat different situation of Riemannian spin manifolds). The method of proof here is different: we use reduce the problem to $\mathbb{T}^p$ by using local coordinates, rather than a doubling construction as employed in \cite{Rennie-2004}.
    
    \begin{lem}\label{kord first lemma} 
        Let $f\in C^{\infty}_c(X).$ We have
        \begin{equation*}
            M_f(D_g+i)^{-1}\in\mathcal{L}_{p,\infty}(L_2\Omega(X,g)).
        \end{equation*}
    \end{lem}
    \begin{proof}
        By using a partition of unity if necessary, we may assume without loss of generality that $f$ is supported in a single chart $(U,h)$ where $h:U\to \mathbb{R}^p$ is a homeomorphism
        onto its image, and since $f$ has compact support we may further assume without loss of generality that $h(U)$ is bounded. Since $h(U)$ is bounded, there is a sufficiently
        large box $[-N,N]^p$ with $h(U)$ in the interior of $[-N,N]^p$. By applying a translation and dilation if necessary, we may assume without loss of generality that $h(U)$
        is contained within the interior of the box $[0,1]^p$. By identifying the edges of $[0,1]^p$, we may view $h$ as a continuous function $h:U\to\mathbb{T}^p$.
%         By identifying the edges of $[-N,N]^p$, we may therefore view $h$ as a continuous function $h:U\to \mathbb{T}^p$.
        
        We define three smooth ``cut-off" functions $\phi_1,\phi_2,\phi_3$ compactly supported in $h(U)$, defined so that for each $j = 1,2,3$ we have $0\leq \phi_j \leq 1$,
        and
        \begin{enumerate}[{\rm (a)}]
            \item{} for all $x \in U$, $\phi_1(h(x))f(x) = f(x)$,
            \item{} we have $\phi_2\phi_1 = \phi_1$,
            \item{} we have $\phi_3\phi_2 = \phi_2$.
        \end{enumerate}
        In other words, on $\mathrm{supp}(f\circ h^{-1})$ we have $\phi_1=1$, and on $\mathrm{supp}(\phi_1)$ we have $\phi_2 = 1$ and on $\mathrm{supp}(\phi_2)$ we have $\phi_3 = 1$.
        
        For $j = 1,2,3$, we also define the function $\psi_j$ by pulling back $\phi_j$ to $X$:
        \begin{equation*}
            \psi_j(x) = \begin{cases}
                                    (\phi_j\circ h)(x),\quad x\in U.\\
                                    0,\quad x\notin U.
                                \end{cases}
        \end{equation*}
        Since $\phi_j$ is compactly supported in $h(U)$, the function $\psi_j$ is smooth and compactly supported in $U$.
        
        Let $g_0$ denote the flat metric on $\mathbb{T}^p$.
        The metric $g$ can be pushed forward by $h$ to a metric $h^*g$ on $h(U)$. We then define a new metric $g_1$ on $\mathbb{T}^p$ by:
        \begin{equation*}
            g_1 := (h^*g)\phi_3+g_0(1-\phi_3).
        \end{equation*}        
        Since $\phi_3$ is compactly supported in $h(U)$, the metric $g_1$ is well defined. Moreover, on $\mathrm{supp}(\phi_2)$ we have $\phi_3 = 1$ so on $\mathrm{supp}(\phi_2)$ the metric
        $g_1$ is identical to $h^*g$.
        
        We define a partial isometry $V:L_2\Omega(X,g)\to L_2\Omega(\mathbb{T}^p,g_1)$ on $\xi \in L_2\Omega(X,g)$, $z \in \mathbb{T}^p$ by:
        \begin{equation*}
            V\xi(z) = \begin{cases}
                            \xi\circ h^{-1}(z),\quad z \in \mathrm{supp}(\phi_2),\\
                            0,\quad z\notin \mathrm{supp}(\phi_2).
                      \end{cases}
        \end{equation*}
        By construction, $V$ induces an isometry from $L_2\Omega(\mathrm{supp}(\psi_2),g)\to L_2\Omega(\mathrm{supp}(\phi_2),g_1)$, and
        \begin{align*}
            VV^* &= M_{\chi_{\mathrm{supp}{\phi_2}}},\\
            V^*V &= M_{\chi_{\mathrm{supp}{\psi_2}}}.
        \end{align*}
        
        We also have that if $j = 1,2$ then
        \begin{equation*}
            M_{\phi_j}V = VM_{\psi_{j}}.
        \end{equation*}
        
        We use the important fact that for $j = 1,2$, we have an equality on $\Omega_c(X)$:
        \begin{equation*}
            V^*D_{g_1}M_{\phi_j} = D_{g}M_{\psi_j}V^*.
        \end{equation*}
        
        Next we consider the following two operators on $L_2\Omega(\mathbb{T}^p,g_1)$.
        \begin{align*}
            P := M_{\phi_1}(D_{g_1}+i)^{-1}M_{\phi_1}\\
            Q := M_{\phi_2}(D_{g_1}+i)^{-1}M_{\phi_2}.
        \end{align*}
        By Lemma \ref{non flat cwikel}, the operators $P$ and $Q$ are in $\mathcal{L}_{p,\infty}(L_2\Omega(\mathbb{T}^p,g_1))$.
        
        We now consider the operator $(D_g+i)M_{\psi_2}V^*PV$. We note that this operator is well defined on $\Omega_c(X)$, since if $u \in \Omega_c(X)$, then $M_{\psi_2}V^*PV$ is 
        smooth and supported in $\mathrm{supp}{\psi_2}$.
        Hence (working on $\Omega_c(X)$):
        \begin{align*}
            (D_g+i)M_{\psi_2}V^*PV &= V^*(D_{g_1}+i)M_{\phi_2}PV\\
                                    &= V^*(D_{g_1}+i)M_{\phi_1}(D_{g_1}+i)^{-1}M_{\phi_1}V\\
                                    &= V^*([D_{g_1},M_{\phi_1}](D_{g_1}+i)^{-1}M_{\phi_1}+M_{\phi_1}^2)V.
        \end{align*}
        Now recalling that $D_{g_1}$ is a local operator, we have $[D_{g_1},M_{\phi_1}] = [D_{g_1},M_{\phi_1}]M_{\phi_2}$.
        
        Moreover, $V^*M_{\phi_1}^2V = M_{\psi_1}^2$ and so
        \begin{equation*}
            (D_g+i)M_{\psi_2}V^*PV = V^*[D_{g_1},M_{\phi_1}]QM_{\phi_1}V+M_{\psi_1}^2.
        \end{equation*}
        Now multiplying on the left by $(D_g+i)^{-1}$, we arrive at:
        \begin{equation*}
            M_{\psi_2}V^*PV = (D_g+i)^{-1}V^*[D_{g_1},M_{\phi_1}]QM_{\phi_1}V+(D_g+i)^{-1}M_{\psi_1}^2.
        \end{equation*}
        The final step is to use the fact that $\psi_2=1$ on the support of $f$, so we may use $M_{\psi_2}^2M_f = M_f$, and multiply on the right by $M_f$
        to obtain:
        \begin{equation*}
            M_{\psi_2}V^*PVM_f = (D_g+i)^{-1}V^*[D_1,M_{\phi_1}]QM_{\phi_1}VM_f+(D_g+i)^{-1}M_f.
        \end{equation*}
        Since both $P$ and $Q$ are in $\mathcal{L}_{p,\infty}(L_2\Omega(\mathbb{T}^p,g_1))$, we finally obtain that $(D_g+i)^{-1}M_f \in \mathcal{L}_{p,\infty}(L_2\Omega(X,g))$.
    \end{proof}
    
    \begin{lem}\label{kord second lemma} 
        Let $f\in C^{\infty}_c(X).$
        Then:
        \begin{equation*}
            \|M_f(D_g+i\lambda)^{-1}\|_{\mathcal{L}_{p+1}(L_2\Omega(X,g))} = O(\lambda^{-\frac{1}{p+1}}),\quad \lambda\to\infty.
        \end{equation*}
    \end{lem}
    \begin{proof} 
        This proof proceeds along similar lines to Lemma \ref{kord first lemma}. We again assume without loss of generality
        that $f$ is supported in a single chart $(U,h)$, and construct the metric $g_1$ on $\mathbb{T}^p$ and the partial isometry $V$ identically to the proof of Lemma \ref{kord first lemma}. We
        also use the same cut-off functions $\phi_1$, $\phi_2$ and $\phi_3$, and $\psi_1,\psi_2,\psi_3$.
        
        In place of the operators $P$ and $Q$, we introduce $P_{\lambda}$ and $Q_{\lambda}$ given by:
        \begin{align*}
            P_{\lambda} &= M_{\phi_1}(D_{g_1}+i\lambda)^{-1}M_{\phi_1}\\
            Q_{\lambda} &= M_{\phi_2}(D_{g_1}+i\lambda)^{-1}M_{\phi_2}.
        \end{align*}
        
        Following the argument of Lemma \ref{kord first lemma} with $P_{\lambda}$ and $Q_\lambda$ in place of $P$ and $Q$, we arrive at:
        \begin{equation*}
            M_{\psi_2}V^*P_{\lambda}VM_f = (D_g+i\lambda)^{-1}V^*[D_1,M_{\phi_1}]Q_{\lambda}M_{\phi_1}VM_f + (D_g+i\lambda)^{-1}M_f.
        \end{equation*}
        
        Due to Lemma \ref{non flat cwikel}.\eqref{non flat cwikel 1}, we have $\|P_\lambda\|_{p+1} = O(\lambda^{-\frac{1}{p+1}})$ and similarly for $Q_\lambda$. Thus,
        \begin{equation*}
            \|(D_g+i\lambda)^{-1}\|_{\mathcal{L}_{p+1}(L_2\Omega(X,g))} = O(\lambda^{-\frac{1}{p+1}}).
        \end{equation*}
    \end{proof}

    We now finally have the results necessary to prove Theorem \ref{manifold theorem}.
    \begin{proof}[Proof of Theorem \ref{manifold theorem}] 
%         {  It is a standard fact that $(C_c^\infty,L_2\Omega(X,g),D_g)$ is a smooth spectral triple.}
        We will instead work with Hypothesis \ref{replacement assumption}, as justified by Theorem \ref{replacement thm}. 
%         We prove the most difficult part of Hypothesis \ref{replacement assumption}, that is part \eqref{rass2}.
                
        First, we show that \ref{replacement assumption}.\eqref{rass1} holds
        for the triple $(C^\infty_c(X),L_2\Omega(X,g),D_g)$.
        
        To this end let $f \in C^\infty_c(X)$.
        We will prove by induction that for all $j\geq 1$, we have
        \begin{equation}\label{manifold inductive eq1}
            M_f(D_g+i)^{-j} \in \mathcal{L}_{\frac{p}{j},\infty}(L_2\Omega(X,g)).
        \end{equation}
        The case $j=1$ is already established by Lemma \ref{kord first lemma}. 
        
        Suppose now that \eqref{manifold inductive eq1} holds for $j \geq 1$. Choose $\phi \in C^\infty_c(X)$ such that $f\phi = f$. Then,
        \begin{align*}
            M_f(D_g+i)^{-j-1} &= M_{\phi}M_f(D_g+i)^{-1}\cdot (D_g+i)^{-j}\\
                              &= -M_{\phi}[(D_g+i)^{-1},M_f](D_g+i)^{-j}+M_{\phi}(D_g+i)^{-1}M_f(D_g+i)^{-j}\\
                              &= M_{\phi}(D_g+i)^{-1}[D,M_f](D_g+i)^{-j-1}+M_{\phi}(D_g+i)^{-1}M_f(D_g+i)^{-j}\\
                              &= M_{\phi}(D_g+i)^{-1}[D,M_f]M_{\phi}(D_g+i)^{-j-1}+M_{\phi}(D_g+i)^{-1}M_f(D_g+i)^{-j}.
        \end{align*}
        Due to Lemma \ref{kord first lemma}, we have $M_{\phi}(D_g+i)^{-1} \in \mathcal{L}_{p,\infty}$, and by the inductive assumption
        we also have $M_{\phi}(D_g+i)^{-j} \in \mathcal{L}_{\frac{p}{j},\infty}$. Then $M_f(D_g+i)^{-j-1} \in \mathcal{L}_{p,\infty}\cdot \mathcal{L}_{\frac{p}{j},\infty}$,
        so applying the H\"older inequality we arrive at $M_f(D_g+i)^{-j-1} \in \mathcal{L}_{\frac{p}{j+1},\infty}$. Taking $j = p$, we get that $M_f(D_g+i)^{-p} \in \mathcal{L}_{1,\infty}(L_2\Omega(X,g))$.
        
        Similarly, since $D_g$ is a local operator, we have that $[D_g,M_f] = [D_g,M_f]M_{\phi}$. Hence, $[D_g,M_f](D_g+i)^{-p} \in \mathcal{L}_{1,\infty}(L_2\Omega(X,g))$. This completes the proof of Hypothesis \ref{replacement assumption}.\eqref{rass1}
        in the case $k=0$.
        
        What remains is to show that Hypothesis \ref{replacement assumption}.\eqref{ass2} holds. We will first deal with the $k=0$ case. To that end, we will show by induction that for all $j\geq 1$:
        \begin{equation}\label{manifold inductive eq2}
            \|M_f(D_g+i\lambda)^{-j}\|_{\frac{p+1}{j}} = O(\lambda^{-\frac{j}{p+1}}),\lambda\to\infty.
        \end{equation}
        The base case $j = 1$ is the result of Lemma \ref{kord second lemma}, and the case $j=p+1$ is what is required for Hypothesis \ref{replacement assumption}.
        
        Suppose now that \eqref{manifold inductive eq2} holds for $j\geq 1$, and again choose $\phi\in C^\infty_c(X)$ such that $f\phi = f$. Then,
        \begin{align*}
            M_f(D_g+i\lambda)^{-j-1}  &= M_{\phi}\cdot M_f(D_g+i\lambda)^{-1}(D_g+i\lambda)^{-j}\\
                                      &= -M_{\phi}[(D_g+i\lambda)^{-1},M_f](D_g+i\lambda)^{-j}\\
                                      &\quad+M_{\phi}(D_g+i\lambda)^{-1}M_f(D_g+i\lambda)^{-j}\\
                                      &= M_{\phi}(D_g+i\lambda)^{-1}[D,M_f]M_{\phi}(D_g+i\lambda)^{-j-1}\\
                                      &\quad+M_{\phi}(D_g+i\lambda)^{-1}M_f(D_g+i\lambda)^{-j}.
        \end{align*}    
        By Lemma \ref{kord second lemma}, we have $\|M_{\phi}(D_g+i\lambda)^{-1}\|_{p+1} = O(\lambda^{-\frac{1}{p+1}})$, and by the inductive assumption
        we also have $\|M_{\phi}(D_g+i\lambda)^{-k}\|_{\frac{p+1}{j}} = O(\lambda^{-\frac{j}{p+1}})$. Then by the Holder inequality,
        \begin{equation*}
            \|M_f(D_g+i\lambda)^{-j-1}\|_{\frac{p+1}{j+1}} \leq O(\lambda^{-\frac{j}{p+1}})\cdot O(\lambda^{-\frac{1}{p+1}}).
        \end{equation*}
        So $\|M_f(D_g+i\lambda)^{-j-1}\|_{\frac{p+1}{j+1}} = O(\lambda^{-\frac{j+1}{p+1}})$. To conclude the same for $\partial(f)$ in place of $f$, we once more use the fact that $[D_g,M_f]M_{\phi} = [D_g,M_f]$.
        This completes the proof of the $k=0$ case of Hypothesis \ref{replacement assumption}.\eqref{rass2}.
        
        For $k > 0$, if $\phi f = f$, then we have
        \begin{equation*}
            \Lambda^k(M_f) = \Lambda^k(M_f)M_{\phi}
        \end{equation*}
        so we may apply the $k=0$ case to $\phi$ to deduce the result. 
        
        That $(C_c^\infty(X),L_2\Omega(X,g),D_g)$ satisfies
        \ref{replacement assumption}.\eqref{rass0} follows from similar reasoning.        
    \end{proof}

