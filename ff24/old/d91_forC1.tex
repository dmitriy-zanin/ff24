The project belongs to the broad field of Non-commutative Analysis. Shortly, this area appeared from the
foundational questions in Quantum Physics. While classically, observables are functions, in quantum world
observables are operators. In abstract terms, observables in a classical system form a commutative von
Neumann algebra, while the ones in a quantum system form a non-commutative von Neumann algebra. Noncommutative
topology and integration theory were developed in 1940-1960. Non-commutative Geometry, the
more specific field of this project, takes its origin in 1980's in the works of Alain Connes.

One of the objectives of this project is a Connes Character Formula. In short, it declares the equality between the
2 cocycles: one appears from the (non-commutative) Riemannian geometry, while another appears from the
conformal geometry. The formula can be found (without any proof) in the book "Noncommutative geometry" by Connes. A number of authors tried to prove it with various degrees of success. For compact non-commutative
manifolds, the final word is due to the CI and co-authors (the paper was published in the Journal of Spectral
Theory in 2016).

The cocycle in the Connes Character Formula is defined via Dixmier trace. Those traces are very special case of
singular traces. The CI co-authored world first monograph on singular traces and their applications and a number
of his papers in this direction are published in highly prestigious journals: Journal f�r die reine und angewandte
Mathematik (the oldest mathematical journal still in publishing), Advances in Mathematics, Journal of Functional
Analysis, etc.

Another objective in this project is to provide a non-commutative version of Minakshisundaram-Plejel theorem,
that is, to find an analogue for the heat kernel expansion. In this expansion, the coefficient in the most important
term is a constant, while the one in the second term is a scalar curvature. Higher order terms can be expressed in
terms of higher order differential invariants. Heuristically, such an expansion is strongly related with singular
traces: the most important term should be equal to the value of the singular trace.

For the simplest non-flat non-commutative manifold (a conformal deformation of a 2-dimensional noncommutative
torus), curvature was computed by Connes and collaborators (Tretkoff, Moscovici, Fathizadeh,
Khalkhali). In this project, we propose a new technique to compute the curvature (and higher order coefficients) by
using theory of Double Operator Integrals. The theory developing since 1970's recently experienced a
breakthrough by the CI and co-authors (to appear in American Journal of Mathematics).

Third objective in this project is to create a spectral triple from a given quantum group. Quantum groups are
certain deformations of well known Lie groups. The latter are, by definition, smooth manifolds. It is of crucial
importance that Dirac operator on such manifolds commutes with the action of the group on itself. A similar
property for quantum groups is called equivariance. A number of attempts were made to construct an equivariant
spectral triple for the simplest possible example, the quantum unitary group. These attempts suffer from the 2
drawbacks: one is dimension drop phenomenon (i.e. spectral dimension does not equal to the homological one),
the other is that the first term in the heat kernel expansion does not recover the Haar state (as it should). The CI
proposes to construct an equivariant spectral triple on quantum groups which is free from these drawbacks.

The CI's believes (and his publication list supports this) that all the stated objectives are plausible.
