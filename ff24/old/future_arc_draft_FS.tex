\documentclass{article}
\usepackage{amsmath,amsfonts,cite,color}
\usepackage[a4paper,margin=2cm]{geometry}

\newtheorem{thm}{Theorem}

\renewcommand{\baselinestretch}{0.95}
\newcommand{\archeading}[1]{\vspace{.3cm} \noindent{\bfseries #1} \vspace{.1cm}   }
\renewcommand{\refname}{{\normalsize References}}


\begin{document}

\parindent=20pt
\pagestyle{empty}
\newpage


PROJECT TITLE
AIMS AND BACKGROUND
FUTURE FELLOWSHIP CANDIDATE
PROPOSED PROJECT QUALITY AND INNOVATION
FEASIBILITY AND STRATEGIC ALIGNMENT
BENEFIT AND COLLABORATION
COMMUNICATION OF RESULTS
MANAGEMENT OF DATA
REFERENCES
ACKNOWLEDGEMENTS (IF REQUIRED)


\archeading{Project Title} Actions of quantum groups and noncommutative manifolds.

\bigskip\archeading{Aims and Background}

{\color{blue} The English below requires a very detailed vetting. Articles are practically absent. Ed or Steven? Or both...}

\paragraph*{Broad Aim} This proposal covers a novel viewpoint on noncommutative Riemannian manifolds and aims bringing together an international team of experts with a new approach to fundamental and long-standing problems in geometry and harmonic analysis.

There exist two theories of Riemannian manifolds: the classical one (which treats manifold as a locally Euclidean space) and the modern one introduced by Connes (which treats manifold as a spectral triple). A spectral triple consists of $\ast-$algebra $\mathcal{A}$ represented on a Hilbert space $H$ and an unbounded self-adjoint operator $D$ on $H.$ {\color{red} It is easy?} to compare both notions of a manifold: in the classical case, one lets $\mathcal{A}=C^{\infty}(X)$ (algebra of smooth functions on $X$), $H=L_2\Omega(X)$ (square-integrable forms on $X$) and lets $D$ to be a Hodge-Dirac operator. Famous Connes Reconstruction Theorem \cite{Connes-reconstruction} tells us that so-defined spectral triple captures all the geometric information about a manifold.

Manifolds are often equipped with a Lie group action. For instance, an orthogonal group ${\rm SO}(n)$ acts on the real sphere $\mathbb{S}^{n-1}.$ Moreover, rotations are exactly those automorphisms of the sphere which commute with Laplace-Beltrami operator. If, in general, a Lie group acts on a manifold $X,$ then one wants to take this action into account when constructing Laplace-type or Dirac-type operator. Specifically, one requires that Dirac-type operator commutes with the action of the group.

In the noncommutative realm, we have an action of a given Lie group (or even quantum group) on a $\ast-$algebra. Natural noncommutative analogue of "commutes with a group action" is an equivariance property. {\color{blue} One more sentence about importance of this property}



The CI will investigate actions of groups (more broadly, quantum groups) on noncommutative manifolds. {\color{red} May be: We shall introduce...}Those actions compel an equivariant Dirac operator which captures all the geometric information about the manifold (famous Connes Reconstruction Theorem). In other words, we suggest a completely different perspectiveon Connes Reconstruction Theorem and obtain a vast generalization of this theorem for non-commutative manifolds admitting equivariant property for actions of quantum groups.


%  which naturally extends the classical independence in the commutative probability theory as well as the free independence developed by Voiculescu in the noncommutative probability theory; develop new approaches to the centrally important concept of cumulant (an analogue of the Fourier transform); find stable random variables; during the project further develop the Law of Large Numbers and the Central Limit Theorem.
%
%On a second front the project will investigate problems in the noncommutative analysis related to the isomorphic embeddings of the Banach spaces. The investigator will introduce the analogue of the Kruglov operator.
%
%This proposal suggests a new and innovative approach to the notion of independence and provides a new path to the solution of fundamental and long-standing problems in the probability theory.
%
%Broadly speaking, there are two distinct theories of probability: the classical theory of random variables (that is of measurable functions on a probability measure space) and that of free probability originating mainly in works of D. Voiculescu and his students and collaborators approximately 40 years ago. In both theories, the notion of independence plays an absolutely fundamental, central role. In fact, one could view a probability theory as a part of general integration theory where elements are subjected to a certain independence condition. The classical notion of independence, which forms the basis of the classical probability theory is replaced in Voiculescu’s theory with a notion of freeness, which carries deep connections with the theory of free groups and their representations (and, hence, various applications in operator algebra theory), random matrices and Wigner’s distribution law (and, hence, non-trivial connections to modern mathematical physics, which ultimately made success and interest to Voiculescu’s free probability theory so wide spread). It is not easy to compare both notions of independence, even though from our point of view such a comparison appears to be paramount to any attempt to unify both theories into a single “theory of probability”. This proposal is the first meaningful attempt of such a unified theory: we aim to present a single viewpoint and a single theory of independence which encompasses, as two extreme cases, the classical and free probability theories and reveals a number of intermediate theories, depending on certain numerical parameters. For various values of such parameters we also recover and, in fact, provide a deeper insight into Nica’s theory [7] of so-called q−convolutions which interpolates between the usual convolution and the free convolution. That theory was a first serious and well-aimed (albeit, eventually unsuccessful) attempt to exhibit various independence laws which range between two cases q = −1 (the free probability case) and q = 1 (the classical case). Another attempt, closely linked to Nica’s work, was made in the paper of Bozeiko and Speicher [1], where the authors discuss q-independent random variables. 
%
%In this project we introduce a new conceptual approach to the notion of independence and suggest new tools for its study. Our approach naturally extends the classical notion of independence in commutative probability theory as well as that of free independence developed by Voiculescu in the noncommutative probability theory. We will develop a new approach to the centrally important concept of cumulant (an analogue of the Fourier transform) and define a notion of stable random variables naturally extending the classical and free versions of that notion. We intend to prove a new (noncommutative) Law of Large Numbers and a new (again, noncommutative) Central Limit Theorem. What will emerge as a result of our work is a new theory of probability, uniting and extending the classical theory, free probability theory, Nica’s theory of q-cumulants and Bozeiko-Speicher’s theory of q-random variables and a number of other scattered results where various authors encounter effects which are not covered by the existing theories and attempt to explain such effects ad hoc.

{\color{red} OMIT}Our proposal gives a firm and unified foundation to many such attempts.


\paragraph*{General Background}

Operator algebras provide a natural edifice to many areas of classical and modern Mathematics. Among them, a particular role is played by $C^{\ast}-$algebras (uniformly closed $\ast-$subalgebras in $B(H)$\footnote{$B(H)$ is the $\ast-$algebra of all bounded operators on a separable Hilbert space $H.$}) and von Neumann algebras (weakly closed subalgebras in $B(H)$). These additional structures make the algebras suitable for studying noncommutative analysis. 

Gelfand-Naimark theorem delivers a duality between category of compact topological spaces and that of commutative $C^{\ast}-$algebras. So, topology is incorporated into a theory of $C^{\ast}-$algebras. A similar result due to von Neumann theorem provides a duality between category of measure spaces and that of commutative von Neumann algebras. In other words, measure theory is a part of von Neumann algebras theory.

A conventional wisdom suggests now a dictionary between classical and quantum worlds.  While topology and measure theory are supplied with natural quantum counter-parts,  geometry was left behind for a long time. Starting from von Neumann himslef, people tried to find a quantum analogue of geometry (that is, to construct a functor from useful geometric categories to treatable quantum categories). So far, the most successful attempt is due to Connes who introduced the notion of spectral triple and promoted it as an analogue of Riemannian manifold as a convenient vehicle for new (noncommutative) geometry comprising as a very special case Riemannian manifolds.  

Notion of a Riemannian manifold is a pillar of a classical differential geometry. Usually, a manifold is defined as a topological space in which every point admits a neighborhood which is homeomorphic to a linear space. A manifold (denoted further by $X$) equipped with a smooth metric tensor (denoted further by $g$) is called Riemannian.

With every Riemannian manifold, one can associate (i) algebra $\mathcal{A}=C^{\infty}(X)$ of all smooth functions on $X;$ (ii) Hilbert space $H=L_2\Omega(X)$ of all square-integrable forms on $X;$ (iii) Hodge-Dirac operator $D$ on $L_2\Omega(X).$ The following properties of the triple $(\mathcal{A},H,D)$ are crucial:
\begin{enumerate}
\item if $\pi:\mathcal{A}\to B(H)$ is the representation of $\mathcal{A}$ by multiplication operators, then the commutator $[D,\pi(a)]$ is bounded for all $a\in\mathcal{A};$
\item the operator $\pi(a)(D+i)^{-1}$ is compact and its singular values decrease as $k^{-\frac1d},$ where $d={\rm dim}(X).$
\end{enumerate}
Recall that $-D^2$ is a Hodge-Laplace operator (denoted further by $\Delta_g$) and its component acting on $0$ order forms is Laplace-Beltrami operator (also denoted by $\Delta_g$) \cite{Rosenberg}.

A triple $(\mathcal{A},H,D)$ satisfying the conditions above is called spectral. Celebrated Reconstruction Theorem due to Connes dictates that every spectral triple with commutative $\mathcal{A}$ (satisfying few natural conditions) comes from a $d-$dimensional Riemannian manifold $X.$
{\color{blue} The following para could be improved and extended}
Hodge-Dirac operator allows an easy description of isometries of the manifold $X.$ If $\gamma:X\to X$ is an isometry and if $U:L_2\Omega(X)\to L_2\Omega(X)$ is a composition (with $\gamma$) operator, then $U$ is unitary, $U$ preserves $\mathcal{A}$ and commutes with $D.$ Conversely, every operator $U$ as above comes from an isometry $\gamma:X\to X.$


As is typical in applications, manifolds are equipped with isometric action of Lie groups. 


------------------------------------------

After the seminal work of Weyl \cite{Weyl}, it become a tradition to measure geometric (and often topological) quantities in terms of heat semi-group expansion. If $(X,g)$ is a compact Riemannian manifold, then resolvent of $\Delta_g$ (Laplace-Beltrami operator) is compact. Moreover, heat semi-group $e^{t\Delta_g}$ belongs to the trace class for every $t>0.$ For every $f\in C^{\infty}(X),$ Minakshisundaram-Plejel theorem\footnote{Among all approaches to Minakshisundaram-Plejel theorem, the most readable is given in \cite{Rosenberg}. Even though Theorem 3.24 there concerns a special case $f=1,$ the proof in general case is very similar.} \cite{Rosenberg} asserts an existence of an asymptotic expansion {\color{blue}$M_f$=?}
\begin{equation}\label{heat asymptotic}
{\rm Tr}(M_fe^{t\Delta_g})\approx (4\pi t)^{-\frac{d}{2}}\cdot \sum_{n\geq0}a_n(f)t^n,\quad t\downarrow0.
\end{equation}
Here, $d$ is the dimension of $X.$ Moreover, there exist functions $A_k\in C^{\infty}(X)$ such that
\begin{equation}\label{normality of coefficients}
a_k(f)=\int_X A_k\cdot f d{\rm vol}_g,
\end{equation}
where ${\rm vol}_g$ is the standard volume element on $X$ given in local coordinates by the formula
$$d{\rm vol}_g=({\rm det}(g))^{\frac12}(x)dx.$$

As established by Weyl, $A_0=1.$ Further computations show that
$$A_1=\frac16 R,$$
where $R$ is the scalar curvature. In particular, $a_1(1)$ is the Einstein-Hilbert action.

Note that $a_0$ extends to a normal state $h$ on $L_{\infty}(X)$ by the obvious formula
$$h(f)=\int_X fd{\rm vol}_g,\quad f\in L_{\infty}(X).$$
Equation \eqref{normality of coefficients} can be re-written as
$$a_k(f)=h(A_k\cdot f),\quad f\in C^{\infty}(X).$$

One of the {\it primary targets} in this project is to extend Minakshisundaram-Plejel theorem (consequently, Weyl theorem) to a vast class of non-commutative manifolds. This grand challenge started in \cite{ConnesTretkoff}, where conformal deformation of the $2-$dimensional non-commutative torus was considered. In \cite{ConnesTretkoff} authors demonstrated that the average of the curvature is $0$ (as suggested by the classical Gauss-Bonnet theorem). Later, curvature (for the conformal deformation of the $2-$dimensional non-commutative torus) was explicitly computed in \cite{ConnesMoscovici_curvature} and \cite{FathizadehKhalkhali}. Later, the term $a_4$ was computed in \cite{ConnesFathizadeh} (intermediate computations include about a million terms!). We briefly restate the whole programme below.  

For non-commutative manifolds, we no longer have such luxury as coordinate system. Suppose that a non-commutative manifold is equipped with a Laplacian $\Delta.$ The task now is to
\begin{enumerate}
\item\label{raph1} find a normal state $h$ on $\mathcal{A}''$ such that
\begin{equation}\label{raph1 eq}
{\rm Tr}(\pi(x)e^{t\Delta})\approx (4\pi t)^{-\frac{d}{2}} h(x),\quad t\downarrow0,
\end{equation}
for every $x\in\mathcal{A}'';$
\item\label{raph2} prove a non-commutative version of Minakshisundaram-Plejel theorem, i.e. generalise the formula \eqref{heat asymptotic} as follows
$${\rm Tr}(\pi(x)e^{t\Delta})\approx (4\pi t)^{-\frac{d}{2}} \sum_{n\geq0}a_n(x)t^n,\quad t\downarrow0,$$
for every $x\in\mathcal{A}'';$
\item\label{raph3} prove normality of coefficients as in \eqref{normality of coefficients} with respect to the volume state, i.e., show that
$$a_k(x)=h(xA_k),\quad x\in\mathcal{A}'',$$
for every $k\geq0$ and for some $A_k\in\mathcal{A};$ 
\item\label{raph4} compute respective $A_k;$
\end{enumerate}
When this mission is accomplished, one can {\it define} a scalar curvature of a non-commutative manifold by setting $R=6A_1.$

In contrast to the above mentioned works, our approach is based not on pseudo-differential calculus, but on Double Operator Integration technique. The particular integral representations  arose in \cite{Connes_team} when geometric measures on Julia sets were recovered by means of the singular traces. They form the core contribution of UNSW team (including the CI) to \cite{Connes_team}.



---------------------------------------------


One of the fundamental tools in noncommutative geometry is the Chern character. The Connes Character Formula (also known as the Hochschild character theorem) provides an expression for the class of the Chern character in Hochschild cohomology, and it is an important    tool in the computation of the Chern character. The formula has been applied to many areas     of noncommutative geometry and its applications: such as the local index formula \cite{ConnesMoscovici}, the spectral characterisation of manifolds \cite{Connes-reconstruction} and recent work in mathematical physics \cite{Connes-Chamseddine-Mukhanov-quanta-of-geometry-2015}. We especially emphasize its applications in \cite{Connes-reconstruction} as particularly relevant to the theme of this application.

In its original formulation, \cite{Connes-original-spectral-1995}, the Character Formula is stated as follows: Let $(\mathcal{A},H,D)$ be a $p$-dimensional compact spectral triple     with (possibly trivial) grading $\Gamma.$ By the definition of a spectral triple, for all $a \in \mathcal{A}$ the commutator $[D,a]$ has an extension to a bounded operator $\partial(a)$ on $H.$ Assume for simplicity that $\ker(D)=\{0\}$ and set $F={\rm sgn}(D).$ For all $a \in \mathcal{A}$ the commutator $[F,a]$ is a compact operator in the weak Schatten ideal $\mathcal{L}_{p,\infty}.$ 

Consider the following two linear maps on the algebraic tensor power $\mathcal{A}^{\otimes(p+1)},$ defined on an elementary tensor $c = a_0\otimes a_1\otimes \cdots \otimes a_p \in \mathcal{A}^{\otimes(p+1)}$ by setting
$$\mathrm{Ch}(c) := \frac{1}{2}\mathrm{Tr}(\Gamma F[F,a_0][F,a_1]\cdots[F,a_p]),\quad \Omega(c) := \Gamma a_0\partial a_1\partial a_2\cdots \partial a_p.$$
Then the Connes Character Formula states that if $c$ is a Hochschild cycle then
\begin{equation*}
\mathrm{Tr}_\omega(\Omega(c)(1+D^2)^{-p/2}) = \mathrm{Ch}(c)
\end{equation*}
for every Dixmier trace $\mathrm{Tr}_\omega$. In other words, the multilinear maps $\mathrm{Ch}$ and $c \mapsto \mathrm{Tr}_\omega(\Omega(c)(1+D^2)^{-p/2})$ define the same class in Hochschild cohomology.
    
There has been great interest in generalising the tools and results of noncommutative geometry to the \lq\lq non-compact\rq\rq (i.e., non-unital) setting. The definition of a spectral triple associated to a non-unital algebra originates with Connes \cite{Connes-reality}, was furthered by the work of Rennie \cite{Rennie}     and Gayral, Gracia-Bond\'ia, Iochum, Sch\"ucker and Varilly \cite{gayral-moyal}. Earlier, similar ideas appeared in the work of Baaj and Julg \cite{Baaj-Julg}. Additional contributions to this area were made by Carey, Gayral, Rennie and Sukochev\cite{CGRS}. The conventional definition  of a non-compact spectral triple is to replace the condition that $(1+D^2)^{-1/2}$ be compact with the assumption that for all $a \in \mathcal{A}$ the operator $a(1+D^2)^{-1/2}$ is compact.
    
This raises an important question: is the Connes Character Formula true for locally compact spectral triples? This question was suggested to the CI by Professor Connes himself during Shanghai conference celebrating 70-th anniversary of A. Connes (Fudan University). During this discussion, Professor Connes suggested that such an extension could be an excellent starting point for several new directions in non-commutative geometry. Firstly, since the Connes Character Formula plays such an important role in the reconstruction theorem for closed Rimannian manifolds, it is expected to play a similar role for locally compact Rimannian manifolds. Secondly, developing the proper techniques needed for a self-contained theory (such as locally compact spectral triples) should also open an avenue for treating the case of manifolds with boundary, in particular punctured manifolds. This suggestion by Professor Connes has been taken seriously by the CI and preliminary work in this direction has already brought substantial fruits. In particular, in the ground-breaking manuscript co-authored by Professor Connes and the CI (and also collaborators from UNSW) the new approach to spectral triples involving {\it symmetric, non-self-adjoint} operators has been propsed \cite{Connes_team_symmetric}. This is precisely the required tools allowing to develop new theory for Riemannian manifolds with boundary. This development indicate the importance and timeliness of the present proposal which {\bf proceed here}
    
    In this project we aim to answer this question, using recently developed techniques of operator integration.
    

------------------------------------------

\paragraph*{Specific aims} There are 6 specific aims.


\noindent{\bf Aim 1:} Investigate when Chern character provides an asymptotic expansion for the heat semi-group. More precisely, when 
\begin{equation}\label{heat eq}
{\rm Tr}(\Omega(c)e^{-s^2D^2})={\rm Ch}(c)s^{-p}+O(s^{1-p}),\quad s\downarrow0,
\end{equation}
for every Hochschild cycle $c\in\mathcal{A}^{\otimes (p+1)}?$ This question is closely related (albeit, not equivalent) to the question about analyticity of the $\zeta-$function. Show that the function (defined a priori for $\Re(z)>p$)
$$z\to {\rm Tr}(\Omega(c)(1+D^2)^{-\frac{z}{2}})$$
admits an analytic extension to the half-plane $\Re(z)>p-1$ so that
$$\lim_{z\to p}(z-p){\rm Tr}(\Omega(c)(1+D^2)^{-\frac{z}{2}})=p{\rm Ch}(c)?$$


\noindent{\bf Aim 2:} The purpose of the Connes Character Formula is to compute the Hochschild class of the Chern character by a "local" formula, here stated in terms of singular traces. Show that
\begin{equation}\label{super main eq}
\varphi(\Omega(c)(1+D^2)^{-\frac{p}{2}})={\rm Ch}(c).
\end{equation}
for every (normalised) trace $\varphi$ on $\mathcal{L}_{1,\infty}$ and for every Hochschild cycle $c\in\mathcal{A}^{\otimes (p+1)}.$ Equivalently, show that
$$\sum_{k=0}^n \lambda(k,\Omega(c)(1+D^2)^{-p/2}) = \mathrm{Ch}(c)\log(n)+O(1),\quad n\to\infty.$$
Here, $\lambda(k,T)$ means the $k-$th eigenvalue (counted with algebraic multiplicity) of a compact operator $T.$

\noindent{\bf Aim 3:} Introduce a generic (i.e., not necessarily a conformal deformation of a flat one) Laplace-Beltrami operator $\Delta$ on the non-commutative torus and non-commutative Euclidean space. Prove a non-commutative version of Minakshisundaram-Plejel theorem (for these manifolds) as outlined above.

\noindent{\bf Aim 4:} Compute explicitly the curvature for a generic Riemannian metric on the non-commutative torus and non-commutative Euclidean space.

\noindent{\bf Aim 5:} Construct a spectral triple (possibly, twisted one) on quantum groups (like ${\rm SU}_q(n)$) and on their homogeneous spaces (like Podles sphere). We aim to have an equivariance property for the Dirac operator in a way that prevents "dimension drop" pathology.

\noindent{\bf Aim 6:} Create a version of Connes Character Formula for the above spectral triples which recovers a non-trivial cocycle.

------------------------------------------

\paragraph*{Approach to specific aims} Here we describe methods in our possession to resolve the problems stated above.

{\bf Approach to Aim 2:} The $\zeta-$function, whose analyticity should be proved in Aim 1 is of the shape $z\to {\rm Tr}(CB^z).$ We have $C=AC$ (hence $C=A^zC$) for a suitable $A$ and, therefore,
$${\rm Tr}(CB^z)={\rm Tr}(CB^zA^z).$$
It is desirable to replace $B^zA^z$ with $(A^{\frac12}BA^{\frac12})^z.$ For this purpose, we use the integral representation
$$B^zA^z-(A^{\frac{1}{2}}BA^{\frac{1}{2}})^z = T_z(0)-\int_{\mathbb{R}} T_z(s)\widehat{g}_z(s)\,ds,$$ 
where $(s,z)\to\widehat{g}_z(s)$ is a sufficiently good scalar-valued function and where
$$T_z(s)= B^{z-1+is}[BA^{\frac{1}{2}},A^{z-\frac{1}{2}+is}]Y^{-is}+B^{is}[BA^{\frac{1}{2}},A^{\frac{1}{2}+is}]Y^{z-1-is}.$$
By using Double Operator Integration technique as developed in \cite{PotapovSukochev}, we aim to prove analyticity of the right hand side and, hence, of the left hand side.

Having established analyticity of the function
$$z\to {\rm Tr}(C(A^{\frac12}BA^{\frac12})^z),\quad \Re(z)>p-1,$$
with simple pole at $z=p,$ we expect that methods from \cite{SUZ-indiana} will lead us to the formula
$$\varphi(CA^{\frac12}BA^{\frac12})=\frac1p\lim_{z\to p}(z-p){\rm Tr}(C(A^{\frac12}BA^{\frac12})^z)$$
for every (normalised) trace on $\mathcal{L}_{1,\infty}.$

{\bf Approach to Aim 3:} In the definition of Laplace-Beltrami operator, one observes a determinant of a matrix-valued function. For a non-coimmutative manifold, matrix-valued function is replaced with a matrix whose components belong to a (usually non-commutative) von Neumann algebra. For such a matrices, there exists a surrogate notion of determinant (due to Fuglede and Kadison). Though it is possible to define Laplace-Beltrami operator using Fuglede-Kadison determinant, our computations show an inconsistency in the formula \eqref{raph1 eq} above.

In fact, we propose to sacrifice the property "determinant is homomorphism" for being able to perform computations making the formula \eqref{raph1 eq} consistent with the definition of Laplace-Beltrami operator. We propose to set 
$$G^{-\frac12}=({\rm det}(g_{ij}))^{-\frac12}\stackrel{def}{=}\pi^{-\frac{d}{2}}\int_{\mathbb{R}^d}e^{-\sum_{i,j}g_{ij}t_it_j}dt.$$

Now, let $h(x)=\tau(xG^{\frac12})$ and consider an inner product $(x,y)\to h(xy^*)$ on the non-commutative torus. Define Laplace-Beltrami operator by the formula
$$-\Delta_g x=M_{G^{-\frac12}}\sum_{i,j=1}^dD_iM_{G^{\frac14}(g^{-1})_{ij}G^{\frac14}}D_j.$$
Our computations show that so-defined operator is self-adjoint (and positive) and that formula \eqref{raph1 eq} becomes consistent.

{\bf Approach to Aim 4:} In the special case of conformal deformation, Laplace-Beltrami operator is (unitarily equivalent to) $M_h\Delta M_h,$ where $\Delta$ is the flat Laplacian. According to the asymptotics in Aim 3, the function
$$z\to {\rm Tr}(M_x(-M_h\Delta M_h)^{-z})$$
admits an analytic extension with at most simple poles at $z=\frac{d}{2},\frac{d}{2}-1,\cdots.$ The curvature term is provided (for $d>2$) by the residue of this function at the point $\frac{d}{2}-1.$ Our function has a shape
$$z\to {\rm Tr}(C(A^{\frac12}BA^{\frac12})^z).$$
It is desirable to replace $(A^{\frac12}BA^{\frac12})^z$ with $B^zA^z.$ Here, we propose to use an integral representation similar to (but more complicated than) the one specified in  the Approach to Aim 2.

In the case of general metric tensor, formulae become much harder and more investigation is required.

\bigskip\archeading{Future Fellowship Candidate}

\begin{thebibliography}{99}
\setlength{\itemsep}{0 pt}
\setlength{\parskip}{0pt}
\setlength{\parsep}{0pt}
\bibitem{Baaj-Julg} Baaj, S., Julg, P. {\it Th\'eorie bivariante de Kasparov et op\'erateurs non born\'es dans les $C^*$-modules hilbertiens.} C. R. Acad. Sci. Paris S\'er. I Math. {\bf 296} (1983), no. 21, 875--878.
\bibitem{CGRS} Carey A., Gayral V., Rennie A., Sukochev F. {\it Integration on locally compact noncommutative spaces.} J. Funct. Anal. {\bf 263} (2012), no. 2, 383--414; Carey A., Gayral V., Rennie A., Sukochev F. {\it Index theory for locally compact noncommutative geometries.} Mem. Amer. Math. Soc. {\bf 231} (2014), no. 1085, vi+130 pp.
\bibitem{Connes-original-spectral-1995} Connes A. {\it Geometry from the spectral point of view.} Lett. Math. Phys., {\bf 34} (3) 203--238, 1995.
\bibitem{Connes-reality} Connes A., {\it Noncommutative geometry and reality.} J. Math. Phys. {\bf 36} (1995), 6194–-6231 .
\bibitem{Connes-reconstruction} Connes A. {\it On the spectral characterization of manifolds.} J. Noncommut. Geom. {\bf 7} (2013), no. 1, 1--82.
\bibitem{Connes-Chamseddine-Mukhanov-quanta-of-geometry-2015} Chamseddine, A., Connes, A., Mukhanov, V. {\it Quanta of geometry: noncommutative aspects.} Phys. Rev. Lett. {\bf 114} (2015), no. 9, 091302, 5pp.
\bibitem{ConnesFathizadeh} Connes A., Fathizadeh F. {\it The term $a_4$ in the heat kernel expansion of noncommutative tori.} arXiv:1611.09815
\bibitem{Connes_team_symmetric} Connes A., Levitina G., McDonald E., Sukochev F., Zanin D. {\it Noncommutative Geometry for Symmetric Non-Self-Adjoint Operators.} arXiv:1808.01772 
\bibitem{ConnesMoscovici} Connes A., Moscovici H. {\it The local index formula in noncommutative geometry.} Geom. Funct. Anal. {\bf 5} (1995), no. 2, 174--243.
\bibitem{ConnesMoscovici_curvature} Connes A., Moscovici H. {\it Modular curvature for noncommutative two-tori.} J. Amer. Math. Soc. {\bf 27} (2014), no. 3, 639--684.
\bibitem{Connes_team} Connes A., Sukochev F., Zanin D. {\it Trace theorem for quasi-Fuchsian groups.} Mat. Sb. {\bf 208} (2017); Connes A., McDonald E., Sukochev F., Zanin D {\it Conformal trace theorem for Julia sets of quadratic polynomials.} Ergodic Theory Dyn. Syst. Published online: 04 December 2017. https://doi.org/10.1017/etds.2017.124
\bibitem{ConnesTretkoff} Connes A., Tretkoff P. {\it The Gauss-Bonnet theorem for the noncommutative two torus.} Noncommutative geometry, arithmetic, and related topics, 141--158, Johns Hopkins Univ. Press, Baltimore, MD, 2011.
\bibitem{FathizadehKhalkhali} Fathizadeh F., Khalkhali M. {\it Scalar curvature for the noncommutative two torus.} J. Noncommut. Geom. {\bf 7} (2013), no. 4, 1145--1183; Fathizadeh F., Khalkhali M. {\it Scalar curvature for noncommutative four-tori.} J. Noncommut. Geom. {\bf 9} (2015), no. 2, 473--503.
\bibitem{gayral-moyal} Gayral V., Gracia-Bondia J., Iochum B., Sch\"ucker T., Varilly J. {\it Moyal planes are spectral triples.} Comm. Math. Phys. {\bf 246} (2004), no. 3, 569--623.
\bibitem{PotapovSukochev} Potapov D., Sukochev F. {\it Operator-Lipschitz functions in Schatten-von Neumann classes.} Acta Math. {\bf 207} (2011), no. 2, 375--389; Potapov D., Sukochev F. {\it Unbounded Fredholm modules and double operator integrals.} J. Reine Angew. Math. {\bf 626} (2009), 159--185.
\bibitem{Rennie} Rennie A. {\it Smoothness and locality for nonunital spectral triples.} K-Theory {\bf 28} (2003), no. 2, 127-–165; Rennie A. {\it Summability for nonunital spectral triples. } K-Theory {\bf 31} (2004), no. 1, 71--100.
\bibitem{Rosenberg} Rosenberg S. {\it The Laplacian on a Riemannian manifold. An introduction to analysis on manifolds.} London Mathematical Society Student Texts, {\bf 31}. Cambridge University Press, Cambridge, 1997.
\bibitem{SUZ-indiana} Sukochev F., Usachev A., Zanin D. {\it Singular traces and residues of the $\zeta$-function.} Indiana U. Math. J., {\bf 66} (2017), no. 4, 1107--1144.
\bibitem{Weyl} Weyl H. {\it Das asymptotische Verteilungsgesetz der Eigenwerte linearer partieller Differentialgleichungen (mit einer Anwendung auf die Theorie der Hohlraumstrahlung).} Math. Ann. {\bf 71} (1912), no. 4, 441--479. 

%\bibitem{AS_embed} Astashkin S., Sukochev F. {\it Orlicz sequence spaces spanned by identically distributed independent random variables in $L_p-$spaces.} J. Math. Anal. Appl. {\bf 413} (2014), no. 1, 1--19.
%\bibitem{JS_ineq} Johnson W., Schechtman G. {\it Sums of independent random variables in rearrangement invariant function spaces.} Ann. Probab. {\bf 17}, No.2, 789--808 (1989).
%\bibitem{Junge_poisson} Junge M. {\it Noncommutative Poisson process} to appear
%\bibitem{Junge_prokhorov} Junge M., Zeng Q. {\it Noncommutative Bennett and Rosenthal inequalities.} Ann. Probab. {\bf 41} (2013), no. 6, 4287--4316.
%\bibitem{LSZ} Lord S., Sukochev F., Zanin D. {\it Singular traces. Theory and applications.} De Gruyter Studies in Mathematics, {\bf 46}. De Gruyter, Berlin, 2013. xvi+452 pp. 
%\bibitem{Nicaqconv} Nica A. {\it A one-parameter family of transforms, linearizing convolution laws for probability distributions.} Comm. Math. Phys. {\bf 168} (1995), no. 1, 187--207.
%\bibitem{Nica_Speicher} Nica A., Speicher R. {\it Lectures on the combinatorics of free probability.} London Mathematical Society Lecture Note Series, Vol. 335 Cambridge University Press, 2006.
%\bibitem{Nica_Speicher_constr} Nica A., Speicher R. {\it On the multiplication of free N-tuples of noncommutative random variables.} Am. J. Math. {\bf 118}, No.4, 799--837 (1996).
%\bibitem{Oravecz} Oravecz F. {\it Nica's $q-$convolution is not positivity preserving.} Comm. Math. Phys. {\bf 258} (2005), no. 2, 475--478.
%\bibitem{RS} Raynaud Y., Sch\"utt C. {\it Some results on symmetric subspaces of $L_1.$} Studia Math. {\bf 89} (1988), no. 1, 27--35.
%\bibitem{Speicher_bad} Speicher R. {\it On universal products.} Fields Institute Communications, Vol. 12 (D. Voiculescu, ed.), AMS, 1997, pp. 257--266.
%\bibitem{SZ_free} Sukochev F., Zanin D. {\it Johnson-Schechtman inequalities in the free probability theory.} J. Funct. Anal. {\bf 263}, No. 10, 2921--2948 (2012).
%\bibitem{VDN} Voiculescu D., Dykema K., Nica A. {\it Free random variables. A noncommutative probability approach to free products with applications to random matrices, operator algebras and harmonic analysis on free groups.} CRM Monograph Series. 1. Providence, RI: American Mathematical Society (AMS). v, 70 p. (1992).
%\bibitem{Zanin_th} Zanin D. {\it Orbits and Khinchine-type inequalities in symmetric spaces}. PhD thesis, 2011, Flinders University.
\end{thebibliography}

\end{document}

