\documentclass[12pt]{article}
\usepackage{amsmath,amsfonts,cite,color}
\usepackage[a4paper,margin=0.5cm]{geometry}
\usepackage{mathptmx}

\newtheorem{thm}{Theorem}

\renewcommand{\baselinestretch}{0.95}
\newcommand{\archeading}[1]{\vspace{.3cm} \noindent{\bfseries #1} \vspace{.1cm}   }
\renewcommand{\refname}{{\normalsize References}}

\begin{document}

\archeading{B8. Research Opportunity and Performance Evidence (ROPE) - Details of the Future Fellowship candidate's career, evidence of research impact and contributions to the field, including those most relevant to this application and evidence of leadership, mentoring and research training or supervision.}

\archeading{Amount of time as an active researcher} I was awarded my PhD in Mathematics 12 years ago in 2011 at Flinders University of South Australia. During those 12 years, I have had no interruptions in research opportunity.

\archeading{Research opportunities} I am employed at UNSW as a Senior Lecturer in the School of Mathematics and Statistics. 

Currently, I hold ARC DP 230100434 ({\color{red} insert money} started in 2023, resulted in 14 publications). In 2018-2021, I held UNSW Scientia Fellowship (\$ 160,000, resulted in 35 publication publications and promotion to a permanent position). In 2015-2018, I held ARC DECRA (\$ 300 000, resulted in 29 publications). 

From 2011 until June 2015 I was employed as a researcher under DP grants DP110100064 and DP120103263 awarded to F. Sukochev (resulted in 17 publications).

From 2007 until December 2010, I was a PhD student at Flinders University.

%I contributed subsequently to both DP110100064 and DP120103263. The positions under Prof. Sukochev afforded me the opportunity to extend my areas of interest and expertise in noncommutative probability and functional analysis to noncommutative geometry, and engage internationally (Germany, USA, Russia, China, Poland) through invited visits and attendance at workshops such as Oberwolfach and Bedlewo. Without a permanent position I have been unable to apply for funding under the ARC Discover Project scheme directly.

%A number of brilliant mathematicians have made, and continue to make, a significant impact on my research career. While completing my undergraduate study in Tashkent, Uzbekistan, I was impressed by Professor Chilin's attitude to Mathematics. It would be fair to say that following his example I decided to be a mathematician. My PhD supervisor Professor Sukochev taught me the foundations of non-commutative integration theory, but also an approach to Mathematics involving strict work ethics, enthusiasm and perseverance. From 2009, I was introduced to the problems on singular traces and was fortunate to work with Professor Nigel Kalton, a giant of functional analysis and Banach Medal winner. I was positively shocked by the depth and speed of this great thinker, whose insight and technical ability dwarfed everyone else I had encountered. The impression is with me still of the modesty of his genius, he was gentle and patient with those who were not grasping the details as quickly as himself. The technical devices introduced by Professor Kalton subsequently inspired my own inventions in this area. Since 2013, I had an opportunity to collaborate with Professor Dykema during my visits to Texas A\&M University and his reciprocal visits to UNSW. Professor Dykema introduced me to the advanced techniques of non-commutative probability which helped me a lot during my DECRA project. From 2015, I regularly collaborate with Professor Connes, a Fields Medal and Crafoord Prize winner, both are equivalents to the Nobel Prize in Mathematics, so needless to say this collaboration has impressed and inspired me deeply. % Work with Professor Connes taught me how to merge a different, seemingly unrelated, areas of Mathematics into a single project.

\archeading{Research achievements and contributions} I work in a broad area of Functional Analysis, more precisely, in (a) theory of singular traces (b) operator theory (c) probability theory (d) harmonic analysis (e) applications to mathematical physics. Below is a detailed {\color{red} bad adjective} account of my research history and some of the most important contributions made in my career.

In 2021-2023, I published (in collaboration with S. Lord, E. McDonald and F. Sukochev) two-volume book "Singular traces" (vol. 1 "Theory" and vol. 2 "Trace Formulas"). This book is based on my previous journal publications and constitutes my contribution to the {\it Theory of Singular Traces.} In this area, I solved a long-standing problem by A. Pietsch on the spectrality of traces. My works in the area are published in e.g. in Crelle's Journal and Advances in Mathematics.

Starting from 2015, I published a number of articles in {\it Non-commutative Probability Theory}. In this area, I developed a new framework of distributional inequalities which allowed final solution to problems proposed by M. Junge and Q. Xu. Publication in this area appear e.g. in the Journal of Functional Analysis. 

I intensively work in the field of {\it Non-commutative Geometry} (collaborating with A. Connes and N. Higson). My publications in the area appear e.g. in Communications in Mathematical Physics and the Journal of Functional Analysis. Among my achievements in this direction is the resolution of the problem proposed by A. Connes in 1994. This area of my research interests is closely related to the proposed Future Fellowship project.

Another area of my interest is {\it Operator Theory}. I obtained an ultimate resolution of the 60-year old problem by M. Krein on operator Lipschitz functions. My works in that area are published e.g. in the  American Journal of Mathematics, Crelle's Journal and Proceedings of the London Mathematical Society.

I am also interested in the {\it Non-commutative Harmonic Analysis} (collaborating  with R. Frank). My papers in this area are published e.g. in Transactions of the American Mathematical Society and in Mathematische Annalen. Some of my works in that direction are related to the proposed Future Fellowship project.

Another direction of my research is {\it Mathematical Physics} (collaborating with F. Gesztesy). My works in that area are published e.g. in the Memoirs of the European Mathematical Society.

I am one of the top contributors to the Journal of Functional Analysis (second in the last 3 years, top 10 in all times). 

I am regularly invited to deliver keynote research talks in the top international conferences. In 2021, I was invited by Professor Connes to deliver a plenary lecture concerning my recent results at the international conference "Cyclic cohomology at 40" at Fields Institute. In 2023 I delivered a plenary lecture in the international conferences "Noncommutative Geometry, Index Theory and Representation Theory" in Kyoto.

I am regular assessor for the submissions to the best international journals such as Journal of Functional Analysis, Advances in Mathematics, Duke Mathematical Journal and Transactions of the American Mathematical Society.

I frequently collaborate with a number of the highest calibre researchers in Australia and overseas (Europe, US, China). My list of collaborators includes Alain Connes (Paris), Kenneth Dykema (Texas A\&M), Rupert Frank (Munich), Friedrich Gesztesy (Vienna), Nigel Higson (Penn State), Marius Junge (Illinois), Albrecht Pietsch (Jena), Fedor Sukochev (UNSW) and Quanhua Xu (Harbin). 


Many of these research achievements underpin the research proposed for this Future Fellowship, and demonstrate my insight and ability in solving hard problems in mathematics using novel methods that often extend to new results in considerably more general contexts.

\end{document}
